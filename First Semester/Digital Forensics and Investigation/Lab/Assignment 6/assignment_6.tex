% This is a Basic Assignment Paper but with like Code and stuff allowed in it, there is also url, hyperlinks from contents included. 

\documentclass[11pt]{article}

% Preamble

\usepackage[margin=1in]{geometry}
\usepackage{amsfonts, amsmath, amssymb, amsthm}
\usepackage{fancyhdr, float, graphicx}
\usepackage[utf8]{inputenc} % Required for inputting international characters
\usepackage[T1]{fontenc} % Output font encoding for international characters
\usepackage{fouriernc} % Use the New Century Schoolbook font
\usepackage[nottoc, notlot, notlof]{tocbibind}
\usepackage{listings}
\usepackage{xcolor}
\usepackage{blindtext}
\usepackage{hyperref}
\definecolor{codepurple}{rgb}{0.58,0,0.952}
\hypersetup{
    colorlinks=true,
    linkcolor=black,
    filecolor=black,      
    urlcolor=codepurple,
    pdfpagemode=FullScreen,
    }

\definecolor{codegreen}{rgb}{0,0.6,0}
\definecolor{codegray}{rgb}{0.5,0.5,0.5}
\definecolor{backcolour}{rgb}{0.95,0.95,0.92}

\lstdefinestyle{mystyle}{
    backgroundcolor=\color{backcolour},   
    commentstyle=\color{codegreen},
    keywordstyle=\color{magenta},
    numberstyle=\tiny\color{codegray},
    stringstyle=\color{codepurple},
    basicstyle=\ttfamily\footnotesize,
    breakatwhitespace=false,         
    breaklines=true,                 
    captionpos=b,                    
    keepspaces=true,                 
    numbers=left,                    
    numbersep=5pt,                  
    showspaces=false,                
    showstringspaces=false,
    showtabs=false,                  
    tabsize=2
}

\lstset{style=mystyle}

% Header and Footer
\pagestyle{fancy}
\fancyhead{}
\fancyfoot{}
\fancyhead[L]{\textit{\Large{Digital Forensics and Investigation - TY. B. Tech}}}
\fancyhead[R]{\textit{Krishnaraj T}}
\fancyfoot[C]{\thepage}
\renewcommand{\footrulewidth}{1pt}
\newtheorem{thm}{Theorem}
\newtheorem{dfn}[thm]{Definition}


% Other Doc Editing
% \parindent 0ex
%\renewcommand{\baselinestretch}{1.5}

\begin{document}

\begin{titlepage}
    \centering

    %---------------------------NAMES-------------------------------

    \huge\textsc{
        MIT World Peace University
    }\\

    \vspace{0.75\baselineskip} % space after Uni Name

    \LARGE{
        Digital Forensics and Investigation\\
        Third Year B. Tech, Semester 5
    }

    \vfill % space after Sub Name

    %--------------------------TITLE-------------------------------

    \rule{\textwidth}{1.6pt}\vspace*{-\baselineskip}\vspace*{2pt}
    \rule{\textwidth}{0.6pt}
    \vspace{0.75\baselineskip} % Whitespace above the title



    \huge{\textsc{
            Audio and Image File Integrity Analysis
        }} \\



    \vspace{0.5\baselineskip} % Whitespace below the title
    \rule{\textwidth}{0.6pt}\vspace*{-\baselineskip}\vspace*{2.8pt}
    \rule{\textwidth}{1.6pt}

    \vspace{1\baselineskip} % Whitespace after the title block

    %--------------------------SUBTITLE --------------------------	

    \LARGE\textsc{
        Lab Assignment 6
    } % Subtitle or further description
    \vfill

    %--------------------------AUTHOR-------------------------------

    Prepared By
    \vspace{0.5\baselineskip} % Whitespace before the editors

    \Large{
        Krishnaraj Thadesar \\
        Cyber Security and Forensics\\
        Batch A1, PA 20
    }


    \vspace{0.5\baselineskip} % Whitespace below the editor list
    \today

\end{titlepage}


\tableofcontents
\thispagestyle{empty}
\clearpage

\setcounter{page}{1}

\section{Aim}
Write a program for identifying tampering in either image or voice data. Use big data as input.

\section{Objectives}
\begin{enumerate}
    \item To learn about audio and image file integrity. 
    \item To learn ways to secure and make sure integrity of audio and image files is maintained. 
\end{enumerate}
\section{Theory}

\subsection{How to Secure Audio Files?}

\begin{enumerate}
    \item Digital Rights Management (DRM) - This is a technology that controls the use and distribution of digital files, such as audio files. DRM can be used to restrict the number of times a file can be played, or to prevent the file from being copied or shared.
    \item Encryption - This is a method of converting the audio file into a code that can only be accessed by those who have the decryption key. This method can be used to protect the file from unauthorized access or copying.
    \item Watermarking - This is a method of adding a unique identifying mark, such as a logo or text, to the audio file. Watermarking can be used to trace the origin of an unauthorized copy of the file.
    \item Copyright Notice - This is a legal notice that informs others that the audio file is protected by copyright law and that unauthorized use of the file is prohibited.
    \item Limited distribution - Keep the distribution of the audio file limited to a specific group or individuals, this can be achieved by sending the audio file via email or cloud storage with restricted access.
    \item Use of a Licensing agreement- Having a licensing agreement in place can set specific terms and conditions on how the audio file can be used, this will help you protect your audio file from unauthorized use.
\end{enumerate}

It is still worth noting that none of these methods can completely prevent unauthorized access or distribution of an audio file, but they can make it more difficult and less attractive for people to do so. It's also important to register your audio file with the Copyright Office for additional legal protection.

\subsection{What is Steganography}
Steganography is the practice of concealing information within other non-secret data, often within media files such as images, audio, or video. This covert technique allows you to hide a message or data within a seemingly innocuous carrier file.

\subsection{Hashing and Methods to check Integrity}
Hashing is a process of converting data into a fixed-size string of characters, which is typically a hexadecimal number. It is used to verify data integrity, detect tampering, and ensure data consistency.

\begin{enumerate}
    \item Checksums: Checksums are simple hashes used to verify the integrity of data. Common examples include CRC32 and MD5. For example, you can generate an MD5 checksum for a file and compare it with the original checksum to detect changes.

    \item Cryptographic Hash Functions: Cryptographic hash functions, like SHA-256 (Secure Hash Algorithm 256-bit), are widely used for secure data verification. They produce unique, fixed-size hashes. For instance, SHA-256 is often used in blockchain technology to validate transactions.

    \item HMAC (Hash-Based Message Authentication Code): HMAC is a technique that combines a cryptographic key with a hash function. It's used for data authentication. An example is HMAC-SHA-256, which adds a secret key to the hashing process.

    \item Digital Signatures: Digital signatures use asymmetric cryptography to hash data and encrypt the hash with a private key. The recipient can decrypt the hash with the sender's public key and compare it to a freshly computed hash. This ensures both integrity and authenticity.

    \item File Verification Tools: Various software tools, such as the `sha256sum` command in Unix-based systems, calculate and compare hashes for file integrity.

\end{enumerate}

\section{Analysis and Experiment}

\subsection{Performing Steganography for Audio and Image Files}

\subsubsection{Audio Files}

Shown here is the audio file 'Alesis-Fusion-Clean-Guitar-C3.wav' that was used to perform this analysis. 

\begin{figure}[H]
    \centering
    \includegraphics[width=.95\textwidth]{audio file.png}
    \caption{Audio File as seen on Sonic Visualizer. }
\end{figure}

\subsubsection{Encrypting Data in the Audio File. }

This was done using a python package called 'stegolsb'.

\begin{figure}[H]
    \centering
    \includegraphics[width=.95\textwidth]{./image.png}
    \caption{Encrypting a Message using Stegolsb}
\end{figure}

\begin{figure}[H]
    \centering
    \includegraphics[width=.95\textwidth]{./audio setg.png}
    \caption{Decrypting a Message using Stegolsb}
\end{figure}

\subsubsection{Text files involved}

Shown here are 2 files. 
\begin{enumerate}
    \item The First file shows the file.txt that was used to encrypt the message. 
    \item The second file, output.txt is where the text output was given. 
\end{enumerate}
\begin{figure}[H]
    \centering
    \includegraphics[width=.95\textwidth]{./text got and given.png}
    \caption{Decrypting a Message using Stegolsb}
\end{figure}

\subsubsection{Image File}

\subsection{Analysing the Differences using Software}

Shown below is the when both the files were opened side by side in sonic visualizer, and no visible difference between the 2 files was noted. The changes were present when checked with hashes, but not visible visually. 

\begin{figure}[H]
    \centering
    \includegraphics[width=.95\textwidth]{no change in sonic.png}
    \caption{This shows that there is no change in both files visually when seen side by side. }
\end{figure}

\section{Platform}
\textbf{Operating System}: Arch Linux x86-64 \\
\textbf{IDEs or Text Editors Used}: Visual Studio Code\\
\textbf{Compilers or Interpreters}: Python 3.12\\

\section{Code}
\lstinputlisting[language=Python, caption="Python Script to Verify Hash"]{./sha.py}

\begin{lstlisting}[language=bash, caption={Output}]
Welcome to DFI Assignment 6, Krishnaraj PT. PA10. CSF
File Modified Detection
First Music File without Tampering: 
Alesis-Fusion-Clean-Guitar-C3.wav
Hash is:  242fff54feface4f9ac8b51887deb4eee052f3d7
Music File after deleting portion of it : 
Hash is:  89755f2b3b0438750818c1ae5fd26a93624cece8
sound_steg.wav
====================================================================================
Image File
Hash of the file is: 
b91bf2a1dd825725f788a7e204cfd38d0a4badd1bf3745ec2efceeb4e3cab591
Modified the File, added hidden data to it.
Hash of the file now is:
d02d021cc9c7771339327216c82540b1796ce1f2d715797907c97ee7c24788e8
\end{lstlisting}

\section{Conclusion}
Thus, we have successfully Analysed the differences between the audio and image file when it is modified. 
\begin{enumerate}
    \item It was found that when an audio file is changed using steganography, the Hash changes. 
    \item A Similar change was noticed when a message was encrypted in an image file as well. 
    \item It was also seen that the wave file of the audio was also modified. The changes were also visually invisible. This was done using sonic visualizer. 
\end{enumerate}
\clearpage

\pagebreak
\begin{thebibliography}{}
    \bibitem{A Novel Steganography Approach for Audio Files}
    \href{https://link.springer.com/article/10.1007/s42979-020-0080-2#:~:text=Therefore%2C%20in%20steganography%20the%20original,6%2C7%2C8%5D.}{A Novel Steganography Approach for Audio Files}\\
    Sazeen T. Abdulrazzaq, Mohammed M. Siddeq, Marcos A. Rodrigues

    \bibitem{https://www.sonicvisualiser.org/download.html}
    \href{https://www.sonicvisualiser.org/download.html}{Sonic Visualizer}

    \bibitem{Documentation for Test Disk Data Recovery Software}
    \href{https://sumit-arora.medium.com/audio-steganography-the-art-of-hiding-secrets-within-earshot-part-2-of-2-c76b1be719b3}{Audio Steganography : The art of hiding secrets within earshot} \\Sumit Kumar Arora

    \bibitem{https://github.com/ragibson/Steganography}
    \href{https://github.com/ragibson/Steganography}{stego-lsb package from python. }

    
\end{thebibliography}
\end{document}