% This is a Basic Assignment Paper but with like Code and stuff allowed in it, there is also url, hyperlinks from contents included. 

\documentclass[11pt]{article}

% Preamble

\usepackage[margin=1in]{geometry}
\usepackage{amsfonts, amsmath, amssymb, amsthm}
\usepackage{fancyhdr, float, graphicx}
\usepackage[utf8]{inputenc} % Required for inputting international characters
\usepackage[T1]{fontenc} % Output font encoding for international characters
\usepackage{fouriernc} % Use the New Century Schoolbook font
\usepackage[nottoc, notlot, notlof]{tocbibind}
\usepackage{listings}
\usepackage{xcolor}
\usepackage{blindtext}
\usepackage{hyperref}
\definecolor{codepurple}{rgb}{0.58,0,0.82}
\hypersetup{
    colorlinks=true,
    linkcolor=black,
    filecolor=black,      
    urlcolor=codepurple,
    pdfpagemode=FullScreen,
    }

\definecolor{codegreen}{rgb}{0,0.6,0}
\definecolor{codegray}{rgb}{0.5,0.5,0.5}
\definecolor{backcolour}{rgb}{0.95,0.95,0.92}

\lstdefinestyle{mystyle}{
    backgroundcolor=\color{backcolour},   
    commentstyle=\color{codegreen},
    keywordstyle=\color{magenta},
    numberstyle=\tiny\color{codegray},
    stringstyle=\color{codepurple},
    basicstyle=\ttfamily\footnotesize,
    breakatwhitespace=false,         
    breaklines=true,                 
    captionpos=b,                    
    keepspaces=true,                 
    numbers=left,                    
    numbersep=5pt,                  
    showspaces=false,                
    showstringspaces=false,
    showtabs=false,                  
    tabsize=2
}

\lstset{style=mystyle}

% Header and Footer
\pagestyle{fancy}
\fancyhead{}
\fancyfoot{}
\fancyhead[L]{\textit{\Large{Digital Forensics and Investigation - TY. B. Tech}}}
\fancyhead[R]{\textit{Krishnaraj T}}
\fancyfoot[C]{\thepage}
\renewcommand{\footrulewidth}{1pt}
\newtheorem{thm}{Theorem}
\newtheorem{dfn}[thm]{Definition}


% Other Doc Editing
% \parindent 0ex
%\renewcommand{\baselinestretch}{1.5}

\begin{document}

\begin{titlepage}
    \centering

    %---------------------------NAMES-------------------------------

    \huge\textsc{
        MIT World Peace University
    }\\

    \vspace{0.75\baselineskip} % space after Uni Name

    \LARGE{
        Digital Forensics and Investigation\\
        Third Year B. Tech, Semester 5
    }

    \vfill % space after Sub Name

    %--------------------------TITLE-------------------------------

    \rule{\textwidth}{1.6pt}\vspace*{-\baselineskip}\vspace*{2pt}
    \rule{\textwidth}{0.6pt}
    \vspace{0.75\baselineskip} % Whitespace above the title



    \huge{\textsc{
            Analysing Home Router Logs
        }} \\



    \vspace{0.5\baselineskip} % Whitespace below the title
    \rule{\textwidth}{0.6pt}\vspace*{-\baselineskip}\vspace*{2.8pt}
    \rule{\textwidth}{1.6pt}

    \vspace{1\baselineskip} % Whitespace after the title block

    %--------------------------SUBTITLE --------------------------	

    \LARGE\textsc{
        Lab Assignment 2
    } % Subtitle or further description
    \vfill

    %--------------------------AUTHOR-------------------------------

    Prepared By
    \vspace{0.5\baselineskip} % Whitespace before the editors

    \Large{
        Krishnaraj Thadesar \\
        Cyber Security and Forensics\\
        Batch A1, PA 20
    }


    \vspace{0.5\baselineskip} % Whitespace below the editor list
    \today

\end{titlepage}


\tableofcontents
\thispagestyle{empty}
\clearpage

\setcounter{page}{1}

\section{Aim}
To Learn about the different tools used in Digital Forensics and Investigation for Mobile Devices.

\section{Objectives}
\begin{enumerate}
    \item Understand the need for Digital Forensics and Investigation for Mobile Devices.
    \item Find out the tools available in the market for Forensics in Mobile Devices.
    \item Experiment with some simple tools.
\end{enumerate}

\section{Theory}

\section{Platform}
\textbf{Operating System}: Windows 11 \\
\textbf{IDEs or Text Editors Used}: Visual Studio Code\\
\textbf{Compilers or Interpreters}: Python 3.10.1\\

% \section{Code}
% \lstinputlisting[language=Python, caption="DSA Signature Validity using PyCrypto Library"]{../Programs/Assignment_7/dsa using lib.py}

\section{Conclusion}
Thus, we were able to learn about the different tools used in Digital Forensics and Investigation for Mobile Devices. We also performed a very basic extraction from ours and our Friend's mobile devices, and managed to get some very basic information from it.

\clearpage

\pagebreak
\begin{thebibliography}{}
    \bibitem{Casey, E., Dietrich, E. (2011)}
    \textit{Digital Evidence and Computer Crime: Forensic Science, Computers, and the Internet.} \\Academic Press.
    \bibitem{Carrier, B. (2014)}
    \textit{File System Forensic Analysis.}
    \\Addison-Wesley.
    \bibitem{Quick, D. (2019)}
    \textit{Mobile Forensic Investigations: A Guide to Evidence Collection, Analysis, and Presentation.} \\Elsevier.
    \bibitem{Vskills - Digital Forensic Tools}
    \href{https://www.vskills.in/certification/tutorial/digital-forensics-tools/}{Vskills - Digital Forensic Tools} \\VSkills
\end{thebibliography}
\end{document}