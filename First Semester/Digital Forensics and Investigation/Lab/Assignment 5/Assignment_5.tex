% This is a Basic Assignment Paper but with like Code and stuff allowed in it, there is also url, hyperlinks from contents included. 

\documentclass[11pt]{article}

% Preamble
\usepackage[margin=1in]{geometry}
\usepackage{amsfonts, amsmath, amssymb, amsthm}
\usepackage{fancyhdr, float, graphicx}
\usepackage[utf8]{inputenc} % Required for inputting international characters
\usepackage[T1]{fontenc} % Output font encoding for international characters
\usepackage{fouriernc} % Use the New Century Schoolbook font
\usepackage[nottoc, notlot, notlof]{tocbibind}
\usepackage{listings}
\usepackage{xcolor}
\usepackage{blindtext}
\usepackage{hyperref}
\definecolor{codepurple}{rgb}{0.58,0,0.952}
\hypersetup{
    colorlinks=true,
    linkcolor=black,
    filecolor=black,      
    urlcolor=codepurple,
    pdfpagemode=FullScreen,
    }

\definecolor{codegreen}{rgb}{0,0.6,0}
\definecolor{codegray}{rgb}{0.5,0.5,0.5}
\definecolor{backcolour}{rgb}{0.95,0.95,0.92}

\lstdefinestyle{mystyle}{
    backgroundcolor=\color{backcolour},   
    commentstyle=\color{codegreen},
    keywordstyle=\color{magenta},
    numberstyle=\tiny\color{codegray},
    stringstyle=\color{codepurple},
    basicstyle=\ttfamily\footnotesize,
    breakatwhitespace=false,         
    breaklines=true,                 
    captionpos=b,                    
    keepspaces=true,                 
    numbers=left,                    
    numbersep=5pt,                  
    showspaces=false,                
    showstringspaces=false,
    showtabs=false,                  
    tabsize=2
}

\lstset{style=mystyle}

% Header and Footer
\pagestyle{fancy}
\fancyhead{}
\fancyfoot{}
\fancyhead[L]{\textit{\Large{Digital Forensics and Investigation - TY. B. Tech}}}
\fancyhead[R]{\textit{Krishnaraj T}}
\fancyfoot[C]{\thepage}
\renewcommand{\footrulewidth}{1pt}
\newtheorem{thm}{Theorem}
\newtheorem{dfn}[thm]{Definition}



\renewcommand{\footrulewidth}{1pt}

\usepackage[breakable]{tcolorbox}
\usepackage{parskip} % Stop auto-indenting (to mimic markdown behaviour)


% Basic figure setup, for now with no caption control since it's done
% automatically by Pandoc (which extracts ![](path) syntax from Markdown).
\usepackage{graphicx}
% Maintain compatibility with old templates. Remove in nbconvert 6.0
\let\Oldincludegraphics\includegraphics
% Ensure that by default, figures have no caption (until we provide a
% proper Figure object with a Caption API and a way to capture that
% in the conversion process - todo).
\usepackage{caption}
\DeclareCaptionFormat{nocaption}{}
\captionsetup{format=nocaption,aboveskip=0pt,belowskip=0pt}

\usepackage{float}
\floatplacement{figure}{H} % forces figures to be placed at the correct location
\usepackage{xcolor} % Allow colors to be defined
\usepackage{enumerate} % Needed for markdown enumerations to work
\usepackage{geometry} % Used to adjust the document margins
\usepackage{amsmath} % Equations
\usepackage{amssymb} % Equations
\usepackage{textcomp} % defines textquotesingle
% Hack from http://tex.stackexchange.com/a/47451/13684:
\AtBeginDocument{%
    \def\PYZsq{\textquotesingle}% Upright quotes in Pygmentized code
}
\usepackage{upquote} % Upright quotes for verbatim code
\usepackage{eurosym} % defines \euro

\usepackage{iftex}
% \ifPDFTeX
% 	\usepackage[T1]{fontenc}
% 	\IfFileExists{alphabeta.sty}{
% 		  \usepackage{alphabeta}
% 	  }{
% 		  \usepackage[mathletters]{ucs}
% 		  \usepackage[utf8x]{inputenc}
% 	  }
% \else
% 	\usepackage{fontspec}
% 	\usepackage{unicode-math}
% \fi

\usepackage{fancyvrb} % verbatim replacement that allows latex
\usepackage{grffile} % extends the file name processing of package graphics
% to support a larger range
\makeatletter % fix for old versions of grffile with XeLaTeX
\@ifpackagelater{grffile}{2019/11/01}
{
    % Do nothing on new versions
}
{
    \def\Gread@@xetex#1{%
        \IfFileExists{"\Gin@base".bb}%
        {\Gread@eps{\Gin@base.bb}}%
        {\Gread@@xetex@aux#1}%
    }
}
\makeatother
\usepackage[Export]{adjustbox} % Used to constrain images to a maximum size
\adjustboxset{max size={0.9\linewidth}{0.9\paperheight}}

% The hyperref package gives us a pdf with properly built
% internal navigation ('pdf bookmarks' for the table of contents,
% internal cross-reference links, web links for URLs, etc.)
\usepackage{hyperref}
% The default LaTeX title has an obnoxious amount of whitespace. By default,
% titling removes some of it. It also provides customization options.
\usepackage{titling}
\usepackage{longtable} % longtable support required by pandoc >1.10
\usepackage{booktabs}  % table support for pandoc > 1.12.2
\usepackage{array}     % table support for pandoc >= 2.11.3
\usepackage{calc}      % table minipage width calculation for pandoc >= 2.11.1
\usepackage[inline]{enumitem} % IRkernel/repr support (it uses the enumerate* environment)
\usepackage[normalem]{ulem} % ulem is needed to support strikethroughs (\sout)
% normalem makes italics be italics, not underlines
\usepackage{mathrsfs}



% Colors for the hyperref package
\definecolor{urlcolor}{rgb}{0,.145,.698}
\definecolor{linkcolor}{rgb}{.71,0.21,0.01}
\definecolor{citecolor}{rgb}{.12,.54,.11}

% ANSI colors
\definecolor{ansi-black}{HTML}{3E424D}
\definecolor{ansi-black-intense}{HTML}{282C36}
\definecolor{ansi-red}{HTML}{E75C58}
\definecolor{ansi-red-intense}{HTML}{B22B31}
\definecolor{ansi-green}{HTML}{00A250}
\definecolor{ansi-green-intense}{HTML}{007427}
\definecolor{ansi-yellow}{HTML}{DDB62B}
\definecolor{ansi-yellow-intense}{HTML}{B27D12}
\definecolor{ansi-blue}{HTML}{208FFB}
\definecolor{ansi-blue-intense}{HTML}{0065CA}
\definecolor{ansi-magenta}{HTML}{D160C4}
\definecolor{ansi-magenta-intense}{HTML}{A03196}
\definecolor{ansi-cyan}{HTML}{60C6C8}
\definecolor{ansi-cyan-intense}{HTML}{258F8F}
\definecolor{ansi-white}{HTML}{C5C1B4}
\definecolor{ansi-white-intense}{HTML}{A1A6B2}
\definecolor{ansi-default-inverse-fg}{HTML}{FFFFFF}
\definecolor{ansi-default-inverse-bg}{HTML}{000000}

% common color for the border for error outputs.
\definecolor{outerrorbackground}{HTML}{FFDFDF}

% commands and environments needed by pandoc snippets
% extracted from the output of `pandoc -s`
\providecommand{\tightlist}{%
    \setlength{\itemsep}{0pt}\setlength{\parskip}{0pt}}
\DefineVerbatimEnvironment{Highlighting}{Verbatim}{commandchars=\\\{\}}
% Add ',fontsize=\small' for more characters per line
\newenvironment{Shaded}{}{}
\newcommand{\KeywordTok}[1]{\textcolor[rgb]{0.00,0.44,0.13}{\textbf{{#1}}}}
\newcommand{\DataTypeTok}[1]{\textcolor[rgb]{0.56,0.13,0.00}{{#1}}}
\newcommand{\DecValTok}[1]{\textcolor[rgb]{0.25,0.63,0.44}{{#1}}}
\newcommand{\BaseNTok}[1]{\textcolor[rgb]{0.25,0.63,0.44}{{#1}}}
\newcommand{\FloatTok}[1]{\textcolor[rgb]{0.25,0.63,0.44}{{#1}}}
\newcommand{\CharTok}[1]{\textcolor[rgb]{0.25,0.44,0.63}{{#1}}}
\newcommand{\StringTok}[1]{\textcolor[rgb]{0.25,0.44,0.63}{{#1}}}
\newcommand{\CommentTok}[1]{\textcolor[rgb]{0.38,0.63,0.69}{\textit{{#1}}}}
\newcommand{\OtherTok}[1]{\textcolor[rgb]{0.00,0.44,0.13}{{#1}}}
\newcommand{\AlertTok}[1]{\textcolor[rgb]{1.00,0.00,0.00}{\textbf{{#1}}}}
\newcommand{\FunctionTok}[1]{\textcolor[rgb]{0.02,0.16,0.49}{{#1}}}
\newcommand{\RegionMarkerTok}[1]{{#1}}
\newcommand{\ErrorTok}[1]{\textcolor[rgb]{1.00,0.00,0.00}{\textbf{{#1}}}}
\newcommand{\NormalTok}[1]{{#1}}

% Additional commands for more recent versions of Pandoc
\newcommand{\ConstantTok}[1]{\textcolor[rgb]{0.53,0.00,0.00}{{#1}}}
\newcommand{\SpecialCharTok}[1]{\textcolor[rgb]{0.25,0.44,0.63}{{#1}}}
\newcommand{\VerbatimStringTok}[1]{\textcolor[rgb]{0.25,0.44,0.63}{{#1}}}
\newcommand{\SpecialStringTok}[1]{\textcolor[rgb]{0.73,0.40,0.53}{{#1}}}
\newcommand{\ImportTok}[1]{{#1}}
\newcommand{\DocumentationTok}[1]{\textcolor[rgb]{0.73,0.13,0.13}{\textit{{#1}}}}
\newcommand{\AnnotationTok}[1]{\textcolor[rgb]{0.38,0.63,0.69}{\textbf{\textit{{#1}}}}}
\newcommand{\CommentVarTok}[1]{\textcolor[rgb]{0.38,0.63,0.69}{\textbf{\textit{{#1}}}}}
\newcommand{\VariableTok}[1]{\textcolor[rgb]{0.10,0.09,0.49}{{#1}}}
\newcommand{\ControlFlowTok}[1]{\textcolor[rgb]{0.00,0.44,0.13}{\textbf{{#1}}}}
\newcommand{\OperatorTok}[1]{\textcolor[rgb]{0.40,0.40,0.40}{{#1}}}
\newcommand{\BuiltInTok}[1]{{#1}}
\newcommand{\ExtensionTok}[1]{{#1}}
\newcommand{\PreprocessorTok}[1]{\textcolor[rgb]{0.74,0.48,0.00}{{#1}}}
\newcommand{\AttributeTok}[1]{\textcolor[rgb]{0.49,0.56,0.16}{{#1}}}
\newcommand{\InformationTok}[1]{\textcolor[rgb]{0.38,0.63,0.69}{\textbf{\textit{{#1}}}}}
\newcommand{\WarningTok}[1]{\textcolor[rgb]{0.38,0.63,0.69}{\textbf{\textit{{#1}}}}}


% Define a nice break command that doesn't care if a line doesn't already
% exist.
\def\br{\hspace*{\fill} \\* }
% Math Jax compatibility definitions
\def\gt{>}
\def\lt{<}
\let\Oldtex\TeX
\let\Oldlatex\LaTeX
\renewcommand{\TeX}{\textrm{\Oldtex}}
\renewcommand{\LaTeX}{\textrm{\Oldlatex}}
% Document parameters
% Document title
\title{Assignment\_1}





% Pygments definitions
\makeatletter
\def\PY@reset{\let\PY@it=\relax \let\PY@bf=\relax%
    \let\PY@ul=\relax \let\PY@tc=\relax%
    \let\PY@bc=\relax \let\PY@ff=\relax}
\def\PY@tok#1{\csname PY@tok@#1\endcsname}
\def\PY@toks#1+{\ifx\relax#1\empty\else%
    \PY@tok{#1}\expandafter\PY@toks\fi}
\def\PY@do#1{\PY@bc{\PY@tc{\PY@ul{%
                \PY@it{\PY@bf{\PY@ff{#1}}}}}}}
\def\PY#1#2{\PY@reset\PY@toks#1+\relax+\PY@do{#2}}

\@namedef{PY@tok@w}{\def\PY@tc##1{\textcolor[rgb]{0.73,0.73,0.73}{##1}}}
\@namedef{PY@tok@c}{\let\PY@it=\textit\def\PY@tc##1{\textcolor[rgb]{0.24,0.48,0.48}{##1}}}
\@namedef{PY@tok@cp}{\def\PY@tc##1{\textcolor[rgb]{0.61,0.40,0.00}{##1}}}
\@namedef{PY@tok@k}{\let\PY@bf=\textbf\def\PY@tc##1{\textcolor[rgb]{0.00,0.50,0.00}{##1}}}
\@namedef{PY@tok@kp}{\def\PY@tc##1{\textcolor[rgb]{0.00,0.50,0.00}{##1}}}
\@namedef{PY@tok@kt}{\def\PY@tc##1{\textcolor[rgb]{0.69,0.00,0.25}{##1}}}
\@namedef{PY@tok@o}{\def\PY@tc##1{\textcolor[rgb]{0.40,0.40,0.40}{##1}}}
\@namedef{PY@tok@ow}{\let\PY@bf=\textbf\def\PY@tc##1{\textcolor[rgb]{0.67,0.13,1.00}{##1}}}
\@namedef{PY@tok@nb}{\def\PY@tc##1{\textcolor[rgb]{0.00,0.50,0.00}{##1}}}
\@namedef{PY@tok@nf}{\def\PY@tc##1{\textcolor[rgb]{0.00,0.00,1.00}{##1}}}
\@namedef{PY@tok@nc}{\let\PY@bf=\textbf\def\PY@tc##1{\textcolor[rgb]{0.00,0.00,1.00}{##1}}}
\@namedef{PY@tok@nn}{\let\PY@bf=\textbf\def\PY@tc##1{\textcolor[rgb]{0.00,0.00,1.00}{##1}}}
\@namedef{PY@tok@ne}{\let\PY@bf=\textbf\def\PY@tc##1{\textcolor[rgb]{0.80,0.25,0.22}{##1}}}
\@namedef{PY@tok@nv}{\def\PY@tc##1{\textcolor[rgb]{0.10,0.09,0.49}{##1}}}
\@namedef{PY@tok@no}{\def\PY@tc##1{\textcolor[rgb]{0.53,0.00,0.00}{##1}}}
\@namedef{PY@tok@nl}{\def\PY@tc##1{\textcolor[rgb]{0.46,0.46,0.00}{##1}}}
\@namedef{PY@tok@ni}{\let\PY@bf=\textbf\def\PY@tc##1{\textcolor[rgb]{0.44,0.44,0.44}{##1}}}
\@namedef{PY@tok@na}{\def\PY@tc##1{\textcolor[rgb]{0.41,0.47,0.13}{##1}}}
\@namedef{PY@tok@nt}{\let\PY@bf=\textbf\def\PY@tc##1{\textcolor[rgb]{0.00,0.50,0.00}{##1}}}
\@namedef{PY@tok@nd}{\def\PY@tc##1{\textcolor[rgb]{0.67,0.13,1.00}{##1}}}
\@namedef{PY@tok@s}{\def\PY@tc##1{\textcolor[rgb]{0.73,0.13,0.13}{##1}}}
\@namedef{PY@tok@sd}{\let\PY@it=\textit\def\PY@tc##1{\textcolor[rgb]{0.73,0.13,0.13}{##1}}}
\@namedef{PY@tok@si}{\let\PY@bf=\textbf\def\PY@tc##1{\textcolor[rgb]{0.64,0.35,0.47}{##1}}}
\@namedef{PY@tok@se}{\let\PY@bf=\textbf\def\PY@tc##1{\textcolor[rgb]{0.67,0.36,0.12}{##1}}}
\@namedef{PY@tok@sr}{\def\PY@tc##1{\textcolor[rgb]{0.64,0.35,0.47}{##1}}}
\@namedef{PY@tok@ss}{\def\PY@tc##1{\textcolor[rgb]{0.10,0.09,0.49}{##1}}}
\@namedef{PY@tok@sx}{\def\PY@tc##1{\textcolor[rgb]{0.00,0.50,0.00}{##1}}}
\@namedef{PY@tok@m}{\def\PY@tc##1{\textcolor[rgb]{0.40,0.40,0.40}{##1}}}
\@namedef{PY@tok@gh}{\let\PY@bf=\textbf\def\PY@tc##1{\textcolor[rgb]{0.00,0.00,0.50}{##1}}}
\@namedef{PY@tok@gu}{\let\PY@bf=\textbf\def\PY@tc##1{\textcolor[rgb]{0.50,0.00,0.50}{##1}}}
\@namedef{PY@tok@gd}{\def\PY@tc##1{\textcolor[rgb]{0.63,0.00,0.00}{##1}}}
\@namedef{PY@tok@gi}{\def\PY@tc##1{\textcolor[rgb]{0.00,0.52,0.00}{##1}}}
\@namedef{PY@tok@gr}{\def\PY@tc##1{\textcolor[rgb]{0.89,0.00,0.00}{##1}}}
\@namedef{PY@tok@ge}{\let\PY@it=\textit}
\@namedef{PY@tok@gs}{\let\PY@bf=\textbf}
\@namedef{PY@tok@gp}{\let\PY@bf=\textbf\def\PY@tc##1{\textcolor[rgb]{0.00,0.00,0.50}{##1}}}
\@namedef{PY@tok@go}{\def\PY@tc##1{\textcolor[rgb]{0.44,0.44,0.44}{##1}}}
\@namedef{PY@tok@gt}{\def\PY@tc##1{\textcolor[rgb]{0.00,0.27,0.87}{##1}}}
\@namedef{PY@tok@err}{\def\PY@bc##1{{\setlength{\fboxsep}{\string -\fboxrule}\fcolorbox[rgb]{1.00,0.00,0.00}{1,1,1}{\strut ##1}}}}
\@namedef{PY@tok@kc}{\let\PY@bf=\textbf\def\PY@tc##1{\textcolor[rgb]{0.00,0.50,0.00}{##1}}}
\@namedef{PY@tok@kd}{\let\PY@bf=\textbf\def\PY@tc##1{\textcolor[rgb]{0.00,0.50,0.00}{##1}}}
\@namedef{PY@tok@kn}{\let\PY@bf=\textbf\def\PY@tc##1{\textcolor[rgb]{0.00,0.50,0.00}{##1}}}
\@namedef{PY@tok@kr}{\let\PY@bf=\textbf\def\PY@tc##1{\textcolor[rgb]{0.00,0.50,0.00}{##1}}}
\@namedef{PY@tok@bp}{\def\PY@tc##1{\textcolor[rgb]{0.00,0.50,0.00}{##1}}}
\@namedef{PY@tok@fm}{\def\PY@tc##1{\textcolor[rgb]{0.00,0.00,1.00}{##1}}}
\@namedef{PY@tok@vc}{\def\PY@tc##1{\textcolor[rgb]{0.10,0.09,0.49}{##1}}}
\@namedef{PY@tok@vg}{\def\PY@tc##1{\textcolor[rgb]{0.10,0.09,0.49}{##1}}}
\@namedef{PY@tok@vi}{\def\PY@tc##1{\textcolor[rgb]{0.10,0.09,0.49}{##1}}}
\@namedef{PY@tok@vm}{\def\PY@tc##1{\textcolor[rgb]{0.10,0.09,0.49}{##1}}}
\@namedef{PY@tok@sa}{\def\PY@tc##1{\textcolor[rgb]{0.73,0.13,0.13}{##1}}}
\@namedef{PY@tok@sb}{\def\PY@tc##1{\textcolor[rgb]{0.73,0.13,0.13}{##1}}}
\@namedef{PY@tok@sc}{\def\PY@tc##1{\textcolor[rgb]{0.73,0.13,0.13}{##1}}}
\@namedef{PY@tok@dl}{\def\PY@tc##1{\textcolor[rgb]{0.73,0.13,0.13}{##1}}}
\@namedef{PY@tok@s2}{\def\PY@tc##1{\textcolor[rgb]{0.73,0.13,0.13}{##1}}}
\@namedef{PY@tok@sh}{\def\PY@tc##1{\textcolor[rgb]{0.73,0.13,0.13}{##1}}}
\@namedef{PY@tok@s1}{\def\PY@tc##1{\textcolor[rgb]{0.73,0.13,0.13}{##1}}}
\@namedef{PY@tok@mb}{\def\PY@tc##1{\textcolor[rgb]{0.40,0.40,0.40}{##1}}}
\@namedef{PY@tok@mf}{\def\PY@tc##1{\textcolor[rgb]{0.40,0.40,0.40}{##1}}}
\@namedef{PY@tok@mh}{\def\PY@tc##1{\textcolor[rgb]{0.40,0.40,0.40}{##1}}}
\@namedef{PY@tok@mi}{\def\PY@tc##1{\textcolor[rgb]{0.40,0.40,0.40}{##1}}}
\@namedef{PY@tok@il}{\def\PY@tc##1{\textcolor[rgb]{0.40,0.40,0.40}{##1}}}
\@namedef{PY@tok@mo}{\def\PY@tc##1{\textcolor[rgb]{0.40,0.40,0.40}{##1}}}
\@namedef{PY@tok@ch}{\let\PY@it=\textit\def\PY@tc##1{\textcolor[rgb]{0.24,0.48,0.48}{##1}}}
\@namedef{PY@tok@cm}{\let\PY@it=\textit\def\PY@tc##1{\textcolor[rgb]{0.24,0.48,0.48}{##1}}}
\@namedef{PY@tok@cpf}{\let\PY@it=\textit\def\PY@tc##1{\textcolor[rgb]{0.24,0.48,0.48}{##1}}}
\@namedef{PY@tok@c1}{\let\PY@it=\textit\def\PY@tc##1{\textcolor[rgb]{0.24,0.48,0.48}{##1}}}
\@namedef{PY@tok@cs}{\let\PY@it=\textit\def\PY@tc##1{\textcolor[rgb]{0.24,0.48,0.48}{##1}}}

\def\PYZbs{\char`\\}
\def\PYZus{\char`\_}
\def\PYZob{\char`\{}
\def\PYZcb{\char`\}}
\def\PYZca{\char`\^}
\def\PYZam{\char`\&}
\def\PYZlt{\char`\<}
\def\PYZgt{\char`\>}
\def\PYZsh{\char`\#}
\def\PYZpc{\char`\%}
\def\PYZdl{\char`\$}
\def\PYZhy{\char`\-}
\def\PYZsq{\char`\'}
\def\PYZdq{\char`\"}
\def\PYZti{\char`\~}
% for compatibility with earlier versions
\def\PYZat{@}
\def\PYZlb{[}
\def\PYZrb{]}
\makeatother


% For linebreaks inside Verbatim environment from package fancyvrb.
\makeatletter
\newbox\Wrappedcontinuationbox
\newbox\Wrappedvisiblespacebox
\newcommand*\Wrappedvisiblespace {\textcolor{red}{\textvisiblespace}}
\newcommand*\Wrappedcontinuationsymbol {\textcolor{red}{\llap{\tiny$\m@th\hookrightarrow$}}}
\newcommand*\Wrappedcontinuationindent {3ex }
\newcommand*\Wrappedafterbreak {\kern\Wrappedcontinuationindent\copy\Wrappedcontinuationbox}
% Take advantage of the already applied Pygments mark-up to insert
% potential linebreaks for TeX processing.
%        {, <, #, %, $, ' and ": go to next line.
%        _, }, ^, &, >, - and ~: stay at end of broken line.
% Use of \textquotesingle for straight quote.
\newcommand*\Wrappedbreaksatspecials {%
    \def\PYGZus{\discretionary{\char`\_}{\Wrappedafterbreak}{\char`\_}}%
    \def\PYGZob{\discretionary{}{\Wrappedafterbreak\char`\{}{\char`\{}}%
    \def\PYGZcb{\discretionary{\char`\}}{\Wrappedafterbreak}{\char`\}}}%
    \def\PYGZca{\discretionary{\char`\^}{\Wrappedafterbreak}{\char`\^}}%
    \def\PYGZam{\discretionary{\char`\&}{\Wrappedafterbreak}{\char`\&}}%
    \def\PYGZlt{\discretionary{}{\Wrappedafterbreak\char`\<}{\char`\<}}%
    \def\PYGZgt{\discretionary{\char`\>}{\Wrappedafterbreak}{\char`\>}}%
    \def\PYGZsh{\discretionary{}{\Wrappedafterbreak\char`\#}{\char`\#}}%
    \def\PYGZpc{\discretionary{}{\Wrappedafterbreak\char`\%}{\char`\%}}%
    \def\PYGZdl{\discretionary{}{\Wrappedafterbreak\char`\$}{\char`\$}}%
    \def\PYGZhy{\discretionary{\char`\-}{\Wrappedafterbreak}{\char`\-}}%
    \def\PYGZsq{\discretionary{}{\Wrappedafterbreak\textquotesingle}{\textquotesingle}}%
    \def\PYGZdq{\discretionary{}{\Wrappedafterbreak\char`\"}{\char`\"}}%
    \def\PYGZti{\discretionary{\char`\~}{\Wrappedafterbreak}{\char`\~}}%
}
% Some characters . , ; ? ! / are not pygmentized.
% This macro makes them "active" and they will insert potential linebreaks
\newcommand*\Wrappedbreaksatpunct {%
    \lccode`\~`\.\lowercase{\def~}{\discretionary{\hbox{\char`\.}}{\Wrappedafterbreak}{\hbox{\char`\.}}}%
    \lccode`\~`\,\lowercase{\def~}{\discretionary{\hbox{\char`\,}}{\Wrappedafterbreak}{\hbox{\char`\,}}}%
    \lccode`\~`\;\lowercase{\def~}{\discretionary{\hbox{\char`\;}}{\Wrappedafterbreak}{\hbox{\char`\;}}}%
    \lccode`\~`\:\lowercase{\def~}{\discretionary{\hbox{\char`\:}}{\Wrappedafterbreak}{\hbox{\char`\:}}}%
    \lccode`\~`\?\lowercase{\def~}{\discretionary{\hbox{\char`\?}}{\Wrappedafterbreak}{\hbox{\char`\?}}}%
    \lccode`\~`\!\lowercase{\def~}{\discretionary{\hbox{\char`\!}}{\Wrappedafterbreak}{\hbox{\char`\!}}}%
    \lccode`\~`\/\lowercase{\def~}{\discretionary{\hbox{\char`\/}}{\Wrappedafterbreak}{\hbox{\char`\/}}}%
    \catcode`\.\active
    \catcode`\,\active
    \catcode`\;\active
    \catcode`\:\active
    \catcode`\?\active
    \catcode`\!\active
    \catcode`\/\active
    \lccode`\~`\~
}
\makeatother

\let\OriginalVerbatim=\Verbatim
\makeatletter
\renewcommand{\Verbatim}[1][1]{%
    %\parskip\z@skip
    \sbox\Wrappedcontinuationbox {\Wrappedcontinuationsymbol}%
    \sbox\Wrappedvisiblespacebox {\FV@SetupFont\Wrappedvisiblespace}%
    \def\FancyVerbFormatLine ##1{\hsize\linewidth
        \vtop{\raggedright\hyphenpenalty\z@\exhyphenpenalty\z@
            \doublehyphendemerits\z@\finalhyphendemerits\z@
            \strut ##1\strut}%
    }%
    % If the linebreak is at a space, the latter will be displayed as visible
    % space at end of first line, and a continuation symbol starts next line.
    % Stretch/shrink are however usually zero for typewriter font.
    \def\FV@Space {%
        \nobreak\hskip\z@ plus\fontdimen3\font minus\fontdimen4\font
        \discretionary{\copy\Wrappedvisiblespacebox}{\Wrappedafterbreak}
        {\kern\fontdimen2\font}%
    }%

    % Allow breaks at special characters using \PYG... macros.
    \Wrappedbreaksatspecials
    % Breaks at punctuation characters . , ; ? ! and / need catcode=\active
    \OriginalVerbatim[#1,codes*=\Wrappedbreaksatpunct]%
}
\makeatother

% Exact colors from NB
\definecolor{incolor}{HTML}{303F9F}
\definecolor{outcolor}{HTML}{D84315}
\definecolor{cellborder}{HTML}{CFCFCF}
\definecolor{cellbackground}{HTML}{F7F7F7}

% prompt
\makeatletter
\newcommand{\boxspacing}{\kern\kvtcb@left@rule\kern\kvtcb@boxsep}
\makeatother
\newcommand{\prompt}[4]{
    {\ttfamily\llap{{\color{#2}[#3]:\hspace{3pt}#4}}\vspace{-\baselineskip}}
}



% Prevent overflowing lines due to hard-to-break entities
\sloppy
% Setup hyperref package
\hypersetup{
    breaklinks=true,  % so long urls are correctly broken across lines
    colorlinks=true,
    urlcolor=urlcolor,
    linkcolor=linkcolor,
    citecolor=citecolor,
}
% Slightly bigger margins than the latex defaults

\geometry{verbose,tmargin=1in,bmargin=1in,lmargin=1in,rmargin=1in}
\hypersetup{
    colorlinks=true,
    linkcolor=black,
    filecolor=magenta,
    urlcolor=cyan,
    pdfpagemode=FullScreen,
}




% Other Doc Editing
% \parindent 0ex
%\renewcommand{\baselinestretch}{1.5}

\begin{document}

\begin{titlepage}
    \centering

    %---------------------------NAMES-------------------------------

    \huge\textsc{
        MIT World Peace University
    }\\

    \vspace{0.75\baselineskip} % space after Uni Name

    \LARGE{
        Digital Forensics and Investigation\\
        Third Year B. Tech, Semester 5
    }

    \vfill % space after Sub Name

    %--------------------------TITLE-------------------------------

    \rule{\textwidth}{1.6pt}\vspace*{-\baselineskip}\vspace*{2pt}
    \rule{\textwidth}{0.6pt}
    \vspace{0.75\baselineskip} % Whitespace above the title



    \huge{\textsc{
            Analysing Email Headers
        }} \\



    \vspace{0.5\baselineskip} % Whitespace below the title
    \rule{\textwidth}{0.6pt}\vspace*{-\baselineskip}\vspace*{2.8pt}
    \rule{\textwidth}{1.6pt}

    \vspace{1\baselineskip} % Whitespace after the title block

    %--------------------------SUBTITLE --------------------------	

    \LARGE\textsc{
        Lab Assignment 5
    } % Subtitle or further description
    \vfill

    %--------------------------AUTHOR-------------------------------

    Prepared By
    \vspace{0.5\baselineskip} % Whitespace before the editors

    \Large{
        Krishnaraj Thadesar \\
        Cyber Security and Forensics\\
        Batch A1, PA 20
    }


    \vspace{0.5\baselineskip} % Whitespace below the editor list
    \today

\end{titlepage}


\tableofcontents
\thispagestyle{empty}
\clearpage

\setcounter{page}{1}

\section{Aim}
To learn about the various types of email headers, and how to analyse them. To perform a live practical upon sent and received mails, while analysing the headers of the same.

\section{Objectives}
\begin{enumerate}
    \item Understand the various types of email headers.
    \item Learn how to analyse email headers.
\end{enumerate}

\section{Theory}

\subsection{Email Clients}

Email clients, also known as email programs or email software, are applications or platforms designed for managing, sending, and receiving email messages. They are essential tools for communication in both personal and professional settings. Here are key points related to email clients:

\begin{enumerate}
    \item \textbf{Purpose of Email Clients:}
          - Email clients are designed to provide a user-friendly interface for managing email communication. They allow users to read, compose, send, and organize emails.

    \item \textbf{Types of Email Clients:}
          - There are various types of email clients, including desktop clients (e.g., Microsoft Outlook, Mozilla Thunderbird), web-based clients (e.g., Gmail, Outlook.com), and mobile clients (e.g., Apple Mail, Gmail app).

    \item \textbf{Features of Email Clients:}
          - Email clients offer a wide range of features, including the ability to access multiple email accounts, organize emails into folders, set up filters and rules, and manage attachments.

          \begin{figure}[H]
              \centering
              \includegraphics[width=.85\textwidth]{./emailclients/email clients_0.jpg}
              \caption{Various kinds of email clients. }
          \end{figure}

    \item \textbf{Integration with Protocols:}
          - Email clients integrate with email protocols such as IMAP (Internet Message Access Protocol) and POP3 (Post Office Protocol) to retrieve and store emails on local devices or remote servers.

    \item \textbf{User Interface:}
          - The user interface of email clients varies, but it typically includes an inbox, folders for organization, a compose window, and options for formatting emails.

    \item \textbf{Cross-Platform Compatibility:}
          - Many email clients are available on multiple platforms, making it easy for users to access their emails on different devices such as computers, smartphones, and tablets.

    \item \textbf{Security and Encryption:}
          - Email clients often support encryption and security features to protect sensitive information, including SSL/TLS for secure connections and S/MIME for email message encryption.

    \item \textbf{Customization:}
          - Users can often customize the appearance and behavior of email clients through themes, plugins, and settings to suit their preferences and needs.

    \item \textbf{Productivity Tools:}
          - Some email clients offer productivity tools like calendars, task management, and contact organization, making them comprehensive communication hubs.

    \item \textbf{Email Standards:}
          - Email clients adhere to email standards such as RFC 5322 for formatting and displaying emails, ensuring compatibility and consistent rendering.

\end{enumerate}

Email clients play a pivotal role in modern communication, providing users with a centralized platform for managing their electronic correspondence.

\begin{figure}[H]
    \centering
    \includegraphics[width=.75\textwidth]{./emailclients/email clients_3.jpg}
    \caption{Gmail}
\end{figure}

\begin{figure}[H]
    \centering
    \includegraphics[width=.75\textwidth]{./emailclients/email clients_7.jpg}
    \caption{Outlook}
\end{figure}

\begin{figure}[H]
    \centering
    \includegraphics[width=.75\textwidth]{./emailclients/email clients_9.jpg}
    \caption{Thunderbird}
\end{figure}

\subsection{Email Headers}

Email headers are crucial components of every email message, providing essential information about the email's origin, routing, and content. They are often hidden from the average email user but are invaluable for analysis and troubleshooting. Here are the key points related to email headers:

\begin{enumerate}
    \item \textbf{Purpose of Email Headers:}
          - Email headers serve the purpose of facilitating the smooth transmission of email messages from the sender to the recipient.

    \item \textbf{Structure of Email Headers:}
          - Email headers are structured as a set of key-value pairs. Each pair is on a separate line, and the header section is separated from the email body by a blank line.

    \item \textbf{Common Header Fields:}
          - Email headers consist of various fields, including "From," "To," "Subject," "Date," and "Message-ID." These fields provide information about the sender, recipient, subject, timestamp, and a unique identifier for the email.

    \item \textbf{Received Headers:}
          - One of the most critical sections of an email header is the "Received" field. It provides a trail of servers through which the email passed, helping trace the email's path and verify its authenticity.

    \item \textbf{Message Routing:}
          - Email headers contain information about how the message traveled from the sender's email client to the recipient's email server. This routing information is useful for diagnosing email delivery issues.

\end{enumerate}

\subsection{Types of Email Headers}

There are several types of email headers, each serving a specific purpose:

\begin{enumerate}
    \item \textbf{MIME Headers:}
          - MIME (Multipurpose Internet Mail Extensions) headers are used to specify the type and structure of the email's content, allowing for the inclusion of multimedia elements like images and attachments.

    \item \textbf{Authentication Headers:}
          - Authentication headers, such as SPF (Sender Policy Framework) and DKIM (DomainKeys Identified Mail), are used to verify the authenticity of the sender and prevent email spoofing.

    \item \textbf{User-Agent Headers:}
          - User-Agent headers provide information about the email client or software used by the sender. This helps in identifying the source of the message.

    \item \textbf{Custom Headers:}
          - Email headers can also include custom fields created by the sender or email servers. These fields are specific to certain applications or organizations.
\end{enumerate}

\subsection{Email Header Analysis}

Email header analysis is a crucial skill in various fields, including cybersecurity, digital forensics, and troubleshooting email delivery issues. Here are key aspects of email header analysis:

\begin{enumerate}
    \item \textbf{IP Address Tracking:}
          - Analyzing email headers allows one to trace the path of the email through IP addresses, helping identify potential issues or suspicious activity.

    \item \textbf{Email Authentication:}
          - Authentication headers like SPF and DKIM can be scrutinized to verify the sender's authenticity, ensuring the email is not a forgery.

    \item \textbf{Server Timestamps:}
          - Email headers include timestamps from servers, which can be compared to identify delays or bottlenecks in email delivery.

    \item \textbf{Identifying Email Spoofing:}
          - By examining routing and authentication information, one can detect instances of email spoofing or phishing attempts.

    \item \textbf{Legal and Compliance:}
          - In legal matters, email headers can serve as evidence, proving the origin and path of an email message.
\end{enumerate}

When analyzing email headers, it's essential to refer to email standards and protocols, such as RFC 5322, for a comprehensive understanding of the header fields and their meanings.


\section{Platform}
\textbf{Operating System}: Arch Linux x86-64 \\
\textbf{IDEs or Text Editors Used}: Visual Studio Code\\
\textbf{Compilers or Interpreters}: Python 3.12\\

\section{Analysis}
\hypertarget{downloading-email-headers}{%
\subsection{Downloading Email Headers}\label{downloading-email-headers}}

    \begin{tcolorbox}[breakable, size=fbox, boxrule=1pt, pad at break*=1mm,colback=cellbackground, colframe=cellborder]
\prompt{In}{incolor}{1}{\boxspacing}
\begin{Verbatim}[commandchars=\\\{\}]
\PY{n}{username} \PY{o}{=} \PY{l+s+s2}{\PYZdq{}}\PY{l+s+s2}{krishnaraj.kpt@outlook.com}\PY{l+s+s2}{\PYZdq{}}
\PY{n}{password} \PY{o}{=} \PY{l+s+s2}{\PYZdq{}}\PY{l+s+s2}{BBQtJTSs8uQh57aw}\PY{l+s+s2}{\PYZdq{}}
\PY{n}{imap\PYZus{}server} \PY{o}{=} \PY{l+s+s2}{\PYZdq{}}\PY{l+s+s2}{outlook.office365.com}\PY{l+s+s2}{\PYZdq{}}
\end{Verbatim}
\end{tcolorbox}

    \begin{tcolorbox}[breakable, size=fbox, boxrule=1pt, pad at break*=1mm,colback=cellbackground, colframe=cellborder]
\prompt{In}{incolor}{5}{\boxspacing}
\begin{Verbatim}[commandchars=\\\{\}]
\PY{k+kn}{import} \PY{n+nn}{imaplib}
\PY{k+kn}{import} \PY{n+nn}{email}
\PY{k+kn}{from} \PY{n+nn}{email}\PY{n+nn}{.}\PY{n+nn}{header} \PY{k+kn}{import} \PY{n}{decode\PYZus{}header}
\PY{k+kn}{from} \PY{n+nn}{email}\PY{n+nn}{.}\PY{n+nn}{parser} \PY{k+kn}{import} \PY{n}{BytesParser}
\PY{k+kn}{import} \PY{n+nn}{json}
\end{Verbatim}
\end{tcolorbox}

    Creating Imap Object

    \begin{tcolorbox}[breakable, size=fbox, boxrule=1pt, pad at break*=1mm,colback=cellbackground, colframe=cellborder]
\prompt{In}{incolor}{6}{\boxspacing}
\begin{Verbatim}[commandchars=\\\{\}]
\PY{c+c1}{\PYZsh{} create an IMAP4 class with SSL}
\PY{n}{imap} \PY{o}{=} \PY{n}{imaplib}\PY{o}{.}\PY{n}{IMAP4\PYZus{}SSL}\PY{p}{(}\PY{n}{imap\PYZus{}server}\PY{p}{)}
\PY{c+c1}{\PYZsh{} authenticate}
\PY{n}{imap}\PY{o}{.}\PY{n}{login}\PY{p}{(}\PY{n}{username}\PY{p}{,} \PY{n}{password}\PY{p}{)}
\PY{n}{status}\PY{p}{,} \PY{n}{messages} \PY{o}{=} \PY{n}{imap}\PY{o}{.}\PY{n}{select}\PY{p}{(}\PY{l+s+s2}{\PYZdq{}}\PY{l+s+s2}{INBOX}\PY{l+s+s2}{\PYZdq{}}\PY{p}{)}

\PY{c+c1}{\PYZsh{} number of top emails to fetch}
\PY{n}{N} \PY{o}{=} \PY{l+m+mi}{15}

\PY{c+c1}{\PYZsh{} total number of emails}
\PY{n}{messages} \PY{o}{=} \PY{n+nb}{int}\PY{p}{(}\PY{n}{messages}\PY{p}{[}\PY{l+m+mi}{0}\PY{p}{]}\PY{p}{)}
\PY{n+nb}{print}\PY{p}{(}\PY{l+s+s2}{\PYZdq{}}\PY{l+s+s2}{The Total Number of Messages in your account are: }\PY{l+s+s2}{\PYZdq{}}\PY{p}{,} \PY{n}{messages}\PY{p}{)}
\end{Verbatim}
\end{tcolorbox}

    \begin{Verbatim}[commandchars=\\\{\}]
The Total Number of Messages in your account are:  1017
    \end{Verbatim}

    Downloading Headers

    \begin{tcolorbox}[breakable, size=fbox, boxrule=1pt, pad at break*=1mm,colback=cellbackground, colframe=cellborder]
\prompt{In}{incolor}{7}{\boxspacing}
\begin{Verbatim}[commandchars=\\\{\}]
\PY{c+c1}{\PYZsh{} create a list to store the emails}
\PY{n}{emails} \PY{o}{=} \PY{p}{[}\PY{p}{]}

\PY{c+c1}{\PYZsh{} fetch the top N email headers}
\PY{k}{for} \PY{n}{i} \PY{o+ow}{in} \PY{n+nb}{range}\PY{p}{(}\PY{n}{messages} \PY{o}{\PYZhy{}} \PY{n}{N}\PY{p}{,} \PY{n}{messages} \PY{o}{+} \PY{l+m+mi}{1}\PY{p}{)}\PY{p}{:}
    \PY{c+c1}{\PYZsh{} fetch the email header}
    \PY{n}{result}\PY{p}{,} \PY{n}{data} \PY{o}{=} \PY{n}{imap}\PY{o}{.}\PY{n}{fetch}\PY{p}{(}\PY{n+nb}{str}\PY{p}{(}\PY{n}{i} \PY{o}{+} \PY{l+m+mi}{1}\PY{p}{)}\PY{p}{,} \PY{l+s+s2}{\PYZdq{}}\PY{l+s+s2}{(RFC822.HEADER)}\PY{l+s+s2}{\PYZdq{}}\PY{p}{)}
    \PY{k}{if} \PY{n}{result} \PY{o}{==} \PY{l+s+s2}{\PYZdq{}}\PY{l+s+s2}{OK}\PY{l+s+s2}{\PYZdq{}}\PY{p}{:}
        \PY{c+c1}{\PYZsh{} parse the email header}
        \PY{n}{email\PYZus{}parser} \PY{o}{=} \PY{n}{BytesParser}\PY{p}{(}\PY{p}{)}
        \PY{n}{email\PYZus{}header} \PY{o}{=} \PY{n}{email\PYZus{}parser}\PY{o}{.}\PY{n}{parsebytes}\PY{p}{(}\PY{n}{data}\PY{p}{[}\PY{l+m+mi}{0}\PY{p}{]}\PY{p}{[}\PY{l+m+mi}{1}\PY{p}{]}\PY{p}{)}

        \PY{c+c1}{\PYZsh{} create a dictionary to store the email header}
        \PY{n}{email} \PY{o}{=} \PY{p}{\PYZob{}}
            \PY{l+s+s2}{\PYZdq{}}\PY{l+s+s2}{subject}\PY{l+s+s2}{\PYZdq{}}\PY{p}{:} \PY{n}{email\PYZus{}header}\PY{p}{[}\PY{l+s+s2}{\PYZdq{}}\PY{l+s+s2}{Subject}\PY{l+s+s2}{\PYZdq{}}\PY{p}{]}\PY{p}{,}
            \PY{l+s+s2}{\PYZdq{}}\PY{l+s+s2}{headers}\PY{l+s+s2}{\PYZdq{}}\PY{p}{:} \PY{p}{[}\PY{p}{]}\PY{p}{,}
        \PY{p}{\PYZcb{}}

        \PY{c+c1}{\PYZsh{} add the email headers to the dictionary}
        \PY{k}{for} \PY{n}{header} \PY{o+ow}{in} \PY{n}{email\PYZus{}header}\PY{o}{.}\PY{n}{items}\PY{p}{(}\PY{p}{)}\PY{p}{:}
            \PY{n}{email}\PY{p}{[}\PY{l+s+s2}{\PYZdq{}}\PY{l+s+s2}{headers}\PY{l+s+s2}{\PYZdq{}}\PY{p}{]}\PY{o}{.}\PY{n}{append}\PY{p}{(}\PY{p}{\PYZob{}}
                \PY{l+s+s2}{\PYZdq{}}\PY{l+s+s2}{header}\PY{l+s+s2}{\PYZdq{}}\PY{p}{:} \PY{n}{header}\PY{p}{[}\PY{l+m+mi}{0}\PY{p}{]}\PY{p}{,}
                \PY{l+s+s2}{\PYZdq{}}\PY{l+s+s2}{value}\PY{l+s+s2}{\PYZdq{}}\PY{p}{:} \PY{n}{header}\PY{p}{[}\PY{l+m+mi}{1}\PY{p}{]}\PY{p}{,}
            \PY{p}{\PYZcb{}}\PY{p}{)}

        \PY{c+c1}{\PYZsh{} add the email to the list of emails}
        \PY{n}{emails}\PY{o}{.}\PY{n}{append}\PY{p}{(}\PY{n}{email}\PY{p}{)}

\PY{c+c1}{\PYZsh{} close the IMAP connection}
\PY{n}{imap}\PY{o}{.}\PY{n}{close}\PY{p}{(}\PY{p}{)}

\PY{c+c1}{\PYZsh{} write the emails to a JSON file}
\PY{k}{with} \PY{n+nb}{open}\PY{p}{(}\PY{l+s+s2}{\PYZdq{}}\PY{l+s+s2}{emails.json}\PY{l+s+s2}{\PYZdq{}}\PY{p}{,} \PY{l+s+s2}{\PYZdq{}}\PY{l+s+s2}{w}\PY{l+s+s2}{\PYZdq{}}\PY{p}{)} \PY{k}{as} \PY{n}{f}\PY{p}{:}
    \PY{n}{json}\PY{o}{.}\PY{n}{dump}\PY{p}{(}\PY{n}{emails}\PY{p}{,} \PY{n}{f}\PY{p}{,} \PY{n}{indent}\PY{o}{=}\PY{l+m+mi}{4}\PY{p}{)}
\end{Verbatim}
\end{tcolorbox}

    Converting the json file to a python dictionary

    \begin{tcolorbox}[breakable, size=fbox, boxrule=1pt, pad at break*=1mm,colback=cellbackground, colframe=cellborder]
\prompt{In}{incolor}{23}{\boxspacing}
\begin{Verbatim}[commandchars=\\\{\}]
\PY{k+kn}{import} \PY{n+nn}{json}
\PY{k}{with} \PY{n+nb}{open}\PY{p}{(}\PY{l+s+s2}{\PYZdq{}}\PY{l+s+s2}{emails.json}\PY{l+s+s2}{\PYZdq{}}\PY{p}{,} \PY{l+s+s2}{\PYZdq{}}\PY{l+s+s2}{r}\PY{l+s+s2}{\PYZdq{}}\PY{p}{)} \PY{k}{as} \PY{n}{f}\PY{p}{:}
    \PY{n}{emails} \PY{o}{=} \PY{n}{json}\PY{o}{.}\PY{n}{load}\PY{p}{(}\PY{n}{f}\PY{p}{)}
\end{Verbatim}
\end{tcolorbox}

    Looking at Header keys of one of the emails

    \begin{tcolorbox}[breakable, size=fbox, boxrule=1pt, pad at break*=1mm,colback=cellbackground, colframe=cellborder]
\prompt{In}{incolor}{29}{\boxspacing}
\begin{Verbatim}[commandchars=\\\{\}]
\PY{n}{header\PYZus{}names} \PY{o}{=} \PY{p}{[}\PY{n}{i}\PY{p}{[}\PY{l+s+s1}{\PYZsq{}}\PY{l+s+s1}{header}\PY{l+s+s1}{\PYZsq{}}\PY{p}{]} \PY{k}{for} \PY{n}{i} \PY{o+ow}{in} \PY{n}{emails}\PY{p}{[}\PY{l+m+mi}{1}\PY{p}{]}\PY{p}{[}\PY{l+s+s2}{\PYZdq{}}\PY{l+s+s2}{headers}\PY{l+s+s2}{\PYZdq{}}\PY{p}{]}\PY{p}{]}
\end{Verbatim}
\end{tcolorbox}

    Focusing on 8 Test Emails

    \begin{tcolorbox}[breakable, size=fbox, boxrule=1pt, pad at break*=1mm,colback=cellbackground, colframe=cellborder]
\prompt{In}{incolor}{24}{\boxspacing}
\begin{Verbatim}[commandchars=\\\{\}]
\PY{n}{emails} \PY{o}{=} \PY{n}{emails}\PY{p}{[}\PY{l+m+mi}{2}\PY{p}{:}\PY{l+m+mi}{10}\PY{p}{]}
\end{Verbatim}
\end{tcolorbox}

    Looking at their Subjects

    \begin{tcolorbox}[breakable, size=fbox, boxrule=1pt, pad at break*=1mm,colback=cellbackground, colframe=cellborder]
\prompt{In}{incolor}{26}{\boxspacing}
\begin{Verbatim}[commandchars=\\\{\}]
\PY{p}{[}\PY{n}{i}\PY{p}{[}\PY{l+s+s1}{\PYZsq{}}\PY{l+s+s1}{subject}\PY{l+s+s1}{\PYZsq{}}\PY{p}{]} \PY{k}{for} \PY{n}{i} \PY{o+ow}{in} \PY{n}{emails}\PY{p}{]}
\end{Verbatim}
\end{tcolorbox}

            \begin{tcolorbox}[breakable, size=fbox, boxrule=.5pt, pad at break*=1mm, opacityfill=0]
\prompt{Out}{outcolor}{26}{\boxspacing}
\begin{Verbatim}[commandchars=\\\{\}]
['Sending from phone gmail app',
 'Self mail from web outlook client',
 'Self test mail from gmail web client',
 'self test mail from mit gmail client',
 'From movile outlook app',
 'Mail sent from phone',
 "Mother's phone Gmail app",
 'Sending from browser with vpn connection on']
\end{Verbatim}
\end{tcolorbox}
        
    Analyse each header and find out what it means

    \begin{tcolorbox}[breakable, size=fbox, boxrule=1pt, pad at break*=1mm,colback=cellbackground, colframe=cellborder]
\prompt{In}{incolor}{59}{\boxspacing}
\begin{Verbatim}[commandchars=\\\{\}]
\PY{n+nb}{print}\PY{p}{(}\PY{l+s+s2}{\PYZdq{}}\PY{l+s+s2}{Total number of headers in provided email: }\PY{l+s+s2}{\PYZdq{}}\PY{p}{,} \PY{n+nb}{len}\PY{p}{(}\PY{n}{header\PYZus{}names}\PY{p}{)}\PY{p}{)}
\end{Verbatim}
\end{tcolorbox}

    \begin{Verbatim}[commandchars=\\\{\}]
Total number of headers in provided email:  77
    \end{Verbatim}

    \begin{tcolorbox}[breakable, size=fbox, boxrule=1pt, pad at break*=1mm,colback=cellbackground, colframe=cellborder]
\prompt{In}{incolor}{45}{\boxspacing}
\begin{Verbatim}[commandchars=\\\{\}]
\PY{c+c1}{\PYZsh{} print header names with their index}
\PY{k}{for} \PY{n}{index}\PY{p}{,} \PY{n}{header} \PY{o+ow}{in} \PY{n+nb}{enumerate}\PY{p}{(}\PY{n}{header\PYZus{}names}\PY{p}{)}\PY{p}{:}
    \PY{n+nb}{print}\PY{p}{(}\PY{n}{index}\PY{p}{,} \PY{n}{header}\PY{p}{)}
\end{Verbatim}
\end{tcolorbox}

    \begin{Verbatim}[commandchars=\\\{\}]
0 MIME-Version
1 Received
2 ARC-Seal
3 ARC-Message-Signature
4 ARC-Authentication-Results
5 Received
6 Received
7 Authentication-Results
8 Received-SPF
9 Received
10 X-IncomingTopHeaderMarker
11 ARC-Seal
12 ARC-Message-Signature
13 ARC-Authentication-Results
14 DKIM-Signature
15 Received
16 Received
17 From
18 To
19 Subject
20 Thread-Topic
21 Thread-Index
22 Date
23 Message-ID
24 Accept-Language
25 Content-Language
26 X-MS-Has-Attach
27 X-MS-TNEF-Correlator
28 msip\_labels
29 x-tmn
30 x-ms-traffictypediagnostic
31 X-MS-Office365-Filtering-Correlation-Id
32 X-Microsoft-Antispam-Untrusted
33 X-Microsoft-Antispam-Message-Info-Original
34 X-MS-Exchange-AntiSpam-MessageData-Original-ChunkCount
35 X-MS-Exchange-AntiSpam-MessageData-Original-0
36 Content-Type
37 X-MS-Exchange-Transport-CrossTenantHeadersStamped
38 X-IncomingHeaderCount
39 Return-Path
40 X-MS-Exchange-Organization-ExpirationStartTime
41 X-MS-Exchange-Organization-ExpirationStartTimeReason
42 X-MS-Exchange-Organization-ExpirationInterval
43 X-MS-Exchange-Organization-ExpirationIntervalReason
44 X-MS-Exchange-Organization-Network-Message-Id
45 X-EOPAttributedMessage
46 X-EOPTenantAttributedMessage
47 X-MS-Exchange-Organization-MessageDirectionality
48 X-MS-Exchange-Transport-CrossTenantHeadersStripped
49 X-MS-Exchange-Transport-CrossTenantHeadersPromoted
50 X-MS-PublicTrafficType
51 X-MS-Exchange-Organization-AuthSource
52 X-MS-Exchange-Organization-AuthAs
53 X-MS-UserLastLogonTime
54 X-MS-Office365-Filtering-Correlation-Id-Prvs
55 X-MS-Exchange-EOPDirect
56 X-Sender-IP
57 X-SID-PRA
58 X-SID-Result
59 X-MS-Exchange-Organization-PCL
60 X-MS-Exchange-Organization-SCL
61 X-Microsoft-Antispam
62 X-MS-Exchange-CrossTenant-OriginalArrivalTime
63 X-MS-Exchange-CrossTenant-Network-Message-Id
64 X-MS-Exchange-CrossTenant-Id
65 X-MS-Exchange-CrossTenant-RMS-PersistedConsumerOrg
66 X-MS-Exchange-CrossTenant-rms-persistedconsumerorg
67 X-MS-Exchange-CrossTenant-AuthSource
68 X-MS-Exchange-CrossTenant-AuthAs
69 X-MS-Exchange-CrossTenant-FromEntityHeader
70 X-MS-Exchange-Transport-CrossTenantHeadersStamped
71 X-MS-Exchange-Transport-EndToEndLatency
72 X-MS-Exchange-Processed-By-BccFoldering
73 X-Microsoft-Antispam-Mailbox-Delivery
74 X-Message-Info
75 X-Message-Delivery
76 X-Microsoft-Antispam-Message-Info
    \end{Verbatim}

    \hypertarget{mime-version}{%
\subsubsection{MIME-Version}\label{mime-version}}

MIME-Version indicates the email's message format. It's important in
investigations to understand how the message is structured and if it
includes multimedia or attachments.

    \begin{tcolorbox}[breakable, size=fbox, boxrule=1pt, pad at break*=1mm,colback=cellbackground, colframe=cellborder]
\prompt{In}{incolor}{38}{\boxspacing}
\begin{Verbatim}[commandchars=\\\{\}]
\PY{n}{emails}\PY{p}{[}\PY{l+m+mi}{1}\PY{p}{]}\PY{p}{[}\PY{l+s+s1}{\PYZsq{}}\PY{l+s+s1}{headers}\PY{l+s+s1}{\PYZsq{}}\PY{p}{]}\PY{p}{[}\PY{l+m+mi}{0}\PY{p}{]}\PY{p}{[}\PY{l+s+s1}{\PYZsq{}}\PY{l+s+s1}{header}\PY{l+s+s1}{\PYZsq{}}\PY{p}{]}\PY{p}{,} \PY{n}{emails}\PY{p}{[}\PY{l+m+mi}{1}\PY{p}{]}\PY{p}{[}\PY{l+s+s1}{\PYZsq{}}\PY{l+s+s1}{headers}\PY{l+s+s1}{\PYZsq{}}\PY{p}{]}\PY{p}{[}\PY{l+m+mi}{0}\PY{p}{]}\PY{p}{[}\PY{l+s+s1}{\PYZsq{}}\PY{l+s+s1}{value}\PY{l+s+s1}{\PYZsq{}}\PY{p}{]}
\end{Verbatim}
\end{tcolorbox}

            \begin{tcolorbox}[breakable, size=fbox, boxrule=.5pt, pad at break*=1mm, opacityfill=0]
\prompt{Out}{outcolor}{38}{\boxspacing}
\begin{Verbatim}[commandchars=\\\{\}]
('MIME-Version', '1.0')
\end{Verbatim}
\end{tcolorbox}
        
    \hypertarget{received}{%
\subsubsection{Received}\label{received}}

Received headers track the path of the email through various servers.
This is crucial for tracing the email's journey, identifying potential
anomalies, or investigating its source.

    \begin{tcolorbox}[breakable, size=fbox, boxrule=1pt, pad at break*=1mm,colback=cellbackground, colframe=cellborder]
\prompt{In}{incolor}{40}{\boxspacing}
\begin{Verbatim}[commandchars=\\\{\}]
\PY{n}{emails}\PY{p}{[}\PY{l+m+mi}{1}\PY{p}{]}\PY{p}{[}\PY{l+s+s1}{\PYZsq{}}\PY{l+s+s1}{headers}\PY{l+s+s1}{\PYZsq{}}\PY{p}{]}\PY{p}{[}\PY{l+m+mi}{1}\PY{p}{]}\PY{p}{[}\PY{l+s+s1}{\PYZsq{}}\PY{l+s+s1}{header}\PY{l+s+s1}{\PYZsq{}}\PY{p}{]}\PY{p}{,} \PY{n}{emails}\PY{p}{[}\PY{l+m+mi}{1}\PY{p}{]}\PY{p}{[}\PY{l+s+s1}{\PYZsq{}}\PY{l+s+s1}{headers}\PY{l+s+s1}{\PYZsq{}}\PY{p}{]}\PY{p}{[}\PY{l+m+mi}{1}\PY{p}{]}\PY{p}{[}\PY{l+s+s1}{\PYZsq{}}\PY{l+s+s1}{value}\PY{l+s+s1}{\PYZsq{}}\PY{p}{]}
\end{Verbatim}
\end{tcolorbox}

    \begin{Verbatim}[commandchars=\\\{\}]
Received from SJ0PR17MB4837.namprd17.prod.outlook.com (2603:10b6:a03:37a::10)
 by DS7PR17MB6730.namprd17.prod.outlook.com with HTTPS; Sun, 29 Oct 2023
 17:32:29 +0000
    \end{Verbatim}

    \hypertarget{arc-seal}{%
\subsubsection{ARC-Seal}\label{arc-seal}}

ARC (Authenticated Received Chain) headers help verify the authenticity
of email forwarding. ARC-Seal ensures the integrity of email headers,
reducing the risk of spoofing.

    \begin{tcolorbox}[breakable, size=fbox, boxrule=1pt, pad at break*=1mm,colback=cellbackground, colframe=cellborder]
\prompt{In}{incolor}{41}{\boxspacing}
\begin{Verbatim}[commandchars=\\\{\}]
\PY{n}{emails}\PY{p}{[}\PY{l+m+mi}{1}\PY{p}{]}\PY{p}{[}\PY{l+s+s1}{\PYZsq{}}\PY{l+s+s1}{headers}\PY{l+s+s1}{\PYZsq{}}\PY{p}{]}\PY{p}{[}\PY{l+m+mi}{2}\PY{p}{]}\PY{p}{[}\PY{l+s+s1}{\PYZsq{}}\PY{l+s+s1}{header}\PY{l+s+s1}{\PYZsq{}}\PY{p}{]}\PY{p}{,} \PY{n}{emails}\PY{p}{[}\PY{l+m+mi}{1}\PY{p}{]}\PY{p}{[}\PY{l+s+s1}{\PYZsq{}}\PY{l+s+s1}{headers}\PY{l+s+s1}{\PYZsq{}}\PY{p}{]}\PY{p}{[}\PY{l+m+mi}{2}\PY{p}{]}\PY{p}{[}\PY{l+s+s1}{\PYZsq{}}\PY{l+s+s1}{value}\PY{l+s+s1}{\PYZsq{}}\PY{p}{]}
\end{Verbatim}
\end{tcolorbox}

            \begin{tcolorbox}[breakable, size=fbox, boxrule=.5pt, pad at break*=1mm, opacityfill=0]
\prompt{Out}{outcolor}{41}{\boxspacing}
\begin{Verbatim}[commandchars=\\\{\}]
('ARC-Seal',
 'i=2; a=rsa-sha256; s=arcselector9901; d=microsoft.com; cv=pass;\textbackslash{}r\textbackslash{}n b=UERZv4Bl
U4weLVUXdoohvbjGYpYf4pb9pFtOwayMK+mtwTbjMhWrqskYHiEhqgH1rrxhnvYgrK7YkvSkKXiEypdA
Oak0f4KcLaHNb/KEBpCVvQoKhVUX2zWzFMxVLsIRkMgoltrKRs0JShcFwrbt6XCvxCZUTbsGQs/hFpaN
0sYlFys1Qu41etiVDrmS8ZYpq4ZnHuXxdBzxW6A8Aql06f5sr4CF2fSeAIjAFu5JB5/tTHlu9wFIYa49
rmvL4i2S8QLHI8IdHsvPpz0oNrK0BVzf8bFat2iF7qnIX2J1lkXo21nWrGEioqJjPP6uusrYJBc5R+Sa
tB5i3rmSYpqn0w==')
\end{Verbatim}
\end{tcolorbox}
        
    \hypertarget{arc-message-signature}{%
\subsubsection{ARC-Message-Signature}\label{arc-message-signature}}

ARC-Message-Signature is part of ARC headers and provides cryptographic
assurance of email header integrity, aiding in detecting email
tampering.

    \begin{tcolorbox}[breakable, size=fbox, boxrule=1pt, pad at break*=1mm,colback=cellbackground, colframe=cellborder]
\prompt{In}{incolor}{42}{\boxspacing}
\begin{Verbatim}[commandchars=\\\{\}]
\PY{n}{emails}\PY{p}{[}\PY{l+m+mi}{1}\PY{p}{]}\PY{p}{[}\PY{l+s+s1}{\PYZsq{}}\PY{l+s+s1}{headers}\PY{l+s+s1}{\PYZsq{}}\PY{p}{]}\PY{p}{[}\PY{l+m+mi}{3}\PY{p}{]}\PY{p}{[}\PY{l+s+s1}{\PYZsq{}}\PY{l+s+s1}{header}\PY{l+s+s1}{\PYZsq{}}\PY{p}{]}\PY{p}{,} \PY{n}{emails}\PY{p}{[}\PY{l+m+mi}{1}\PY{p}{]}\PY{p}{[}\PY{l+s+s1}{\PYZsq{}}\PY{l+s+s1}{headers}\PY{l+s+s1}{\PYZsq{}}\PY{p}{]}\PY{p}{[}\PY{l+m+mi}{3}\PY{p}{]}\PY{p}{[}\PY{l+s+s1}{\PYZsq{}}\PY{l+s+s1}{value}\PY{l+s+s1}{\PYZsq{}}\PY{p}{]}
\end{Verbatim}
\end{tcolorbox}

            \begin{tcolorbox}[breakable, size=fbox, boxrule=.5pt, pad at break*=1mm, opacityfill=0]
\prompt{Out}{outcolor}{42}{\boxspacing}
\begin{Verbatim}[commandchars=\\\{\}]
('ARC-Message-Signature',
 'i=2; a=rsa-sha256; c=relaxed/relaxed; d=microsoft.com;\textbackslash{}r\textbackslash{}n
s=arcselector9901;\textbackslash{}r\textbackslash{}n h=From:Date:Subject:Message-ID:Content-Type:MIME-
Version:X-MS-Exchange-AntiSpam-MessageData-ChunkCount:X-MS-Exchange-AntiSpam-
MessageData-0:X-MS-Exchange-AntiSpam-MessageData-1;\textbackslash{}r\textbackslash{}n
bh=6+cClN22oIrc2YH61nPV02IScRJCb64tf03+C6e0M4c=;\textbackslash{}r\textbackslash{}n b=eM0/ekjq2KehxMnRBghHFpHhg
VKFTyGOnc/ccplTJ7KonRY/xFz58qfR9ixNv6igINRBz+QQaWFKprBg57YvrEJRPljNUKG0WmKJXKF1C
zif79KmQpGcyxrjMBkNfga0hZWdPpPgOAVbNEG0z2uYUQ4zIqpiqq0wJ69EUaestV84DGs3O14jqflXj
ihcktRdZBX6zL/WD1gnOdr6ParkWfeJUaCV1BjMcpoUFXMnCZwlB2ST1aNgUdXHrqvxty4c2Q0/1uDWW
r9Wu/Vx5h/kN9LqqjvwNmvzTLuRE1oPLaIch8T2dBnIf6leWeKENffAjFK+Kba43a68mbAnbxlvsg=='
)
\end{Verbatim}
\end{tcolorbox}
        
    \hypertarget{arc-authentication-results}{%
\subsubsection{ARC-Authentication-Results}\label{arc-authentication-results}}

These headers indicate the email's authentication status. Investigators
can use this to assess the email's legitimacy and potential for
phishing.

    \begin{tcolorbox}[breakable, size=fbox, boxrule=1pt, pad at break*=1mm,colback=cellbackground, colframe=cellborder]
\prompt{In}{incolor}{43}{\boxspacing}
\begin{Verbatim}[commandchars=\\\{\}]
\PY{n}{emails}\PY{p}{[}\PY{l+m+mi}{1}\PY{p}{]}\PY{p}{[}\PY{l+s+s1}{\PYZsq{}}\PY{l+s+s1}{headers}\PY{l+s+s1}{\PYZsq{}}\PY{p}{]}\PY{p}{[}\PY{l+m+mi}{4}\PY{p}{]}\PY{p}{[}\PY{l+s+s1}{\PYZsq{}}\PY{l+s+s1}{header}\PY{l+s+s1}{\PYZsq{}}\PY{p}{]}\PY{p}{,} \PY{n}{emails}\PY{p}{[}\PY{l+m+mi}{1}\PY{p}{]}\PY{p}{[}\PY{l+s+s1}{\PYZsq{}}\PY{l+s+s1}{headers}\PY{l+s+s1}{\PYZsq{}}\PY{p}{]}\PY{p}{[}\PY{l+m+mi}{4}\PY{p}{]}\PY{p}{[}\PY{l+s+s1}{\PYZsq{}}\PY{l+s+s1}{value}\PY{l+s+s1}{\PYZsq{}}\PY{p}{]}
\end{Verbatim}
\end{tcolorbox}

            \begin{tcolorbox}[breakable, size=fbox, boxrule=.5pt, pad at break*=1mm, opacityfill=0]
\prompt{Out}{outcolor}{43}{\boxspacing}
\begin{Verbatim}[commandchars=\\\{\}]
('ARC-Authentication-Results',
 'i=2; mx.microsoft.com 1; spf=pass (sender ip is\textbackslash{}r\textbackslash{}n 40.92.20.10)
smtp.rcpttodomain=outlook.com smtp.mailfrom=outlook.com;\textbackslash{}r\textbackslash{}n dmarc=pass (p=none
sp=quarantine pct=100) action=none\textbackslash{}r\textbackslash{}n header.from=outlook.com; dkim=pass
(signature was verified)\textbackslash{}r\textbackslash{}n header.d=outlook.com; arc=pass (0 oda=0 ltdi=1)')
\end{Verbatim}
\end{tcolorbox}
        
    \hypertarget{dkim-signature}{%
\subsubsection{DKIM-Signature}\label{dkim-signature}}

DKIM (DomainKeys Identified Mail) verifies that the email content hasn't
been altered in transit. It's crucial for email integrity checks and
source verification.

    \begin{tcolorbox}[breakable, size=fbox, boxrule=1pt, pad at break*=1mm,colback=cellbackground, colframe=cellborder]
\prompt{In}{incolor}{46}{\boxspacing}
\begin{Verbatim}[commandchars=\\\{\}]
\PY{n}{emails}\PY{p}{[}\PY{l+m+mi}{1}\PY{p}{]}\PY{p}{[}\PY{l+s+s1}{\PYZsq{}}\PY{l+s+s1}{headers}\PY{l+s+s1}{\PYZsq{}}\PY{p}{]}\PY{p}{[}\PY{l+m+mi}{14}\PY{p}{]}\PY{p}{[}\PY{l+s+s1}{\PYZsq{}}\PY{l+s+s1}{header}\PY{l+s+s1}{\PYZsq{}}\PY{p}{]}\PY{p}{,} \PY{n}{emails}\PY{p}{[}\PY{l+m+mi}{1}\PY{p}{]}\PY{p}{[}\PY{l+s+s1}{\PYZsq{}}\PY{l+s+s1}{headers}\PY{l+s+s1}{\PYZsq{}}\PY{p}{]}\PY{p}{[}\PY{l+m+mi}{14}\PY{p}{]}\PY{p}{[}\PY{l+s+s1}{\PYZsq{}}\PY{l+s+s1}{value}\PY{l+s+s1}{\PYZsq{}}\PY{p}{]}
\end{Verbatim}
\end{tcolorbox}

            \begin{tcolorbox}[breakable, size=fbox, boxrule=.5pt, pad at break*=1mm, opacityfill=0]
\prompt{Out}{outcolor}{46}{\boxspacing}
\begin{Verbatim}[commandchars=\\\{\}]
('DKIM-Signature',
 'v=1; a=rsa-sha256; c=relaxed/relaxed; d=outlook.com;\textbackslash{}r\textbackslash{}n s=selector1;\textbackslash{}r\textbackslash{}n
h=From:Date:Subject:Message-ID:Content-Type:MIME-Version:X-MS-Exchange-
SenderADCheck;\textbackslash{}r\textbackslash{}n bh=6+cClN22oIrc2YH61nPV02IScRJCb64tf03+C6e0M4c=;\textbackslash{}r\textbackslash{}n b=aZNfYl
N+/lbQqIqDR1Cblhb9/x28HgNh+pAywoD+43Be+F/5cGPhWuKGP6InowbprEEutN/A5RLJ20qRurKixC
RrzUMNtv/QnJXQFkFCvHBOWiZyEANGeA5iUHQ38WwdNG0IsLWgyb4s82CjKshEyYgkgfPcpP854CN9Is
qJ1EUA3TIayHNbGwRilfGqwSFICBp4EODwvlXl+WWU4ihYg7HwlSaKgC/gn2WSnmIo3G/L3YbZkI9X41
B3ttxHTCeCTjAktqE29Ww13fLEgMB4F+gD8xJSmNQlYvSAcM6unCowj5AjXRLPHJ5aCseQL0Mnq5aFVu
foGrR9yFUn45RoYA==')
\end{Verbatim}
\end{tcolorbox}
        
    \hypertarget{from}{%
\subsubsection{From}\label{from}}

The ``From'' header shows the sender's email address. It's essential for
identifying the sender, although it can be spoofed.

    \begin{tcolorbox}[breakable, size=fbox, boxrule=1pt, pad at break*=1mm,colback=cellbackground, colframe=cellborder]
\prompt{In}{incolor}{47}{\boxspacing}
\begin{Verbatim}[commandchars=\\\{\}]
\PY{n}{emails}\PY{p}{[}\PY{l+m+mi}{1}\PY{p}{]}\PY{p}{[}\PY{l+s+s1}{\PYZsq{}}\PY{l+s+s1}{headers}\PY{l+s+s1}{\PYZsq{}}\PY{p}{]}\PY{p}{[}\PY{l+m+mi}{17}\PY{p}{]}\PY{p}{[}\PY{l+s+s1}{\PYZsq{}}\PY{l+s+s1}{header}\PY{l+s+s1}{\PYZsq{}}\PY{p}{]}\PY{p}{,} \PY{n}{emails}\PY{p}{[}\PY{l+m+mi}{1}\PY{p}{]}\PY{p}{[}\PY{l+s+s1}{\PYZsq{}}\PY{l+s+s1}{headers}\PY{l+s+s1}{\PYZsq{}}\PY{p}{]}\PY{p}{[}\PY{l+m+mi}{17}\PY{p}{]}\PY{p}{[}\PY{l+s+s1}{\PYZsq{}}\PY{l+s+s1}{value}\PY{l+s+s1}{\PYZsq{}}\PY{p}{]}
\end{Verbatim}
\end{tcolorbox}

            \begin{tcolorbox}[breakable, size=fbox, boxrule=.5pt, pad at break*=1mm, opacityfill=0]
\prompt{Out}{outcolor}{47}{\boxspacing}
\begin{Verbatim}[commandchars=\\\{\}]
('From', 'Krishnaraj Thadesar <Krishnaraj.kpt@outlook.com>')
\end{Verbatim}
\end{tcolorbox}
        
    \hypertarget{to}{%
\subsubsection{To}\label{to}}

The ``To'' header reveals the email's recipient, which is significant
for understanding the email's target and potential threat actors.

    \begin{tcolorbox}[breakable, size=fbox, boxrule=1pt, pad at break*=1mm,colback=cellbackground, colframe=cellborder]
\prompt{In}{incolor}{48}{\boxspacing}
\begin{Verbatim}[commandchars=\\\{\}]
\PY{n}{emails}\PY{p}{[}\PY{l+m+mi}{1}\PY{p}{]}\PY{p}{[}\PY{l+s+s1}{\PYZsq{}}\PY{l+s+s1}{headers}\PY{l+s+s1}{\PYZsq{}}\PY{p}{]}\PY{p}{[}\PY{l+m+mi}{18}\PY{p}{]}\PY{p}{[}\PY{l+s+s1}{\PYZsq{}}\PY{l+s+s1}{header}\PY{l+s+s1}{\PYZsq{}}\PY{p}{]}\PY{p}{,} \PY{n}{emails}\PY{p}{[}\PY{l+m+mi}{1}\PY{p}{]}\PY{p}{[}\PY{l+s+s1}{\PYZsq{}}\PY{l+s+s1}{headers}\PY{l+s+s1}{\PYZsq{}}\PY{p}{]}\PY{p}{[}\PY{l+m+mi}{18}\PY{p}{]}\PY{p}{[}\PY{l+s+s1}{\PYZsq{}}\PY{l+s+s1}{value}\PY{l+s+s1}{\PYZsq{}}\PY{p}{]}
\end{Verbatim}
\end{tcolorbox}

            \begin{tcolorbox}[breakable, size=fbox, boxrule=.5pt, pad at break*=1mm, opacityfill=0]
\prompt{Out}{outcolor}{48}{\boxspacing}
\begin{Verbatim}[commandchars=\\\{\}]
('To', 'Krishnaraj Thadesar <krishnaraj.kpt@outlook.com>')
\end{Verbatim}
\end{tcolorbox}
        
    \hypertarget{subject}{%
\subsubsection{Subject}\label{subject}}

The subject line provides insight into the email's content, which is
crucial for assessing the email's purpose and relevance to an
investigation.

    \begin{tcolorbox}[breakable, size=fbox, boxrule=1pt, pad at break*=1mm,colback=cellbackground, colframe=cellborder]
\prompt{In}{incolor}{49}{\boxspacing}
\begin{Verbatim}[commandchars=\\\{\}]
\PY{n}{emails}\PY{p}{[}\PY{l+m+mi}{1}\PY{p}{]}\PY{p}{[}\PY{l+s+s1}{\PYZsq{}}\PY{l+s+s1}{headers}\PY{l+s+s1}{\PYZsq{}}\PY{p}{]}\PY{p}{[}\PY{l+m+mi}{19}\PY{p}{]}\PY{p}{[}\PY{l+s+s1}{\PYZsq{}}\PY{l+s+s1}{header}\PY{l+s+s1}{\PYZsq{}}\PY{p}{]}\PY{p}{,} \PY{n}{emails}\PY{p}{[}\PY{l+m+mi}{1}\PY{p}{]}\PY{p}{[}\PY{l+s+s1}{\PYZsq{}}\PY{l+s+s1}{headers}\PY{l+s+s1}{\PYZsq{}}\PY{p}{]}\PY{p}{[}\PY{l+m+mi}{19}\PY{p}{]}\PY{p}{[}\PY{l+s+s1}{\PYZsq{}}\PY{l+s+s1}{value}\PY{l+s+s1}{\PYZsq{}}\PY{p}{]}
\end{Verbatim}
\end{tcolorbox}

            \begin{tcolorbox}[breakable, size=fbox, boxrule=.5pt, pad at break*=1mm, opacityfill=0]
\prompt{Out}{outcolor}{49}{\boxspacing}
\begin{Verbatim}[commandchars=\\\{\}]
('Subject', 'Self mail from web outlook client')
\end{Verbatim}
\end{tcolorbox}
        
    \hypertarget{date}{%
\subsubsection{Date}\label{date}}

The date header shows when the email was sent. It's valuable for
establishing timelines and correlations with other events.

    \begin{tcolorbox}[breakable, size=fbox, boxrule=1pt, pad at break*=1mm,colback=cellbackground, colframe=cellborder]
\prompt{In}{incolor}{50}{\boxspacing}
\begin{Verbatim}[commandchars=\\\{\}]
\PY{n}{emails}\PY{p}{[}\PY{l+m+mi}{1}\PY{p}{]}\PY{p}{[}\PY{l+s+s1}{\PYZsq{}}\PY{l+s+s1}{headers}\PY{l+s+s1}{\PYZsq{}}\PY{p}{]}\PY{p}{[}\PY{l+m+mi}{22}\PY{p}{]}\PY{p}{[}\PY{l+s+s1}{\PYZsq{}}\PY{l+s+s1}{header}\PY{l+s+s1}{\PYZsq{}}\PY{p}{]}\PY{p}{,} \PY{n}{emails}\PY{p}{[}\PY{l+m+mi}{1}\PY{p}{]}\PY{p}{[}\PY{l+s+s1}{\PYZsq{}}\PY{l+s+s1}{headers}\PY{l+s+s1}{\PYZsq{}}\PY{p}{]}\PY{p}{[}\PY{l+m+mi}{22}\PY{p}{]}\PY{p}{[}\PY{l+s+s1}{\PYZsq{}}\PY{l+s+s1}{value}\PY{l+s+s1}{\PYZsq{}}\PY{p}{]}
\end{Verbatim}
\end{tcolorbox}

            \begin{tcolorbox}[breakable, size=fbox, boxrule=.5pt, pad at break*=1mm, opacityfill=0]
\prompt{Out}{outcolor}{50}{\boxspacing}
\begin{Verbatim}[commandchars=\\\{\}]
('Date', 'Sun, 29 Oct 2023 17:32:25 +0000')
\end{Verbatim}
\end{tcolorbox}
        
    \hypertarget{message-id}{%
\subsubsection{Message-ID}\label{message-id}}

The Message-ID is unique to each email and can be used for tracking and
associating related messages in an investigation.

    \begin{tcolorbox}[breakable, size=fbox, boxrule=1pt, pad at break*=1mm,colback=cellbackground, colframe=cellborder]
\prompt{In}{incolor}{51}{\boxspacing}
\begin{Verbatim}[commandchars=\\\{\}]
\PY{n}{emails}\PY{p}{[}\PY{l+m+mi}{1}\PY{p}{]}\PY{p}{[}\PY{l+s+s1}{\PYZsq{}}\PY{l+s+s1}{headers}\PY{l+s+s1}{\PYZsq{}}\PY{p}{]}\PY{p}{[}\PY{l+m+mi}{23}\PY{p}{]}\PY{p}{[}\PY{l+s+s1}{\PYZsq{}}\PY{l+s+s1}{header}\PY{l+s+s1}{\PYZsq{}}\PY{p}{]}\PY{p}{,} \PY{n}{emails}\PY{p}{[}\PY{l+m+mi}{1}\PY{p}{]}\PY{p}{[}\PY{l+s+s1}{\PYZsq{}}\PY{l+s+s1}{headers}\PY{l+s+s1}{\PYZsq{}}\PY{p}{]}\PY{p}{[}\PY{l+m+mi}{23}\PY{p}{]}\PY{p}{[}\PY{l+s+s1}{\PYZsq{}}\PY{l+s+s1}{value}\PY{l+s+s1}{\PYZsq{}}\PY{p}{]}
\end{Verbatim}
\end{tcolorbox}

            \begin{tcolorbox}[breakable, size=fbox, boxrule=.5pt, pad at break*=1mm, opacityfill=0]
\prompt{Out}{outcolor}{51}{\boxspacing}
\begin{Verbatim}[commandchars=\\\{\}]
('Message-ID',
 '\textbackslash{}r\textbackslash{}n <DS7PR17MB6730EE7BE9EA14814BB3AFE980A2A@DS7PR17MB6730.namprd17.prod.outlo
ok.com>')
\end{Verbatim}
\end{tcolorbox}
        
    \hypertarget{content-type}{%
\subsubsection{Content-Type}\label{content-type}}

Content-Type specifies the format of the email content. It helps
investigators interpret the email's structure and potential for
malicious attachments.

    \begin{tcolorbox}[breakable, size=fbox, boxrule=1pt, pad at break*=1mm,colback=cellbackground, colframe=cellborder]
\prompt{In}{incolor}{52}{\boxspacing}
\begin{Verbatim}[commandchars=\\\{\}]
\PY{n}{emails}\PY{p}{[}\PY{l+m+mi}{1}\PY{p}{]}\PY{p}{[}\PY{l+s+s1}{\PYZsq{}}\PY{l+s+s1}{headers}\PY{l+s+s1}{\PYZsq{}}\PY{p}{]}\PY{p}{[}\PY{l+m+mi}{36}\PY{p}{]}\PY{p}{[}\PY{l+s+s1}{\PYZsq{}}\PY{l+s+s1}{header}\PY{l+s+s1}{\PYZsq{}}\PY{p}{]}\PY{p}{,} \PY{n}{emails}\PY{p}{[}\PY{l+m+mi}{1}\PY{p}{]}\PY{p}{[}\PY{l+s+s1}{\PYZsq{}}\PY{l+s+s1}{headers}\PY{l+s+s1}{\PYZsq{}}\PY{p}{]}\PY{p}{[}\PY{l+m+mi}{36}\PY{p}{]}\PY{p}{[}\PY{l+s+s1}{\PYZsq{}}\PY{l+s+s1}{value}\PY{l+s+s1}{\PYZsq{}}\PY{p}{]}
\end{Verbatim}
\end{tcolorbox}

            \begin{tcolorbox}[breakable, size=fbox, boxrule=.5pt, pad at break*=1mm, opacityfill=0]
\prompt{Out}{outcolor}{52}{\boxspacing}
\begin{Verbatim}[commandchars=\\\{\}]
('Content-Type',
 'multipart/alternative;\textbackslash{}r\textbackslash{}n\textbackslash{}tboundary="\_000\_DS7PR17MB6730EE7BE9EA14814BB3AFE980
A2ADS7PR17MB6730namp\_"')
\end{Verbatim}
\end{tcolorbox}
        
    \hypertarget{return-path}{%
\subsubsection{Return-Path}\label{return-path}}

Return-Path indicates where undeliverable emails should be sent. It can
assist in identifying email redirection or bouncing patterns.

    \begin{tcolorbox}[breakable, size=fbox, boxrule=1pt, pad at break*=1mm,colback=cellbackground, colframe=cellborder]
\prompt{In}{incolor}{54}{\boxspacing}
\begin{Verbatim}[commandchars=\\\{\}]
\PY{n}{emails}\PY{p}{[}\PY{l+m+mi}{1}\PY{p}{]}\PY{p}{[}\PY{l+s+s1}{\PYZsq{}}\PY{l+s+s1}{headers}\PY{l+s+s1}{\PYZsq{}}\PY{p}{]}\PY{p}{[}\PY{l+m+mi}{39}\PY{p}{]}\PY{p}{[}\PY{l+s+s1}{\PYZsq{}}\PY{l+s+s1}{header}\PY{l+s+s1}{\PYZsq{}}\PY{p}{]}\PY{p}{,} \PY{n}{emails}\PY{p}{[}\PY{l+m+mi}{1}\PY{p}{]}\PY{p}{[}\PY{l+s+s1}{\PYZsq{}}\PY{l+s+s1}{headers}\PY{l+s+s1}{\PYZsq{}}\PY{p}{]}\PY{p}{[}\PY{l+m+mi}{39}\PY{p}{]}\PY{p}{[}\PY{l+s+s1}{\PYZsq{}}\PY{l+s+s1}{value}\PY{l+s+s1}{\PYZsq{}}\PY{p}{]}
\end{Verbatim}
\end{tcolorbox}

            \begin{tcolorbox}[breakable, size=fbox, boxrule=.5pt, pad at break*=1mm, opacityfill=0]
\prompt{Out}{outcolor}{54}{\boxspacing}
\begin{Verbatim}[commandchars=\\\{\}]
('Return-Path', 'krishnaraj.kpt@outlook.com')
\end{Verbatim}
\end{tcolorbox}
        
    \hypertarget{x-sender-ip}{%
\subsubsection{X-Sender-IP}\label{x-sender-ip}}

This header contains the IP address of the sender, which is essential
for tracking the origin of the email and potential geolocation.

    \begin{tcolorbox}[breakable, size=fbox, boxrule=1pt, pad at break*=1mm,colback=cellbackground, colframe=cellborder]
\prompt{In}{incolor}{55}{\boxspacing}
\begin{Verbatim}[commandchars=\\\{\}]
\PY{n}{emails}\PY{p}{[}\PY{l+m+mi}{1}\PY{p}{]}\PY{p}{[}\PY{l+s+s1}{\PYZsq{}}\PY{l+s+s1}{headers}\PY{l+s+s1}{\PYZsq{}}\PY{p}{]}\PY{p}{[}\PY{l+m+mi}{56}\PY{p}{]}\PY{p}{[}\PY{l+s+s1}{\PYZsq{}}\PY{l+s+s1}{header}\PY{l+s+s1}{\PYZsq{}}\PY{p}{]}\PY{p}{,} \PY{n}{emails}\PY{p}{[}\PY{l+m+mi}{1}\PY{p}{]}\PY{p}{[}\PY{l+s+s1}{\PYZsq{}}\PY{l+s+s1}{headers}\PY{l+s+s1}{\PYZsq{}}\PY{p}{]}\PY{p}{[}\PY{l+m+mi}{56}\PY{p}{]}\PY{p}{[}\PY{l+s+s1}{\PYZsq{}}\PY{l+s+s1}{value}\PY{l+s+s1}{\PYZsq{}}\PY{p}{]}
\end{Verbatim}
\end{tcolorbox}

            \begin{tcolorbox}[breakable, size=fbox, boxrule=.5pt, pad at break*=1mm, opacityfill=0]
\prompt{Out}{outcolor}{55}{\boxspacing}
\begin{Verbatim}[commandchars=\\\{\}]
('X-Sender-IP', '40.92.20.10')
\end{Verbatim}
\end{tcolorbox}
        
    \hypertarget{x-ms-exchange-transport-endtoendlatency}{%
\subsubsection{X-MS-Exchange-Transport-EndToEndLatency}\label{x-ms-exchange-transport-endtoendlatency}}

End-to-end latency is crucial for assessing the email's delivery speed,
which might reveal anomalies or delays in transit.

    \begin{tcolorbox}[breakable, size=fbox, boxrule=1pt, pad at break*=1mm,colback=cellbackground, colframe=cellborder]
\prompt{In}{incolor}{56}{\boxspacing}
\begin{Verbatim}[commandchars=\\\{\}]
\PY{n}{emails}\PY{p}{[}\PY{l+m+mi}{1}\PY{p}{]}\PY{p}{[}\PY{l+s+s1}{\PYZsq{}}\PY{l+s+s1}{headers}\PY{l+s+s1}{\PYZsq{}}\PY{p}{]}\PY{p}{[}\PY{l+m+mi}{71}\PY{p}{]}\PY{p}{[}\PY{l+s+s1}{\PYZsq{}}\PY{l+s+s1}{header}\PY{l+s+s1}{\PYZsq{}}\PY{p}{]}\PY{p}{,} \PY{n}{emails}\PY{p}{[}\PY{l+m+mi}{1}\PY{p}{]}\PY{p}{[}\PY{l+s+s1}{\PYZsq{}}\PY{l+s+s1}{headers}\PY{l+s+s1}{\PYZsq{}}\PY{p}{]}\PY{p}{[}\PY{l+m+mi}{71}\PY{p}{]}\PY{p}{[}\PY{l+s+s1}{\PYZsq{}}\PY{l+s+s1}{value}\PY{l+s+s1}{\PYZsq{}}\PY{p}{]}
\end{Verbatim}
\end{tcolorbox}

            \begin{tcolorbox}[breakable, size=fbox, boxrule=.5pt, pad at break*=1mm, opacityfill=0]
\prompt{Out}{outcolor}{56}{\boxspacing}
\begin{Verbatim}[commandchars=\\\{\}]
('X-MS-Exchange-Transport-EndToEndLatency', '00:00:02.4306131')
\end{Verbatim}
\end{tcolorbox}
        
    \hypertarget{x-microsoft-antispam-mailbox-delivery}{%
\subsubsection{X-Microsoft-Antispam-Mailbox-Delivery}\label{x-microsoft-antispam-mailbox-delivery}}

This header provides information about the email's delivery and its
classification as spam or not, aiding in filtering and threat analysis.

    \begin{tcolorbox}[breakable, size=fbox, boxrule=1pt, pad at break*=1mm,colback=cellbackground, colframe=cellborder]
\prompt{In}{incolor}{57}{\boxspacing}
\begin{Verbatim}[commandchars=\\\{\}]
\PY{n}{emails}\PY{p}{[}\PY{l+m+mi}{1}\PY{p}{]}\PY{p}{[}\PY{l+s+s1}{\PYZsq{}}\PY{l+s+s1}{headers}\PY{l+s+s1}{\PYZsq{}}\PY{p}{]}\PY{p}{[}\PY{l+m+mi}{73}\PY{p}{]}\PY{p}{[}\PY{l+s+s1}{\PYZsq{}}\PY{l+s+s1}{header}\PY{l+s+s1}{\PYZsq{}}\PY{p}{]}\PY{p}{,} \PY{n}{emails}\PY{p}{[}\PY{l+m+mi}{1}\PY{p}{]}\PY{p}{[}\PY{l+s+s1}{\PYZsq{}}\PY{l+s+s1}{headers}\PY{l+s+s1}{\PYZsq{}}\PY{p}{]}\PY{p}{[}\PY{l+m+mi}{73}\PY{p}{]}\PY{p}{[}\PY{l+s+s1}{\PYZsq{}}\PY{l+s+s1}{value}\PY{l+s+s1}{\PYZsq{}}\PY{p}{]}
\end{Verbatim}
\end{tcolorbox}

            \begin{tcolorbox}[breakable, size=fbox, boxrule=.5pt, pad at break*=1mm, opacityfill=0]
\prompt{Out}{outcolor}{57}{\boxspacing}
\begin{Verbatim}[commandchars=\\\{\}]
('X-Microsoft-Antispam-Mailbox-Delivery',
 '\textbackslash{}r\textbackslash{}n\textbackslash{}tucf:0;jmr:0;ex:0;auth:1;dest:I;OFR:SpamFilterPass;ENG:(5062000305)(92022
1119095)(90000117)(920221120095)(90013020)(91025020)(91040095)(9050020)(9065024)
(9100341)(1000006)(944500132)(2008001134)(4810010)(4910033)(9920006)(9510006)(10
105021)(9320005)(9230038)(120001);')
\end{Verbatim}
\end{tcolorbox}
        
    \hypertarget{x-message-info-and-x-message-delivery}{%
\subsubsection{X-Message-Info and
X-Message-Delivery}\label{x-message-info-and-x-message-delivery}}

These headers contain miscellaneous information about the email's
handling, which can be valuable for tracking and understanding the
email's journey.

    \begin{tcolorbox}[breakable, size=fbox, boxrule=1pt, pad at break*=1mm,colback=cellbackground, colframe=cellborder]
\prompt{In}{incolor}{58}{\boxspacing}
\begin{Verbatim}[commandchars=\\\{\}]
\PY{n}{emails}\PY{p}{[}\PY{l+m+mi}{1}\PY{p}{]}\PY{p}{[}\PY{l+s+s1}{\PYZsq{}}\PY{l+s+s1}{headers}\PY{l+s+s1}{\PYZsq{}}\PY{p}{]}\PY{p}{[}\PY{l+m+mi}{75}\PY{p}{]}\PY{p}{[}\PY{l+s+s1}{\PYZsq{}}\PY{l+s+s1}{header}\PY{l+s+s1}{\PYZsq{}}\PY{p}{]}\PY{p}{,} \PY{n}{emails}\PY{p}{[}\PY{l+m+mi}{1}\PY{p}{]}\PY{p}{[}\PY{l+s+s1}{\PYZsq{}}\PY{l+s+s1}{headers}\PY{l+s+s1}{\PYZsq{}}\PY{p}{]}\PY{p}{[}\PY{l+m+mi}{75}\PY{p}{]}\PY{p}{[}\PY{l+s+s1}{\PYZsq{}}\PY{l+s+s1}{value}\PY{l+s+s1}{\PYZsq{}}\PY{p}{]}
\end{Verbatim}
\end{tcolorbox}

            \begin{tcolorbox}[breakable, size=fbox, boxrule=.5pt, pad at break*=1mm, opacityfill=0]
\prompt{Out}{outcolor}{58}{\boxspacing}
\begin{Verbatim}[commandchars=\\\{\}]
('X-Message-Delivery', 'Vj0xLjE7dXM9MDtsPTA7YT0xO0Q9MTtHRD0xO1NDTD0tMQ==')
\end{Verbatim}
\end{tcolorbox}
        
    Defining a function to get information about an ip address for sender ip
analysis

    \begin{tcolorbox}[breakable, size=fbox, boxrule=1pt, pad at break*=1mm,colback=cellbackground, colframe=cellborder]
\prompt{In}{incolor}{65}{\boxspacing}
\begin{Verbatim}[commandchars=\\\{\}]
\PY{k+kn}{import} \PY{n+nn}{requests}

\PY{k}{def} \PY{n+nf}{get\PYZus{}ip\PYZus{}information}\PY{p}{(}\PY{n}{ip\PYZus{}address}\PY{p}{)}\PY{p}{:}
    \PY{k}{def} \PY{n+nf}{get\PYZus{}ip\PYZus{}location}\PY{p}{(}\PY{p}{)}\PY{p}{:}
        \PY{c+c1}{\PYZsh{} Make a GET request to ipinfo.io with the IP address}
        \PY{n}{url} \PY{o}{=} \PY{l+s+sa}{f}\PY{l+s+s2}{\PYZdq{}}\PY{l+s+s2}{https://ipinfo.io/}\PY{l+s+si}{\PYZob{}}\PY{n}{ip\PYZus{}address}\PY{l+s+si}{\PYZcb{}}\PY{l+s+s2}{/json}\PY{l+s+s2}{\PYZdq{}}
        \PY{n}{response} \PY{o}{=} \PY{n}{requests}\PY{o}{.}\PY{n}{get}\PY{p}{(}\PY{n}{url}\PY{p}{)}

        \PY{k}{if} \PY{n}{response}\PY{o}{.}\PY{n}{status\PYZus{}code} \PY{o}{==} \PY{l+m+mi}{200}\PY{p}{:}
            \PY{n}{data} \PY{o}{=} \PY{n}{response}\PY{o}{.}\PY{n}{json}\PY{p}{(}\PY{p}{)}
            \PY{k}{return} \PY{n}{data}
        \PY{k}{else}\PY{p}{:}
            \PY{k}{return} \PY{k+kc}{None}

    \PY{n}{location\PYZus{}info} \PY{o}{=} \PY{n}{get\PYZus{}ip\PYZus{}location}\PY{p}{(}\PY{p}{)}
    \PY{c+c1}{\PYZsh{} print(location\PYZus{}info)}
    \PY{k}{if} \PY{n}{location\PYZus{}info}\PY{p}{:}
        \PY{c+c1}{\PYZsh{} Print the location information}
        \PY{n+nb}{print}\PY{p}{(}\PY{l+s+sa}{f}\PY{l+s+s2}{\PYZdq{}}\PY{l+s+s2}{IP Address: }\PY{l+s+si}{\PYZob{}}\PY{n}{location\PYZus{}info}\PY{p}{[}\PY{l+s+s1}{\PYZsq{}}\PY{l+s+s1}{ip}\PY{l+s+s1}{\PYZsq{}}\PY{p}{]}\PY{l+s+si}{\PYZcb{}}\PY{l+s+s2}{\PYZdq{}}\PY{p}{)}
        \PY{n+nb}{print}\PY{p}{(}\PY{l+s+sa}{f}\PY{l+s+s2}{\PYZdq{}}\PY{l+s+s2}{Hostname: }\PY{l+s+si}{\PYZob{}}\PY{n}{location\PYZus{}info}\PY{p}{[}\PY{l+s+s1}{\PYZsq{}}\PY{l+s+s1}{hostname}\PY{l+s+s1}{\PYZsq{}}\PY{p}{]}\PY{l+s+si}{\PYZcb{}}\PY{l+s+s2}{\PYZdq{}}\PY{p}{)}
        \PY{n+nb}{print}\PY{p}{(}\PY{l+s+sa}{f}\PY{l+s+s2}{\PYZdq{}}\PY{l+s+s2}{City: }\PY{l+s+si}{\PYZob{}}\PY{n}{location\PYZus{}info}\PY{p}{[}\PY{l+s+s1}{\PYZsq{}}\PY{l+s+s1}{city}\PY{l+s+s1}{\PYZsq{}}\PY{p}{]}\PY{l+s+si}{\PYZcb{}}\PY{l+s+s2}{\PYZdq{}}\PY{p}{)}
        \PY{c+c1}{\PYZsh{} print(f\PYZdq{}Region: \PYZob{}location\PYZus{}info[\PYZsq{}region\PYZsq{}]\PYZcb{}\PYZdq{})}
        \PY{n+nb}{print}\PY{p}{(}\PY{l+s+sa}{f}\PY{l+s+s2}{\PYZdq{}}\PY{l+s+s2}{Country: }\PY{l+s+si}{\PYZob{}}\PY{n}{location\PYZus{}info}\PY{p}{[}\PY{l+s+s1}{\PYZsq{}}\PY{l+s+s1}{country}\PY{l+s+s1}{\PYZsq{}}\PY{p}{]}\PY{l+s+si}{\PYZcb{}}\PY{l+s+s2}{\PYZdq{}}\PY{p}{)}
        \PY{n+nb}{print}\PY{p}{(}\PY{l+s+sa}{f}\PY{l+s+s2}{\PYZdq{}}\PY{l+s+s2}{Location: }\PY{l+s+si}{\PYZob{}}\PY{n}{location\PYZus{}info}\PY{p}{[}\PY{l+s+s1}{\PYZsq{}}\PY{l+s+s1}{loc}\PY{l+s+s1}{\PYZsq{}}\PY{p}{]}\PY{l+s+si}{\PYZcb{}}\PY{l+s+s2}{\PYZdq{}}\PY{p}{)}
        \PY{c+c1}{\PYZsh{} print(f\PYZdq{}Organization: \PYZob{}location\PYZus{}info[\PYZsq{}org\PYZsq{}]\PYZcb{}\PYZdq{})}
        \PY{n+nb}{print}\PY{p}{(}\PY{l+s+sa}{f}\PY{l+s+s2}{\PYZdq{}}\PY{l+s+s2}{Timezone: }\PY{l+s+si}{\PYZob{}}\PY{n}{location\PYZus{}info}\PY{p}{[}\PY{l+s+s1}{\PYZsq{}}\PY{l+s+s1}{timezone}\PY{l+s+s1}{\PYZsq{}}\PY{p}{]}\PY{l+s+si}{\PYZcb{}}\PY{l+s+s2}{\PYZdq{}}\PY{p}{)}
    \PY{k}{else}\PY{p}{:}
        \PY{n+nb}{print}\PY{p}{(}\PY{l+s+s2}{\PYZdq{}}\PY{l+s+s2}{Unable to retrieve location information for the IP address.}\PY{l+s+s2}{\PYZdq{}}\PY{p}{)}
\end{Verbatim}
\end{tcolorbox}

    Analysing Sender IP Address for all emails

    \begin{tcolorbox}[breakable, size=fbox, boxrule=1pt, pad at break*=1mm,colback=cellbackground, colframe=cellborder]
\prompt{In}{incolor}{66}{\boxspacing}
\begin{Verbatim}[commandchars=\\\{\}]
\PY{k}{for} \PY{n}{email} \PY{o+ow}{in} \PY{n}{emails}\PY{p}{:}
    \PY{k}{for} \PY{n}{header} \PY{o+ow}{in} \PY{n}{email}\PY{p}{[}\PY{l+s+s1}{\PYZsq{}}\PY{l+s+s1}{headers}\PY{l+s+s1}{\PYZsq{}}\PY{p}{]}\PY{p}{:}
        \PY{k}{if} \PY{n}{header}\PY{p}{[}\PY{l+s+s1}{\PYZsq{}}\PY{l+s+s1}{header}\PY{l+s+s1}{\PYZsq{}}\PY{p}{]} \PY{o}{==} \PY{l+s+s1}{\PYZsq{}}\PY{l+s+s1}{X\PYZhy{}Sender\PYZhy{}IP}\PY{l+s+s1}{\PYZsq{}}\PY{p}{:}
            \PY{n+nb}{print}\PY{p}{(}\PY{l+s+s2}{\PYZdq{}}\PY{l+s+s2}{Subject of Email: }\PY{l+s+s2}{\PYZdq{}}\PY{p}{,} \PY{n}{email}\PY{p}{[}\PY{l+s+s1}{\PYZsq{}}\PY{l+s+s1}{subject}\PY{l+s+s1}{\PYZsq{}}\PY{p}{]}\PY{p}{)}
            \PY{n+nb}{print}\PY{p}{(}\PY{l+s+s2}{\PYZdq{}}\PY{l+s+s2}{IP Address Information:}\PY{l+s+s2}{\PYZdq{}}\PY{p}{)}
            \PY{n}{get\PYZus{}ip\PYZus{}information}\PY{p}{(}\PY{n}{header}\PY{p}{[}\PY{l+s+s1}{\PYZsq{}}\PY{l+s+s1}{value}\PY{l+s+s1}{\PYZsq{}}\PY{p}{]}\PY{p}{)}

            \PY{n+nb}{print}\PY{p}{(}\PY{p}{)}
\end{Verbatim}
\end{tcolorbox}

    \begin{Verbatim}[commandchars=\\\{\}]
Subject of Email:  Sending from phone gmail app
IP Address Information:
IP Address: 209.85.218.48
Hostname: mail-ej1-f48.google.com
City: Oudeschip
Country: NL
Location: 53.4300,6.8264
Timezone: Europe/Amsterdam

Subject of Email:  Self mail from web outlook client
IP Address Information:
IP Address: 40.92.20.10
Hostname: mail-bn8nam11olkn2010.outbound.protection.outlook.com
City: Boydton
Country: US
Location: 36.6676,-78.3875
Timezone: America/New\_York

Subject of Email:  Self test mail from gmail web client
IP Address Information:
IP Address: 209.85.167.178
Hostname: mail-oi1-f178.google.com
City: Tulsa
Country: US
Location: 36.1540,-95.9928
Timezone: America/Chicago

Subject of Email:  self test mail from mit gmail client
IP Address Information:
IP Address: 209.85.219.47
Hostname: mail-qv1-f47.google.com
City: Raleigh
Country: US
Location: 35.7721,-78.6386
Timezone: America/New\_York

Subject of Email:  From movile outlook app
IP Address Information:
IP Address: 40.92.242.29
Hostname: mail-ps2kor01olkn2029.outbound.protection.outlook.com
City: Busan
Country: KR
Location: 35.1017,129.0300
Timezone: Asia/Seoul

Subject of Email:  Mail sent from phone
IP Address Information:
IP Address: 209.85.160.175
Hostname: mail-qt1-f175.google.com
City: Charlotte
Country: US
Location: 35.2271,-80.8431
Timezone: America/New\_York

Subject of Email:  Mother's phone Gmail app
IP Address Information:
IP Address: 209.85.160.171
Hostname: mail-qt1-f171.google.com
City: Charlotte
Country: US
Location: 35.2271,-80.8431
Timezone: America/New\_York

Subject of Email:  Sending from browser with vpn connection on
IP Address Information:
IP Address: 209.85.208.52
Hostname: mail-ed1-f52.google.com
City: Oudeschip
Country: NL
Location: 53.4300,6.8264
Timezone: Europe/Amsterdam

    \end{Verbatim}


\section{Analysing Effect of Network Connection on IP Headers}

\subsection{Using VPN}
Emails were sent by "parthzarekar3@outlook.com" to "krishnaraj.kpt@outlook.com" with connection to a VPN. 

The IP Information was taken from the headers. 

\begin{verbatim}
{
    "header": "X-Sender-IP",
    "value": "209.85.208.52"
},
\end{verbatim}

From This the IP Address was taken and searched on iplookup.com

The IP Information was: 
\begin{verbatim}
IP Address Information:
IP Address: 209.85.208.52
Hostname: mail-ed1-f52.google.com
City: Oudeschip
Country: NL
Location: 53.4300,6.8264
Timezone: Europe/Amsterdam
\end{verbatim}

\begin{figure}[H]
    \centering
    \includegraphics[width=.85\textwidth]{52.png}
    \caption{}
\end{figure}

\subsection{Without using VPN, on Wifi}

\begin{verbatim}
    {
        "header": "X-Sender-IP",
        "value": "209.85.160.171"
    },
    \end{verbatim}
    
    From This the IP Address was taken and searched on iplookup.com
    
    The IP Information was: 
    \begin{verbatim}
    Subject of Email:  Mother's phone Gmail app
    IP Address Information:
    IP Address: 209.85.160.171
    Hostname: mail-qt1-f171.google.com
    City: Charlotte
    Country: US
    Location: 35.2271,-80.8431
    Timezone: America/New\_York
    \end{verbatim}
    
    \begin{figure}[H]
        \centering
        \includegraphics[width=.85\textwidth]{171.png}
        \caption{}
    \end{figure}

    
\subsection{Without using VPN, on Mobile Data}

\begin{verbatim}
    {
        "header": "X-Sender-IP",
        "value": "209.85.208.41"
    },
    \end{verbatim}
    
    From This the IP Address was taken and searched on iplookup.com
    
    The IP Information was: 
    \begin{verbatim}
    Subject of Email:  Sending from browser using mobile hotspot
    IP Address Information:
    IP Address: 209.85.208.41
    Hostname: mail-ed1-f41.google.com
    City: Oudeschip
    Country: NL
    Location: 53.4300,6.8264
    Timezone: Europe/Amsterdam
    \end{verbatim}
    
    \begin{figure}[H]
        \centering
        \includegraphics[width=.85\textwidth]{41.png}
        \caption{}
    \end{figure}



\begin{figure}[H]
    \centering
    \includegraphics[width=.45\textwidth]{parth gmail sent.png}
    \caption{Emails Sent from Parth to Krish}
\end{figure}
    

\begin{figure}[H]
    \centering
    \includegraphics[width=.45\textwidth]{parth outlook sent.png}
    \caption{Emails sent from Parth to Krish in outlook.}
\end{figure}


\begin{figure}[H]
    \centering
    \includegraphics[width=.45\textwidth]{krish outlook received.png}
    \caption{Images Received by Krish. }
\end{figure}

% \section{Code}
% \lstinputlisting[language=Python, caption="DSA Signature Validity using PyCrypto Library"]{../Programs/Assignment_7/dsa using lib.py}

\section{Conclusion}
We have explored how to analyze email headers and understand their significance:

\begin{enumerate}
    \item Email headers contain vital information about the email's origin, routing, and content.
    \item Analyzing email headers is crucial for diagnosing email delivery issues, ensuring authenticity, and enhancing cybersecurity.
    \item Key header fields include "From," "To," "Received," and "Message-ID."
\end{enumerate}

We have also learned about email routing:

\begin{enumerate}
    \item Regardless of the user's connection method (VPN, Mobile Data, or WiFi), emails are routed through servers.
    \item Standard email protocol involves servers sending emails from one server to another.
\end{enumerate}

Additionally, it was discovered that Outlook has security vulnerabilities:

\begin{enumerate}
    \item Outlook's security, especially regarding password protection, has been found lacking.
    \item Even after enabling 2 Factor Authentication (2FA), a simple Python script using the imaplib package can extract emails with just the password.
\end{enumerate}

\clearpage

\pagebreak
\begin{thebibliography}{}
    \bibitem{RFC 5322}
    \href{https://tools.ietf.org/html/rfc5322}{Internet Message Format}\\
    Internet Message Format

    \bibitem{Email Headers}
    \href{https://www.emailonacid.com/blog/article/email-development/mail-headers-what-they-are-and-how-to-analyze-them/}{Email Headers} \\"Email Headers: What They Are and How to Analyze Them" by Email on Acid

    \bibitem{How to Read Email Headers}
    \href{https://mxtoolbox.com/Public/Content/EmailHeaders/}{How to Read Email Headers} \\MX Toolbox

\end{thebibliography}
\end{document}