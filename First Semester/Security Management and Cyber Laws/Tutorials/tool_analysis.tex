\documentclass[11pt]{article}

% Preamble

\usepackage[margin=1in]{geometry}
\usepackage{amsfonts, amsmath, amssymb}
\usepackage{fancyhdr, float, graphicx}
\usepackage[utf8]{inputenc} % Required for inputting international characters
\usepackage[T1]{fontenc} % Output font encoding for international characters
\usepackage{fouriernc} % Use the New Century Schoolbook font
\usepackage[nottoc, notlot, notlof]{tocbibind}
\usepackage{listings}
\usepackage{xcolor}
\usepackage{blindtext}
\usepackage{hyperref}
\hypersetup{
    colorlinks=true,
    linkcolor=black,
    filecolor=magenta,
    urlcolor=cyan,
    pdfpagemode=FullScreen,
}

\definecolor{codegreen}{rgb}{0,0.6,0}
\definecolor{codegray}{rgb}{0.5,0.5,0.5}
\definecolor{codepurple}{rgb}{0.58,0,0.82}
\definecolor{backcolour}{rgb}{0.95,0.95,0.92}

\lstdefinestyle{mystyle}{
    backgroundcolor=\color{backcolour},
    commentstyle=\color{codegreen},
    keywordstyle=\color{magenta},
    numberstyle=\tiny\color{codegray},
    stringstyle=\color{codepurple},
    basicstyle=\ttfamily\footnotesize,
    breakatwhitespace=false,
    breaklines=true,
    captionpos=b,
    keepspaces=true,
    numbers=left,
    numbersep=5pt,
    showspaces=false,
    showstringspaces=false,
    showtabs=false,
    tabsize=2
}

\lstset{style=mystyle}

% Header and Footer
\pagestyle{fancy}
\fancyhead{}
\fancyfoot{}
\fancyhead[L]{\textit{\Large{Security Management and Cyber Laws - 3nd Year B. Tech}}}
\fancyhead[R]{\textit{Krishnaraj T}}
\fancyfoot[C]{\thepage}
\renewcommand{\footrulewidth}{1pt}

% Other Doc Editing
% \parindent 0ex
%\renewcommand{\baselinestretch}{1.5}

\begin{document}

\begin{titlepage}
    \centering

    %---------------------------NAMES-------------------------------

    \huge\textsc{
        MIT World Peace University
    }\\

    \vspace{0.75\baselineskip} % space after Uni Name

    \LARGE{
        Security Management and Cyber Laws\\
        Third Year B. Tech, Semester 5
    }

    \vfill % space after Sub Name

    %--------------------------TITLE-------------------------------

    \rule{\textwidth}{1.6pt}\vspace{-\baselineskip}\vspace{2pt}
    \rule{\textwidth}{0.6pt}
    \vspace{0.75\baselineskip} % Whitespace above the title

    \huge{\textsc{
            Tools and Techniques used in Ethical Hacking
        }} \\

    \vspace{0.5\baselineskip} % Whitespace below the title
    \rule{\textwidth}{0.6pt}\vspace{-\baselineskip}\vspace{2.8pt}
    \rule{\textwidth}{1.6pt}

    \vspace{1\baselineskip} % Whitespace after the title block

    %--------------------------SUBTITLE --------------------------	

    \LARGE\textsc{
        SMCL Graded Tutorial for Unit 5\\
        John the Ripper
    } % Subtitle or further description
    \vfill

    %--------------------------AUTHOR-------------------------------

    Prepared By \vspace{0.5\baselineskip} % Whitespace before the editors

    \Large{
        Krishnaraj Thadesar \\
        Cyber Security and Forensics\\
        Batch A1, PA 10
    }

    \vspace{0.5\baselineskip} % Whitespace below the editor list
    \today

\end{titlepage}

\tableofcontents
\thispagestyle{empty}
\clearpage

\setcounter{page}{1}

\section{Aim}
To study and use John the Ripper to crack passwords.
\section{Objectives}
\begin{enumerate}
    \item To study the different features of John the Ripper.
    \item To study the different ways to crack passwords.
    \item To study the need of John the Ripper.
\end{enumerate}

\section{What is John the Ripper?}
John the Ripper is a widely used open-source password cracking tool. It is designed to identify weak passwords by employing various password cracking techniques, including dictionary attacks, brute force attacks, and more. John the Ripper is highly flexible and supports multiple hash algorithms.

\subsection{Features}
John the Ripper offers the following key features:
\begin{itemize}
    \item Support for various password hash algorithms such as DES, MD5, SHA-1, SHA-256, etc.
    \item Configurability for different attack modes and methods.
    \item Compatibility with different platforms, including Unix, Windows, and more.
    \item Multiple attack modes, including dictionary attacks, brute force attacks, and hybrid attacks.
    \item Availability of community-contributed patches and enhancements.
\end{itemize}

\subsection{Ways to Crack Passwords}
\begin{enumerate}
    \item \textbf{Dictionary Attack:} Uses a wordlist to try potential passwords.
    \item \textbf{Brute Force Attack:} Tries all possible combinations of characters systematically.
    \item \textbf{Hybrid Attack:} Combines dictionary and brute force attacks for increased effectiveness.
    \item \textbf{Rainbow Table Attack:} Uses precomputed tables of hashes for quick password lookup.
    \item \textbf{Rule-based Attack:} Applies a set of rules to generate passwords.
    \item \textbf{Mask Attack:} Uses a predefined mask to generate passwords.
    \item \textbf{Permutation Attack:} Generates passwords using permutations of characters.
    \item \textbf{Markov Attack:} Uses Markov chains for intelligent password guessing.
    \item \textbf{PRINCE Attack:} Utilizes a PRINCE algorithm for password generation.
\end{enumerate}

\subsection{Need of John the Ripper}
John the Ripper is crucial for:
\begin{itemize}
    \item Assessing password security by identifying weak or easily guessable passwords.
    \item Conducting security audits to uncover vulnerabilities in password policies.
    \item Educating users about the importance of strong and secure passwords.
    \item Strengthening overall cybersecurity by proactively addressing password weaknesses.
\end{itemize}

\subsection{Installation}
To install John the Ripper on your system, follow these steps:
\begin{enumerate}
    \item Download the latest version from the official website.
    \item Extract the downloaded archive.
    \item Navigate to the extracted directory.
    \item Run the installation command appropriate for your platform.
\end{enumerate}

\section{Usage}
We will now explore different methods to crack passwords using John the Ripper.
\subsection{Dictionary Attack}

\subsubsection{JTR Command}

\begin{verbatim}
john --wordlist=dictionary.txt hash_file
\end{verbatim}

\subsubsection{Pros}

\begin{enumerate}
    \item Fast and efficient for passwords based on dictionary words.
    \item Utilizes a predefined list of potential passwords.
    \item Suitable for users who might use common words as passwords.
\end{enumerate}

\subsubsection{Cons}

\begin{enumerate}
    \item Less effective against passwords with complex structures or variations.
    \item Limited by the contents of the chosen wordlist.
    \item Might not be successful against passwords with additional characters or patterns.
\end{enumerate}

\subsubsection{Steps}

\begin{enumerate}
    \item Replace \texttt{hash\_file} with the actual name of your password hash file.
    \item Ensure that \texttt{dictionary.txt} contains a comprehensive wordlist.
    \item Run the provided JTR command in your terminal.
\end{enumerate}


\subsubsection{Screenshots}

\subsection{Brute Force Attack}

\subsubsection{JTR Command}

\begin{verbatim}
john --incremental hash_file
\end{verbatim}

\subsubsection{Pros}

\begin{enumerate}
    \item Exhaustive and covers all possible combinations.
    \item Guarantees success given enough time and resources.
\end{enumerate}

\subsubsection{Cons}

\begin{enumerate}
    \item Time-consuming, especially for complex passwords.
    \item Resource-intensive, requires substantial computing power.
\end{enumerate}

\subsubsection{Steps}

\begin{enumerate}
    \item Replace \texttt{hash\_file} with the actual name of your password hash file.
    \item Execute the provided JTR command in your terminal.
\end{enumerate}

\subsubsection{Screenshots}

\subsection{Hybrid Attack}

\subsubsection{JTR Command}

\begin{verbatim}
john --incremental --wordlist=dictionary.txt hash_file
\end{verbatim}

\subsubsection{Pros}

\begin{enumerate}
    \item Combines the thoroughness of brute force with the efficiency of a wordlist.
    \item More time-effective than a pure brute force approach.
\end{enumerate}

\subsubsection{Cons}

\begin{enumerate}
    \item Success highly depends on the quality of the wordlist.
    \item May not be as fast as dictionary attacks alone.
\end{enumerate}

\subsubsection{Steps}

\begin{enumerate}
    \item Replace \texttt{hash\_file} with the actual name of your password hash file.
    \item Provide a comprehensive wordlist in \texttt{dictionary.txt}.
    \item Run the specified JTR command.
\end{enumerate}

\subsubsection{Screenshots}

% Repeat the above structure for the remaining attack methods

\subsection{Rainbow Table Attack}

\subsubsection{JTR Command}

\begin{verbatim}
john --format=raw-md5 --external=rainbow_table.rt hash_file
\end{verbatim}

\subsubsection{Pros}

\begin{enumerate}
    \item Rapid lookup for precomputed hashes.
    \item Efficient for commonly used passwords.
\end{enumerate}

\subsubsection{Cons}

\begin{enumerate}
    \item Limited to precomputed tables.
    \item Requires significant storage for extensive tables.
\end{enumerate}

\subsubsection{Steps}

\begin{enumerate}
    \item Replace \texttt{hash\_file} with the actual name of your password hash file.
    \item Use an appropriate precomputed rainbow table (\texttt{rainbow\_table.rt}).
    \item Execute the provided JTR command.
\end{enumerate}

\subsubsection{Screenshots}

% Continue this pattern for the remaining attacks

\subsection{Rule-based Attack}

\subsubsection{JTR Command}

\begin{verbatim}
john --wordlist=dictionary.txt --rules hash_file
\end{verbatim}

\subsubsection{Pros}

\begin{enumerate}
    \item Dynamically applies rules for password generation.
    \item Efficient for complex password structures.
\end{enumerate}

\subsubsection{Cons}

\begin{enumerate}
    \item Success depends on the effectiveness of the rule set.
    \item May require fine-tuning for optimal results.
\end{enumerate}

\subsubsection{Steps}

\begin{enumerate}
    \item Replace \texttt{hash\_file} with the actual name of your password hash file.
    \item Customize and provide rules for password generation.
    \item Run the specified JTR command.
\end{enumerate}

\subsubsection{Screenshots}

\subsection{Mask Attack}

\subsubsection{JTR Command}

\begin{verbatim}
john --mask=?l?d?u hash_file
\end{verbatim}

\subsubsection{Pros}

\begin{enumerate}
    \item Allows customization based on known patterns.
    \item Useful for structured password cracking.
\end{enumerate}

\subsubsection{Cons}

\begin{enumerate}
    \item Resource-intensive for complex mask patterns.
    \item Success depends on accurately defining the password structure.
\end{enumerate}

\subsubsection{Steps}

\begin{enumerate}
    \item Replace \texttt{hash\_file} with the actual name of your password hash file.
    \item Adjust the mask pattern (\texttt{?l}, \texttt{?d}, \texttt{?u} for lowercase, digit, uppercase) based on the expected password structure.
    \item Execute the specified JTR command.
\end{enumerate}

\subsubsection{Screenshots}

\subsection{Permutation Attack}

\subsubsection{JTR Command}

\begin{verbatim}
john --incremental=perm hash_file
\end{verbatim}

\subsubsection{Pros}

\begin{enumerate}
    \item Comprehensive by trying all permutations of characters.
    \item Useful for cases where the password structure is not well-defined.
\end{enumerate}

\subsubsection{Cons}

\begin{enumerate}
    \item Highly resource-intensive and time-consuming.
    \item Success depends on the length and complexity of the password.
\end{enumerate}

\subsubsection{Steps}

\begin{enumerate}
    \item Replace \texttt{hash\_file} with the actual name of your password hash file.
    \item Execute the specified JTR command.
\end{enumerate}

\subsubsection{Screenshots}

\begin{figure}[H]
    \centering
    \includegraphics[width=.95\textwidth]{zip pass.png}
    \caption{Making a Zip password}
\end{figure}

\begin{figure}[H]
    \centering
    \includegraphics[width=.95\textwidth]{zip cracked.png}
    \caption{Crackign a Zip password with JTR}
\end{figure}

\begin{figure}[H]
    \centering
    \includegraphics[width=.95\textwidth]{zip_with_new.png}
    \caption{New Zip Password Cracked with Dictionary Attack}
\end{figure}

\begin{figure}[H]
    \centering
    \includegraphics[width=.95\textwidth]{new user.png}
    \caption{Making a new user with a simple password}
\end{figure}

\begin{figure}[H]
    \centering
    \includegraphics[width=.95\textwidth]{shadow.png}
    \caption{Getting the shadow file. }
\end{figure}

\begin{figure}[H]
    \centering
    \includegraphics[width=.95\textwidth]{john shadow.png}
    \caption{Trying to crack the file with Jtr}
\end{figure}

\begin{figure}[H]
    \centering
    \includegraphics[width=.95\textwidth]{md5hashcrack.png}
    \caption{Cracking MD5 Hashes with JTR}
\end{figure}

\section{Code}

\lstinputlisting[language=Python, caption=Script to make hashes of passwords. ]{save_pass_hash.py}

\section{Platform}
\textbf{Operating System}: Arch Linux x86-64 \\
\textbf{IDEs or Text Editors Used}: Visual Studio Code\\
\textbf{Compilers or Interpreters}: Python 3.10.1\\

\section{Conclusion}
Thus, we have successfully installed and used John the Ripper to crack
passwords, and also learnt about the need of John the Ripper.

\clearpage

\begin{thebibliography}{99}
    \bibitem{johntheripper}
    John the Ripper. \\
    Website: \url{https://www.openwall.com/john/}

    \bibitem{passwordcracking}
    Password Cracking Techniques. \\
    Website: \url{https://www.cybersecurity-insiders.com/password-cracking-techniques/}

    \bibitem{dictionaryattack}
    Dictionary Attack. \\
    Website: \url{https://www.techopedia.com/definition/1779/dictionary-attack}

    \bibitem{bruteforceattack}
    Brute Force Attack. \\
    Website: \url{https://www.imperva.com/learn/application-security/brute-force-attack/}

    \bibitem{hybridattack}
    Hybrid Attack. \\
    Website: \url{https://www.techopedia.com/definition/1779/dictionary-attack}

    \bibitem{rainbowtableattack}
    Rainbow Table Attack. \\
    Website: \url{https://www.techopedia.com/definition/1779/dictionary-attack}

    \bibitem{rulebasedattack}
    Rule-based Attack. \\
    Website: \url{https://www.techopedia.com/definition/1779/dictionary-attack}

    \bibitem{maskattack}
    Mask Attack. \\
    Website: \url{https://www.techopedia.com/definition/1779/dictionary-attack}

    \bibitem{permutationattack}
    Permutation Attack. \\
    Website: \url{https://www.techopedia.com/definition/1779/dictionary-attack}

    \bibitem{mitworldpeaceuniversity}
    MIT World Peace University. \\
    Website: \url{https://mitwpu.edu.in/}

    \bibitem{securitymanagementandcyberlaws}
    Security Management and Cyber Laws. \\
    Website: \url{https://mitwpu.edu.in/course/b-tech-cyber-security/}
\end{thebibliography}

\end{document}