% This is a Basic Assignment Paper but with like Code and stuff allowed in it, there is also url, hyperlinks from contents included. 

\documentclass[11pt]{article}

% Preamble

\usepackage[margin=1in]{geometry}
\usepackage{amsfonts, amsmath, amssymb, amsthm}
\usepackage{fancyhdr, float, graphicx}
\usepackage[utf8]{inputenc} % Required for inputting international characters
\usepackage[T1]{fontenc} % Output font encoding for international characters
\usepackage{fouriernc} % Use the New Century Schoolbook font
\usepackage[nottoc, notlot, notlof]{tocbibind}
\usepackage{listings}
\usepackage{xcolor}
\usepackage{blindtext}
\usepackage{hyperref}
\hypersetup{
    colorlinks=true,
    linkcolor=black,
    filecolor=magenta,      
    urlcolor=cyan,
    pdfpagemode=FullScreen,
    }

\definecolor{codegreen}{rgb}{0,0.6,0}
\definecolor{codegray}{rgb}{0.5,0.5,0.5}
\definecolor{codepurple}{rgb}{0.58,0,0.82}
\definecolor{backcolour}{rgb}{0.95,0.95,0.92}

\lstdefinestyle{mystyle}{
    backgroundcolor=\color{backcolour},   
    commentstyle=\color{codegreen},
    keywordstyle=\color{magenta},
    numberstyle=\tiny\color{codegray},
    stringstyle=\color{codepurple},
    basicstyle=\ttfamily\footnotesize,
    breakatwhitespace=false,         
    breaklines=true,                 
    captionpos=b,                    
    keepspaces=true,                 
    numbers=left,                    
    numbersep=5pt,                  
    showspaces=false,                
    showstringspaces=false,
    showtabs=false,                  
    tabsize=2
}

\lstset{style=mystyle}

% Header and Footer
\pagestyle{fancy}
\fancyhead{}
\fancyfoot{}
\fancyhead[L]{\textit{\Large{AI and ML Techniques - 3rd Year}}}
\fancyhead[R]{\textit{Krishnaraj T}}
\fancyfoot[C]{\thepage}
\renewcommand{\footrulewidth}{1pt}
\newtheorem{thm}{Theorem}
\newtheorem{dfn}[thm]{Definition}


% Other Doc Editing
% \parindent 0ex
%\renewcommand{\baselinestretch}{1.5}

\begin{document}

\begin{titlepage}
    \centering

    %---------------------------NAMES-------------------------------

    \huge\textsc{
        MIT World Peace University
    }\\

    \vspace{0.75\baselineskip} % space after Uni Name

    \LARGE{
        Artificial Intelligence and Machine Learning Techniques\\
        Third Year B. Tech, Semester 5
    }

    \vfill % space after Sub Name

    %--------------------------TITLE-------------------------------

    \rule{\textwidth}{1.6pt}\vspace*{-\baselineskip}\vspace*{2pt}
    \rule{\textwidth}{0.6pt}
    \vspace{0.75\baselineskip} % Whitespace above the title



    \huge{\textsc{
            State Space Search
        }} \\



    \vspace{0.5\baselineskip} % Whitespace below the title
    \rule{\textwidth}{0.6pt}\vspace*{-\baselineskip}\vspace*{2.8pt}
    \rule{\textwidth}{1.6pt}

    \vspace{1\baselineskip} % Whitespace after the title block

    %--------------------------SUBTITLE --------------------------	

    \LARGE\textsc{
        Lab Assignment 1
    } % Subtitle or further description
    \vfill

    %--------------------------AUTHOR-------------------------------

    Prepared By
    \vspace{0.5\baselineskip} % Whitespace before the editors

    \Large{
        Krishnaraj Thadesar \\
        Cyber Security and Forensics\\
        Batch A1, PA 20
    }


    \vspace{0.5\baselineskip} % Whitespace below the editor list
    \today

\end{titlepage}


\tableofcontents
\thispagestyle{empty}
\clearpage

\setcounter{page}{1}

\section{Aim}
Study state space representation for AI problem solving.

\section{Theory}

\subsection{State Space Search}
\begin{dfn}
    \textbf{State Space} is a set of all possible configurations of the system. Each configuration is called a \textbf{state} of the system. The state space is the set of all possible states reachable from the initial state by any sequence of actions.\\

    \noindent
    It is a way to mathematically represent a problem by defining all the possible states
    in which the problem can be. This is used in search algorithms to represent the initial state, goal
    state, and current state of the problem. Each state in the state space is represented using a set of
    variables.
\end{dfn}

\begin{dfn}
    The \textbf{efficiency} of the search algorithm greatly depends on the size of the state space, and it is
    important to choose an appropriate representation and search strategy to search the state space
    efficiently.
\end{dfn}

One of the most well-known state space search algorithms is the $A^*$ algorithm. Other commonly
used state space search algorithms include breadth-first search (BFS), depth-first search
(DFS), hill climbing, simulated annealing, and genetic algorithms.

\subsection{Features of State Space Search}
State space search has several features that make it an effective problem-solving technique in
Artificial Intelligence. These features include:

\begin{enumerate}
    \item \textbf{Exhaustiveness}:
          State space search explores all possible states of a problem to find a solution.
    \item \textbf{Completeness}:
          If a solution exists, state space search will find it.
    \item \textbf{Optimality}:
          Searching through a state space results in an optimal solution.
    \item \textbf{Uninformed and Informed           Search}:
          State space search in artificial intelligence can be classified as uninformed if it provides
          additional information about the problem.
\end{enumerate}

\subsection{Steps in State Space Search}

The steps involved in state space search are as follows:

\begin{enumerate}
    \item To begin the search process, we set the current state to the initial state.
    \item We then check if the current state is the goal state. If it is, we terminate the algorithm and
          return the result.
    \item If the current state is not the goal state, we generate the set of possible successor states
          that can be reached from the current state.
    \item For each successor state, we check if it has already been visited. If it has, we skip it, else
          we add it to the queue of states to be visited.
          Next, we set the next state in the queue as the current state and check if it's the goal state.
    \item If it is, we return the result. If not, we repeat the previous step until we find the goal state
          or explore all the states.
    \item If all possible states have been explored and the goal state still needs to be found, we
          return with no solution.
\end{enumerate}

\subsection{State Space Representation}

\textbf{State space Representation} involves defining an INITIAL STATE and a GOAL STATE and
then determining a sequence of actions, called states, to follow.

\begin{figure}[H]
    \centering
    \includegraphics[width=.85\textwidth]{./diagram1.png}
    \caption{State Space Representation}
\end{figure}

\subsection{More Standard Definitions}

\begin{dfn}
    \textbf{State}:
    A state can be an Initial State, a Goal State, or any other possible state that can be
    generated by applying rules between them.

\end{dfn}
\begin{dfn}
    \textbf{Space}:
    In an AI problem, space refers to the exhaustive collection of all conceivable states.
\end{dfn}
\begin{dfn}
    \textbf{Search}:
    This technique moves from the beginning state to the desired state by applying good rules
    while traversing the space of all possible states.
\end{dfn}
\begin{dfn}
    \textbf{Search Tree}:
    To visualize the search issue, a search tree is used, which is a tree-like structure that
    represents the problem. The initial state is represented by the root node of the search tree,
    which is the starting point of the tree.
\end{dfn}
\begin{dfn}
    \textbf{Transition Model}:
    This describes what each action does, while Path Cost assigns a cost value to each path,
    an activity sequence that connects the beginning node to the end node. The optimal
    option has the lowest cost among all alternatives.
\end{dfn}

\section{Platform}
\textbf{Operating System}: Arch Linux x86-64 \\
\textbf{IDEs or Text Editors Used}: Visual Studio Code\\
\textbf{Compilers or Interpreters}: Not Required. \\

\section{Rules Written for Towers of Hanoi}
\lstinputlisting[caption=`Rules that were written for an agent for solving the towers of Hanoi Problem and creating the State Space Tree`]{../../Programs/Towers_of_Hanoi.md}

% \section{Input and Output}


% \section{Code}
% \lstinputlisting[language=Python, caption="DSA Signature Validity using PyCrypto Library"]{../Programs/Assignment_7/dsa using lib.py}

\section{Conclusion}
State Space Search is a problem-solving technique used in AI to find a solution
to a problem by exploring all possible states of the problem.

\clearpage

\section{FAQ}
\begin{enumerate}
    \item \textbf{State space representation of 8 Puzzle Problem}\\

          \begin{enumerate}
              \item The \textbf{8-puzzle} problem is a commonly used example of a state space search. It is a sliding
                    puzzle game consisting of 8 numbered tiles arranged in a \textbf{3x3} grid and one blank space.
                    The game aims to rearrange the tiles from their initial state to a final goal state by sliding
                    them into the blank space.
              \item To represent the state space in this problem, we use the nine tiles in the puzzle and their
                    respective positions in the grid. Each state in the state space is represented by a \textbf{3x3} array
                    with values ranging from 1 to 8, and the blank space is represented as an empty tile.
              \item The initial state of the puzzle represents the starting configuration of the tiles, while the
                    goal state represents the desired configuration. \textbf{Search algorithms} utilize the state space
                    to find a sequence of moves that will transform the initial state into the goal state.
          \end{enumerate}
          \begin{figure}[H]
              \centering
              \includegraphics[width=.45\textwidth]{8_queens1.png}
              \caption{8 Puzzle Initial and Goal State}
          \end{figure}
          \begin{figure}[H]
              \centering
              \includegraphics[width=.55\textwidth]{8_queens2.png}
              \caption{8 Puzzle State Space}
          \end{figure}
    \item \textbf{Application of Stata Space Search}\\
          \begin{enumerate}
              \item \textbf{State space search} algorithms are used in various fields, such as robotics, game playing,
                    computer networks, operations research, bioinformatics, cryptography, and supply chain
                    management. In artificial intelligence, state space search algorithms can solve problems
                    like pathfinding, planning, and scheduling.
              \item They are also useful in planning robot motion and finding the best sequence of actions to
                    achieve a goal. In games, state space search algorithms can help determine the best move
                    for a player given a particular game state.
              \item \textbf{State space search algorithms} can optimize routing and resource allocation in computer
                    networks and operations research.
              \item In \textbf{Bioinformatics}, state space search algorithms can help find patterns in biological data
                    and predict protein structures.
              \item In \textbf{Cryptography}, state space search algorithms are used to break codes and find
                    cryptographic keys.
          \end{enumerate}

\end{enumerate}

\end{document}