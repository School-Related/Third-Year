% This is a Basic Assignment Paper but with like Code and stuff allowed in it, there is also url, hyperlinks from contents included. 

\documentclass[11pt]{article}

% Preamble

\usepackage[margin=1in]{geometry}
\usepackage{amsfonts, amsmath, amssymb}
\usepackage{fancyhdr, float, graphicx}
\usepackage[utf8]{inputenc} % Required for inputting international characters
\usepackage[T1]{fontenc} % Output font encoding for international characters
\usepackage{fouriernc} % Use the New Century Schoolbook font
\usepackage[nottoc, notlot, notlof]{tocbibind}
\usepackage{listings}
\usepackage{xcolor}
\usepackage{blindtext}
\usepackage{hyperref}
\hypersetup{
    colorlinks=true,
    linkcolor=black,
    filecolor=magenta,      
    urlcolor=cyan,
    pdfpagemode=FullScreen,
    }

\definecolor{codegreen}{rgb}{0,0.6,0}
\definecolor{codegray}{rgb}{0.5,0.5,0.5}
\definecolor{codepurple}{rgb}{0.58,0,0.82}
\definecolor{backcolour}{rgb}{0.95,0.95,0.92}

\lstdefinestyle{mystyle}{
    backgroundcolor=\color{backcolour},   
    commentstyle=\color{codegreen},
    keywordstyle=\color{magenta},
    numberstyle=\tiny\color{codegray},
    stringstyle=\color{codepurple},
    basicstyle=\ttfamily\footnotesize,
    breakatwhitespace=false,         
    breaklines=true,                 
    captionpos=b,                    
    keepspaces=true,                 
    numbers=left,                    
    numbersep=5pt,                  
    showspaces=false,                
    showstringspaces=false,
    showtabs=false,                  
    tabsize=2
}

\lstset{style=mystyle}

% Header and Footer
\pagestyle{fancy}
\fancyhead{}
\fancyfoot{}
\fancyhead[L]{\textit{\Large{Full Stack Development - 3nd Year B. Tech}}}
%\fancyhead[R]{\textit{something}}
\fancyfoot[C]{\thepage}
\renewcommand{\footrulewidth}{1pt}
\newtheorem{thm}{Theorem}
\newtheorem{dfn}[thm]{Definition}


%define Javascript language
\lstdefinelanguage{JavaScript}{
keywords={typeof, new, true, false, catch, function, return, null, catch, switch, var, if, in, while, do, else, case, break},
keywordstyle=\color{blue}\bfseries,
ndkeywords={class, export, boolean, throw, implements, import, this},
ndkeywordstyle=\color{darkgray}\bfseries,
identifierstyle=\color{black},
sensitive=false,
comment=[l]{//},
morecomment=[s]{/*}{*/},
commentstyle=\color{purple}\ttfamily,
stringstyle=\color{red}\ttfamily,
morestring=[b]',
morestring=[b]"
}

\lstset{
language=JavaScript,
extendedchars=true,
basicstyle=\footnotesize\ttfamily,
showstringspaces=false,
showspaces=false,
numbers=left,
numberstyle=\footnotesize,
numbersep=9pt,
tabsize=2,
breaklines=true,
showtabs=false,
captionpos=b
}



% Other Doc Editing
% \parindent 0ex
%\renewcommand{\baselinestretch}{1.5}

\begin{document}

\begin{titlepage}
    \centering

    %---------------------------NAMES-------------------------------

    \huge\textsc{
        MIT World Peace University
    }\\

    \vspace{0.75\baselineskip} % space after Uni Name

    \LARGE{
        Full Stack Development\\
        Third Year B. Tech, Semester 5
    }

    \vfill % space after Sub Name

    %--------------------------TITLE-------------------------------

    \rule{\textwidth}{1.6pt}\vspace*{-\baselineskip}\vspace*{2pt}
    \rule{\textwidth}{0.6pt}
    \vspace{0.75\baselineskip} % Whitespace above the title



    \huge{\textsc{
            Implementation of Form Validation, JQuery, and Ajax
        }} \\



    \vspace{0.5\baselineskip} % Whitespace below the title
    \rule{\textwidth}{0.6pt}\vspace*{-\baselineskip}\vspace*{2.8pt}
    \rule{\textwidth}{1.6pt}

    \vspace{1\baselineskip} % Whitespace after the title block

    %--------------------------SUBTITLE --------------------------	

    \LARGE\textsc{
        Lab Assignment 3
    } % Subtitle or further description
    \vfill

    %--------------------------AUTHOR-------------------------------

    Prepared By
    \vspace{0.5\baselineskip} % Whitespace before the editors

    \Large{
        Krishnaraj Thadesar \\
        Cyber Security and Forensics\\
        Batch A1, PA 20
    }


    \vspace{0.5\baselineskip} % Whitespace below the editor list
    \today

\end{titlepage}


\tableofcontents
\thispagestyle{empty}
\clearpage

\setcounter{page}{1}

\section{Aim}
Write a program to perform following form validations using JavaScript, where all fields are mandatory, and some fields like Phone number, Email Address, Zip code Validation etc. are included. Also include JavaScript to access and manipulate Document Object Model (DOM) objects in an HTML web page. Include JQuery to develop to develop your application as an Ajax based application.


\section{Objectives}
\begin{itemize}
    \item To understand what is form validation.
    \item To learn basic functioning of DOM objects.
    \item To learn how to apply various techniques to implement it.
\end{itemize}

\section{Problem Statement}
Write a program to design Student registration form by using HTML, CSS having following
fields: Username, Email, Phone number, Password, Confirm Password and write external
javascript code to achieve following validations:
\begin{enumerate}
    \item Fields should not be empty. If spaces are entered those should be considered empty
    \item Phone number must accept only numeric values and it should be 10 digits
    \item Password length must be at least 7 and it should contain at least one capital letter, one digit and one special character from the set (\&,\$,\#, \@)
    \item Value entered in password field and confirm password fields must match
\end{enumerate}

Email address must contain @ sign and a ., there should be few letters before the @ sign, should be three letters between @ sign and a . There must be 3 or 2 letters after the . (hint: Use
regular expression)


\section{Theory}

\section{Types of Form Validation}

\begin{dfn}
    Form validation is the process of checking that a form has been filled in correctly before it is submitted. It is used to prevent users from submitting invalid data, such as an email address without an @ sign or a password that is too short.
\end{dfn}


\subsection{Types}
\begin{enumerate}
    \item \textbf{Client-side validation:} Client-side validation is performed by the web browser before the form is submitted to the server. It is used to provide immediate feedback to the user about any errors in their input.
    \item \textbf{Server-side validation:} Server-side validation is performed by the web server after the form has been submitted. It is used to ensure that the data submitted by the user is valid and safe to process.
    \item \textbf{Database validation:} Database validation is performed by the database management system after the data has been stored in the database. It is used to ensure that the data is consistent with the database schema and any constraints defined on the table.
    \item \textbf{Business logic validation:} Business logic validation is performed by the application after the data has been retrieved from the database. It is used to ensure that the data is consistent with the business rules defined by the application.
\end{enumerate}


\section{The Document Object Model}

\begin{dfn}
    The Document Object Model (DOM) is a programming interface for HTML and XML documents. It defines the logical structure of a document and the way it is accessed and manipulated.
\end{dfn}

\begin{enumerate}
    \item The DOM is used by JavaScript to access and manipulate HTML documents.
    \item It is also used by other programming languages, such as PHP and Python, to generate HTML documents dynamically.
    \item The DOM is an object-oriented model that represents HTML documents as a collection of objects.
    \item Each object has properties and methods that can be used to access and manipulate the document.
    \item The DOM is a platform-independent model that can be used with any programming language.
\end{enumerate}

\subsection{DOM Tree}

\begin{dfn}
    The DOM tree is a hierarchical representation of the HTML document. It consists of nodes that represent elements, attributes, text, and comments.
\end{dfn}

\begin{figure}[H]
    \centering
    \includegraphics[width=.95\textwidth]{0*Sk5AAj4ze_bDFPA0.png}
    \caption{The DOM Tree}
\end{figure}


\subsection{History}

\begin{itemize}
    \item The DOM was first introduced in 1998 by the World Wide Web Consortium (W3C).
    \item It was designed to be a platform-independent model for representing HTML documents.
    \item It is now supported by all major web browsers, including Internet Explorer, Firefox, Chrome, Safari, and Opera.

\end{itemize}


\subsection{Advantages}

\begin{enumerate}
    \item The DOM is a platform-independent model that can be used with any programming language.
    \item It is a standard model that is supported by all major web browsers.
    \item It is an object-oriented model that represents HTML documents as a collection of objects.
    \item It is a hierarchical model that represents the logical structure of an HTML document.
    \item It is a dynamic model that can be used to access and manipulate HTML documents.
    \item It is a flexible model that can be extended to support new features.
    \item It is a powerful model that can be used to create interactive web applications.
    \item It is a simple model that is easy to learn and understand.
    \item It is a scalable model that can be used to create large web applications.
\end{enumerate}

\section{JQuery}

\begin{dfn}
    JQuery is a JavaScript library that simplifies the process of creating interactive web pages. It is used to manipulate HTML documents and handle events.
\end{dfn}

\subsection{History and Significance}

\begin{enumerate}
    \item JQuery was first released in 2006 by John Resig.
    \item It was designed to simplify the process of creating interactive web pages.
    \item It is now supported by all major web browsers, including Internet Explorer, Firefox, Chrome, Safari, and Opera.
    \item It is used by many popular websites, such as Google, Facebook, Twitter, and YouTube.
    \item It is used by many popular web applications, such as WordPress, Drupal, Joomla, and Magento.
\end{enumerate}

\subsection{Advantages}

\begin{enumerate}

    \item \textbf{Simplified DOM Manipulation:} jQuery simplifies DOM (Document Object Model) manipulation tasks. It provides easy-to-use methods to select, modify, and manipulate HTML elements.

          \begin{lstlisting}[language=HTML, caption=Example: Selecting and Modifying Elements]
<!-- HTML -->
<button id="myButton">Click Me</button>

<!-- jQuery -->
<script src="https://code.jquery.com/jquery-3.6.0.min.js"></script>
<script>
    // Selecting an element by ID
    var button = $("#myButton");

    // Modifying its text
    button.text("New Text");
</script>
\end{lstlisting}

    \item \textbf{Event Handling:} jQuery simplifies event handling by providing methods like `.on()` and `.click()` for attaching event handlers to elements.

          \begin{lstlisting}[language=HTML, caption=Example: Event Handling]
<!-- HTML -->
<button id="myButton">Click Me</button>

<!-- jQuery -->
<script src="https://code.jquery.com/jquery-3.6.0.min.js"></script>
<script>
    // Event handling with jQuery
    $("#myButton").click(function() {
        alert("Button Clicked!");
    });
</script>
\end{lstlisting}

    \item \textbf{AJAX Support:} jQuery streamlines AJAX (Asynchronous JavaScript and XML) requests, making it easier to fetch data from a server without reloading the entire page.

          \begin{lstlisting}[language=JavaScript, caption=Example: AJAX Request]
// jQuery AJAX request
$.ajax({
    url: "https://api.example.com/data",
    method: "GET",
    success: function(data) {
        console.log(data);
    },
    error: function(error) {
        console.error(error);
    }
});
\end{lstlisting}

    \item \textbf{Cross-Browser Compatibility:} jQuery abstracts away browser-specific differences, ensuring consistent behavior across various browsers.

    \item \textbf{Rich Set of Plugins:} jQuery offers a vast ecosystem of plugins that extend its functionality, allowing developers to add features like sliders, date pickers, and more without reinventing the wheel.

    \item \textbf{Animation and Effects:} jQuery provides methods for creating animations and adding visual effects to web pages with ease.

          \begin{lstlisting}[language=HTML, caption=Example: Animation]
<!-- HTML -->
<div id="box">Click Me</div>

<!-- jQuery -->
<script src="https://code.jquery.com/jquery-3.6.0.min.js"></script>
<script>
    // jQuery animation
    $("#box").click(function() {
        $(this).animate({ left: "200px" }, "slow");
    });
</script>
\end{lstlisting}

\end{enumerate}

\subsection{JQuery Selectors}

jQuery provides a variety of selectors that allow you to target and manipulate specific elements in the HTML DOM. Here are some commonly used jQuery selectors:

\begin{enumerate}

    \item \textbf{Element Selector:} Selects all elements with a specified tag name.

          \begin{lstlisting}[language=HTML, caption=Example: Element Selector]
<!-- HTML -->
<ul>
    <li>Item 1</li>
    <li>Item 2</li>
    <li>Item 3</li>
</ul>

<!-- jQuery -->
<script src="https://code.jquery.com/jquery-3.6.0.min.js"></script>
<script>
    // Select all <li> elements
    var listItems = $("li");
</script>
\end{lstlisting}

    \item \textbf{ID Selector:} Selects a single element with a specific ID attribute.

          \begin{lstlisting}[language=HTML, caption=Example: ID Selector]
<!-- HTML -->
<div id="myDiv">Hello, World!</div>

<!-- jQuery -->
<script src="https://code.jquery.com/jquery-3.6.0.min.js"></script>
<script>
    // Select the element with ID "myDiv"
    var myElement = $("#myDiv");
</script>
\end{lstlisting}

    \item \textbf{Class Selector:} Selects all elements with a specific class attribute.

          \begin{lstlisting}[language=HTML, caption=Example: Class Selector]
<!-- HTML -->
<p class="highlight">This is a highlighted text.</p>
<p>This is a normal text.</p>

<!-- jQuery -->
<script src="https://code.jquery.com/jquery-3.6.0.min.js"></script>
<script>
    // Select all elements with class "highlight"
    var highlightedElements = $(".highlight");
</script>
\end{lstlisting}

    \item \textbf{Attribute Selector:} Selects elements with a specific attribute and optional value.

          \begin{lstlisting}[language=HTML, caption=Example: Attribute Selector]
<!-- HTML -->
<input type="text" id="username" name="username" value="John" />

<!-- jQuery -->
<script src="https://code.jquery.com/jquery-3.6.0.min.js"></script>
<script>
    // Select element with attribute "name"
    var usernameInput = $("[name='username']");
</script>
\end{lstlisting}

    \item \textbf{Descendant Selector:} Selects all elements that are descendants of a specified element.

          \begin{lstlisting}[language=HTML, caption=Example: Descendant Selector]
<!-- HTML -->
<ul id="myList">
    <li>Item 1</li>
    <li>Item 2</li>
</ul>

<!-- jQuery -->
<script src="https://code.jquery.com/jquery-3.6.0.min.js"></script>
<script>
    // Select all <li> elements inside the element with ID "myList"
    var listItems = $("#myList li");
</script>
\end{lstlisting}

\end{enumerate}


\section{Platform}
\textbf{Operating System}: Arch Linux x86-64 \\
\textbf{IDEs or Text Editors Used}: Visual Studio Code\\
\textbf{Compilers or Interpreters}: Brave Browser (Chromium v117.0.5938.88.) \\

\section{Input and Output}

\begin{enumerate}
    \item The Student Validaiton form was created, and the relevant fields were added.
    \item The validation was done on client side using Javascript Successfully.
    \item The screenshots of the same are given below.
\end{enumerate}

\section{Screenshots}

\begin{figure}[H]
    \centering
    \includegraphics[width=.95\textwidth]{02.png}
    \caption{The Student Validation form with Mandatory Fields. You cant submit without filling them. The Phone Number field only accepts numbers.}
\end{figure}

\begin{figure}[H]
    \centering
    \includegraphics[width=.95\textwidth]{03.png}
    \caption{The Country Code Selection Field. The Countries REST api is called by using Jquery AJAX to get the list of countries. }
\end{figure}

\begin{figure}[H]
    \centering
    \includegraphics[width=.95\textwidth]{04.png}
    \caption{Email Validation done on client side JS}
\end{figure}

\begin{figure}[H]
    \centering
    \includegraphics[width=.95\textwidth]{05.png}
    \caption{Form being submitted after entering everything. Notice the button is blue. }
\end{figure}

\begin{figure}[H]
    \centering
    \includegraphics[width=.95\textwidth]{06.png}
    \caption{The Color of the button changes to green for 3 seconds, as instructed in the problem statement. Only then the form submits. }
\end{figure}

\section{Code}
\lstinputlisting[language=HTML, caption="index.html"]{../../Programs/Assignment 3/index.html}

\section{Conclusion}
Thus, we have successfully implemented form validation using Javascript and used Jquery as well as JQuery Ajax.
\clearpage

\section{FAQ}
\begin{enumerate}
    \item \textbf{Reasons for Form Validations:}
          \begin{itemize}
              \item \textbf{Data Integrity:} Form validations ensure that the data submitted by users is accurate and follows the expected format. This prevents incorrect or malicious data from being processed, maintaining the integrity of your application's database.
              \item \textbf{Enhanced User Experience:} Validations provide immediate feedback to users, helping them correct errors before submission. This enhances the user experience and reduces frustration.
              \item \textbf{Security:} Proper form validations can prevent common security vulnerabilities like SQL injection and cross-site scripting (XSS) attacks by rejecting potentially harmful input.
          \end{itemize}

    \item \textbf{Modifying Attribute Value using DOM:} In JavaScript, you can modify an attribute value using the DOM (Document Object Model) as follows:
          \begin{lstlisting}[language=JavaScript, caption=Example: Modifying Attribute Value using DOM]
// Assuming 'element' is a reference to the HTML element you want to modify.

// To change the 'src' attribute of an image:
element.setAttribute('src', 'new_image.jpg');

// To modify the 'href' attribute of a link:
element.setAttribute('href', 'new_url.html');
        \end{lstlisting}


    \item \textbf{jQuery Ajax:} jQuery Ajax is a feature in the jQuery library that allows you to make asynchronous HTTP requests to a server and handle the responses without requiring a full page reload. It offers a simplified way to perform tasks like fetching data from a server, sending data to a server, or updating a web page without the need for extensive JavaScript code.

          \begin{lstlisting}[language=JavaScript, caption=Example: AJAX Request]
// jQuery AJAX request
$.ajax({
    url: "https://api.example.com/data",
    method: "GET",
    success: function(data) {
        console.log(data);
    },
    error: function(error) {
        console.error(error);
    }
});
        \end{lstlisting}
\end{enumerate}
\end{document}