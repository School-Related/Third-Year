% This is a Basic Assignment Paper but with like Code and stuff allowed in it, there is also url, hyperlinks from contents included. 

\documentclass[11pt]{article}

% Preamble

\usepackage[margin=1in]{geometry}
\usepackage{amsfonts, amsmath, amssymb}
\usepackage{fancyhdr, float, graphicx}
\usepackage[utf8]{inputenc} % Required for inputting international characters
\usepackage[T1]{fontenc} % Output font encoding for international characters
\usepackage{fouriernc} % Use the New Century Schoolbook font
\usepackage[nottoc, notlot, notlof]{tocbibind}
\usepackage{listings}
\usepackage{xcolor}
\usepackage{blindtext}
\usepackage{hyperref}
\hypersetup{
    colorlinks=true,
    linkcolor=black,
    filecolor=magenta,      
    urlcolor=cyan,
    pdfpagemode=FullScreen,
    }

\definecolor{codegreen}{rgb}{0,0.6,0}
\definecolor{codegray}{rgb}{0.5,0.5,0.5}
\definecolor{codepurple}{rgb}{0.58,0,0.82}
\definecolor{backcolour}{rgb}{0.95,0.95,0.92}
\definecolor{mygreen}{rgb}{0,0.6,0}
\definecolor{mygray}{rgb}{0.5,0.5,0.5}
\definecolor{mymauve}{rgb}{0.58,0,0.82}

\lstdefinestyle{mystyle}{
    backgroundcolor=\color{backcolour},   
    commentstyle=\color{codegreen},
    keywordstyle=\color{magenta},
    numberstyle=\tiny\color{codegray},
    stringstyle=\color{codepurple},
    basicstyle=\ttfamily\footnotesize,
    breakatwhitespace=false,         
    breaklines=true,                 
    captionpos=b,                    
    keepspaces=true,                 
    numbers=left,                    
    numbersep=5pt,                  
    showspaces=false,                
    showstringspaces=false,
    showtabs=false,                  
    tabsize=2
}

\lstset{style=mystyle}

% Header and Footer
\pagestyle{fancy}
\fancyhead{}
\fancyfoot{}
\fancyhead[L]{\textit{\Large{Full Stack Development - 3nd Year B. Tech}}}
%\fancyhead[R]{\textit{something}}
\fancyfoot[C]{\thepage}
\renewcommand{\footrulewidth}{1pt}
\newtheorem{thm}{Theorem}
\newtheorem{dfn}[thm]{Definition}

\definecolor{darkgray}{rgb}{.4,.4,.4}
\definecolor{purple}{rgb}{0.65, 0.12, 0.82}
 
%define Javascript language
\lstdefinelanguage{JavaScript}{
keywords={typeof, new, true, false, catch, function, return, null, catch, switch, var, if, in, while, do, else, case, break},
keywordstyle=\color{blue}\bfseries,
ndkeywords={class, export, boolean, throw, implements, import, this},
ndkeywordstyle=\color{darkgray}\bfseries,
identifierstyle=\color{black},
sensitive=false,
comment=[l]{//},
morecomment=[s]{/*}{*/},
commentstyle=\color{purple}\ttfamily,
stringstyle=\color{red}\ttfamily,
morestring=[b]',
morestring=[b]"
}
 
\lstset{
language=JavaScript,
extendedchars=true,
basicstyle=\footnotesize\ttfamily,
showstringspaces=false,
showspaces=false,
numbers=left,
numberstyle=\footnotesize,
numbersep=9pt,
tabsize=2,
breaklines=true,
showtabs=false,
captionpos=b
}

% Other Doc Editing
% \parindent 0ex
%\renewcommand{\baselinestretch}{1.5}

\begin{document}

\begin{titlepage}
    \centering

    %---------------------------NAMES-------------------------------

    \huge\textsc{
        MIT World Peace University
    }\\

    \vspace{0.75\baselineskip} % space after Uni Name

    \LARGE{
        Full Stack Development\\
        Third Year B. Tech, Semester 5
    }

    \vfill % space after Sub Name

    %--------------------------TITLE-------------------------------

    \rule{\textwidth}{1.6pt}\vspace*{-\baselineskip}\vspace*{2pt}
    \rule{\textwidth}{0.6pt}
    \vspace{0.75\baselineskip} % Whitespace above the title



    \huge{\textsc{
            Developing a FSD Application using MERN Stack. \\
            Novel Tea Library
        }} \\



    \vspace{0.5\baselineskip} % Whitespace below the title
    \rule{\textwidth}{0.6pt}\vspace*{-\baselineskip}\vspace*{2.8pt}
    \rule{\textwidth}{1.6pt}

    \vspace{1\baselineskip} % Whitespace after the title block

    %--------------------------SUBTITLE --------------------------	

    \LARGE\textsc{
        Lab Assignment 7
    } % Subtitle or further description
    \vfill

    %--------------------------AUTHOR-------------------------------

    Prepared By
    \vspace{0.5\baselineskip} % Whitespace before the editors

    \Large{
        Krishnaraj Thadesar \\
        Cyber Security and Forensics\\
        Batch A1, PA 20
    }


    \vspace{0.5\baselineskip} % Whitespace below the editor list
    \today

\end{titlepage}


\tableofcontents
\thispagestyle{empty}
\clearpage

\setcounter{page}{1}

\section{Aim}
Develop a full stack web application using MERN stack to perform CRUD operations.

\section{Objectives}
\begin{itemize}
   \item To develop full-stack web projects using the MERN stack.
   \item To learn database connectivity using fetch api.
   \item To perform insert, update, delete and search operations on database.
\end{itemize}

\section{Problem Statement}
Student can create a React form or use existing/ implemented HTML form for Library
Management System with the fields mentioned: Book name, ISBN No, Book title, Author name,
Publisher name and perform following operations

\begin{enumerate}
    \item Insert Book details -Book name, ISBN No, Book title, Author name, Publisher name
    \item Delete the Book records based on ISBN No
    \item Update the Book details based on ISBN No- Example students can update wrong entered book details based on searching the record with ISBN No.
    \item Display the Updated Book details or View the Book Details records in tabular format.
\end{enumerate}

\section{Theory}

\subsection{The MERN Stack}
What is MERN stack?
Use of Fetch api.
\begin{dfn}
    The MERN stack is a JavaScript stack that’s designed to make the development process smoother. MERN includes four open-source components: MongoDB, Express, React, and Node.js. These components provide an end-to-end framework for developers to work in. MERN stack is a combination of four technologies: MongoDB, Express, React, and Node.js. It is designed to make the development process smoother and easier.
\end{dfn}

\subsection{Components of the Mern Stack}
\subsection{Features}

\begin{enumerate}
    \item \textbf{Statelessness:} REST APIs are stateless, meaning each request from a client to a server must contain all the information needed to understand and fulfill the request. There is no session state stored on the server between requests, making it scalable and easy to cache responses.

    \item \textbf{Resource-Based:} REST treats every piece of information or functionality as a resource, which can be uniquely identified using URLs. Resources can represent objects, data, or services.

    \item \textbf{HTTP Methods:} REST uses standard HTTP methods such as GET, POST, PUT, DELETE, etc., to perform actions on resources. These methods provide a uniform interface for interacting with resources.

    \item \textbf{Representation:} Data in REST is represented in a format, such as JSON or XML. Clients can request specific representations of resources, allowing flexibility in data exchange.

    \item \textbf{Stateless Communication:} RESTful communication between the client and server is stateless, which means each request from a client to a server must contain all the information needed. The server does not store information about the client's state between requests.

\end{enumerate}
\begin{figure}[H]
    \centering
    \includegraphics[width=.95\textwidth]{rest api/rest api_2.jpg}
    \caption{REST APIs Explained}
\end{figure}


\subsection{Advantages}

\begin{enumerate}
    \item \textbf{Simplicity:} REST APIs are easy to understand and use due to their simplicity. They use standard HTTP methods and rely on URLs to access resources.

    \item \textbf{Scalability:} Stateless nature and resource-based architecture make REST APIs highly scalable. They can handle a large number of concurrent requests.

    \item \textbf{Flexibility:} REST APIs support multiple data formats (JSON, XML, etc.) and can be used with various programming languages. This flexibility makes them suitable for diverse applications.

    \item \textbf{Interoperability:} RESTful Web services enable interoperability between different systems and platforms, making them suitable for building distributed and heterogeneous applications.

    \item \textbf{Caching:} REST APIs can leverage caching mechanisms, improving response times and reducing server load for frequently accessed resources.

\end{enumerate}

\subsection{History and Significance}

\begin{enumerate}
    \item \textbf{Origin:} The term "REST" was introduced by Roy Fielding in his Ph.D. dissertation in 2000. It draws inspiration from the principles of the World Wide Web and HTTP.

    \item \textbf{Significance:} REST has become the predominant architectural style for designing networked applications and web services due to its simplicity, scalability, and flexibility. It has been widely adopted for building APIs for web and mobile applications.

    \item \textbf{Continued Evolution:} Over the years, RESTful practices and conventions have evolved, and new technologies and tools have emerged to support the development and consumption of RESTful APIs.

\end{enumerate}

\subsection{Steps to Create a REST API using Node and Express}

Creating a RESTful API with Node.js and Express is a common task for building server-side applications. Here are the steps to create a REST API using these technologies:

\subsubsection{Set Up a Node.js Project}

\begin{enumerate}
    \item Ensure you have Node.js installed. If not, download and install it from the official website: \url{https://nodejs.org/}.
    \item Create a new project directory for your REST API.
    \item Open a terminal or command prompt and navigate to your project directory.
    \item Initialize a Node.js project by running the following command:
          \begin{verbatim}
    npm init -y
    \end{verbatim}
          This will create a `package.json` file.
\end{enumerate}

\subsubsection{Install Express}

\begin{enumerate}
    \item Install Express as a dependency for your project using npm:
          \begin{verbatim}
    npm install express
    \end{verbatim}
\end{enumerate}

\subsubsection{Create an Express Application}

\begin{enumerate}
    \item Create a JavaScript file (e.g., `app.js`) for your Express application.
    \item In `app.js`, import Express and create an instance of the Express application:
          \begin{verbatim}
    const express = require('express');
    const app = express();
    \end{verbatim}
\end{enumerate}

\subsubsection{Define Routes and Handlers}

\begin{enumerate}
    \item Define routes and request handlers for your API. For example, to create a simple endpoint that returns a JSON response:
          \begin{verbatim}
    app.get('/api', (req, res) => {
        res.json({ message: 'Welcome to the REST API!' });
    });
    \end{verbatim}
          You can create more complex routes and handlers as needed for your application.
\end{enumerate}

\subsubsection{Start the Server}

\begin{enumerate}
    \item Start the Express server by specifying a port and listening for incoming requests:
          \begin{verbatim}
    const port = process.env.PORT || 3000;
    app.listen(port, () => {
        console.log(`Server is running on port ${port}`);
    });
    \end{verbatim}
\end{enumerate}

\subsubsection{Test Your API}

\begin{enumerate}
    \item Use a tool like Postman or your web browser to test your API by sending HTTP requests to the defined endpoints (e.g., `http://localhost:3000/api`).
    \item Verify that your API is responding as expected.
\end{enumerate}

\subsubsection{Add Database Integration (Optional)}

\begin{enumerate}
    \item If your application requires database storage, integrate a database system like MongoDB or PostgreSQL. You can use libraries like Mongoose (for MongoDB) or Sequelize (for SQL databases) to interact with the database.
    \item Define database models and routes for CRUD operations (Create, Read, Update, Delete) on your data.
\end{enumerate}

\subsubsection{Deploy Your API (Optional)}

\begin{enumerate}
    \item Deploy your REST API to a hosting platform or server. Popular options include Heroku, AWS, Azure, and DigitalOcean.
    \item Set up any necessary environment variables and configure your server for production use.
\end{enumerate}

\section{Platform}
\textbf{Operating System}: Arch Linux x86-64 \\
\textbf{IDEs or Text Editors Used}: Visual Studio Code\\
\textbf{Compilers or Interpreters}: Brave Browser (Chromium v117.0.5938.88.) \\

\section{Input and Output}

\begin{enumerate}
    \item A Webpage was created for searching for images. The user can enter a search term and the number of images to be displayed. The images are fetched from the Unsplash, Pexels and Pixabay APIs, and merged into a single array. The images are then displayed on the webpage.
    \item This was done so as to resolve the problem of having to browser multiple webpages to find high quality images for making presentations. This webpage allows the user to search for images from multiple sources at once.
    \item The webpage was created using React and Tailwindcss. The images are fetched using the Axios library. The screenshot of the webpage is shown below.
\end{enumerate}

\section{Screenshots}

\subsection{React Frontend}

\subsection{React Frontend}
\begin{figure}[H]
    \centering
    \includegraphics[width=.95\textwidth]{Screenshots/home.png}
    \caption{The Home Page of the Novel Tea Library}
\end{figure}

\begin{figure}[H]
    \centering
    \includegraphics[width=.95\textwidth]{Screenshots/books.png}
    \caption{The Books retreived from /getbooks.php and shown on the Frontend}
\end{figure}

\begin{figure}[H]
    \centering
    \includegraphics[width=.95\textwidth]{Screenshots/add.png}
    \caption{The Add Page}
\end{figure}

\begin{figure}[H]
    \centering
    \includegraphics[width=.95\textwidth]{Screenshots/delete.png}
    \caption{The Delete Page}
\end{figure}
\begin{figure}[H]

    \centering
    \includegraphics[width=.95\textwidth]{Screenshots/update.png}
    \caption{The Update Page, where any field can be updated}
\end{figure}

\begin{figure}[H]
    \centering
    \includegraphics[width=.95\textwidth]{Screenshots/search.png}
    \caption{The Search Page showing results of search term "H"}
\end{figure}

\subsection{Requests using Postman}

\begin{figure}[H]
    \centering
    \includegraphics[width=.95\textwidth]{Screenshots/postman 1.png}
    \caption{Requests sent using Postman}
\end{figure}

\begin{figure}[H]
    \centering
    \includegraphics[width=.95\textwidth]{Screenshots/postman 2.png}
    \caption{Requests sent using Postman}
\end{figure}


\subsection{Node and Express Backend}

\begin{figure}[H]
    \centering
    \includegraphics[width=.95\textwidth]{Screenshots/backend 1.png}
    \caption{The Node and Express Backend server running. It is listening on port 3000.}
\end{figure}
\begin{figure}[H]
    \centering
    \includegraphics[width=.95\textwidth]{Screenshots/backend 2.png}
    \caption{The Node and Express Backend server running and serving request. }
\end{figure}

\section{Code}
\lstinputlisting[language=JavaScript, caption=package.json]{../../Programs/Assignment 6/backend/package.json}
\lstinputlisting[language=JavaScript, caption=app.js managing the app. ]{../../Programs/Assignment 6/backend/app.js}
\lstinputlisting[language=JavaScript, caption=api.js defining the routes. ]{../../Programs/Assignment 6/backend/routes/api.js}
\lstinputlisting[language=JavaScript, caption=Book.js interacting with MongoDB]{../../Programs/Assignment 6/backend/models/Book.js}
\lstinputlisting[language=HTML, caption=books.js serving routes. ]{../../Programs/Assignment 6/backend/controllers/books.js}

\section{Conclusion}
Thus, we have successfully created a REST API using Node.js and Express.js. We have also created a frontend using React.js and Tailwindcss. The frontend interacts with the backend using Axios. The backend interacts with the MongoDB database using Mongoose. The frontend and backend are hosted on Heroku. The database is hosted on MongoDB Atlas. The frontend is hosted at \url{https://noveltea.surge.sh/} and the backend is hosted at \url{https://novel-tea-library-backend.herokuapp.com/}.

\clearpage

\section{FAQ}
\begin{enumerate}
    \item \textit{What are HTTP Request types?}\\

          \begin{dfn}
              HTTP (Hypertext Transfer Protocol) is a protocol used for communication between web servers and clients. There are several types of HTTP requests that can be used to interact with a web server:
          \end{dfn}

          \begin{enumerate}
              \item GET: retrieves a resource from the server. This is the most common type of request, and it is used to retrieve web pages, images, and other types of content.
              \item POST: submits data to the server. This is commonly used for submitting form data, such as login credentials or search queries.
              \item PUT: updates a resource on the server. This is used to update an existing resource, such as a file or a database record.
              \item DELETE: deletes a resource from the server. This is used to delete an existing resource, such as a file or a database record.
              \item HEAD: retrieves the headers for a resource. This is used to retrieve metadata about a resource, such as its content type and length, without actually retrieving the resource itself.
              \item OPTIONS: retrieves the supported HTTP methods for a resource. This is used to determine which HTTP methods are supported by a resource, such as GET, POST, PUT, and DELETE.
          \end{enumerate}
          Each HTTP request consists of a request method, a URL, and optional headers and data. The server responds to each request with an HTTP status code, which indicates whether the request was successful or not.


\end{enumerate}

\end{document}