% This is a Basic Assignment Paper but with like Code and stuff allowed in it, there is also url, hyperlinks from contents included. 

\documentclass[11pt]{article}

% Preamble

\usepackage[margin=1in]{geometry}
\usepackage{amsfonts, amsmath, amssymb}
\usepackage{fancyhdr, float, graphicx}
\usepackage[utf8]{inputenc} % Required for inputting international characters
\usepackage[T1]{fontenc} % Output font encoding for international characters
\usepackage{fouriernc} % Use the New Century Schoolbook font
\usepackage[nottoc, notlot, notlof]{tocbibind}
\usepackage{listings}
\usepackage{xcolor}
\usepackage{blindtext}
\usepackage{hyperref}
\hypersetup{
    colorlinks=true,
    linkcolor=black,
    filecolor=magenta,
    urlcolor=blue,
    pdfpagemode=FullScreen,
}

\definecolor{codegreen}{rgb}{0,0.6,0}
\definecolor{codegray}{rgb}{0.5,0.5,0.5}
\definecolor{codepurple}{rgb}{0.58,0,0.82}
\definecolor{backcolour}{rgb}{0.95,0.95,0.92}

\lstdefinestyle{mystyle}{
    backgroundcolor=\color{backcolour},
    commentstyle=\color{codegreen},
    keywordstyle=\color{magenta},
    numberstyle=\tiny\color{codegray},
    stringstyle=\color{codepurple},
    basicstyle=\ttfamily\footnotesize,
    breakatwhitespace=false,
    breaklines=true,
    captionpos=b,
    keepspaces=true,
    numbers=left,
    numbersep=5pt,
    showspaces=false,
    showstringspaces=false,
    showtabs=false,
    tabsize=2
}

\lstset{style=mystyle}

% Header and Footer
\pagestyle{fancy}
\fancyhead{}
\fancyfoot{}
\fancyhead[L]{\textit{\Large{Wireless Devices and Mobile Security - 3nd Year B. Tech}}}
\fancyhead[R]{\textit{Krishnaraj T}}
\fancyfoot[C]{\thepage}
\renewcommand{\footrulewidth}{1pt}

% Other Doc Editing
% \parindent 0ex
%\renewcommand{\baselinestretch}{1.5}

\begin{document}

\begin{titlepage}
    \centering

    %---------------------------NAMES-------------------------------

    \huge\textsc{
        MIT World Peace University
    }\\

    \vspace{0.75\baselineskip} % space after Uni Name

    \LARGE{
        Wireless Devices and Mobile Security\\
        Third Year B. Tech, Semester 5
    }

    \vfill % space after Sub Name

    %--------------------------TITLE-------------------------------

    \rule{\textwidth}{1.6pt}\vspace*{-\baselineskip}\vspace*{2pt}
    \rule{\textwidth}{0.6pt}
    \vspace{0.75\baselineskip} % Whitespace above the title

    \huge{\textsc{
            Program to Send OTP to Mobile Phone using Python and Twilio API
        }} \\

    \vspace{0.5\baselineskip} % Whitespace below the title
    \rule{\textwidth}{0.6pt}\vspace*{-\baselineskip}\vspace*{2.8pt}
    \rule{\textwidth}{1.6pt}

    \vspace{1\baselineskip} % Whitespace after the title block

    %--------------------------SUBTITLE --------------------------	

    \LARGE\textsc{
        Lab Assignment 6
    } % Subtitle or further description
    \vfill

    %--------------------------AUTHOR-------------------------------

    Prepared By \vspace{0.5\baselineskip} % Whitespace before the editors

    \Large{
        Krishnaraj Thadesar \\
        Cyber Security and Forensics\\
        Batch A1, PA 10
    }

    \vspace{0.5\baselineskip} % Whitespace below the editor list
    \today

\end{titlepage}

\tableofcontents
\thispagestyle{empty}
\clearpage

\setcounter{page}{1}

\section{Aim}
To write a program to send OTP to mobile phone using Python and Twilio API.
\section{Objectives}
\begin{enumerate}
    \item To learn how to use Twilio API to send SMS.
    \item To learn how to use Python to send SMS.
    \item To learn how to use Python to generate OTP.
\end{enumerate}
\section{Theory}

\subsection{Twilio API}
\begin{enumerate}
    \item \textbf{Overview:} The Twilio API is a cloud communications platform that allows developers to integrate messaging, voice, and video capabilities into their applications. It provides a set of RESTful APIs for building communication solutions.

    \item \textbf{Key Features:}
          \begin{itemize}
              \item Sending and receiving SMS and MMS messages.
              \item Making and receiving voice calls.
              \item Video conferencing capabilities.
              \item Integration with various programming languages.
          \end{itemize}

    \item \textbf{Use Cases:}
          \begin{itemize}
              \item Implementing two-factor authentication (2FA).
              \item Building notification systems.
              \item Creating interactive voice response (IVR) systems.
          \end{itemize}

\end{enumerate}


\begin{figure}[H]
    \centering
    \includegraphics[width=.95\textwidth]{twilio/twilio_6.jpg}
    \caption{Twilio Features}
\end{figure}

\subsection{Pricing of Twilio API}
\begin{enumerate}
    \item \textbf{Billing Model:} Twilio charges based on usage, with costs associated with each message, call, or other communication type.

    \item \textbf{Factors Affecting Pricing:}
          \begin{itemize}
              \item Message type (SMS, MMS).
              \item Destination country for calls and messages.
              \item Type of phone number used (local, toll-free).
              \item Volume of usage.
          \end{itemize}

    \item \textbf{Pricing Details:} Twilio provides a detailed pricing page on their official website, allowing users to estimate costs based on their specific use case.

\end{enumerate}

\subsection{OTP Generation with Python}
\begin{enumerate}
    \item \textbf{Python Libraries:} Use libraries like `pyotp` or `onetimepass` to generate OTPs (One-Time Passwords) in Python.

    \item \textbf{Time-based OTP (TOTP):} TOTP is a widely used algorithm for generating OTPs based on the current time.

    \item \textbf{Implementation:} Sample Python code involves importing the library, creating an OTP object, and generating OTPs based on the provided key.

    \item \textbf{Security Considerations:} Ensure the secure storage of secret keys and follow best practices for OTP implementation.

\end{enumerate}

\begin{figure}[H]
    \centering
    \includegraphics[width=.75\textwidth]{otp.png}
    \caption{OTP Example SMS}
\end{figure}

\subsection{Working of OTPs for Enhanced Security}
\begin{enumerate}
    \item \textbf{Two-Factor Authentication (2FA):} OTPs are commonly used as a second factor to enhance security along with passwords.

    \item \textbf{Dynamic Authentication Codes:} OTPs change dynamically at regular intervals, providing a time-sensitive layer of security.

    \item \textbf{Use in Identity Verification:} OTPs are employed in identity verification processes, ensuring that the entity accessing the system has possession of the valid OTP.

    \item \textbf{Avoiding Replay Attacks:} OTPs are designed to be used only once, mitigating the risk of replay attacks.
\end{enumerate}


\section{Platform}
\textbf{Operating System}: Arch Linux x86-64 \\
\textbf{IDEs or Text Editors Used}: Visual Studio Code\\
\textbf{Compilers or Interpreters}: Python 3.10.1\\

\section{Input and Output}

\begin{figure}[H]
    \centering
    \includegraphics[width=.95\textwidth]{./output.jpeg}
    \caption{Terminal Input and Output}
\end{figure}

\begin{figure}[H]
    \centering
    \includegraphics[width=.45\textwidth]{./message.jpeg}
    \caption{Message Received on Phone (+919834312135)}
\end{figure}

\begin{figure}[H]
    \centering
    \includegraphics[width=.45\textwidth]{./otps.jpeg}
    \caption{Other Messages Received on Phone (+919834312135)}
\end{figure}

\section{Code}
\lstinputlisting[language=Python, caption=Script to Send SMS via Twilio API]{../../Programs/Assignment 6/send_sms.py}

\section{Conclusion}
Thus, we have successfully used Twilio API to send OTP to a mobile phone using
Python, and verified it on the script. \clearpage

\begin{thebibliography}{99}
    \bibitem{twiliodocs}
    Twilio Documentation. \\
    \url{https://www.twilio.com/docs}

    \bibitem{twiliopricing}
    Twilio Pricing. \\
    \url{https://www.twilio.com/pricing}

    \bibitem{pyotp}
    PyOTP Documentation. \\
    \url{https://github.com/pyauth/pyotp}

    \bibitem{onetimepass}
    onetimepass Documentation. \\
    \url{https://github.com/tadeck/onetimepass}

    \bibitem{nistguidelines}
    NIST Digital Identity Guidelines. \\
    \url{https://www.nist.gov/publications/digital-identity-guidelines}
\end{thebibliography}

\end{document}