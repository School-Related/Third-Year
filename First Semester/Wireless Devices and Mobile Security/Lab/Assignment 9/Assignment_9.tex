% This is a Basic Assignment Paper but with like Code and stuff allowed in it, there is also url, hyperlinks from contents included. 

\documentclass[11pt]{article}

% Preamble

\usepackage[margin=1in]{geometry}
\usepackage{amsfonts, amsmath, amssymb}
\usepackage{fancyhdr, float, graphicx}
\usepackage[utf8]{inputenc} % Required for inputting international characters
\usepackage[T1]{fontenc} % Output font encoding for international characters
\usepackage{fouriernc} % Use the New Century Schoolbook font
\usepackage[nottoc, notlot, notlof]{tocbibind}
\usepackage{listings}
\usepackage{xcolor}
\usepackage{blindtext}
\usepackage{hyperref}
\hypersetup{
	colorlinks=true,
	linkcolor=black,
	filecolor=magenta,
	urlcolor=blue,
	pdfpagemode=FullScreen,
}

\definecolor{codegreen}{rgb}{0,0.6,0}
\definecolor{codegray}{rgb}{0.5,0.5,0.5}
\definecolor{codepurple}{rgb}{0.58,0,0.82}
\definecolor{backcolour}{rgb}{0.95,0.95,0.92}

\lstdefinestyle{mystyle}{
	backgroundcolor=\color{backcolour},
	commentstyle=\color{codegreen},
	keywordstyle=\color{magenta},
	numberstyle=\tiny\color{codegray},
	stringstyle=\color{codepurple},
	basicstyle=\ttfamily\footnotesize,
	breakatwhitespace=false,
	breaklines=true,
	captionpos=b,
	keepspaces=true,
	numbers=left,
	numbersep=5pt,
	showspaces=false,
	showstringspaces=false,
	showtabs=false,
	tabsize=2
}

\lstset{style=mystyle}

% Header and Footer
\pagestyle{fancy}
\fancyhead{}
\fancyfoot{}
\fancyhead[L]{\textit{\Large{Wireless Devices and Mobile Security - 3nd Year B. Tech}}}
\fancyhead[R]{\textit{Krishnaraj T}}
\fancyfoot[C]{\thepage}
\renewcommand{\footrulewidth}{1pt}

% Other Doc Editing
% \parindent 0ex
%\renewcommand{\baselinestretch}{1.5}

\begin{document}

\begin{titlepage}
	\centering

	%---------------------------NAMES-------------------------------

	\huge\textsc{
		MIT World Peace University
	}\\

	\vspace{0.75\baselineskip} % space after Uni Name

	\LARGE{
		Wireless Devices and Mobile Security\\
		Third Year B. Tech, Semester 5
	}

	\vfill % space after Sub Name

	%--------------------------TITLE-------------------------------

	\rule{\textwidth}{1.6pt}\vspace*{-\baselineskip}\vspace*{2pt}
	\rule{\textwidth}{0.6pt}
	\vspace{0.75\baselineskip} % Whitespace above the title

	\huge{\textsc{
			Installation and Configuration of Any Wifi Traffic Analyser Tool.
		}} \\

	\vspace{0.5\baselineskip} % Whitespace below the title
	\rule{\textwidth}{0.6pt}\vspace*{-\baselineskip}\vspace*{2.8pt}
	\rule{\textwidth}{1.6pt}

	\vspace{1\baselineskip} % Whitespace after the title block

	%--------------------------SUBTITLE --------------------------	

	\LARGE\textsc{
		Lab Assignment 9
	} % Subtitle or further description
	\vfill

	%--------------------------AUTHOR-------------------------------

	Prepared By \vspace{0.5\baselineskip} % Whitespace before the editors

	\Large{
		Krishnaraj Thadesar \\
		Cyber Security and Forensics\\
		Batch A1, PA 10
	}

	\vspace{0.5\baselineskip} % Whitespace below the editor list
	\today

\end{titlepage}

\tableofcontents
\thispagestyle{empty}
\clearpage

\setcounter{page}{1}

\section{Aim}
Install, configure and demonstrate any one Wi-Fi traffic analyzer using
sniffing tools such as Wireshark, AirCrack, AirSnort, etc.

\section{Objectives}
\begin{enumerate}
	\item To install Wireshark on the system.
	\item To capture packets using Wireshark.
	\item To analyse the captured packets.
\end{enumerate}

\section{Theory}

\subsection{Wireshark}

\begin{figure}[H]
	\centering
	\includegraphics[width=.95\textwidth]{wireshark/wireshark_4.jpg}
	\caption{Wireshark GUI}
\end{figure}

\subsubsection{Installation}
\begin{itemize}
	\item \textbf{Procedure:} Wireshark can be installed on various operating systems, including Windows, macOS, and Linux. Visit the official Wireshark website (\url{https://www.wireshark.org/}) and follow the installation instructions for your specific platform.
	\item \textbf{Dependencies:} Wireshark may require the installation of WinPcap (Windows), libpcap (Linux), or npcap (Windows) for packet capture.
\end{itemize}

\subsubsection{Working}
\begin{itemize}
	\item Wireshark captures and analyzes packets on a network in real-time.
	\item Users can apply various filters to focus on specific types of traffic.
	\item The captured data can be displayed in different formats, facilitating detailed
	      protocol analysis.
\end{itemize}

\subsubsection{Pros}
\begin{itemize}
	\item User-friendly interface with powerful features.
	\item Extensive protocol support for in-depth analysis.
	\item Active community and regular updates.
\end{itemize}

\subsubsection{Cons}
\begin{itemize}
	\item May consume significant system resources during packet capture.
	\item Beginners might find the wealth of features overwhelming.
	\item Limited to the capabilities of the network interface card (NIC).
\end{itemize}

\subsection{AirCrack}
\begin{figure}[H]
	\centering
	\includegraphics[width=.95\textwidth]{aircrackng/aircrackng_3.jpg}
	\caption{Aircrack}
\end{figure}

\subsubsection{Installation}
\begin{itemize}
	\item \textbf{Procedure:} AirCrack-ng, a suite of wireless network security tools, can be installed on various platforms. Detailed installation instructions are available on the official website (\url{https://www.aircrack-ng.org/}).
	\item \textbf{Dependencies:} AirCrack-ng relies on libpcap and other libraries for packet capture and analysis.
\end{itemize}

\subsubsection{Working}
\begin{itemize}
	\item AirCrack-ng is primarily used for assessing the security of Wi-Fi networks.
	\item It includes tools for capturing, analyzing, and cracking WEP and WPA/WPA2-PSK
	      keys.
	\item Supports various attacks like packet injection and de-authentication to test
	      network vulnerabilities.
\end{itemize}

\subsubsection{Pros}
\begin{itemize}
	\item Comprehensive suite for wireless network security.
	\item Active development community and frequent updates.
	\item Capable of testing the security of WEP and WPA/WPA2-PSK.
\end{itemize}

\subsubsection{Cons}
\begin{itemize}
	\item Requires a good understanding of wireless networks and security concepts.
	\item Use in unauthorized networks may violate ethical and legal standards.
	\item Effectiveness is dependent on the strength of encryption used.
\end{itemize}

\subsection{AirSnort}

\begin{figure}[H]
	\centering
	\includegraphics[width=.95\textwidth]{airsnort/airsnort_6.jpg}
	\caption{Airsnort}
\end{figure}

\subsubsection{Installation}
\begin{itemize}
	\item \textbf{Procedure:} AirSnort, a wireless LAN (WLAN) tool, is no longer actively maintained. Installation may vary based on the available repositories or archived versions.
	\item \textbf{Dependencies:} Originally designed for Linux, it relies on libpcap and other libraries for packet capture.
\end{itemize}

\subsubsection{Working}
\begin{itemize}
	\item AirSnort was designed to crack WEP encryption keys by capturing data packets
	      and analyzing them.
	\item It focused on exploiting weaknesses in the WEP algorithm to recover network
	      passwords.
	\item Due to its outdated nature, it may not be effective against modern, more secure
	      encryption standards.
\end{itemize}

\subsubsection{Pros}
\begin{itemize}
	\item Historically used for educational purposes to highlight WEP vulnerabilities.
	\item Provided insights into the weaknesses of early wireless encryption
\end{itemize}

\subsubsection{Cons}
\begin{itemize}
	\item Outdated and no longer actively maintained.
	\item Limited effectiveness against modern and more secure Wi-Fi encryption.
	\item Not recommended for practical use in contemporary security assessments.
\end{itemize}

\section{Platform}
\textbf{Operating System}: Arch Linux x86-64 \\
\textbf{IDEs or Text Editors Used}: Visual Studio Code\\
\textbf{Compilers or Interpreters}: Python 3.10.1\\

\section{Working Screenshots}

\begin{figure}[H]
	\centering
	\includegraphics[width=.95\textwidth]{01.png}
	\caption{The command line window is showing that the wlane wireless interface has been put into monitor mode, and that two processes that could interfere with this mode have been killed.}
\end{figure}

\begin{figure}[H]
	\centering
	\includegraphics[width=.95\textwidth]{02.png}
	\caption{Wifi Password Key Found}
\end{figure}

\begin{figure}[H]
	\centering
	\includegraphics[width=.95\textwidth]{03.png}
	\caption{WPA handshake captured!}
\end{figure}

\begin{figure}[H]
	\centering
	\includegraphics[width=.95\textwidth]{04.png}
	\caption{Wi-Fi traffic capturing using Wireshark.}
\end{figure}

\begin{figure}[H]
	\centering
	\includegraphics[width=.95\textwidth]{05.png}
	\caption{802.11 frame capture in progress.}
\end{figure}

\begin{figure}[H]
	\centering
	\includegraphics[width=.95\textwidth]{06.png}
	\caption{Wi-Fi network scan results.}
\end{figure}

\begin{figure}[H]
	\centering
	\includegraphics[width=.95\textwidth]{07.png}
	\caption{Preparing to capture Wi-Fi traffic.}
\end{figure}

\begin{figure}[H]
	\centering
	\includegraphics[width=.95\textwidth]{08.png}
	\caption{A command-line window executing the aireplay-ng-death tool to deauthenticate clients from a Wi-Fi network. }
\end{figure}

\begin{figure}[H]
	\centering
	\includegraphics[width=.95\textwidth]{09.png}
	\caption{Wi-Fi network scan results on a Kali Linux system.	}
\end{figure}

\section{Conclusion}
Thus, the installation and configuration of Any Wifi Traffic Analyser Tool was
successfully done. We installed Wireshark, captured packets and analysed them.

\section{FAQ}

\begin{enumerate}
	\item \textit{List the different open source tool to capture packet. Also, write its
		      features.}\\

	      \textbf{Packet Capture Tools:}
	      \begin{itemize}
		      \item \textit{Wireshark:}
		            \begin{itemize}
			            \item \textbf{Features:} Wireshark is a widely-used open-source packet analyzer. It allows real-time packet capture and display.
			            \item \textbf{Additional Features:} Protocol analysis, deep inspection of hundreds of protocols, live capture, and offline analysis.
			            \item \textbf{Reference:} \cite{wireshark}
		            \end{itemize}

		      \item \textit{Tshark:}
		            \begin{itemize}
			            \item \textbf{Features:} Tshark is the command-line version of Wireshark. It offers similar features for packet capture and analysis.
			            \item \textbf{Additional Features:} Scriptable using Lua, supports various capture file formats.
			            \item \textbf{Reference:} \cite{tshark}
		            \end{itemize}

		      \item \textit{Tcpdump:}
		            \begin{itemize}
			            \item \textbf{Features:} Tcpdump is a command-line packet analyzer for Unix-like systems.
			            \item \textbf{Additional Features:} Filters for specific protocols, customizable output formats.
			            \item \textbf{Reference:} \cite{tcpdump}
		            \end{itemize}
	      \end{itemize}

	\item \textit{Which mode NIC uses for Ethereal/packet sniffing?}\\

	      \textbf{NIC Modes for Ethereal/Packet Sniffing:} NIC primarily uses the \textit{Promiscuous Mode} for Ethereal/packet sniffing. In this mode, the NIC captures all traffic on the network, regardless of the destination address.
	\item \textit{Which wireshark filter can be used to monitor outgoing packets from a specific
		      system on the network?}\\

	      \textbf{Wireshark Filter for Monitoring Outgoing Packets:} To monitor outgoing packets from a specific system on the network using Wireshark, you can use the following filter:
	      \begin{verbatim}
            ip.src == <source_IP_address>
        \end{verbatim}
	      Replace \texttt{<source\_IP\_address>} with the actual IP address of the system
	      you want to monitor.
\end{enumerate}

\clearpage
\begin{thebibliography}{99}
	\bibitem{wireshark}
	Wireshark. \\
	Website: \url{https://www.wireshark.org/}

	\bibitem{tshark}
	Tshark. \\
	Website: \url{https://www.wireshark.org/docs/man-pages/tshark.html}

	\bibitem{tcpdump}
	Tcpdump. \\
	Website: \url{https://www.tcpdump.org/}

	\bibitem{aircrack}
	AirCrack-ng. \\
	Website: \url{https://www.aircrack-ng.org/}

	\bibitem{airsnort}
	AirSnort. \\
	Website: \url{https://sourceforge.net/projects/airsnort/}
\end{thebibliography}

\end{document}