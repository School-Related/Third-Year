% This is a Basic Assignment Paper but with like Code and stuff allowed in it, there is also url, hyperlinks from contents included. 

\documentclass[11pt]{article}

% Preamble

\usepackage[margin=1in]{geometry}
\usepackage{amsfonts, amsmath, amssymb}
\usepackage{fancyhdr, float, graphicx}
\usepackage[utf8]{inputenc} % Required for inputting international characters
\usepackage[T1]{fontenc} % Output font encoding for international characters
\usepackage{fouriernc} % Use the New Century Schoolbook font
\usepackage[nottoc, notlot, notlof]{tocbibind}
\usepackage{listings}
\usepackage{xcolor}
\usepackage{blindtext}
\usepackage{hyperref}
\hypersetup{
    colorlinks=true,
    linkcolor=black,
    filecolor=magenta,
    urlcolor=blue,
    pdfpagemode=FullScreen,
}

\definecolor{codegreen}{rgb}{0,0.6,0}
\definecolor{codegray}{rgb}{0.5,0.5,0.5}
\definecolor{codepurple}{rgb}{0.58,0,0.82}
\definecolor{backcolour}{rgb}{0.95,0.95,0.92}

\lstdefinestyle{mystyle}{
    backgroundcolor=\color{backcolour},
    commentstyle=\color{codegreen},
    keywordstyle=\color{magenta},
    numberstyle=\tiny\color{codegray},
    stringstyle=\color{codepurple},
    basicstyle=\ttfamily\footnotesize,
    breakatwhitespace=false,
    breaklines=true,
    captionpos=b,
    keepspaces=true,
    numbers=left,
    numbersep=5pt,
    showspaces=false,
    showstringspaces=false,
    showtabs=false,
    tabsize=2
}

\lstset{style=mystyle}

% Header and Footer
\pagestyle{fancy}
\fancyhead{}
\fancyfoot{}
\fancyhead[L]{\textit{\Large{Wireless Devices and Mobile Security - 3nd Year B. Tech}}}
\fancyhead[R]{\textit{Krishnaraj T}}
\fancyfoot[C]{\thepage}
\renewcommand{\footrulewidth}{1pt}

% Other Doc Editing
% \parindent 0ex
%\renewcommand{\baselinestretch}{1.5}

\begin{document}

\begin{titlepage}
    \centering

    %---------------------------NAMES-------------------------------

    \huge\textsc{
        MIT World Peace University
    }\\

    \vspace{0.75\baselineskip} % space after Uni Name

    \LARGE{
        Wireless Devices and Mobile Security\\
        Third Year B. Tech, Semester 5
    }

    \vfill % space after Sub Name

    %--------------------------TITLE-------------------------------

    \rule{\textwidth}{1.6pt}\vspace*{-\baselineskip}\vspace*{2pt}
    \rule{\textwidth}{0.6pt}
    \vspace{0.75\baselineskip} % Whitespace above the title

    \huge{\textsc{
            Encryption and Decryption of Files and Text in an Android App
        }} \\

    \vspace{0.5\baselineskip} % Whitespace below the title
    \rule{\textwidth}{0.6pt}\vspace*{-\baselineskip}\vspace*{2.8pt}
    \rule{\textwidth}{1.6pt}

    \vspace{1\baselineskip} % Whitespace after the title block

    %--------------------------SUBTITLE --------------------------	

    \LARGE\textsc{
        Lab Assignment 5
    } % Subtitle or further description
    \vfill

    %--------------------------AUTHOR-------------------------------

    Prepared By \vspace{0.5\baselineskip} % Whitespace before the editors

    \Large{
        Krishnaraj Thadesar \\
        Cyber Security and Forensics\\
        Batch A1, PA 10
    }

    \vspace{0.5\baselineskip} % Whitespace below the editor list
    \today

\end{titlepage}

\tableofcontents
\thispagestyle{empty}
\clearpage

\setcounter{page}{1}

\section{Aim}
Write an android program to encrypt and decrypt text file. Use bouncy castle
library API or java cryptography API.

\section{Objectives}

\begin{enumerate}
    \item To understand the working of encryption and decryption of files and text in an
          android app.
    \item To understand the working of bouncy castle library API or java cryptography
          API.
    \item To understand the working of android app development.
    \item To understand the working of android studio.
    \item To understand the working of android emulator.
\end{enumerate}

\section{Theory}

\subsection{Android Studio}

\begin{figure}[H]
    \centering
    \includegraphics[width=.45\textwidth]{android studio/android studio_0.jpg}
    \caption{Android Studio Logo}
\end{figure}
\begin{enumerate}
    \item \textbf{Overview:} Android Studio is the official integrated development environment (IDE) for Android app development. It is based on IntelliJ IDEA and provides a comprehensive set of tools for designing, building, testing, and debugging Android applications.

    \item \textbf{Features:}
          \begin{itemize}
              \item Android Studio includes a visual layout editor for designing user interfaces.
              \item It supports multiple languages, including Java and Kotlin.
              \item The built-in emulator allows developers to test their apps on various Android
                    devices.
              \item Integration with version control systems like Git simplifies collaborative
                    development.
          \end{itemize}

    \item \textbf{Advantages:}
          \begin{itemize}
              \item Rich set of templates for common Android app components.
              \item Seamless integration with Google services and libraries.
              \item Robust debugging tools for identifying and fixing issues.
          \end{itemize}

          \begin{figure}[H]
              \centering
              \includegraphics[width=.95\textwidth]{android studio/android studio_2.jpg}
              \caption{Android Studio Interface}
          \end{figure}
\end{enumerate}

\subsection{Cryptography Libraries}

\subsubsection{Bouncy Castle Library API}
\begin{enumerate}
    \item \textbf{Introduction:} Bouncy Castle is a cryptography library that provides APIs for various cryptographic operations. It is written in Java and supports a wide range of algorithms and protocols.

    \item \textbf{Key Features:}
          \begin{itemize}
              \item Bouncy Castle supports both symmetric and asymmetric encryption algorithms.
              \item It includes implementations for various cryptographic standards like PKCS,
                    OpenPGP, and S/MIME.
              \item The library provides a flexible and extensible architecture for cryptographic
                    operations.
          \end{itemize}

    \item \textbf{Use Cases:}
          \begin{itemize}
              \item Commonly used in Java applications for secure communication.
              \item Integration with other security protocols and frameworks.
          \end{itemize}

\end{enumerate}

\begin{figure}[H]
    \centering
    \includegraphics[width=.95\textwidth]{decryption/decryption_0.jpg}
    \caption{Encryption and Decryption}
\end{figure}

\subsubsection{Java Cryptography API}
\begin{enumerate}
    \item \textbf{Overview:} The Java Cryptography Architecture (JCA) is a framework for handling cryptographic operations in Java applications. It includes the Java Cryptography Extension (JCE), which provides implementations for cryptographic algorithms.

    \item \textbf{Key Components:}
          \begin{itemize}
              \item \textbf{Message Digests and Digital Signatures:} JCA supports various algorithms for creating message digests and digital signatures.
              \item \textbf{Key Management:} Provides classes for key generation, key storage, and key exchange.
              \item \textbf{Secure Random Number Generation:} Ensures the generation of secure random numbers.
          \end{itemize}

    \item \textbf{Integration with Bouncy Castle:} Java Cryptography API can be integrated with the Bouncy Castle library for extended cryptographic functionalities.

\end{enumerate}

\section{Android App Permissions}

\subsection{Permissions Required}
\begin{enumerate}
    \item \textbf{Android Media Store API:}
          \begin{itemize}
              \item The \texttt{READ\_EXTERNAL\_STORAGE} permission is required to read from
                    external storage, including media files.
              \item For writing media files, the \texttt{WRITE\_EXTERNAL\_STORAGE} permission is
                    necessary.
              \item To capture photos or videos using the device's camera, the \texttt{CAMERA}
                    permission is required.
          \end{itemize}

    \item \textbf{Usage in AndroidManifest.xml:}
          \begin{verbatim}
      <uses-permission android:name="android.permission.READ_EXTERNAL_STORAGE" />
      <uses-permission android:name="android.permission.WRITE_EXTERNAL_STORAGE" />
      <uses-permission android:name="android.permission.CAMERA" />
    \end{verbatim}

    \item \textbf{Best Practices:}
          \begin{itemize}
              \item Request these permissions at runtime on devices running Android 6.0 (API level
                    23) and higher.
              \item Handle permission responses gracefully to ensure a smooth user experience.
          \end{itemize}
\end{enumerate}

\subsubsection{Android Manifest.xml}
\lstinputlisting[language=xml, caption=Manifest of Filesealer]{../../Programs/Assignment 5/app/src/main/AndroidManifest.xml}

\subsection{App - Filesealer}

\begin{figure}[H]
    \centering
    \includegraphics[width=.95\textwidth]{screenshots/presentation.png}
    \caption{Filesealer App Presentation}
\end{figure}

\subsubsection*{Link of the app}
\url{https://play.google.com/store/apps/details?id=com.krishnaraj.filesealer}

\section{Platform}
\textbf{Operating System}: Arch Linux x86 64 \\
\textbf{IDEs or Text Editors Used}: Visual Studio Code\\
\textbf{Compilers or Interpreters}: Python 3.10.1\\

\section{Screenshots}

\begin{figure}[H]
    \centering
    \includegraphics[width=.45\textwidth]{screenshots/WhatsApp Image 2023-11-20 at 20.50.02.jpeg}
    \caption{Text Encryption and Decryption}
\end{figure}

\section{Code}
\lstinputlisting[language=Python, caption=MainActivity.java]{../../Programs/Assignment 5/app/src/main/java/com/krishnaraj/filesealer/MainActivity.java}

\subsection{Encryption using Bouncy Castle Library API}
\begin{lstlisting}[language=java]
private String encryptBouncyCastle(String strToEncrypt, String secretKey, Context context) {
    // make sure nothing is empty
    if (strToEncrypt.isEmpty()) {
        showToast(context, "Please enter a string to encrypt.");
        return strToEncrypt;
    }

    String encryptionKey = secretKey;

    if (encryptionKey.isEmpty()) {
        showToast(context, "Please enter a key.");
        return encryptionKey;
    }

    if (encryptionKey.length() < 32) {
        int keyLength = encryptionKey.length();
        int repeatKey = 32 / keyLength;
        encryptionKey = new String(new char[repeatKey]).replace("\0", encryptionKey);
        int newKeyLength = encryptionKey.length();
        int addKey = 32 - newKeyLength;
        encryptionKey += encryptionKey.substring(0, addKey);
    }

    Log.d("EncryptFragment", "Encryption Key: " + encryptionKey);
    Log.d("EncryptFragment", "String to Encrypt: " + strToEncrypt);

    Security.addProvider(new BouncyCastleProvider());
    byte[] keyBytes;

    try {
        keyBytes = encryptionKey.getBytes(StandardCharsets.UTF_8);
        SecretKeySpec skey = new SecretKeySpec(keyBytes, "AES");
        byte[] input = strToEncrypt.getBytes(StandardCharsets.UTF_8);

        synchronized (Cipher.class) {
            @SuppressLint("GetInstance") Cipher cipher = Cipher.getInstance("AES/ECB/PKCS7Padding", "BC");
            cipher.init(Cipher.ENCRYPT_MODE, skey);

            byte[] cipherText = new byte[cipher.getOutputSize(input.length)];
            int ctLength = cipher.update(input, 0, input.length, cipherText, 0);
            ctLength += cipher.doFinal(cipherText, ctLength);
            Log.d("EncryptFragment", "ctLength: " + ctLength);
            // log the encrypted string
            return Base64.encodeToString(cipherText, Base64.DEFAULT);
        }
    } catch (NoSuchAlgorithmException | NoSuchPaddingException | NoSuchProviderException |
                InvalidKeyException | BadPaddingException | IllegalBlockSizeException e) {
        e.printStackTrace();
        Log.d("EncryptFragment", "Exception: " + e.getMessage());
        showToast(context, "Error: Unable to Encode this Text");
    } catch (ShortBufferException e) {
        throw new RuntimeException(e);
    }
    return encryptionKey;
}
\end{lstlisting}

\subsection{Decryption using Bouncy Castle Library API}

\begin{lstlisting}[language=java]
private String decryptWithAES(String key, String strToDecrypt, Context context) {
    Security.addProvider(new BouncyCastleProvider());
    byte[] keyBytes;

    String encryptionKey = key;

    if (encryptionKey.isEmpty()) {
        showToast(context, "Please enter a key.");
        return encryptionKey;
    }

    if (encryptionKey.length() < 32) {
        int keyLength = encryptionKey.length();
        int repeatKey = 32 / keyLength;
        encryptionKey = new String(new char[repeatKey]).replace("\0", encryptionKey);
        int newKeyLength = encryptionKey.length();
        int addKey = 32 - newKeyLength;
        encryptionKey += encryptionKey.substring(0, addKey);
    }

    Log.d("DecryptFragment", "Encryption Key: " + encryptionKey);

    try {
        keyBytes = encryptionKey.getBytes(StandardCharsets.UTF_8);
        SecretKeySpec skey = new SecretKeySpec(keyBytes, "AES");
        byte[] input = android.util.Base64.decode(strToDecrypt.trim(), android.util.Base64.DEFAULT);

        synchronized (Cipher.class) {
            @SuppressLint("GetInstance") Cipher cipher = Cipher.getInstance("AES/ECB/PKCS7Padding", "BC");
            cipher.init(Cipher.DECRYPT_MODE, skey);

            byte[] plainText = new byte[cipher.getOutputSize(input.length)];
            int ptLength = cipher.update(input, 0, input.length, plainText, 0);
            Log.d("DecryptFragment", "ptLength: " + ptLength);
            Log.d("DecryptFragment", "plainText: " + Arrays.toString(plainText));
            ptLength += cipher.doFinal(plainText, ptLength);
            Log.d("DecryptFragment", "ptLength: " + ptLength);
            // make the plaintext based on the pt length
            String decryptedString = new String(plainText, 0, ptLength);
            // log the decrypted string
            Log.d("DecryptFragment", "Decrypted String: " + decryptedString);
            return decryptedString;
        }
    } catch (NoSuchAlgorithmException | NoSuchPaddingException | NoSuchProviderException |
                InvalidKeyException | BadPaddingException | IllegalBlockSizeException e) {
        e.printStackTrace();
        Log.d("DecryptFragment", "Exception: " + e.getMessage());
        showToast(context, "Error: Unable to Decode this Text");
    } catch (ShortBufferException e) {
        throw new RuntimeException(e);
    }

    return "";
}
\end{lstlisting}

\section{Conclusion}
Thus, we have studied and implemented encryption using bouncy castle API.
\clearpage

\section{FAQ}

\begin{enumerate}
    \item \textit{1. What is Bouncy Castle used for?}
          \begin{itemize}
              \item \textbf{Purpose:} Bouncy Castle is primarily used as a cryptography library in Java applications.
              \item \textbf{Functionality:} It provides APIs for various cryptographic operations, including both symmetric and asymmetric encryption algorithms.
              \item \textbf{Use Cases:} Bouncy Castle is commonly employed for ensuring the security of data during communication and storage in Java applications.
          \end{itemize}

    \item \textit{2. What do you mean by message digest? List different algorithms.}
          \begin{itemize}
              \item \textbf{Message Digest:} A message digest is a fixed-size hash value computed from the input data, commonly used for ensuring data integrity.
              \item \textbf{Algorithms:}
                    \begin{enumerate}
                        \item \textbf{MD5 (Message Digest Algorithm 5):} Produces a 128-bit hash value.
                        \item \textbf{SHA-1 (Secure Hash Algorithm 1):} Generates a 160-bit hash value. Note: It's now considered insecure for cryptographic purposes.
                        \item \textbf{SHA-256, SHA-384, and SHA-512:} Part of the SHA-2 family, producing hash values of 256, 384, and 512 bits, respectively.
                    \end{enumerate}
          \end{itemize}

          \begin{figure}[H]
              \centering
              \includegraphics[width=.85\textwidth]{message digest/message digest_1.jpg}
              \caption{Example of a Message Digest}
          \end{figure}

\end{enumerate}

\clearpage

\begin{thebibliography}{99} % 99 is just a placeholder for the maximum number of entries; adjust as needed

    \bibitem{androidstudio}
    Official Android Studio documentation. \\
    \url{https://developer.android.com/studio}

    \bibitem{bouncycastle}
    Official Bouncy Castle documentation. \\
    \url{https://www.bouncycastle.org/documentation.html}

    \bibitem{androidpermissions}
    Android Developer Guide on Permissions Overview. \\
    \url{https://developer.android.com/guide/topics/permissions/overview}

    \bibitem{permissionsruntime}
    Android Developer Guide on Requesting Permissions at Run Time. \\
    \url{https://developer.android.com/training/permissions/requesting}

    \bibitem{javacrypto}
    Java Cryptography Architecture (JCA) Reference Guide. \\
    \url{https://docs.oracle.com/javase/8/docs/technotes/guides/security/crypto/CryptoSpec.html}

\end{thebibliography}

\end{document}
