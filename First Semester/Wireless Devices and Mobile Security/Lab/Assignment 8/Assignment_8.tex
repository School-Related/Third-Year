% This is a Basic Assignment Paper but with like Code and stuff allowed in it, there is also url, hyperlinks from contents included. 

\documentclass[11pt]{article}

% Preamble

\usepackage[margin=1in]{geometry}
\usepackage{amsfonts, amsmath, amssymb}
\usepackage{fancyhdr, float, graphicx}
\usepackage[utf8]{inputenc} % Required for inputting international characters
\usepackage[T1]{fontenc} % Output font encoding for international characters
\usepackage{fouriernc} % Use the New Century Schoolbook font
\usepackage[nottoc, notlot, notlof]{tocbibind}
\usepackage{listings}
\usepackage{xcolor}
\usepackage{blindtext}
\usepackage{hyperref}
\hypersetup{
  colorlinks=true,
  linkcolor=black,
  filecolor=magenta,
  urlcolor=cyan,
  pdfpagemode=FullScreen,
}

\definecolor{codegreen}{rgb}{0,0.6,0}
\definecolor{codegray}{rgb}{0.5,0.5,0.5}
\definecolor{codepurple}{rgb}{0.58,0,0.82}
\definecolor{backcolour}{rgb}{0.95,0.95,0.92}

\lstdefinestyle{mystyle}{
  backgroundcolor=\color{backcolour},
  commentstyle=\color{codegreen},
  keywordstyle=\color{magenta},
  numberstyle=\tiny\color{codegray},
  stringstyle=\color{codepurple},
  basicstyle=\ttfamily\footnotesize,
  breakatwhitespace=false,
  breaklines=true,
  captionpos=b,
  keepspaces=true,
  numbers=left,
  numbersep=5pt,
  showspaces=false,
  showstringspaces=false,
  showtabs=false,
  tabsize=2
}

\lstset{style=mystyle}

% Header and Footer
\pagestyle{fancy}
\fancyhead{}
\fancyfoot{}
\fancyhead[L]{\textit{\Large{Wireless Devices and Mobile Security - 3nd Year B. Tech}}}
\fancyhead[R]{\textit{Krishnaraj T}}
\fancyfoot[C]{\thepage}
\renewcommand{\footrulewidth}{1pt}

% Other Doc Editing
% \parindent 0ex
%\renewcommand{\baselinestretch}{1.5}

\begin{document}

\begin{titlepage}
  \centering

  %---------------------------NAMES-------------------------------

  \huge\textsc{
    MIT World Peace University
  }\\

  \vspace{0.75\baselineskip} % space after Uni Name

  \LARGE{
    Wireless Devices and Mobile Security\\
    Third Year B. Tech, Semester 5
  }

  \vfill % space after Sub Name

  %--------------------------TITLE-------------------------------

  \rule{\textwidth}{1.6pt}\vspace*{-\baselineskip}\vspace*{2pt}
  \rule{\textwidth}{0.6pt}
  \vspace{0.75\baselineskip} % Whitespace above the title

  \huge{\textsc{
      Study and Comparison of Different types of APN routers, like CISCO, TP Link, D-Link, etc.
    }} \\

  \vspace{0.5\baselineskip} % Whitespace below the title
  \rule{\textwidth}{0.6pt}\vspace*{-\baselineskip}\vspace*{2.8pt}
  \rule{\textwidth}{1.6pt}

  \vspace{1\baselineskip} % Whitespace after the title block

  %--------------------------SUBTITLE --------------------------	

  \LARGE\textsc{
    Lab Assignment 8
  } % Subtitle or further description
  \vfill

  %--------------------------AUTHOR-------------------------------

  Prepared By \vspace{0.5\baselineskip} % Whitespace before the editors

  \Large{
    Krishnaraj Thadesar \\
    Cyber Security and Forensics\\
    Batch A1, PA 10
  }
  \vspace{0.5\baselineskip} % Whitespace below the editor list
  \today

\end{titlepage}

\tableofcontents
\thispagestyle{empty}
\clearpage

\setcounter{page}{1}

\section{Aim}
To learn about different types of APN routers, like CISCO, TP Link, D-Link, and
compare their features.

\section{Objectives}
\begin{enumerate}
  \item To learn about different types of APN routers, like CISCO, TP Link, D-Link,
        etc.
  \item To compare their features.
  \item To learn about the security features of these routers.
\end{enumerate}

\section{Theory}

\subsection{What is a Router?}
\begin{enumerate}
  \item \textbf{Definition:} A router is a network device that connects multiple computer networks together and directs data traffic between them. It operates at the network layer of the OSI model and is a key component in home and enterprise networking.

  \item \textbf{Functionality:} Routers forward data packets based on their destination IP addresses, making decisions about the optimal path for data transmission.

  \item \textbf{Key Features:} Many routers also include built-in features like firewalls, DHCP servers, and wireless access points.

  \item \textbf{Use Cases:} Routers are essential for enabling communication between devices on different networks, providing internet access, and ensuring data security.
\end{enumerate}

\begin{figure}[H]
  \centering
  \includegraphics[width=.45\textwidth]{routers/routers_0.jpg}
  \caption{A Router}
\end{figure}

\begin{figure}[H]
  \centering
  \includegraphics[width=.45\textwidth]{routers/routers_7.jpg}
  \caption{A Router}
\end{figure}

\subsection{What is an APN Router?}
\begin{enumerate}
  \item \textbf{Definition:} An APN router, or Access Point Name router, is a specialized router designed to manage the connection between a mobile network and the internet. It plays a crucial role in enabling data communication for mobile devices.

  \item \textbf{Mobile Network Integration:} APN routers facilitate the integration of mobile devices with the internet by providing a gateway for data communication.

  \item \textbf{Configuration:} Users can configure APN settings on these routers to specify the network to which they want to connect.

  \item \textbf{Use Cases:} APN routers are commonly used in scenarios where reliable mobile data connectivity is required, such as in remote locations or for mobile hotspots.
\end{enumerate}

\subsection{What is the Difference between a Router and an APN Router?}
\begin{enumerate}
  \item \textbf{Scope:} A standard router manages data communication between different computer networks, while an APN router specifically handles the connection between mobile networks and the internet.

  \item \textbf{Functionality:} Routers focus on directing data packets between networks based on IP addresses, while APN routers specialize in managing mobile data connections for devices like smartphones and IoT devices.

  \item \textbf{Configuration:} APN routers involve configuring settings related to mobile network access, including APN settings, which are not typically found in standard routers.

  \item \textbf{Use Cases:} Routers are used in general networking scenarios, while APN routers are tailored for mobile data communication, making them suitable for mobile hotspots, remote locations, and IoT applications.
\end{enumerate}

\subsection{Different Types of Routers Available in the Market}
\subsubsection{CISCO}
\begin{enumerate}
  \item \textbf{Enterprise Focus:} CISCO routers are widely used in enterprise-level networks, providing advanced features and scalability.

  \item \textbf{Security Features:} CISCO routers often include robust security features, making them suitable for securing large networks.

  \item \textbf{Variety of Models:} The product range includes routers for small businesses, branch offices, and large data centers.
\end{enumerate}

\begin{figure}[H]
  \centering
  \includegraphics[width=.45\textwidth]{cisco router/cisco router_2.jpg}
  \caption{Cisco Routers}
\end{figure}

\begin{figure}[H]
  \centering
  \includegraphics[width=.95\textwidth]{cisco router admin page/cisco router admin page_0.jpg}
  \caption{Cisco Router Admin page}
\end{figure}

\subsubsection{TP Link}
\begin{enumerate}
  \item \textbf{Consumer and Small Business Focus:} TP-Link routers are popular among consumers and small businesses, offering a balance between features and affordability.

  \item \textbf{Wireless Technologies:} Many TP-Link routers support advanced wireless technologies, including Wi-Fi 6.

  \item \textbf{User-Friendly Interface:} TP-Link routers often feature a user-friendly interface, making them accessible for home users.
\end{enumerate}

\begin{figure}[H]
  \centering
  \includegraphics[width=.45\textwidth]{tp link router/tp link router_7.jpg}
  \caption{TP Link Routers}
\end{figure}

\begin{figure}[H]
  \centering
  \includegraphics[width=.95\textwidth]{tp link admin page/tp link admin page_6.jpg}
  \caption{TP Link Router Admin page}
\end{figure}

\subsubsection{D-Link}
\begin{enumerate}
  \item \textbf{Home and Small Office Use:} D-Link routers cater to home users and small office environments.

  \item \textbf{Affordability:} D-Link routers are known for their affordability and straightforward setups.

  \item \textbf{Wireless Connectivity:} Many D-Link models support wireless connectivity, making them suitable for homes with multiple devices.
\end{enumerate}

\begin{figure}[H]
  \centering
  \includegraphics[width=.45\textwidth]{d link router/d link router_1.jpg}
  \caption{D Link Routers}
\end{figure}

\begin{figure}[H]
  \centering
  \includegraphics[width=.95\textwidth]{dlink admin page/dlink admin page_3.jpg}
  \caption{D Link Admin Page}
\end{figure}

\subsubsection{Netgear}
\begin{enumerate}
  \item \textbf{Home Networking Focus:} Netgear routers are designed for home networking, emphasizing ease of use and reliability.

  \item \textbf{Mesh Wi-Fi Systems:} Netgear offers mesh Wi-Fi systems for extended coverage in larger homes.

  \item \textbf{Parental Controls:} Some Netgear routers include robust parental control features for managing internet access.
\end{enumerate}

\begin{figure}[H]
  \centering
  \includegraphics[width=.45\textwidth]{netgear routers/netgear routers_8.jpg}
  \caption{Netgear Routers}
\end{figure}

\begin{figure}[H]
  \centering
  \includegraphics[width=.95\textwidth]{netgear routers admin page/netgear routers admin page_0.jpg}
  \caption{Netgear Router Admin Page}
\end{figure}
\subsubsection{Asus}
\begin{enumerate}
  \item \textbf{Gaming and High-Performance:} ASUS routers often target gamers and users seeking high-performance networking solutions.

  \item \textbf{Advanced Features:} ASUS routers may include features like gaming acceleration, VPN support, and AiMesh for mesh networking.

  \item \textbf{Quality of Service (QoS):} ASUS routers may prioritize gaming traffic using QoS features for an enhanced gaming experience.
\end{enumerate}

\begin{figure}[H]
  \centering
  \includegraphics[width=.45\textwidth]{asus routers/asus routers_3.jpg}
  \caption{Asus Routers}
\end{figure}

\begin{figure}[H]
  \centering
  \includegraphics[width=.95\textwidth]{asus router admin page/asus router admin page_9.jpg}
  \caption{Asus Router Admin Page}
\end{figure}

\subsubsection{Linksys}
\begin{enumerate}
  \item \textbf{Home and Small Business:} Linksys routers cater to both home users and small businesses.

  \item \textbf{Mesh Networking:} Linksys offers mesh networking solutions for whole-home coverage.

  \item \textbf{Open Source Firmware:} Some Linksys models support open-source firmware, allowing advanced users to customize their router's functionality.
\end{enumerate}

\begin{figure}[H]
  \centering
  \includegraphics[width=.45\textwidth]{linksys routers/linksys routers_9.jpg}
  \caption{Linksys Routers}
\end{figure}

\begin{figure}[H]
  \centering
  \includegraphics[width=.95\textwidth]{linksys routers admin page/linksys routers admin page_7.jpg}
  \caption{Linksys Router Admin Page}
\end{figure}

\subsubsection{MikroTik}
\begin{enumerate}
  \item \textbf{Affordable and Versatile:} MikroTik routers are known for their affordability and versatility.

  \item \textbf{RouterOS:} MikroTik routers run on RouterOS, a Linux-based operating system, providing extensive configuration options.

  \item \textbf{Enterprise Solutions:} MikroTik offers routers suitable for both small networks and larger enterprise solutions.
\end{enumerate}

\begin{figure}[H]
  \centering
  \includegraphics[width=.45\textwidth]{microtik routers/microtik routers_4.jpg}
  \caption{Microtik Routers}
\end{figure}

\begin{figure}[H]
  \centering
  \includegraphics[width=.95\textwidth]{mikrotik router admin page/mikrotik router admin page_5.jpg}
  \caption{Mikrotik Router Admin Page}
\end{figure}

\subsubsection{Ubiquiti}
\begin{enumerate}
  \item \textbf{Enterprise Networking:} Ubiquiti routers focus on providing solutions for enterprise-level networking.

  \item \textbf{Unified Networking:} Ubiquiti offers routers that integrate with their broader ecosystem of networking products, providing a unified network management experience.

  \item \textbf{Scalability:} Ubiquiti routers are scalable and suitable for building large, complex networks.
\end{enumerate}

\begin{figure}[H]
  \centering
  \includegraphics[width=.45\textwidth]{ubiquiti routers/ubiquiti routers_0.jpg}
  \caption{Ubiquiti Routers}
\end{figure}

\begin{figure}[H]
  \centering
  \includegraphics[width=.95\textwidth]{ubiquiti router admin page/ubiquiti router admin page_2.jpg}
  \caption{Ubiquti Router Admin Page}
\end{figure}

\subsubsection{Huawei}
\begin{enumerate}
  \item \textbf{Diverse Product Range:} Huawei offers a diverse range of routers, including models for consumers, small businesses, and large enterprises.

  \item \textbf{5G Routers:} Huawei is known for its 5G routers, providing high-speed internet connectivity.

  \item \textbf{Advanced Technologies:} Huawei routers often incorporate advanced technologies, making them suitable for modern networking requirements.
\end{enumerate}

\begin{figure}[H]
  \centering
  \includegraphics[width=.45\textwidth]{huawei routers/huawei routers_8.jpg}
  \caption{Huawei Routers}
\end{figure}

\begin{figure}[H]
  \centering
  \includegraphics[width=.95\textwidth]{huawei router admin page/huawei router admin page_2.jpg}
  \caption{Huaewi Router Admin Page}
\end{figure}

\section{Router Comparison}

\subsection{Cisco Routers}
\textbf{Pros:}
\begin{itemize}
  \item Advanced features suitable for enterprise networks.
  \item Robust security features.
  \item Variety of models catering to different network sizes.
\end{itemize}
\textbf{Cons:}
\begin{itemize}
  \item Higher cost, especially for enterprise-level models.
  \item Steeper learning curve for configuration.
\end{itemize}

\subsection{TP-Link Routers}
\textbf{Pros:}
\begin{itemize}
  \item Affordable options for home and small business users.
  \item Support for advanced wireless technologies.
  \item User-friendly interface.
\end{itemize}
\textbf{Cons:}
\begin{itemize}
  \item May lack some advanced features compared to enterprise-level routers.
  \item Limited scalability for large networks.
\end{itemize}

\subsection{D-Link Routers}
\textbf{Pros:}
\begin{itemize}
  \item Cost-effective for home and small office use.
  \item Straightforward setups.
  \item Wireless connectivity for multiple devices.
\end{itemize}
\textbf{Cons:}
\begin{itemize}
  \item Limited feature set compared to more advanced routers.
  \item May not be suitable for larger networks.
\end{itemize}

\subsection{Netgear Routers}
\textbf{Pros:}
\begin{itemize}
  \item Designed for home networking with a focus on ease of use.
  \item Mesh Wi-Fi systems for extended coverage.
  \item Some models offer robust parental control features.
\end{itemize}
\textbf{Cons:}
\begin{itemize}
  \item Features may be less advanced compared to routers targeting enterprise or
        gaming markets.
  \item Limited options for advanced configurations.
\end{itemize}

\subsection{ASUS Routers}
\textbf{Pros:}
\begin{itemize}
  \item Targeted towards gaming and high-performance users.
  \item Advanced features like gaming acceleration and AiMesh support.
  \item Quality of Service (QoS) prioritization for an enhanced gaming experience.
\end{itemize}
\textbf{Cons:}
\begin{itemize}
  \item Higher cost compared to basic home routers.
  \item May have a steeper learning curve for users unfamiliar with advanced networking
        concepts.
\end{itemize}

\subsection{Linksys Routers}
\textbf{Pros:}
\begin{itemize}
  \item Suitable for both home users and small businesses.
  \item Mesh networking solutions for whole-home coverage.
  \item Some models support open-source firmware for customization.
\end{itemize}
\textbf{Cons:}
\begin{itemize}
  \item Features may not be as advanced as routers designed for specific purposes like
        gaming or enterprise networking.
  \item Limited scalability for larger networks.
\end{itemize}

\subsection{MikroTik Routers}
\textbf{Pros:}
\begin{itemize}
  \item Affordable and versatile routers.
  \item RouterOS provides extensive configuration options.
  \item Suitable for both small networks and larger enterprise solutions.
\end{itemize}
\textbf{Cons:}
\begin{itemize}
  \item May have a steeper learning curve due to advanced configuration options.
  \item Limited brand recognition compared to more mainstream options.
\end{itemize}

\subsection{Ubiquiti Routers}
\textbf{Pros:}
\begin{itemize}
  \item Focused on enterprise networking solutions.
  \item Integration with a broader ecosystem of networking products.
  \item Scalable for building large, complex networks.
\end{itemize}
\textbf{Cons:}
\begin{itemize}
  \item Higher cost compared to routers designed for home use.
  \item May be overkill for smaller networks or home users.
\end{itemize}

\subsection{Huawei Routers}
\textbf{Pros:}
\begin{itemize}
  \item Diverse product range catering to consumers, small businesses, and large
        enterprises.
  \item Known for 5G routers providing high-speed internet connectivity.
  \item Incorporates advanced technologies suitable for modern networking requirements.
\end{itemize}
\textbf{Cons:}
\begin{itemize}
  \item Limited availability in certain regions compared to more globally recognized
        brands.
  \item Some models may be priced higher than comparable options from other brands.
\end{itemize}

\section{Platform}
\textbf{Operating System}: Arch Linux x86-64 \\
\textbf{IDEs or Text Editors Used}: Visual Studio Code\\
\textbf{Compilers or Interpreters}: Python 3.10.1\\

\section{Conclusion}
Thus, we have successfully studied and compared different types of APN routers,
like CISCO, TP Link, D-Link, etc.

\clearpage

\begin{thebibliography}{99}
  \bibitem{ciscorouters}
  Cisco Routers. \\
  Website: \url{https://www.cisco.com/c/en/us/products/routers/index.html}

  \bibitem{tplinkrouters}
  TP-Link Routers. \\
  Website: \url{https://www.tp-link.com/us/home-networking/wifi-router/}

  \bibitem{dlinkrouters}
  D-Link Routers. \\
  Website: \url{https://www.dlink.com/en/consumer/routers}

  \bibitem{netgearrouters}
  Netgear Routers. \\
  Website: \url{https://www.netgear.com/home/products/networking/wifi-routers/}

  \bibitem{asusrouters}
  ASUS Routers. \\
  Website: \url{https://www.asus.com/Networking-IoT-Servers/WiFi-Routers-Products/}

  \bibitem{linksysrouters}
  Linksys Routers. \\
  Website: \url{https://www.linksys.com/us/routers/}

  \bibitem{mikrotikrouters}
  MikroTik Routers. \\
  Website: \url{https://mikrotik.com/products/group/routers}

  \bibitem{ubiquitirouters}
  Ubiquiti Routers. \\
  Website: \url{https://www.ui.com/products/#routing}

  \bibitem{huaweirouters}
  Huawei Routers. \\
  Website: \url{https://consumer.huawei.com/en/routers/}
\end{thebibliography}

\end{document}