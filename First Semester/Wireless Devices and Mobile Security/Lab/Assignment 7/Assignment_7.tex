% This is a Basic Assignment Paper but with like Code and stuff allowed in it, there is also url, hyperlinks from contents included. 

\documentclass[11pt]{article}

% Preamble

\usepackage[margin=1in]{geometry}
\usepackage{amsfonts, amsmath, amssymb}
\usepackage{fancyhdr, float, graphicx}
\usepackage[utf8]{inputenc} % Required for inputting international characters
\usepackage[T1]{fontenc} % Output font encoding for international characters
\usepackage{fouriernc} % Use the New Century Schoolbook font
\usepackage[nottoc, notlot, notlof]{tocbibind}
\usepackage{listings}
\usepackage{xcolor}
\usepackage{blindtext}
\usepackage{hyperref}
\hypersetup{
    colorlinks=true,
    linkcolor=black,
    filecolor=magenta,
    urlcolor=blue,
    pdfpagemode=FullScreen,
}

\definecolor{codegreen}{rgb}{0,0.6,0}
\definecolor{codegray}{rgb}{0.5,0.5,0.5}
\definecolor{codepurple}{rgb}{0.58,0,0.82}
\definecolor{backcolour}{rgb}{0.95,0.95,0.92}

\lstdefinestyle{mystyle}{
    backgroundcolor=\color{backcolour},
    commentstyle=\color{codegreen},
    keywordstyle=\color{magenta},
    numberstyle=\tiny\color{codegray},
    stringstyle=\color{codepurple},
    basicstyle=\ttfamily\footnotesize,
    breakatwhitespace=false,
    breaklines=true,
    captionpos=b,
    keepspaces=true,
    numbers=left,
    numbersep=5pt,
    showspaces=false,
    showstringspaces=false,
    showtabs=false,
    tabsize=2
}

\lstset{style=mystyle}

% Header and Footer
\pagestyle{fancy}
\fancyhead{}
\fancyfoot{}
\fancyhead[L]{\textit{\Large{Wireless Devices and Mobile Security - 3nd Year B. Tech}}}
\fancyhead[R]{\textit{Krishnaraj T}}
\fancyfoot[C]{\thepage}
\renewcommand{\footrulewidth}{1pt}

% Other Doc Editing
% \parindent 0ex
%\renewcommand{\baselinestretch}{1.5}

\begin{document}

\begin{titlepage}
    \centering

    %---------------------------NAMES-------------------------------

    \huge\textsc{
        MIT World Peace University
    }\\

    \vspace{0.75\baselineskip} % space after Uni Name

    \LARGE{
        Wireless Devices and Mobile Security\\
        Third Year B. Tech, Semester 5
    }

    \vfill % space after Sub Name

    %--------------------------TITLE-------------------------------

    \rule{\textwidth}{1.6pt}\vspace*{-\baselineskip}\vspace*{2pt}
    \rule{\textwidth}{0.6pt}
    \vspace{0.75\baselineskip} % Whitespace above the title

    \huge{\textsc{
            Configuration of APN of a router, and manage its access control for Security.
        }} \\

    \vspace{0.5\baselineskip} % Whitespace below the title
    \rule{\textwidth}{0.6pt}\vspace*{-\baselineskip}\vspace*{2.8pt}
    \rule{\textwidth}{1.6pt}

    \vspace{1\baselineskip} % Whitespace after the title block

    %--------------------------SUBTITLE --------------------------	

    \LARGE\textsc{
        Lab Assignment 7
    } % Subtitle or further description
    \vfill

    %--------------------------AUTHOR-------------------------------

    Prepared By \vspace{0.5\baselineskip} % Whitespace before the editors

    \Large{
        Krishnaraj Thadesar \\
        Cyber Security and Forensics\\
        Batch A1, PA 10
    }

    \vspace{0.5\baselineskip} % Whitespace below the editor list
    \today

\end{titlepage}

\tableofcontents
\thispagestyle{empty}
\clearpage

\setcounter{page}{1}

\section{Aim}
To Learn about the Configuration of APN of a router, and manage its access
control for Security.
\section{Objectives}
\begin{enumerate}
    \item To learn how to configure APN of a router.
    \item To learn how to manage access control of a router.
    \item To learn how to secure a router.
    \item To learn how to connect a device to a router.
\end{enumerate}

\section{Theory}
\subsection{Access Point Name (APN)}
\begin{enumerate}
  \item \textbf{Definition:} APN, or Access Point Name, is a gateway between a mobile network and another computer network. It is used to connect mobile devices to the internet and other resources.

  \item \textbf{Configuration:} Users can configure APN settings on their devices, specifying the network to which they want to connect.

  \item \textbf{Use in Mobile Networks:} APNs play a crucial role in enabling data communication for mobile devices, facilitating internet access and multimedia messaging.

  \item \textbf{Security Considerations:} Configuring APN settings securely is important to prevent unauthorized access and potential security vulnerabilities.
\end{enumerate}

\subsection{Access Control}
\begin{enumerate}
  \item \textbf{Definition:} Access control refers to the practice of restricting access to a system or resource only to authorized entities and preventing unauthorized access.

  \item \textbf{Key Components:} Access control systems typically include authentication, authorization, and auditing mechanisms.

  \item \textbf{Implementation:} Access control can be implemented through methods like role-based access control (RBAC), mandatory access control (MAC), or discretionary access control (DAC).

  \item \textbf{Importance in Security:} Proper access control is crucial for protecting sensitive information, ensuring privacy, and preventing unauthorized activities.

\end{enumerate}

\subsection{Router Security}
\begin{enumerate}
  \item \textbf{Router Configuration:} Securing a router involves configuring settings such as passwords, firewalls, and firmware updates.

  \item \textbf{Firewall Settings:} Routers often include built-in firewalls that can be configured to filter incoming and outgoing traffic.

  \item \textbf{Firmware Updates:} Regularly updating router firmware is essential to patch vulnerabilities and improve overall security.

  \item \textbf{Guest Network Considerations:} Many routers offer guest network features, allowing users to separate guest and private networks for enhanced security.

\end{enumerate}

\subsection{Connecting a Device to a Router}
\begin{enumerate}
  \item \textbf{SSID and Password:} To connect a device to a router, users typically need to enter the router's SSID (Service Set Identifier) and the associated password.

  \item \textbf{Wi-Fi Protected Setup (WPS):} Some routers support WPS, a simplified method for connecting devices without entering a password.

  \item \textbf{Security Considerations:} Using strong passwords and avoiding public Wi-Fi networks are important for securing device connections.

  \item \textbf{Troubleshooting:} In case of connection issues, troubleshooting steps may involve checking router settings, restarting devices, or updating network drivers.

\end{enumerate}

\subsection{Different Wi-Fi Security Protocols}
\begin{enumerate}
  \item \textbf{WEP (Wired Equivalent Privacy):} An older and less secure protocol, susceptible to various attacks.

  \item \textbf{WPA (Wi-Fi Protected Access):} Introduced as a more secure replacement for WEP, with variations like WPA2 and WPA3.

  \item \textbf{WPA3:} The latest Wi-Fi security protocol, providing stronger encryption and improved security features.

  \item \textbf{Choosing Security Protocols:} Users should select the most secure protocol supported by their devices and routers.

\end{enumerate}



\section{Implementation}

\subsection{Configuration Instructions}

\begin{figure}[H]
    \centering
    \includegraphics[width=.85\textwidth]{lab router/1.png}
    % \caption{}
\end{figure}

\begin{figure}[H]
    \centering
    \includegraphics[width=.85\textwidth]{lab router/2.png}
    % \caption{}
\end{figure}

\begin{figure}[H]
    \centering
    \includegraphics[width=.85\textwidth]{lab router/3.png}
    % \caption{}
\end{figure}

\subsection{Set up Instructions}

\begin{figure}[H]
    \centering
    \includegraphics[width=.85\textwidth]{lab router/connection.png}
\end{figure}

\subsection{The Router}

\begin{figure}[H]
    \centering
    \includegraphics[width=.85\textwidth]{lab router/router 3.jpeg}
    \caption{Ports of the router}
\end{figure}

\begin{figure}[H]
    \centering
    \includegraphics[width=.85\textwidth]{lab router/router 4.jpeg}
    \caption{The Router}
\end{figure}

\begin{figure}[H]
    \centering
    \includegraphics[width=.85\textwidth]{lab router/tp link 1.png}
    \caption{The Admin page of the TP Link Router}
\end{figure}

\begin{figure}[H]
    \centering
    \includegraphics[width=.85\textwidth]{lab router/tp link 2.png}
    \caption{The Admin page of the TP Link Router}
\end{figure}

\begin{figure}[H]
    \centering
    \includegraphics[width=.85\textwidth]{lab router/tp link 3.png}
    \caption{The Admin page of the TP Link Router}
\end{figure}

\section{Platform}
\textbf{Operating System}: Arch Linux x86-64 \\
\textbf{IDEs or Text Editors Used}: Visual Studio Code\\
\textbf{Compilers or Interpreters}: Python 3.10.1\\

\section{Conclusion}
Thus, we have learnt about the Configuration of APN of a router, and manage its access control for Security.

\clearpage

\begin{thebibliography}{99}
    \bibitem{gsmaapn}
      GSM Association (GSMA) APN Configuration Guidelines. \\
      \url{https://www.gsma.com/technicalprojects/wp-content/uploads/2019/05/IR-82-v16.0.pdf}
  
    \bibitem{nistaccesscontrol}
      NIST Special Publication 800-53: Access Control. \\
      \url{https://csrc.nist.gov/publications/detail/sp/800-53/rev-5/final}
  
    \bibitem{fccroutersecurity}
      Federal Communications Commission (FCC) Router Security Tips. \\
      \url{https://www.fcc.gov/general/router-security-tips}
  
    \bibitem{wifiwps}
      Wi-Fi Alliance: Wi-Fi Protected Setup (WPS) Overview. \\
      \url{https://www.wi-fi.org/discover-wi-fi/wi-fi-protected-setup}
  
    \bibitem{wifisecurityprotocols}
      Wi-Fi Alliance: Wi-Fi Security Protocols. \\
      \url{https://www.wi-fi.org/discover-wi-fi/security}
  \end{thebibliography}
\end{document}