% This is a Basic Assignment Paper but with like Code and stuff allowed in it, there is also url, hyperlinks from contents included. 

\documentclass[11pt]{article}

% Preamble

\usepackage[margin=1in]{geometry}
\usepackage{amsfonts, amsmath, amssymb, amsthm}
\usepackage{fancyhdr, float, graphicx}
\usepackage[utf8]{inputenc} % Required for inputting international characters
\usepackage[T1]{fontenc} % Output font encoding for international characters
\usepackage{fouriernc} % Use the New Century Schoolbook font
\usepackage[nottoc, notlot, notlof]{tocbibind}
\usepackage{listings}
\usepackage{xcolor}
\usepackage{blindtext}
\usepackage{hyperref}
\definecolor{codepurple}{rgb}{0.58,0,0.82}
\hypersetup{
    colorlinks=true,
    linkcolor=black,
    filecolor=black,      
    urlcolor=codepurple,
    pdfpagemode=FullScreen,
    }

\definecolor{codegreen}{rgb}{0,0.6,0}
\definecolor{codegray}{rgb}{0.5,0.5,0.5}
\definecolor{backcolour}{rgb}{0.95,0.95,0.92}

\lstdefinestyle{mystyle}{
    backgroundcolor=\color{backcolour},   
    commentstyle=\color{codegreen},
    keywordstyle=\color{magenta},
    numberstyle=\tiny\color{codegray},
    stringstyle=\color{codepurple},
    basicstyle=\ttfamily\footnotesize,
    breakatwhitespace=false,         
    breaklines=true,                 
    captionpos=b,                    
    keepspaces=true,                 
    numbers=left,                    
    numbersep=5pt,                  
    showspaces=false,                
    showstringspaces=false,
    showtabs=false,                  
    tabsize=2
}

\lstset{style=mystyle}

% Header and Footer
\pagestyle{fancy}
\fancyhead{}
\fancyfoot{}
\fancyhead[L]{\textit{\Large{Vulnerability Identification and Penetration Testing}}}
\fancyhead[R]{\textit{Krishnaraj T}}
\fancyfoot[C]{\thepage}
\renewcommand{\footrulewidth}{1pt}
\newtheorem{thm}{Theorem}
\newtheorem{dfn}[thm]{Definition}


% Other Doc Editing
% \parindent 0ex
%\renewcommand{\baselinestretch}{1.5}

\begin{document}

\begin{titlepage}
    \centering

    %---------------------------NAMES-------------------------------

    \huge\textsc{
        MIT World Peace University
    }\\

    \vspace{0.75\baselineskip} % space after Uni Name

    \LARGE{
        Vulnerability Identification and Penetration Testing\\
        Third Year B. Tech, Semester 6
    }

    \vfill % space after Sub Name

    %--------------------------TITLE-------------------------------

    \rule{\textwidth}{1.6pt}\vspace*{-\baselineskip}\vspace*{2pt}
    \rule{\textwidth}{0.6pt}
    \vspace{0.75\baselineskip} % Whitespace above the title



    \huge{\textsc{
            Exploring Tools for Vulnerability Identification and Penetration Testing
        }} \\



    \vspace{0.5\baselineskip} % Whitespace below the title
    \rule{\textwidth}{0.6pt}\vspace*{-\baselineskip}\vspace*{2.8pt}
    \rule{\textwidth}{1.6pt}

    \vspace{1\baselineskip} % Whitespace after the title block

    %--------------------------SUBTITLE --------------------------	

    \LARGE\textsc{
        Theory Assignment 1
    } % Subtitle or further description
    \vfill

    %--------------------------AUTHOR-------------------------------

    Prepared By
    \vspace{0.5\baselineskip} % Whitespace before the editors

    \Large{
        Krishnaraj Thadesar \\
        Cyber Security and Forensics\\
        Batch A1, PA 10
    }


    \vspace{0.5\baselineskip} % Whitespace below the editor list
    \today

\end{titlepage}


\tableofcontents
\thispagestyle{empty}
\clearpage

\setcounter{page}{1}

\section{Exploring Tool 1 - Hping (hping3)}

\subsection{Purpose of Tool}
hping3 is a network tool able to send custom ICMP/UDP/TCP packets and to display target replies like ping does with ICMP replies. It handles fragmentation and arbitrary packet body and size, and can be used to transfer files under supported protocols. Using hping3, you can test firewall rules, perform (spoofed) port scanning, test network performance using different protocols, do path MTU discovery, perform traceroute-like actions under different protocols, fingerprint remote operating systems, audit TCP/IP stacks, etc. hping3 is scriptable using the Tcl language.

\subsection{Command 1 - Scanning}

\subsubsection*{Syntax}
\begin{verbatim}
$ sudo hping3 --scan ports -S target_ip
\end{verbatim}

\subsubsection*{Command}
\begin{verbatim}
$sudo hping3 --scan 1-30,70-90 -S www.target.host
\end{verbatim}

\subsubsection*{Purpose}
To scan for open ports on the target machine.

\subsubsection*{Output}
\begin{figure}[H]
    \centering
    \includegraphics[width=0.8\textwidth]{hping scan.jpg}
    \caption{To scan open ports}
    \label{fig:1}
\end{figure}

\subsection{Command 2 - Traceroute}

\subsubsection*{Syntax}
\begin{verbatim}
$ sudo hping3 --traceroute target_ip
\end{verbatim}

\subsubsection*{Command}
\begin{verbatim}
$sudo hping3 --traceroute krishnarajt.surge.sh
\end{verbatim}


\subsubsection*{Purpose}
To trace the route to the target machine.

\subsubsection*{Output}
\begin{figure}[H]
    \centering
    \includegraphics[width=0.8\textwidth]{hping trace.jpg}
    \caption{To trace the route to the target machine}
    \label{fig:2}
\end{figure}

\subsection{Command 3 - Flood}

\subsubsection*{Syntax}
\begin{verbatim}
$ sudo hping3 --flood target_ip
\end{verbatim}

\subsubsection*{Command}
\begin{verbatim}
$sudo hping3 --flood krishnarajt.surge.sh
\end{verbatim}

\subsubsection*{Purpose}
To flood the target machine with packets.

\subsubsection*{Output}
\begin{figure}[H]
    \centering
    \includegraphics[width=0.8\textwidth]{hping flood.jpg}
    \caption{To flood the target machine with packets}
    \label{fig:3}
\end{figure}

\subsection{Command 4 - Ping}

\subsubsection*{Syntax}
\begin{verbatim}
$ sudo hping3 --icmp target_ip
\end{verbatim}

\subsubsection*{Command}
\begin{verbatim}
$sudo hping3 --icmp krishnarajt.surge.sh
\end{verbatim}

\subsubsection*{Purpose}
To ping the target machine.

\subsubsection*{Output}
\begin{figure}[H]
    \centering
    \includegraphics[width=0.8\textwidth]{hping ping.jpg}
    \caption{To ping the target machine}
    \label{fig:4}
\end{figure}

\subsection{Command 5 - Syn Flood}

\subsubsection*{Syntax}
\begin{verbatim}
$ sudo hping3 --flood --rand-source -S target_ip
\end{verbatim}

\subsubsection*{Command}
\begin{verbatim}
$sudo hping3 --flood --rand-source -S krishnarajt.surge.sh
\end{verbatim}

\subsubsection*{Purpose}
To flood the target machine with SYN packets.

\subsubsection*{Output}
\begin{figure}[H]
    \centering
    \includegraphics[width=0.8\textwidth]{hping syn flood.jpg}
    \caption{To flood the target machine with SYN packets}
    \label{fig:5}

\end{figure}

\subsection{Command 6 - UDP Flood}

\subsubsection*{Syntax}
\begin{verbatim}
$ sudo hping3 --flood --rand-source -2 target_ip
\end{verbatim}

\subsubsection*{Command}
\begin{verbatim}
$sudo hping3 --flood --rand-source -2 krishnarajt.surge.sh
\end{verbatim}

\subsubsection*{Purpose}
To flood the target machine with UDP packets.

\subsubsection*{Output}
\begin{figure}[H]
    \centering
    \includegraphics[width=0.8\textwidth]{hping udp flood.jpg}
    \caption{To flood the target machine with UDP packets}
    \label{fig:6}
\end{figure}

\subsection{Command 7 - TCP Flood}

\subsubsection*{Syntax}
\begin{verbatim}
$ sudo hping3 --flood --rand-source -1 target_ip
\end{verbatim}

\subsubsection*{Command}
\begin{verbatim}
$sudo hping3 --flood --rand-source -1 krishnarajt.surge.sh
\end{verbatim}

\subsubsection*{Purpose}
To flood the target machine with TCP packets.

\subsubsection*{Output}
\begin{figure}[H]
    \centering
    \includegraphics[width=0.8\textwidth]{hping tcp flood.jpg}
    \caption{To flood the target machine with TCP packets}
    \label{fig:7}
\end{figure}

\subsection{Command 8 - HTTP Flood}

\subsubsection*{Syntax}
\begin{verbatim}
$ sudo hping3 --flood --rand-source -F target_ip
\end{verbatim}

\subsubsection*{Command}
\begin{verbatim}
$sudo hping3 --flood --rand-source -F krishnarajt.surge.sh
\end{verbatim}

\subsubsection*{Purpose}
To flood the target machine with HTTP packets.

\subsubsection*{Output}
\begin{figure}[H]
    \centering
    \includegraphics[width=0.8\textwidth]{hping http flood.jpg}
    \caption{To flood the target machine with HTTP packets}
    \label{fig:8}
\end{figure}

\section{Exploring Tool 2 - p0f}
p0f is a tool that utilizes an array of sophisticated, purely passive traffic fingerprinting mechanisms to identify the players behind any incidental TCP/IP communications (often as little as a single normal SYN) without interfering in any way. Version 3 is a complete rewrite of the original codebase, incorporating a significant number of improvements to network-level fingerprinting, and introducing the ability to reason about application-level payloads (e.g., HTTP).
\section{Exploring Tool 3 - httprint}
\section{Exploring Tool 4 - brutus}

\section{Platform}
\textbf{Operating System}: Kali Linux Rolling on WSL\\
\textbf{IDEs or Text Editors Used}: Visual Studio Code\\
% \textbf{Compilers or Interpreters}: Python 3.10.1\\

% \section{Code}
% \lstinputlisting[language=Python, caption="DSA Signature Validity using PyCrypto Library"]{../Programs/Assignment_7/dsa using lib.py}

\section{Conclusion}
Thus, we have successfully Explored several Tools for Vulnerability Identification and Penetration Testing.

\clearpage

\pagebreak

\end{document}