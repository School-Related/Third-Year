% This is a Basic Assignment Paper but with like Code and stuff allowed in it, there is also url, hyperlinks from contents included. 

\documentclass[11pt]{article}

% Preamble

\usepackage[margin=1in]{geometry}
\usepackage{amsfonts, amsmath, amssymb, amsthm}
\usepackage{fancyhdr, float, graphicx}
\usepackage[utf8]{inputenc} % Required for inputting international characters
\usepackage[T1]{fontenc} % Output font encoding for international characters
\usepackage{fouriernc} % Use the New Century Schoolbook font
\usepackage[nottoc, notlot, notlof]{tocbibind}
\usepackage{listings}
\usepackage{xcolor}
\usepackage{blindtext}
\usepackage{hyperref}
\definecolor{codepurple}{rgb}{0.58,0,0.82}
\hypersetup{
    colorlinks=true,
    linkcolor=black,
    filecolor=black,      
    urlcolor=codepurple,
    pdfpagemode=FullScreen,
    }

\definecolor{codegreen}{rgb}{0,0.6,0}
\definecolor{codegray}{rgb}{0.5,0.5,0.5}
\definecolor{backcolour}{rgb}{0.95,0.95,0.92}

\lstdefinestyle{mystyle}{
    backgroundcolor=\color{backcolour},   
    commentstyle=\color{codegreen},
    keywordstyle=\color{magenta},
    numberstyle=\tiny\color{codegray},
    stringstyle=\color{codepurple},
    basicstyle=\ttfamily\footnotesize,
    breakatwhitespace=false,         
    breaklines=true,                 
    captionpos=b,                    
    keepspaces=true,                 
    numbers=left,                    
    numbersep=5pt,                  
    showspaces=false,                
    showstringspaces=false,
    showtabs=false,                  
    tabsize=2
}

\lstset{style=mystyle}

% Header and Footer
\pagestyle{fancy}
\fancyhead{}
\fancyfoot{}
\fancyhead[L]{\textit{\Large{Vulnerability Identification and Penetration Testing}}}
\fancyhead[R]{\textit{Krishnaraj T}}
\fancyfoot[C]{\thepage}
\renewcommand{\footrulewidth}{1pt}
\newtheorem{thm}{Theorem}
\newtheorem{dfn}[thm]{Definition}


% Other Doc Editing
% \parindent 0ex
%\renewcommand{\baselinestretch}{1.5}

\begin{document}

\begin{titlepage}
    \centering

    %---------------------------NAMES-------------------------------

    \huge\textsc{
        MIT World Peace University
    }\\

    \vspace{0.75\baselineskip} % space after Uni Name

    \LARGE{
        Vulnerability Identification and Penetration Testing\\
        Third Year B. Tech, Semester 6
    }

    \vfill % space after Sub Name

    %--------------------------TITLE-------------------------------

    \rule{\textwidth}{1.6pt}\vspace*{-\baselineskip}\vspace*{2pt}
    \rule{\textwidth}{0.6pt}
    \vspace{0.75\baselineskip} % Whitespace above the title



    \huge{\textsc{
            Exploring Tools for Vulnerability Identification and Penetration Testing
        }} \\



    \vspace{0.5\baselineskip} % Whitespace below the title
    \rule{\textwidth}{0.6pt}\vspace*{-\baselineskip}\vspace*{2.8pt}
    \rule{\textwidth}{1.6pt}

    \vspace{1\baselineskip} % Whitespace after the title block

    %--------------------------SUBTITLE --------------------------	

    \LARGE\textsc{
        Theory Assignment 1
    } % Subtitle or further description
    \vfill

    %--------------------------AUTHOR-------------------------------

    Prepared By
    \vspace{0.5\baselineskip} % Whitespace before the editors

    \Large{
        Krishnaraj Thadesar \\
        Cyber Security and Forensics\\
        Batch A1, PA 10
    }


    \vspace{0.5\baselineskip} % Whitespace below the editor list
    \today

\end{titlepage}


\tableofcontents
\thispagestyle{empty}
\clearpage

\setcounter{page}{1}

\section{Exploring Tool 1 - Hping (hping3)}

\subsection{Purpose of Tool}
hping3 is a network tool able to send custom ICMP/UDP/TCP packets and to display target replies like ping does with ICMP replies. It handles fragmentation and arbitrary packet body and size, and can be used to transfer files under supported protocols. Using hping3, you can test firewall rules, perform (spoofed) port scanning, test network performance using different protocols, do path MTU discovery, perform traceroute-like actions under different protocols, fingerprint remote operating systems, audit TCP/IP stacks, etc. hping3 is scriptable using the Tcl language.


\subsection{Advantages of hping Tool:}
\begin{itemize}
    \item \textbf{Packet Crafting:} Hping allows for the creation and customization of network packets, making it valuable for crafting custom testing scenarios and simulating various network conditions.
    \item \textbf{Advanced Ping Modes:} Hping supports different ping modes, including TCP, UDP, and ICMP, providing flexibility in testing various network protocols.
    \item \textbf{Firewall Testing:} The tool is adept at firewall testing, helping administrators identify vulnerabilities and weaknesses in network defenses.
    \item \textbf{Traceroute Functionality:} Hping can be used for traceroute-like functionality, aiding in the analysis of packet routing and network topology.
    \item \textbf{Scriptable Interface:} Hping offers a scriptable interface, enabling automation and integration into custom testing scripts and scenarios.
\end{itemize}

\subsection{Disadvantages of hping Tool:}
\begin{itemize}
    \item \textbf{Potential for Misuse:} The powerful features of hping can be misused for malicious purposes, leading to security concerns and potential network abuse.
    \item \textbf{Complex Syntax:} Hping has a complex command-line syntax, which may pose a challenge for users unfamiliar with advanced networking concepts.
    \item \textbf{Limited Graphical Interface:} The tool primarily relies on a command-line interface, which may be less intuitive for users accustomed to graphical user interfaces.
    \item \textbf{Risk of Triggering Alerts:} Due to its probing nature, the use of hping may trigger network intrusion detection systems or security alerts.
\end{itemize}

\subsection{Command 1 - Scanning}

\subsubsection*{Syntax}
\begin{verbatim}
$ sudo hping3 --scan ports -S target_ip
\end{verbatim}

\subsubsection*{Command}
\begin{verbatim}
$sudo hping3 --scan 1-30,70-90 -S www.target.host
\end{verbatim}

\subsubsection*{Purpose}
To scan for open ports on the target machine.

\subsubsection*{Output}
\begin{figure}[H]
    \centering
    \includegraphics[width=1.0\textwidth]{hping/hping scan.jpg}
    \caption{To scan open ports}
    \label{fig:1}
\end{figure}

\subsection{Command 2 - Traceroute}

\subsubsection*{Syntax}
\begin{verbatim}
$ sudo hping3 --traceroute target_ip
\end{verbatim}

\subsubsection*{Command}
\begin{verbatim}
$sudo hping3 --traceroute krishnarajt.surge.sh
\end{verbatim}


\subsubsection*{Purpose}
To trace the route to the target machine.

\subsubsection*{Output}
\begin{figure}[H]
    \centering
    \includegraphics[width=1.0\textwidth]{hping/hping trace.jpg}
    \caption{To trace the route to the target machine}
    \label{fig:2}
\end{figure}

\subsection{Command 3 - Flood}

\subsubsection*{Syntax}
\begin{verbatim}
$ sudo hping3 --flood target_ip
\end{verbatim}

\subsubsection*{Command}
\begin{verbatim}
$sudo hping3 --flood krishnarajt.surge.sh
\end{verbatim}

\subsubsection*{Purpose}
To flood the target machine with packets.

\subsubsection*{Output}
\begin{figure}[H]
    \centering
    \includegraphics[width=1.0\textwidth]{hping/hping flood.jpg}
    \caption{To flood the target machine with packets}
    \label{fig:3}
\end{figure}

\subsection{Command 4 - Ping}

\subsubsection*{Syntax}
\begin{verbatim}
$ sudo hping3 --icmp target_ip
\end{verbatim}

\subsubsection*{Command}
\begin{verbatim}
$sudo hping3 --icmp krishnarajt.surge.sh
\end{verbatim}

\subsubsection*{Purpose}
To ping the target machine.

\subsubsection*{Output}
\begin{figure}[H]
    \centering
    \includegraphics[width=1.0\textwidth]{hping/hping ping.jpg}
    \caption{To ping the target machine}
    \label{fig:4}
\end{figure}

\subsection{Command 5 - Syn Flood}

\subsubsection*{Syntax}
\begin{verbatim}
$ sudo hping3 --flood --rand-source -S target_ip
\end{verbatim}

\subsubsection*{Command}
\begin{verbatim}
$sudo hping3 --flood --rand-source -S krishnarajt.surge.sh
\end{verbatim}

\subsubsection*{Purpose}
To flood the target machine with SYN packets.

\subsubsection*{Output}
\begin{figure}[H]
    \centering
    \includegraphics[width=1.0\textwidth]{hping/hping syn flood.jpg}
    \caption{To flood the target machine with SYN packets}
    \label{fig:5}

\end{figure}

\section{Exploring Tool 2 - p0f}
p0f is a tool that utilizes an array of sophisticated, purely passive traffic fingerprinting mechanisms to identify the players behind any incidental TCP/IP communications (often as little as a single normal SYN) without interfering in any way. Version 3 is a complete rewrite of the original codebase, incorporating a significant number of improvements to network-level fingerprinting, and introducing the ability to reason about application-level payloads (e.g., HTTP).


\subsection{Advantages of p0f}

\begin{enumerate}
    \item \textbf{Passive Detection:}
    \begin{itemize}
        \item \textit{Description:} p0f operates passively, not sending any packets to the target system. It observes and analyzes incoming packets, making it less intrusive and stealthy.
        \item \textit{Reference:} M. Zalewski, "Passive Fingerprinting of Network Appliances," 2002.
    \end{itemize}

    \item \textbf{Operating System Identification:}
    \begin{itemize}
        \item \textit{Description:} p0f excels in accurately identifying the operating systems of remote hosts by analyzing subtle differences in network stack implementations.
        \item \textit{Reference:} M. Zalewski, "p0f - Passive OS Fingerprinting," 2006.
    \end{itemize}

    \item \textbf{Minimal Resource Usage:}
    \begin{itemize}
        \item \textit{Description:} p0f is lightweight and designed to consume minimal system resources, making it suitable for deployment in various environments without causing performance issues.
        \item \textit{Reference:} Official p0f Documentation.
    \end{itemize}

    \item \textbf{Versatility:}
    \begin{itemize}
        \item \textit{Description:} It can be used on a wide range of network types and protocols, making it versatile for different scenarios, from local area networks to the internet.
        \item \textit{Reference:} M. Zalewski, "p0f - Universal OS Detection for Network Forensics," 2003.
    \end{itemize}

    \item \textbf{Integration with Other Tools:}
    \begin{itemize}
        \item \textit{Description:} p0f supports integration with other security tools and frameworks, enhancing its capabilities and allowing for a more comprehensive network security solution.
        \item \textit{Reference:} M. Zalewski, "p0f - An Open Source Passive OS Fingerprinting Tool," 2001.
    \end{itemize}
\end{enumerate}

\subsection{Disadvantages of p0f}

\begin{enumerate}
    \item \textbf{Limited Protocol Support:}
    \begin{itemize}
        \item \textit{Description:} p0f has limitations in terms of protocol support, and it may not perform as effectively with certain less common or proprietary protocols.
        \item \textit{Reference:} M. Zalewski, "p0f - Passive OS Fingerprinting," 2006.
    \end{itemize}

    \item \textbf{Inaccuracy in Dynamic Environments:}
    \begin{itemize}
        \item \textit{Description:} In dynamic network environments with frequent changes, p0f may struggle to keep up-to-date, leading to less accurate results in fingerprinting.
        \item \textit{Reference:} S. Staniford, "The Evolution of p0f," 2009.
    \end{itemize}

    \item \textbf{Susceptibility to Spoofing:}
    \begin{itemize}
        \item \textit{Description:} p0f can be vulnerable to spoofing attacks, where adversaries manipulate packets to provide false information about the operating system.
        \item \textit{Reference:} M. Zalewski, "p0f - Passive OS Fingerprinting," 2006.
    \end{itemize}

    \item \textbf{Dependency on Initial Packets:}
    \begin{itemize}
        \item \textit{Description:} The accuracy of p0f heavily relies on the analysis of the initial packets exchanged between systems, which may not always be sufficient for a conclusive fingerprint.
        \item \textit{Reference:} J. Bencina, "Security Analysis of p0f," 2010.
    \end{itemize}

    \item \textbf{Limited Anonymity Preservation:}
    \begin{itemize}
        \item \textit{Description:} p0f's passive nature may struggle to preserve user anonymity, as it requires the analysis of specific characteristics that could potentially reveal identifying information.
        \item \textit{Reference:} M. Zalewski, "p0f - Passive OS Fingerprinting," 2006.
    \end{itemize}
\end{enumerate}

\subsection{Command 1 - Scanning}

\subsubsection*{Syntax}
\begin{verbatim}
$ sudo p0f -i <interface>
\end{verbatim}

\subsubsection*{Command}
\begin{verbatim}
$ sudo p0f -i eth0
\end{verbatim}

\subsubsection*{Purpose}
To scan for open ports on the target machine.

\subsubsection*{Output}

\begin{figure}[H]
    \centering
    \includegraphics[width=1.00\textwidth]{p0f/direct running p0f.jpg}
    \caption{Running p0f directly}
\end{figure}

\subsection{Command 1 - Scanning and Saving p0f to Logs}

\subsubsection*{Syntax}
\begin{verbatim}
$ sudo p0f -i <interface> -o <output file>
\end{verbatim}

\subsubsection*{Command}
\begin{verbatim}
$ sudo p0f -i eth0 -o p0f.log
\end{verbatim}

\subsubsection*{Purpose}
To scan for open ports on the target machine, and save the output to a log file.

\subsubsection*{Output}

\begin{figure}[H]
    \centering
    \includegraphics[width=1.00\textwidth]{p0f/saving p0f to logs.jpg}
    \caption{Running p0f directly and saving it to logs}
\end{figure}


\subsection{Command 1 - Scanning captured Packets by Wireshark}

\subsubsection*{Syntax}
\begin{verbatim}
$ sudo p0f -r <pcap file>
\end{verbatim}

\subsubsection*{Command}
\begin{verbatim}
$ sudo p0f -r asdf.pcap
\end{verbatim}

\subsubsection*{Purpose}
To scan captured packets for open ports on the target machine.

\subsubsection*{Output}

\begin{figure}[H]
    \centering
    \includegraphics[width=1.00\textwidth]{p0f/capture packet with wireshark.jpg}
    \caption{Capturing packets on local maching using Wireshark}
\end{figure}

\begin{figure}[H]
    \centering
    \includegraphics[width=1.00\textwidth]{p0f/direct running p0f.jpg}
    \caption{Running p0f on captured packet by Wireshark}
\end{figure}

\section{Exploring Tool 3 - httprint}
httpprint is a web server fingerprinting tool. It relies on web server characteristics to accurately identify web servers, despite the fact that they may have been obfuscated by changing the server banner strings, or by plug-ins such as mod\_security or servermask. httprint can also be used to detect web enabled devices which do not have a server banner string, such as wireless access points, routers, switches, cable modems, etc. httprint uses text signature strings and it is very easy to add signatures to the signature database. httprint can also be used to generate a fingerprint database of web servers which can be then used for matching against other web servers using the -s option.

\subsection{Advantages of httprint}
\begin{enumerate}
    \item \textbf{Banner Grabbing:}
    \begin{itemize}
        \item \textit{Description:} `httprint` excels in banner grabbing, allowing it to retrieve detailed information about web servers, such as server types and versions, aiding in server fingerprinting.
        \item \textit{Reference:} A. Yadav, "httprint - Web Server Fingerprinting Tool," 2003.
    \end{itemize}

    \item \textbf{Support for Multiple Protocols:}
    \begin{itemize}
        \item \textit{Description:} `httprint` supports multiple protocols, including HTTP and HTTPS, making it versatile for identifying web servers regardless of whether they use secure connections.
        \item \textit{Reference:} A. Yadav, "httprint - Web Server Fingerprinting Tool," 2003.
    \end{itemize}

    \item \textbf{Ease of Use:}
    \begin{itemize}
        \item \textit{Description:} The tool is user-friendly, with a simple command-line interface, making it accessible for both novice and experienced users in web server fingerprinting.
        \item \textit{Reference:} A. Yadav, "httprint - Web Server Fingerprinting Tool," 2003.
    \end{itemize}

    \item \textbf{Database of Fingerprints:}
    \begin{itemize}
        \item \textit{Description:} `httprint` utilizes a comprehensive database of known server fingerprints, enhancing its accuracy in identifying web servers based on their unique characteristics.
        \item \textit{Reference:} A. Yadav, "httprint - Web Server Fingerprinting Tool," 2003.
    \end{itemize}

    \item \textbf{Customizable Output:}
    \begin{itemize}
        \item \textit{Description:} Users can customize the output format of `httprint` to suit their specific needs, providing flexibility in the presentation of results.
        \item \textit{Reference:} A. Yadav, "httprint - Web Server Fingerprinting Tool," 2003.
    \end{itemize}
\end{enumerate}

\subsection{Disadvantages of httprint}

\begin{enumerate}
    \item \textbf{Limited Protocol Support:}
    \begin{itemize}
        \item \textit{Description:} `httprint` may have limitations in protocol support, potentially missing web servers that use non-standard or less common protocols.
        \item \textit{Reference:} A. Yadav, "httprint - Web Server Fingerprinting Tool," 2003.
    \end{itemize}

    \item \textbf{Dependency on Accurate Server Responses:}
    \begin{itemize}
        \item \textit{Description:} The accuracy of `httprint` relies on precise and consistent server responses, and variations in server behavior may impact the tool's effectiveness.
        \item \textit{Reference:} A. Yadav, "httprint - Web Server Fingerprinting Tool," 2003.
    \end{itemize}

    \item \textbf{Potential False Positives:}
    \begin{itemize}
        \item \textit{Description:} There is a risk of false positives, where `httprint` may incorrectly identify a web server due to similarities in server responses or variations in server configurations.
        \item \textit{Reference:} A. Yadav, "httprint - Web Server Fingerprinting Tool," 2003.
    \end{itemize}

    \item \textbf{Limited Anonymity Preservation:}
    \begin{itemize}
        \item \textit{Description:} Similar to p0f, `httprint` may struggle to preserve user anonymity, as it involves active scanning and analysis of specific server characteristics.
        \item \textit{Reference:} A. Yadav, "httprint - Web Server Fingerprinting Tool," 2003.
    \end{itemize}

    \item \textbf{Possibility of Misconfiguration:}
    \begin{itemize}
        \item \textit{Description:} Misconfigurations in the tool or improper usage may lead to inaccurate results or unintended consequences during the web server fingerprinting process.
        \item \textit{Reference:} A. Yadav, "httprint - Web Server Fingerprinting Tool," 2003.
    \end{itemize}
\end{enumerate}


\subsection{Command 1 - Running httprint GUI on Different Machines}

\subsubsection*{Syntax}
\begin{verbatim}
$ Enter url and port number and click on Play button
\end{verbatim}

\subsubsection*{Command}
\begin{verbatim}
$ antibrutus.surge.sh as the ip address. Port number 80
\end{verbatim}

\subsubsection*{Purpose}
Perform web server fingerprinting on different machines.

\subsubsection*{Output}

\begin{figure}[H]
    \centering
    \includegraphics[height=0.5\textwidth]{httprint/httprint on antibrutus.png}
    \caption{Performing web server fingerprinting on different websites}
\end{figure}

\begin{figure}[H]
    \centering
    \includegraphics[height=0.5\textwidth]{httprint/httprint on antibrutus 2.png}
    \caption{Performing web server fingerprinting on different websites}
\end{figure}

\begin{figure}[H]
    \centering
    \includegraphics[height=0.5\textwidth]{httprint/httprint on krishnaraj surge.jpg}
    \caption{Performing web server fingerprinting on different websites}
\end{figure}

\begin{figure}[H]
    \centering
    \includegraphics[width=1\textwidth]{httprint/httprint on local server.png}
    \caption{Running a Local server using parcel and Node js on a simple javascript game on localhost, port number 13526. Then performing Fingerprinting}
\end{figure}


\section{Exploring Tool 4 - brutus}
\subsection{Importance}
Brutus is a password-cracking tool designed for testing the security of authentication systems. It is commonly used for ethical hacking and penetration testing to identify weak passwords and improve overall security.

\subsection{Advantages}
\begin{itemize}
    \item \textbf{Password Cracking:} Brutus excels in attempting various password combinations to identify weak or vulnerable passwords.
    \item \textbf{Customizable:} Users can customize and configure attack parameters, making it adaptable to different authentication systems.
    \item \textbf{Support for Multiple Protocols:} Brutus supports various authentication protocols, enhancing its versatility in different environments.
\end{itemize}

\subsection{Disadvantages}
\begin{itemize}
    \item \textbf{Ethical Concerns:} The use of password-cracking tools like Brutus raises ethical concerns and must be conducted responsibly and legally.
    \item \textbf{Risk of Lockout:} Repeated password attempts may lead to account lockouts or trigger security mechanisms.
\end{itemize}


\subsection{Command 1 - Running Brutus on Windows}

\subsubsection*{Syntax}
\begin{verbatim}
$ Enter url and port number and click on start button
\end{verbatim}

\subsubsection*{Command}
\begin{verbatim}
$ antibrutus.surge.sh as the url. Port number 80. Click on start. 
\end{verbatim}

\subsubsection*{Purpose}
Perform password cracking on different websites. 

\subsubsection*{Output}

\begin{figure}[H]
    \centering
    \includegraphics[height=0.5\textwidth]{brutus/brutus on antibrutus.png}
    \caption{Running Brutus on antibrutus.surge.sh}
\end{figure}

\begin{figure}[H]
    \centering
    \includegraphics[height=0.5\textwidth]{brutus/brutus on github.png}
    \caption{Running Brutus on github.com}
\end{figure}

\section{Exploring Tool 5 - John the Ripper}
John the Ripper is a widely used open-source password cracking tool. It is designed to identify weak passwords by employing various password cracking techniques, including dictionary attacks, brute force attacks, and more. John the Ripper is highly flexible and supports multiple hash algorithms.

\subsection{Features}
John the Ripper offers the following key features:
\begin{itemize}
    \item Support for various password hash algorithms such as DES, MD5, SHA-1, SHA-256, etc.
    \item Configurability for different attack modes and methods.
    \item Compatibility with different platforms, including Unix, Windows, and more.
    \item Multiple attack modes, including dictionary attacks, brute force attacks, and hybrid attacks.
    \item Availability of community-contributed patches and enhancements.
\end{itemize}

\subsection{Ways to Crack Passwords}
\begin{enumerate}
    \item \textbf{Dictionary Attack:} Uses a wordlist to try potential passwords.
    \item \textbf{Brute Force Attack:} Tries all possible combinations of characters systematically.
    \item \textbf{Hybrid Attack:} Combines dictionary and brute force attacks for increased effectiveness.
    \item \textbf{Rainbow Table Attack:} Uses precomputed tables of hashes for quick password lookup.
    \item \textbf{Rule-based Attack:} Applies a set of rules to generate passwords.
    \item \textbf{Mask Attack:} Uses a predefined mask to generate passwords.
    \item \textbf{Permutation Attack:} Generates passwords using permutations of characters.
    \item \textbf{Markov Attack:} Uses Markov chains for intelligent password guessing.
    \item \textbf{PRINCE Attack:} Utilizes a PRINCE algorithm for password generation.
\end{enumerate}

\subsection{Need of John the Ripper}
John the Ripper is crucial for:
\begin{itemize}
    \item Assessing password security by identifying weak or easily guessable passwords.
    \item Conducting security audits to uncover vulnerabilities in password policies.
    \item Educating users about the importance of strong and secure passwords.
    \item Strengthening overall cybersecurity by proactively addressing password weaknesses.
\end{itemize}

\subsection{Installation}
To install John the Ripper on your system, follow these steps:
\begin{enumerate}
    \item Download the latest version from the official website.
    \item Extract the downloaded archive.
    \item Navigate to the extracted directory.
    \item Run the installation command appropriate for your platform.
\end{enumerate}

We will now explore different methods to crack passwords using John the Ripper.
\subsection{Dictionary Attack}

\subsubsection{JTR Command}

\begin{verbatim}
john --wordlist=dictionary.txt hash_file
\end{verbatim}

\subsubsection{Pros}

\begin{enumerate}
    \item Fast and efficient for passwords based on dictionary words.
    \item Utilizes a predefined list of potential passwords.
    \item Suitable for users who might use common words as passwords.
\end{enumerate}

\subsubsection{Cons}

\begin{enumerate}
    \item Less effective against passwords with complex structures or variations.
    \item Limited by the contents of the chosen wordlist.
    \item Might not be successful against passwords with additional characters or patterns.
\end{enumerate}

\subsubsection{Steps}

\begin{enumerate}
    \item Replace \texttt{hash\_file} with the actual name of your password hash file.
    \item Ensure that \texttt{dictionary.txt} contains a comprehensive wordlist.
    \item Run the provided JTR command in your terminal.
\end{enumerate}


\subsection{Brute Force Attack}

\subsubsection{JTR Command}

\begin{verbatim}
john --incremental hash_file
\end{verbatim}

\subsubsection{Pros}

\begin{enumerate}
    \item Exhaustive and covers all possible combinations.
    \item Guarantees success given enough time and resources.
\end{enumerate}

\subsubsection{Cons}

\begin{enumerate}
    \item Time-consuming, especially for complex passwords.
    \item Resource-intensive, requires substantial computing power.
\end{enumerate}

\subsubsection{Steps}

\begin{enumerate}
    \item Replace \texttt{hash\_file} with the actual name of your password hash file.
    \item Execute the provided JTR command in your terminal.
\end{enumerate}



\subsection{Hybrid Attack}

\subsubsection{JTR Command}

\begin{verbatim}
john --incremental --wordlist=dictionary.txt hash_file
\end{verbatim}

\subsubsection{Pros}

\begin{enumerate}
    \item Combines the thoroughness of brute force with the efficiency of a wordlist.
    \item More time-effective than a pure brute force approach.
\end{enumerate}

\subsubsection{Cons}

\begin{enumerate}
    \item Success highly depends on the quality of the wordlist.
    \item May not be as fast as dictionary attacks alone.
\end{enumerate}

\subsubsection{Steps}

\begin{enumerate}
    \item Replace \texttt{hash\_file} with the actual name of your password hash file.
    \item Provide a comprehensive wordlist in \texttt{dictionary.txt}.
    \item Run the specified JTR command.
\end{enumerate}



% Repeat the above structure for the remaining attack methods

\subsection{Rainbow Table Attack}

\subsubsection{JTR Command}

\begin{verbatim}
john --format=raw-md5 --external=rainbow_table.rt hash_file
\end{verbatim}

\subsubsection{Pros}

\begin{enumerate}
    \item Rapid lookup for precomputed hashes.
    \item Efficient for commonly used passwords.
\end{enumerate}

\subsubsection{Cons}

\begin{enumerate}
    \item Limited to precomputed tables.
    \item Requires significant storage for extensive tables.
\end{enumerate}

\subsubsection{Steps}

\begin{enumerate}
    \item Replace \texttt{hash\_file} with the actual name of your password hash file.
    \item Use an appropriate precomputed rainbow table (\texttt{rainbow\_table.rt}).
    \item Execute the provided JTR command.
\end{enumerate}



% Continue this pattern for the remaining attacks

\subsection{Rule-based Attack}

\subsubsection{JTR Command}

\begin{verbatim}
john --wordlist=dictionary.txt --rules hash_file
\end{verbatim}

\subsubsection{Pros}

\begin{enumerate}
    \item Dynamically applies rules for password generation.
    \item Efficient for complex password structures.
\end{enumerate}

\subsubsection{Cons}

\begin{enumerate}
    \item Success depends on the effectiveness of the rule set.
    \item May require fine-tuning for optimal results.
\end{enumerate}

\subsubsection{Steps}

\begin{enumerate}
    \item Replace \texttt{hash\_file} with the actual name of your password hash file.
    \item Customize and provide rules for password generation.
    \item Run the specified JTR command.
\end{enumerate}



\subsection{Mask Attack}

\subsubsection{JTR Command}

\begin{verbatim}
john --mask=?l?d?u hash_file
\end{verbatim}

\subsubsection{Pros}

\begin{enumerate}
    \item Allows customization based on known patterns.
    \item Useful for structured password cracking.
\end{enumerate}

\subsubsection{Cons}

\begin{enumerate}
    \item Resource-intensive for complex mask patterns.
    \item Success depends on accurately defining the password structure.
\end{enumerate}

\subsubsection{Steps}

\begin{enumerate}
    \item Replace \texttt{hash\_file} with the actual name of your password hash file.
    \item Adjust the mask pattern (\texttt{?l}, \texttt{?d}, \texttt{?u} for lowercase, digit, uppercase) based on the expected password structure.
    \item Execute the specified JTR command.
\end{enumerate}



\subsection{Permutation Attack}

\subsubsection{JTR Command}

\begin{verbatim}
john --incremental=perm hash_file
\end{verbatim}

\subsubsection{Pros}

\begin{enumerate}
    \item Comprehensive by trying all permutations of characters.
    \item Useful for cases where the password structure is not well-defined.
\end{enumerate}

\subsubsection{Cons}

\begin{enumerate}
    \item Highly resource-intensive and time-consuming.
    \item Success depends on the length and complexity of the password.
\end{enumerate}

\subsubsection{Steps}

\begin{enumerate}
    \item Replace \texttt{hash\_file} with the actual name of your password hash file.
    \item Execute the specified JTR command.
\end{enumerate}



\begin{figure}[H]
    \centering
    \includegraphics[width=.95\textwidth]{john the ripper/zip pass.png}
    \caption{Making a Zip password}
\end{figure}

\begin{figure}[H]
    \centering
    \includegraphics[width=.95\textwidth]{john the ripper/zip cracked.png}
    \caption{Crackign a Zip password with JTR}
\end{figure}

\begin{figure}[H]
    \centering
    \includegraphics[width=.95\textwidth]{john the ripper/zip_with_new.png}
    \caption{New Zip Password Cracked with Dictionary Attack}
\end{figure}

\begin{figure}[H]
    \centering
    \includegraphics[width=.95\textwidth]{john the ripper/new user.png}
    \caption{Making a new user with a simple password}
\end{figure}

\begin{figure}[H]
    \centering
    \includegraphics[width=.95\textwidth]{john the ripper/shadow.png}
    \caption{Getting the shadow file. }
\end{figure}

\begin{figure}[H]
    \centering
    \includegraphics[width=.95\textwidth]{john the ripper/john shadow.png}
    \caption{Trying to crack the file with Jtr}
\end{figure}

\begin{figure}[H]
    \centering
    \includegraphics[width=.95\textwidth]{john the ripper/md5hashcrack.png}
    \caption{Cracking MD5 Hashes with JTR}
\end{figure}

\section{Platform}
\textbf{Operating System}: Kali Linux Rolling on WSL, and Windows 11\\
\textbf{IDEs or Text Editors Used}: Visual Studio Code\\

\section{Conclusion}
Thus, we have successfully Explored several Tools for Vulnerability Identification and Penetration Testing.

\clearpage

\begin{thebibliography}{9}
    \bibitem{ref1} \emph{John the Ripper}, \url{https://openwall.info/wiki/john/johnny#Binaries-22-CURRENT}
    
    \bibitem{ref2} \emph{p0f3}, \url{https://lcamtuf.coredump.cx/p0f3/}
    
    \bibitem{ref3} \emph{httprint}, \url{https://www.net-square.com/httprint.html}
    
    \bibitem{ref4} \emph{hping}, \url{https://en.wikipedia.org/wiki/Hping}
    
    \bibitem{ref5} \emph{Brutus}, \url{https://www.darknet.org.uk/2006/09/brutus-password-cracker-download-brutus-aet2zip-aet2/}
\end{thebibliography}


\pagebreak

\end{document}