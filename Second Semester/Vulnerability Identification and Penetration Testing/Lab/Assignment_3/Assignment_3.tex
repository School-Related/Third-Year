% This is a Basic Assignment Paper but with like Code and stuff allowed in it, there is also url, hyperlinks from contents included. 

\documentclass[11pt]{article}

% Preamble

\usepackage[margin=1in]{geometry}
\usepackage{amsfonts, amsmath, amssymb, amsthm}
\usepackage{fancyhdr, float, graphicx}
\usepackage[utf8]{inputenc} % Required for inputting international characters
\usepackage[T1]{fontenc} % Output font encoding for international characters
% \usepackage{fouriernc} % Use the New Century Schoolbook font
\usepackage[nottoc, notlot, notlof]{tocbibind}
\usepackage{listings}
\usepackage{xcolor}
\usepackage{blindtext}
\usepackage{hyperref}
\definecolor{codepurple}{rgb}{0.58,0,0.82}
\hypersetup{
    colorlinks=true,
    linkcolor=black,
    filecolor=black,      
    urlcolor=codepurple,
    pdfpagemode=FullScreen,
    }

\definecolor{codegreen}{rgb}{0,0.6,0}
\definecolor{codegray}{rgb}{0.5,0.5,0.5}
\definecolor{backcolour}{rgb}{0.95,0.95,0.92}

\lstdefinestyle{mystyle}{
    backgroundcolor=\color{backcolour},   
    commentstyle=\color{codegreen},
    keywordstyle=\color{magenta},
    numberstyle=\tiny\color{codegray},
    stringstyle=\color{codepurple},
    basicstyle=\ttfamily\footnotesize,
    breakatwhitespace=false,         
    breaklines=true,                 
    captionpos=b,                    
    keepspaces=true,                 
    numbers=left,                    
    numbersep=5pt,                  
    showspaces=false,                
    showstringspaces=false,
    showtabs=false,                  
    tabsize=2
}

\lstset{style=mystyle}

% Header and Footer
\pagestyle{fancy}
\fancyhead{}
\fancyfoot{}
\fancyhead[L]{\textit{\Large{Vulnerability Identification and Penetration Testing}}}
\fancyhead[R]{\textit{Krishnaraj T}}
\fancyfoot[C]{\thepage}
\renewcommand{\footrulewidth}{1pt}
\newtheorem{thm}{Theorem}
\newtheorem{dfn}[thm]{Definition}


% Other Doc Editing
% \parindent 0ex
%\renewcommand{\baselinestretch}{1.5}

\begin{document}

\begin{titlepage}
    \centering

    %---------------------------NAMES-------------------------------

    \huge\textsc{
        MIT World Peace University
    }\\

    \vspace{0.75\baselineskip} % space after Uni Name

    \LARGE{
        Vulnerability Identification and Penetration Testing\\
        Third Year B. Tech, Semester 6
    }

    \vfill % space after Sub Name

    %--------------------------TITLE-------------------------------

    \rule{\textwidth}{1.6pt}\vspace*{-\baselineskip}\vspace*{2pt}
    \rule{\textwidth}{0.6pt}
    \vspace{0.75\baselineskip} % Whitespace above the title



    \huge{\textsc{
            Gathering Network Information from Attackers Perspective
        }} \\



    \vspace{0.5\baselineskip} % Whitespace below the title
    \rule{\textwidth}{0.6pt}\vspace*{-\baselineskip}\vspace*{2.8pt}
    \rule{\textwidth}{1.6pt}

    \vspace{1\baselineskip} % Whitespace after the title block

    %--------------------------SUBTITLE --------------------------	

    \LARGE\textsc{
        Assignment 3
    } % Subtitle or further description
    \vfill

    %--------------------------AUTHOR-------------------------------

    Prepared By
    \vspace{0.5\baselineskip} % Whitespace before the editors

    \Large{
        Krishnaraj Thadesar \\
        Cyber Security and Forensics\\
        Batch A1, PA 10
    }


    \vspace{0.5\baselineskip} % Whitespace below the editor list
    \today

\end{titlepage}


\tableofcontents
\thispagestyle{empty}
\clearpage

\setcounter{page}{1}

\section{Aim}
To use various tools to gather network information from an attackers perspective.

\section{Objectives}
\begin{enumerate}
    \item To understand the importance of gathering network information.
    \item To use various tools to gather network information.
    \item To understand the importance of network information in penetration testing.
    \item To understand the importance of network information in vulnerability identification.
\end{enumerate}
\section{Theory}
\section{Network Information Gathering}

\subsection{nmap}
nmap is a network scanning tool that is used to discover hosts and services on a computer network. It is used to create a "map" of the network by sending specially crafted packets to the target host and analyzing the responses.

\subsection{Uses}

\begin{itemize}
    \item Discovering hosts on a network.
    \item Identifying open ports and services running on target systems.
    \item Detecting operating systems and software versions.
    \item Performing security audits and vulnerability assessments.
    \item Monitoring network performance and availability.
    \item Troubleshooting network connectivity issues.
    \item Investigating suspicious or malicious activities.
\end{itemize}


\subsection{Advantages}
\begin{itemize}
    \item Comprehensive scanning capabilities for discovering hosts, open ports, and services running on target systems.
    \item Flexible and customizable scanning options, allowing users to tailor scans according to their specific requirements.
    \item Support for scripting and automation, enabling efficient and repeatable scanning processes.
    \item Active development and community support, ensuring continuous improvement and updates to the tool.
    \item Cross-platform compatibility, making it available for use on various operating systems.
\end{itemize}

\subsection{Disadvantages}
\begin{itemize}
    \item Requires expertise to interpret scan results accurately and effectively identify vulnerabilities.
    \item Scanning can be resource-intensive and may lead to network congestion or disruptions if not managed properly.
    \item Limited effectiveness against well-configured and hardened systems that actively block or disguise scanning attempts.
    \item Possibility of triggering intrusion detection and prevention systems or being flagged as suspicious activity by network administrators.
    \item Legal considerations and potential ethical concerns related to unauthorized scanning of networks without proper authorization.
\end{itemize}

\subsection{Metasploit}
Metasploit is a penetration testing framework that simplifies hacking. It is an open-source tool used for developing, testing, and executing exploit code against remote target machines to exploit vulnerabilities in a network.

\subsection{Uses}

\begin{itemize}
    \item Identifying and exploiting vulnerabilities in target systems for penetration testing purposes.
    \item Developing and testing custom exploits for known vulnerabilities in software and systems.
    \item Post-exploitation activities, such as privilege escalation, lateral movement, and data exfiltration.
    \item Generating and delivering payloads to compromised systems for remote access and control.
    \item Conducting security assessments, red teaming exercises, and ethical hacking engagements.
    \item Researching and analyzing security vulnerabilities and attack techniques to improve defense strategies.
    \item Collaborating with other security tools and frameworks to enhance testing capabilities and coverage.
    \item Training and education in offensive security techniques and methodologies for security professionals.
    \item Contributing to the security community through the development and sharing of exploit code and research.
    \item Supporting incident response and forensic investigations by simulating real-world attack scenarios.
    \item Demonstrating the impact of security vulnerabilities and the importance of proactive defense measures.
\end{itemize}

\subsection{Advantages}
\begin{itemize}
    \item Extensive library of exploit modules and payloads for targeting a wide range of vulnerabilities across different systems and applications.
    \item User-friendly interface with intuitive features and workflows, suitable for both beginners and experienced penetration testers.
    \item Integration with other security tools and frameworks, enhancing its capabilities and interoperability in complex testing scenarios.
    \item Active community and regular updates, providing access to new exploits, features, and improvements.
    \item Extensibility through custom scripting and module development, allowing users to extend its functionality according to their needs.
\end{itemize}

\subsection{Disadvantages}
\begin{itemize}
    \item Requires caution and ethical considerations due to the potential for causing harm or damage if used maliciously or irresponsibly.
    \item Complexity of exploit development and usage may pose challenges for inexperienced users, requiring a steep learning curve.
    \item Reliance on publicly available exploit code may lead to detection by security solutions or failure to exploit patched vulnerabilities.
    \item Limited effectiveness against well-defended networks with robust security measures and up-to-date patching practices.
    \item Legal implications and regulatory compliance considerations, particularly when conducting penetration testing without proper authorization or consent.
\end{itemize}


\section{Implementation}
\subsection{Nmap Scan to Identify Operating System of Target Host}

\subsubsection*{Syntax}
\begin{verbatim}
$ nmap -O <target_ip>
\end{verbatim}

\subsubsection*{Command}
\begin{verbatim}
$ nmap -O 192.168.114.1
\end{verbatim}

\subsubsection*{Purpose}
This command is used to identify the operating system of the target host with the IP address
\subsubsection*{Output}
\begin{figure}[H]
    \centering
    \includegraphics[width=0.99\textwidth]{a3_ss .png}
    \caption{Output of the Command}
\end{figure}

\subsection{Nmap Scan to Identify Target OS with verbose output}

\subsubsection*{Syntax}
\begin{verbatim}
$ nmap -O <target_ip> -v
\end{verbatim}

\subsubsection*{Command}
\begin{verbatim}
$ nmap -O 192.168.114.1 -v
\end{verbatim}

\subsubsection*{Purpose}
This command is used to identify the operating system of the target host with the IP address and display verbose output.
\subsubsection*{Output}
\begin{figure}[H]
    \centering
    \includegraphics[width=0.99\textwidth]{a3_ss (1).png}
    \caption{Output of the Command}
\end{figure}

\subsection{Nmap Scan to Identify OS and Services Running on Target Host with verbose output}

\subsubsection*{Syntax}
\begin{verbatim}
$ nmap -sV -O -v <target_ip>
\end{verbatim}

\subsubsection*{Command}
\begin{verbatim}
$ nmap -sV -O -v 192.168.114.1
\end{verbatim}

\subsubsection*{Purpose}
This command is used to identify the operating system and services running on the target host with the IP address and display verbose output.
\subsubsection*{Output}
\begin{figure}[H]
    \centering
    \includegraphics[width=0.99\textwidth]{a3_ss (2).png}
    \caption{Output of the Command}
\end{figure}
\begin{figure}[H]
    \centering
    \includegraphics[width=0.99\textwidth]{a3_ss (3).png}
    \caption{Output of the Command}
\end{figure}

\begin{figure}[H]
    \centering
    \includegraphics[width=0.99\textwidth]{a3_ss (4).png}
    \caption{Output of the Command}
\end{figure}

\subsection{Performing SYN Scan on Target Host with Specific Port}

\subsubsection*{Syntax}
\begin{verbatim}
$ nmap -sS -p <port> <target_ip>
\end{verbatim}

\subsubsection*{Command}
\begin{verbatim}
$ nmap -sS -p 80 192.168.114.1
\end{verbatim}

\subsubsection*{Purpose}
This command is used to perform a SYN scan on port 80 of the target host with the IP address.
\subsubsection*{Output}
\begin{figure}[H]
    \centering
    \includegraphics[width=0.99\textwidth]{a3_ss (5).png}
    \caption{Output of the Command}
\end{figure}

\subsection{Performing SYN Scan on Specific Ports of Target Host}

\subsubsection*{Syntax}
\begin{verbatim}
$ nmap -sS -p <port1,port2,...> <target_ip>
\end{verbatim}

\subsubsection*{Command}
\begin{verbatim}
$ nmap -sS -p 20,21,22 192.168.114.1
\end{verbatim}

\subsubsection*{Purpose}
This command is used to perform a SYN scan on specific ports (20, 21, 22) of the target host with the IP address.
\subsubsection*{Output}
\begin{figure}[H]
    \centering
    \includegraphics[width=0.99\textwidth]{a3_ss (6).png}
    \caption{Output of the Command}
\end{figure}
\subsection{Performing SYN Scan on Top Ports of Target Host}

\subsubsection*{Syntax}
\begin{verbatim}
$ nmap --top-port <number_of_ports> <target_ip>
\end{verbatim}

\subsubsection*{Command}
\begin{verbatim} 
$ nmap --top-port 20 192.168.114.1
\end{verbatim}

\subsubsection*{Purpose}
This command is used to perform a SYN scan on the top 20 ports of the target host with the IP address.
\subsubsection*{Output}
\begin{figure}[H]
    \centering
    \includegraphics[width=0.99\textwidth]{a3_ss (7).png}
    \caption{Output of the Command}
\end{figure}
\subsection{Performing a Vulnerability Scan on Target Host using Nmap Scripts}

\subsubsection*{Syntax}
\begin{verbatim}
$ nmap -sV --script=http-malware-host <target_ip> -Pn
\end{verbatim}

\subsubsection*{Command}
\begin{verbatim}
$ nmap -sV --script=http-malware-host 192.168.114.1 -Pn
\end{verbatim}

\subsubsection*{Purpose}
This command is used to perform a vulnerability scan on the target host with the IP address using nmap scripts.
\subsubsection*{Output}
\begin{figure}[H]
    \centering
    \includegraphics[width=0.99\textwidth]{a3_ss (8).png}
    \caption{Output of the Command}
\end{figure}

\begin{figure}[H]
    \centering
    \includegraphics[width=0.99\textwidth]{a3_ss (9).png}
    \caption{Output of the Command}
\end{figure}

\begin{figure}[H]
    \centering
    \includegraphics[width=0.99\textwidth]{a3_ss (10).png}
    \caption{Output of the Command}
\end{figure}

\subsection{Performing a Vulnerability Scan on Target Host using Nmap Scripts}

\subsubsection*{Syntax}
\begin{verbatim}
$ nmap -Pn --script vuln <target_ip>
\end{verbatim}

\subsubsection*{Command}
\begin{verbatim}
$ nmap -Pn --script vuln 192.168.114.1
\end{verbatim}

\subsubsection*{Purpose}
This command is used to perform a vulnerability scan on the target host with the IP address using nmap scripts.
\subsubsection*{Output}
\begin{figure}[H]
    \centering
    \includegraphics[width=0.99\textwidth]{a3_ss (12).png}
    \caption{Output of the Command}
\end{figure}
\begin{figure}[H]
    \centering
    \includegraphics[width=0.99\textwidth]{a3_ss (15).png}
    \caption{Output of the Command}
\end{figure}
\subsection{Django server logs while performing vuln script scan}

\subsubsection*{Output}
\begin{figure}[H]
    \centering
    \includegraphics[width=0.99\textwidth]{a3_ss (13).png}
    \caption{Output of the Command}
\end{figure}


\subsection{Running Nmap to scan for vulnerabilities}

\subsubsection*{Syntax}
\begin{verbatim}
$ nmap -sV <target_ip> -Pn
\end{verbatim}

\subsubsection*{Command}
\begin{verbatim}
$ nmap -sV 192.168.114.1 -Pn
\end{verbatim}

\subsubsection*{Purpose}
This command is used to perform a vulnerability scan on the target host with the IP address using nmap scripts.
\subsubsection*{Output}
\begin{figure}[H]
    \centering
    \includegraphics[width=0.99\textwidth]{a3_ss (16).png}
    \caption{Output of the Command}
\end{figure}
\subsection{Searching for versions on metasploit}

\subsubsection*{Syntax}
\begin{verbatim}
$ search <service_name>
\end{verbatim}

\subsubsection*{Command}
\begin{verbatim}
$ search <different services>
\end{verbatim}

\subsubsection*{Purpose}
This command is used to search for available exploits related to the specified service or software version on Metasploit.
\subsubsection*{Output}
\begin{figure}[H]
    \centering
    \includegraphics[width=0.99\textwidth]{a3_ss (17).png}
    \caption{Output of the Command}
\end{figure}
\subsection{Searching for exploits on metasploit}

\begin{figure}[H]
    \centering
    \includegraphics[width=0.99\textwidth]{a3_ss (18).png}
    \caption{Output of the Command}
\end{figure}

\begin{figure}[H]
    \centering
    \includegraphics[width=0.99\textwidth]{a3_ss (19).png}
    \caption{Output of the Command}
\end{figure}
\subsection{Using an Exploit}

\subsubsection*{Syntax}
\begin{verbatim}
$ use <exploit_name>
\end{verbatim}

\subsubsection*{Command}
\begin{verbatim}
$ use 13
\end{verbatim}

\subsubsection*{Purpose}
This command is used to select an exploit with the ID 13 for further configuration and execution.
\subsubsection*{Output}
\begin{figure}[H]
    \centering
    \includegraphics[width=0.99\textwidth]{a3_ss (20).png}
    \caption{Output of the Command}
\end{figure}
\subsection{Checking options}

\subsubsection*{Syntax}
\begin{verbatim}
$ show options
\end{verbatim}

\subsubsection*{Command}
\begin{verbatim}
$ show options
\end{verbatim}

\subsubsection*{Purpose}
This command will display the available options and their current values for the selected exploit.
\subsubsection*{Output}
\begin{figure}[H]
    \centering
    \includegraphics[width=0.99\textwidth]{a3_ss (21).png}
    \caption{Output of the Command}
\end{figure}
\subsection{Setting the RHOST for the Exploit}

\subsubsection*{Syntax}
\begin{verbatim}
$ set rhosts <target_ip>
\end{verbatim}

\subsubsection*{Command}
\begin{verbatim}
$ set rhosts 192.168.114.1
\end{verbatim}

\subsubsection*{Purpose}
This command is used to set the target host IP address for the selected exploit.
\subsubsection*{Output}
\begin{figure}[H]
    \centering
    \includegraphics[width=0.99\textwidth]{a3_ss (22).png}
    \caption{Output of the Command}
\end{figure}
\subsection{Running the Exploit (Failed)}

\subsubsection*{Syntax}
\begin{verbatim}
$ exploit
\end{verbatim}

\subsubsection*{Command}
\begin{verbatim}
$ exploit
\end{verbatim}

\subsubsection*{Purpose}
This command will execute the selected exploit against the target host with the specified configuration.
\subsubsection*{Output}
\begin{figure}[H]
    \centering
    \includegraphics[width=0.99\textwidth]{a3_ss (23).png}
    \caption{Output of the Command}
\end{figure}
\subsection{Trying another Exploit (karaf password disclosure)}

\subsubsection*{Syntax}
\begin{verbatim}
$ use <exploit_name>
\end{verbatim}

\subsubsection*{Command}
\begin{verbatim}
$ use 4
\end{verbatim}

\subsubsection*{Purpose}
This command is used to select an exploit with the ID 4 for further configuration and execution.
\subsubsection*{Output}
\begin{figure}[H]
    \centering
    \includegraphics[width=0.99\textwidth]{a3_ss (28).png}
    \caption{To get port number}
\end{figure}
\begin{figure}[H]
    \centering
    \includegraphics[width=0.99\textwidth]{a3_ss (29).png}
    \caption{Output of the Command}
\end{figure}
\begin{figure}[H]
    \centering
    \includegraphics[width=0.99\textwidth]{a3_ss (24).png}
    \caption{Output of the Command}
\end{figure}
\subsection{Setting the RHOST for the Exploit (karaf password disclosure)}

\subsubsection*{Syntax}
\begin{verbatim}
$ set rhosts <target_ip>
\end{verbatim}

\subsubsection*{Command}
\begin{verbatim}
$ set rhosts 192.168.114.1
\end{verbatim}

\subsubsection*{Purpose}
This command is used to set the target host IP address for the selected exploit.

\subsubsection*{Output}

\begin{figure}[H]
    \centering
    \includegraphics[width=0.99\textwidth]{a3_ss (25).png}
    \caption{Output of the Command}
\end{figure}
\subsection{Successful password disclosure exploit}

\subsubsection*{Syntax}
\begin{verbatim}
$ exploit
\end{verbatim}

\subsubsection*{Command}
\begin{verbatim}
$ exploit
\end{verbatim}

\subsubsection*{Purpose}
This command will execute the selected exploit against the target host with the specified configuration.
\subsubsection*{Output}
\begin{figure}[H]
    \centering
    \includegraphics[width=0.99\textwidth]{a3_ss (27).png}
    \caption{Creating the userpass file}
\end{figure}
\begin{figure}[H]
    \centering
    \includegraphics[width=0.99\textwidth]{a3_ss (26).png}
    \caption{Output of the Command}
\end{figure}


\section{Platform}
\textbf{Operating System}: Arch Linux X8664 \\
\textbf{IDEs or Text Editors Used}: Visual Studio Code\\
% \textbf{Compilers or Interpreters}: Python 3.10.1\\

% \section{Code}
% \lstinputlisting[language=Python, caption="DSA Signature Validity using PyCrypto Library"]{../Programs/Assignment_7/dsa using lib.py}
\section{FAQs}
\begin{enumerate}
    \item Explanation of vulnerabilities:
          \begin{itemize}
              \item CSRF (Cross-Site Request Forgery) - Exploits a user's authenticated session to perform unauthorized actions.
              \item SSRF (Server-Side Request Forgery) - Allows attackers to send crafted requests from the server, potentially accessing internal resources.
              \item XSS (Cross-Site Scripting) - Injects malicious scripts into web pages viewed by other users, leading to data theft or manipulation.
              \item DOM-based XSS - Occurs when client-side scripts manipulate the DOM in an unsafe way, leading to XSS vulnerabilities.
              \item Slowloris Attack - Exploits server-side vulnerabilities by keeping many connections open simultaneously, exhausting resources.
              \item SQL Injection - Exploits vulnerabilities in database query interfaces to execute arbitrary SQL code.
              \item Remote Code Execution (RCE) - Allows attackers to execute arbitrary code on a target system remotely.
              \item Directory Traversal - Exploits insecure file path handling to access files outside of the intended directory.
          \end{itemize}

    \item Usage of Metasploit Framework:
          \begin{itemize}
              \item Metasploit Framework is used for penetration testing and exploit development, allowing testers to identify and exploit vulnerabilities in target systems.
          \end{itemize}

    \item Modules supported by Metasploit Framework:
          \begin{itemize}
              \item Metasploit Framework supports various modules for exploit development, post-exploitation, payload generation, auxiliary scanning, and evasion techniques.
          \end{itemize}
\end{enumerate}

\section{Conclusion}
In this Assignment, we explored the importance of gathering network information from an attacker's perspective using tools like nmap and Metasploit. We discussed the uses, advantages, and disadvantages of these tools and demonstrated their implementation through practical examples. By understanding the capabilities and limitations of these tools, security professionals can enhance their network reconnaissance and penetration testing activities to identify vulnerabilities and improve defense strategies.
\clearpage

\pagebreak


\end{document}