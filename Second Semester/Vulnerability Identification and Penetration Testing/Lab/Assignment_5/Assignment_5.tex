% This is a Basic Assignment Paper but with like Code and stuff allowed in it, there is also url, hyperlinks from contents included. 

\documentclass[11pt]{article}

% Preamble

\usepackage[margin=1in]{geometry}
\usepackage{amsfonts, amsmath, amssymb, amsthm}
\usepackage{fancyhdr, float, graphicx}
\usepackage[utf8]{inputenc} % Required for inputting international characters
\usepackage[T1]{fontenc} % Output font encoding for international characters
% \usepackage{fouriernc} % Use the New Century Schoolbook font
\usepackage[nottoc, notlot, notlof]{tocbibind}
\usepackage{listings}
\usepackage{xcolor}
\usepackage{blindtext}
\usepackage{hyperref}
\definecolor{codepurple}{rgb}{0.58,0,0.82}
\hypersetup{
    colorlinks=true,
    linkcolor=black,
    filecolor=black,      
    urlcolor=codepurple,
    pdfpagemode=FullScreen,
    }

\definecolor{codegreen}{rgb}{0,0.6,0}
\definecolor{codegray}{rgb}{0.5,0.5,0.5}
\definecolor{backcolour}{rgb}{0.95,0.95,0.92}

\lstdefinestyle{mystyle}{
    backgroundcolor=\color{backcolour},   
    commentstyle=\color{codegreen},
    keywordstyle=\color{magenta},
    numberstyle=\tiny\color{codegray},
    stringstyle=\color{codepurple},
    basicstyle=\ttfamily\footnotesize,
    breakatwhitespace=false,         
    breaklines=true,                 
    captionpos=b,                    
    keepspaces=true,                 
    numbers=left,                    
    numbersep=5pt,                  
    showspaces=false,                
    showstringspaces=false,
    showtabs=false,                  
    tabsize=2
}

\lstset{style=mystyle}

% Header and Footer
\pagestyle{fancy}
\fancyhead{}
\fancyfoot{}
\fancyhead[L]{\textit{\Large{Vulnerability Identification and Penetration Testing}}}
\fancyhead[R]{\textit{Krishnaraj T}}
\fancyfoot[C]{\thepage}
\renewcommand{\footrulewidth}{1pt}
\newtheorem{thm}{Theorem}
\newtheorem{dfn}[thm]{Definition}


% Other Doc Editing
% \parindent 0ex
%\renewcommand{\baselinestretch}{1.5}

\begin{document}

\begin{titlepage}
    \centering

    %---------------------------NAMES-------------------------------

    \huge\textsc{
        MIT World Peace University
    }\\

    \vspace{0.75\baselineskip} % space after Uni Name

    \LARGE{
        Vulnerability Identification and Penetration Testing\\
        Third Year B. Tech, Semester 6
    }

    \vfill % space after Sub Name

    %--------------------------TITLE-------------------------------

    \rule{\textwidth}{1.6pt}\vspace*{-\baselineskip}\vspace*{2pt}
    \rule{\textwidth}{0.6pt}
    \vspace{0.75\baselineskip} % Whitespace above the title



    \huge{\textsc{
            Creating protocol-specific target lists for vulnerability discovery
        }} \\



    \vspace{0.5\baselineskip} % Whitespace below the title
    \rule{\textwidth}{0.6pt}\vspace*{-\baselineskip}\vspace*{2.8pt}
    \rule{\textwidth}{1.6pt}

    \vspace{1\baselineskip} % Whitespace after the title block

    %--------------------------SUBTITLE --------------------------	

    \LARGE\textsc{
        Assignment 5
    } % Subtitle or further description
    \vfill

    %--------------------------AUTHOR-------------------------------

    Prepared By
    \vspace{0.5\baselineskip} % Whitespace before the editors

    \Large{
        Krishnaraj Thadesar \\
        Cyber Security and Forensics\\
        Batch A1, PA 10
    }


    \vspace{0.5\baselineskip} % Whitespace below the editor list
    \today

\end{titlepage}


\tableofcontents
\thispagestyle{empty}
\clearpage

\setcounter{page}{1}

\section{Aim}
To create protocol-specific target lists for vulnerability discovery.
\section{Objectives}
\begin{enumerate}
    \item To understand the concept of protocol-specific target lists.
    \item To create a target list for a specific protocol.
    \item To use Grep to search for vulnerabilities in the target list.
\end{enumerate}

\section{Theory}
\subsection{grep}

\subsection{Uses}
\begin{itemize}
    \item Searching for patterns or text within files.
    \item Filtering command output based on specific criteria.
    \item Extracting relevant information from large datasets.
    \item Automating tasks through scripting and regular expressions.
\end{itemize}

\subsection{Advantages}
\begin{itemize}
    \item Fast and efficient searching capability.
    \item Support for regular expressions, enabling complex pattern matching.
    \item Versatility in handling various file formats and input sources.
    \item Integration with other command-line tools and scripting languages.
\end{itemize}

\subsection{Disadvantages}
\begin{itemize}
    \item Limited graphical user interface (GUI), requiring familiarity with command-line usage.
    \item Learning curve for mastering regular expressions and advanced search techniques.
    \item May produce overwhelming output in certain scenarios, requiring careful filtering.
    \item Reliance on textual data, making it less suitable for analyzing binary or structured data.
\end{itemize}


\section{Implementation}

\subsection{Greppable Scan for Port 22 on Target IP Address}

\subsubsection*{Syntax}
\begin{verbatim}
$ nmap -p <port> -oG <target>
\end{verbatim}

\subsubsection*{Command}
\begin{verbatim}
$ nmap -p 22 -oG 192.168.114.1 
\end{verbatim}

\subsubsection*{Purpose}
This command performs a grepable scan specifically for port 22 on the target IP address
\subsubsection*{Output}
\begin{figure}[H]
    \centering
    \includegraphics[width=0.99\textwidth]{Screenshot 2024-04-20 141054.png}
    \caption{Output of the command}
    \label{fig:1}
\end{figure}


\section{Platform}
\textbf{Operating System}: Arch Linux X8664 \\
\textbf{IDEs or Text Editors Used}: Visual Studio Code\\
% \textbf{Compilers or Interpreters}: Python 3.10.1\\

% \section{Code}
% \lstinputlisting[language=Python, caption="DSA Signature Validity using PyCrypto Library"]{../Programs/Assignment_7/dsa using lib.py}

\section{FAQs}
\begin{enumerate}
    \item \textbf{What the different scanning flags are there in nmap? Explain them.}
          \begin{itemize}
              \item \textbf{-sS (TCP SYN scan)}: Stealthy scan technique that sends SYN packets to target ports to determine their state.
              \item \textbf{-sT (TCP connect scan)}: Establishes full TCP connections to target ports, suitable for systems where SYN packets are blocked.
              \item \textbf{-sU (UDP scan)}: Scans for open UDP ports by sending UDP packets to target ports and analyzing responses.
              \item \textbf{-sF (TCP FIN scan)}: Sends TCP FIN packets to target ports to determine their state, useful for evading detection by firewalls.
              \item \textbf{-sX (Xmas scan)}: Sends TCP packets with FIN, PSH, and URG flags set to target ports, used for detecting firewall filtering rules.
          \end{itemize}

    \item \textbf{What is a grepable format?}
          \begin{itemize}
              \item Grepable format: A machine-readable format produced by Nmap that allows for easy parsing and filtering of scan results using command-line tools like grep.
              \item Each line in the output corresponds to a single host or port, with fields separated by tabs for easy processing.
          \end{itemize}

    \item \textbf{Write a command to perform a grepable scan for port number 80 to check status as closed.}
          \begin{itemize}
              \item \texttt{nmap -p80 --open -oG scanresults.txt <target>} (Replace <target> with the target IP address or hostname)
              \item This command performs a grepable scan specifically for port 80 to check if it is closed, saving the results to a file named "scanresults.txt".
          \end{itemize}
\end{enumerate}


\section{Conclusion}
In this assignment, we learned about creating protocol-specific target lists for vulnerability discovery. We explored the concept of grepable formats and how they can be used to search for vulnerabilities in target lists. By creating custom target lists and using grep to filter the results, we can identify potential security issues and take appropriate actions to mitigate them.
\clearpage

\pagebreak


\end{document}