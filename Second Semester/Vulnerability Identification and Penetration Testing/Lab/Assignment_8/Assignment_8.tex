% This is a Basic Assignment Paper but with like Code and stuff allowed in it, there is also url, hyperlinks from contents included. 

\documentclass[11pt]{article}

% Preamble

\usepackage[margin=1in]{geometry}
\usepackage{amsfonts, amsmath, amssymb, amsthm}
\usepackage{fancyhdr, float, graphicx}
\usepackage[utf8]{inputenc} % Required for inputting international characters
\usepackage[T1]{fontenc} % Output font encoding for international characters
% \usepackage{fouriernc} % Use the New Century Schoolbook font
\usepackage[nottoc, notlot, notlof]{tocbibind}
\usepackage{listings}
\usepackage{xcolor}
\usepackage{blindtext}
\usepackage{hyperref}
\definecolor{codepurple}{rgb}{0.58,0,0.82}
\hypersetup{
    colorlinks=true,
    linkcolor=black,
    filecolor=black,      
    urlcolor=codepurple,
    pdfpagemode=FullScreen,
    }

\definecolor{codegreen}{rgb}{0,0.6,0}
\definecolor{codegray}{rgb}{0.5,0.5,0.5}
\definecolor{backcolour}{rgb}{0.95,0.95,0.92}

\lstdefinestyle{mystyle}{
    backgroundcolor=\color{backcolour},   
    commentstyle=\color{codegreen},
    keywordstyle=\color{magenta},
    numberstyle=\tiny\color{codegray},
    stringstyle=\color{codepurple},
    basicstyle=\ttfamily\footnotesize,
    breakatwhitespace=false,         
    breaklines=true,                 
    captionpos=b,                    
    keepspaces=true,                 
    numbers=left,                    
    numbersep=5pt,                  
    showspaces=false,                
    showstringspaces=false,
    showtabs=false,                  
    tabsize=2
}

\lstset{style=mystyle}

% Header and Footer
\pagestyle{fancy}
\fancyhead{}
\fancyfoot{}
\fancyhead[L]{\textit{\Large{Vulnerability Identification and Penetration Testing}}}
\fancyhead[R]{\textit{Krishnaraj T}}
\fancyfoot[C]{\thepage}
\renewcommand{\footrulewidth}{1pt}
\newtheorem{thm}{Theorem}
\newtheorem{dfn}[thm]{Definition}


% Other Doc Editing
% \parindent 0ex
%\renewcommand{\baselinestretch}{1.5}

\begin{document}

\begin{titlepage}
    \centering

    %---------------------------NAMES-------------------------------

    \huge\textsc{
        MIT World Peace University
    }\\

    \vspace{0.75\baselineskip} % space after Uni Name

    \LARGE{
        Vulnerability Identification and Penetration Testing\\
        Third Year B. Tech, Semester 6
    }

    \vfill % space after Sub Name

    %--------------------------TITLE-------------------------------

    \rule{\textwidth}{1.6pt}\vspace*{-\baselineskip}\vspace*{2pt}
    \rule{\textwidth}{0.6pt}
    \vspace{0.75\baselineskip} % Whitespace above the title



    \huge{\textsc{
            IT Audit, Malware Analysis and Vulnerability Assessment for Report Generation
        }} \\



    \vspace{0.5\baselineskip} % Whitespace below the title
    \rule{\textwidth}{0.6pt}\vspace*{-\baselineskip}\vspace*{2.8pt}
    \rule{\textwidth}{1.6pt}

    \vspace{1\baselineskip} % Whitespace after the title block

    %--------------------------SUBTITLE --------------------------	

    \LARGE\textsc{
        Assignment 8
    } % Subtitle or further description
    \vfill

    %--------------------------AUTHOR-------------------------------

    Prepared By
    \vspace{0.5\baselineskip} % Whitespace before the editors

    \Large{
        Krishnaraj Thadesar \\
        Cyber Security and Forensics\\
        Batch A1, PA 10
    }


    \vspace{0.5\baselineskip} % Whitespace below the editor list
    \today

\end{titlepage}


\tableofcontents
\thispagestyle{empty}
\clearpage

\setcounter{page}{1}

\section{Aim}
To perform IT Audit, Malware Analysis and Vulnerability Assessment for Report Generation.
\section{Objectives}
\begin{enumerate}
    \item To understand the concept of IT Audit, Malware Analysis and Vulnerability Assessment.
    \item To perform IT Audit, Malware Analysis and Vulnerability Assessment.
\end{enumerate}

\section{Theory}
\subsection{IT Audit}
An IT audit is the examination and evaluation of an organization's information technology infrastructure, policies and operations. Information technology audits determine whether IT controls protect corporate assets, ensure data integrity and are aligned with the business's overall goals.

\subsection{Malware Analysis}

Malware analysis is the process of determining the functionality, origin and potential impact of a given malware sample such as a virus, worm, trojan horse, rootkit, or backdoor. Malware analysts typically do not have access to the source code of the malware, which means they must extract as much information as possible from the compiled code.

\subsection{Vulnerability Assessment}
A vulnerability assessment is the process of defining, identifying, classifying and prioritizing vulnerabilities in computer systems, applications and network infrastructures and providing the organization doing the assessment with the necessary knowledge, awareness and risk background to understand the threats to its environment and react appropriately.

\subsection{IT Audit Tools}
\begin{enumerate}
    \item Nikto: Web server scanner for identifying vulnerabilities and misconfigurations.
    \item Wireshark: Network protocol analyzer for capturing and analyzing network traffic.
    \item Nessus: Vulnerability scanner for identifying security vulnerabilities in networks, systems, and applications.
    \item OpenVAS: Open-source vulnerability scanner for detecting security issues in networks and web applications.
    \item Nexpose: Vulnerability management solution for discovering, assessing, and prioritizing vulnerabilities.
    \item Retina: Vulnerability management and assessment tool for identifying and remediating security risks.
    \item Qualys: Cloud-based security and compliance platform offering vulnerability management and threat protection.
    \item SAINT: Security assessment and vulnerability management tool for identifying and mitigating security risks.
    \item Core Impact: Penetration testing tool for simulating real-world attack scenarios and identifying vulnerabilities.
    \item Metasploit: Penetration testing framework for exploiting security vulnerabilities in target systems.
    \item Nmap: Network scanner for discovering hosts and services on a network and identifying potential security issues.
\end{enumerate}


\subsection{Nikto}

\subsection{Uses}
\begin{itemize}
    \item Performs comprehensive web server scanning to identify potential vulnerabilities and misconfigurations.
    \item Checks for common security issues such as outdated software, insecure server configurations, and known vulnerabilities.
    \item Provides detailed reports and recommendations for improving web server security posture.
\end{itemize}

\subsection{Advantages}
\begin{itemize}
    \item Free and open-source tool, widely used in the cybersecurity community.
    \item Supports scanning of multiple web servers and protocols, including HTTP and HTTPS.
    \item Regularly updated with new checks and vulnerability signatures to detect the latest threats.
\end{itemize}

\subsection{Disadvantages}
\begin{itemize}
    \item May produce false positives or miss certain vulnerabilities in complex web applications.
    \item Limited to web server scanning and may not cover all aspects of network security assessment.
    \item Requires careful interpretation of scan results and manual verification of findings.
\end{itemize}

\subsection{Wireshark}

\subsection{Uses}
\begin{itemize}
    \item Captures and analyzes network traffic to troubleshoot network issues and investigate security incidents.
    \item Provides real-time monitoring of network packets, allowing for deep inspection of protocols and traffic patterns.
    \item Supports various protocols and data types, including TCP, UDP, HTTP, and VoIP.
\end{itemize}

\subsection{Advantages}
\begin{itemize}
    \item Free and open-source packet analyzer, available for multiple platforms including Windows, macOS, and Linux.
    \item User-friendly interface with powerful filtering and analysis capabilities for both novice and experienced users.
    \item Offers extensive protocol support and customizable display options for detailed packet analysis.
\end{itemize}

\subsection{Disadvantages}
\begin{itemize}
    \item High learning curve for beginners due to the complexity of network protocols and packet analysis techniques.
    \item May capture sensitive information if not configured properly, leading to privacy concerns.
    \item Requires adequate system resources for capturing and processing large volumes of network traffic.
\end{itemize}

\section{Implementation}

\subsection{}

\subsubsection*{Syntax}
\begin{verbatim}
$ nikto -h <target> -Tuning x
\end{verbatim}

\subsubsection*{Command}
\begin{verbatim}
$ nikto -h www.thisislegal.com -Tuning x
\end{verbatim}

\subsubsection*{Purpose}
This command performs a web server scan on the target host "www.thisislegal.com".

Tuning Levels in Nikto:
\begin{itemize}
    \item 0: File Upload
    \item 1: Interesting File / Seen in logs
    \item 2: Misconfiguration / Default File
    \item 3: Information Disclosure
    \item 4: Injection (XSS/Script/HTML)
    \item 5: Remote File Retrieval - Inside Web Root
    \item 6: Denial of Service
    \item 7: Remote File Retrieval - Server Wide
    \item 8: Command Execution / Remote Shell
    \item 9: SQL Injection
\end{itemize}

Meaning of Tuning Level x, is that it will scan for all the vulnerabilities from level 0 to level 9.

\subsubsection*{Output}
\begin{figure}[H]
    \centering
    \includegraphics[width=0.99\textwidth]{assignment 8.png}
    \caption{Output of the command}
\end{figure}

\subsection{Running Nikto with Tuning Level 6}

\subsubsection*{Syntax}
\begin{verbatim}
$ nikto -h <target> -Tuning x 6
\end{verbatim}

\subsubsection*{Command}
\begin{verbatim}
$ nikto -Tuning x 6 -h example.com
\end{verbatim}

\subsubsection*{Purpose}
This command will perform a web server scan on the target host "example.com" with all tuning levels except for 6 (Denial of Service).
\subsubsection*{Output}
\begin{figure}[H]
    \centering
    \includegraphics[width=0.99\textwidth]{assignment 8 (5).png}
    \caption{Output of the command}
\end{figure}

\subsection{}

\subsubsection*{Syntax}
\begin{verbatim}
$ nikto -Tuning <number> -h <target>
\end{verbatim}

\subsubsection*{Command}
\begin{verbatim}
$ nikto -Tuning 9 -h example.com
\end{verbatim}

\subsubsection*{Purpose}
This command will perform a web server scan on the target host "example.com" with tuning level 9 (SQL Injection).
\subsubsection*{Output}
\begin{figure}[H]
    \centering
    \includegraphics[width=0.99\textwidth]{assignment 8 (6).png}
    \caption{Output of the command}
\end{figure}

\subsection{}

\subsubsection*{Syntax}
\begin{verbatim}
$ nikto -h <target> -o <output> -F txt
\end{verbatim}

\subsubsection*{Command}
\begin{verbatim}
$ nikto -h www.thisislegal.com -o outputnikto -F txt
\end{verbatim}

\subsubsection*{Purpose}
This command performs a web server scan on the target host "www.thisislegal.com" and saves the output in a text file named "outputnikto".
\subsubsection*{Output}
\begin{figure}[H]
    \centering
    \includegraphics[width=0.99\textwidth]{assignment 8 (7).png}
    \caption{Output of the command}
\end{figure}

\subsection{outputnikto file}

\begin{figure}[H]
    \centering
    \includegraphics[width=0.99\textwidth]{assignment 8 (8).png}
    \caption{Output of the command}
\end{figure}


\subsection{Using Wireshark to Capture Packet and get password details. }

\begin{figure}[H]
    \centering
    \includegraphics[width=0.99\textwidth]{assignment 8 (1).png}
    \caption{Filtering by target ip}
\end{figure}
\begin{figure}[H]
    \centering
    \includegraphics[width=0.99\textwidth]{assignment 8 (3).png}
    \caption{Login Request from ip}
\end{figure}

\begin{figure}[H]
    \centering
    \includegraphics[width=0.99\textwidth]{assignment 8 (2).png}
    \caption{Viewing Http packets to get username and password}
\end{figure}


\section{Platform}
\textbf{Operating System}: Arch Linux X8664 \\
\textbf{IDEs or Text Editors Used}: Visual Studio Code\\
% \textbf{Compilers or Interpreters}: Python 3.10.1\\

% \section{Code}
% \lstinputlisting[language=Python, caption="DSA Signature Validity using PyCrypto Library"]{../Programs/Assignment_7/dsa using lib.py}

\section{FAQs}
\begin{enumerate}
    \item \textbf{Detailing of output file “outputnikto”. (EXPLAIN ALL VULNERABILITIES):}

          The output file "outputnikto" contains the results of a scan performed by the Nikto web server scanner on the target host "www.thisislegal.com" on port 80. Below are explanations for each vulnerability identified:

          \begin{itemize}
              \item The anti-clickjacking X-Frame-Options header is not present.
                    \begin{itemize}
                        \item This vulnerability exposes the website to clickjacking attacks, where an attacker can trick users into clicking on malicious elements by overlaying them on legitimate web content.
                    \end{itemize}

              \item The X-Content-Type-Options header is not set.
                    \begin{itemize}
                        \item This vulnerability may allow attackers to manipulate the content type of the website, potentially leading to content spoofing or other attacks.
                    \end{itemize}

              \item PHP/8.0.7 appears to be outdated.
                    \begin{itemize}
                        \item Outdated software versions may contain known vulnerabilities that could be exploited by attackers to compromise the web server.
                    \end{itemize}

              \item Apache/2.4.48 appears to be outdated.
                    \begin{itemize}
                        \item Similarly, outdated versions of web server software like Apache may contain vulnerabilities that could be exploited by attackers.
                    \end{itemize}

              \item OpenSSL/1.1.1k appears to be outdated.
                    \begin{itemize}
                        \item Outdated versions of OpenSSL may contain security vulnerabilities that could be exploited to intercept or manipulate encrypted communications.
                    \end{itemize}

              \item HTTP TRACE method is active which suggests the host is vulnerable to XST.
                    \begin{itemize}
                        \item The HTTP TRACE method can be exploited in cross-site tracing (XST) attacks to steal sensitive information from users' cookies.
                    \end{itemize}

              \item Retrieved x-powered-by header: PHP/8.0.7.
                    \begin{itemize}
                        \item Revealing server details like the PHP version in HTTP headers can provide attackers with information to tailor their attacks.
                    \end{itemize}

              \item Cookie PHPSESSID created without the httponly flag.
                    \begin{itemize}
                        \item Cookies without the httponly flag may be accessible to client-side scripts, increasing the risk of cookie theft via cross-site scripting (XSS) attacks.
                    \end{itemize}

              \item phpMyAdmin is for managing MySQL databases, and should be protected or limited to authorized hosts.
                    \begin{itemize}
                        \item Exposing phpMyAdmin to unauthorized access may allow attackers to gain unauthorized access to the MySQL database and its contents.
                    \end{itemize}

              \item Directory indexing
          \end{itemize}

    \item \textbf{What is OSVDB. Enlist top 10 vulnerabilities supported by OSVDB:}

          OSVDB (Open Sourced Vulnerability Database) was a project dedicated to collecting and sharing information about security vulnerabilities. However, it has been discontinued, and its data is no longer maintained. Nevertheless, here are the top 10 vulnerabilities commonly supported by OSVDB:

          \begin{itemize}
              \item SQL Injection (SQLi)
              \item Cross-Site Scripting (XSS)
              \item Remote Code Execution (RCE)
              \item Directory Traversal
              \item Authentication Bypass
              \item Information Disclosure
              \item Buffer Overflow
              \item Cross-Site Request Forgery (CSRF)
              \item Command Injection
              \item Denial of Service (DoS)
          \end{itemize}

    \item \textbf{Write one page information on Wireshark (basic terminologies and working):}

          Wireshark is a network protocol analyzer used for capturing and analyzing network traffic. Below is an overview of basic terminologies and working of Wireshark:

          \begin{itemize}
              \item \textbf{Packet}: Unit of data transmitted over a network.
              \item \textbf{Protocol}: Set of rules defining how data is transmitted and received.
              \item \textbf{Capture Filter}: Criteria used to select specific packets for analysis.
              \item \textbf{Display Filter}: Criteria used to filter and display specific packets in Wireshark.

                    Wireshark captures network traffic through a process called packet capture. It then analyzes these captured packets to understand network communication. Users can apply capture and display filters to focus on relevant packets and interpret packet details such as source, destination, protocol, and payload. Wireshark offers features like live capture, packet decoding, protocol support, and export and save options for offline analysis and reporting.
          \end{itemize}
\end{enumerate}

\section{Conclusion}

In this assignment, we performed IT Audit, Malware Analysis, and Vulnerability Assessment using tools like Nikto and Wireshark. We identified potential vulnerabilities in a web server and analyzed network traffic to understand network communication. These activities are essential for maintaining the security and integrity of information systems and protecting against cyber threats.
\clearpage

\pagebreak


\end{document}