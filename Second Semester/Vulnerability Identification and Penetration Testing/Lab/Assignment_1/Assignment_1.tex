% This is a Basic Assignment Paper but with like Code and stuff allowed in it, there is also url, hyperlinks from contents included. 

\documentclass[11pt]{article}

% Preamble

\usepackage[margin=1in]{geometry}
\usepackage{amsfonts, amsmath, amssymb, amsthm}
\usepackage{fancyhdr, float, graphicx}
\usepackage[utf8]{inputenc} % Required for inputting international characters
\usepackage[T1]{fontenc} % Output font encoding for international characters
% \usepackage{fouriernc} % Use the New Century Schoolbook font
\usepackage[nottoc, notlot, notlof]{tocbibind}
\usepackage{listings}
\usepackage{xcolor}
\usepackage{blindtext}
\usepackage{hyperref}
\definecolor{codepurple}{rgb}{0.58,0,0.82}
\hypersetup{
    colorlinks=true,
    linkcolor=black,
    filecolor=black,      
    urlcolor=codepurple,
    pdfpagemode=FullScreen,
    }

\definecolor{codegreen}{rgb}{0,0.6,0}
\definecolor{codegray}{rgb}{0.5,0.5,0.5}
\definecolor{backcolour}{rgb}{0.95,0.95,0.92}

\lstdefinestyle{mystyle}{
    backgroundcolor=\color{backcolour},   
    commentstyle=\color{codegreen},
    keywordstyle=\color{magenta},
    numberstyle=\tiny\color{codegray},
    stringstyle=\color{codepurple},
    basicstyle=\ttfamily\footnotesize,
    breakatwhitespace=false,         
    breaklines=true,                 
    captionpos=b,                    
    keepspaces=true,                 
    numbers=left,                    
    numbersep=5pt,                  
    showspaces=false,                
    showstringspaces=false,
    showtabs=false,                  
    tabsize=2
}

\lstset{style=mystyle}

% Header and Footer
\pagestyle{fancy}
\fancyhead{}
\fancyfoot{}
\fancyhead[L]{\textit{\Large{Vulnerability Identification and Penetration Testing}}}
\fancyhead[R]{\textit{Krishnaraj T}}
\fancyfoot[C]{\thepage}
\renewcommand{\footrulewidth}{1pt}
\newtheorem{thm}{Theorem}
\newtheorem{dfn}[thm]{Definition}


% Other Doc Editing
% \parindent 0ex
%\renewcommand{\baselinestretch}{1.5}

\begin{document}

\begin{titlepage}
    \centering

    %---------------------------NAMES-------------------------------

    \huge\textsc{
        MIT World Peace University
    }\\

    \vspace{0.75\baselineskip} % space after Uni Name

    \LARGE{
        Vulnerability Identification and Penetration Testing\\
        Third Year B. Tech, Semester 6
    }

    \vfill % space after Sub Name

    %--------------------------TITLE-------------------------------

    \rule{\textwidth}{1.6pt}\vspace*{-\baselineskip}\vspace*{2pt}
    \rule{\textwidth}{0.6pt}
    \vspace{0.75\baselineskip} % Whitespace above the title



    \huge{\textsc{
            BruteForce password Attack using Burpsuite
        }} \\



    \vspace{0.5\baselineskip} % Whitespace below the title
    \rule{\textwidth}{0.6pt}\vspace*{-\baselineskip}\vspace*{2.8pt}
    \rule{\textwidth}{1.6pt}

    \vspace{1\baselineskip} % Whitespace after the title block

    %--------------------------SUBTITLE --------------------------	

    \LARGE\textsc{
        Assignment 1
    } % Subtitle or further description
    \vfill

    %--------------------------AUTHOR-------------------------------

    Prepared By
    \vspace{0.5\baselineskip} % Whitespace before the editors

    \Large{
        Krishnaraj Thadesar \\
        Cyber Security and Forensics\\
        Batch A1, PA 10
    }


    \vspace{0.5\baselineskip} % Whitespace below the editor list
    \today

\end{titlepage}


\tableofcontents
\thispagestyle{empty}
\clearpage

\setcounter{page}{1}

\section{Aim}
To perform a brute force attack using Burpsuite.

\section{Objectives}
\begin{enumerate}
    \item To learn about the Burpsuite tool.
    \item To perform a brute force attack using Burpsuite.
    \item To understand the concept of brute force attacks.
\end{enumerate}

\section{Theory}

\subsection{Burpsuite}
Burp Suite is a leading cybersecurity tool that provides web application security testing. It is developed by PortSwigger, an IT security company based in the UK. Burp Suite is widely used by security professionals to perform various security testing tasks, such as scanning, crawling, and testing web applications for vulnerabilities.

\subsection{Advantages}
\begin{itemize}
    \item BurpSuite provides a user-friendly interface, making it accessible to both beginners and experienced users.
    \item It offers a wide range of features for web application testing, including proxy, scanner, and intruder, among others.
    \item BurpSuite supports various platforms, including Windows, macOS, and Linux, enhancing its versatility and usability.
    \item The tool allows for easy customization and extension through the use of plugins, enabling users to tailor their testing environment according to their needs.
    \item BurpSuite provides detailed reports and logs, facilitating thorough analysis and documentation of test results.
\end{itemize}

\subsection{Disadvantages}
\begin{itemize}
    \item While BurpSuite offers a free version, its full set of features is only available in the paid version, which may be a barrier for some users.
    \item Due to its extensive functionality, BurpSuite has a steep learning curve, requiring time and effort to master all of its features effectively.
    \item The tool's reliance on Java may lead to performance issues, especially when dealing with large-scale or complex testing scenarios.
    \item BurpSuite's automated scanning capabilities may produce false positives or miss certain vulnerabilities, requiring manual verification by users.
    \item The graphical user interface (GUI) of BurpSuite, while intuitive for many, may not be suitable for users who prefer command-line interfaces or scripted testing approaches.
\end{itemize}


\section{Implementation}

\subsection{Launching Burpsuite}

% \subsubsection*{Syntax}
% \begin{verbatim}
% $
% \end{verbatim}

% \subsubsection*{Command}
% \begin{verbatim}
% $
% \end{verbatim}

% \subsubsection*{Purpose}

% \subsubsection*{Output}
\begin{figure}[H]
    \centering
    \includegraphics[width=0.95\textwidth]{burpsuite (1).png}
    \caption{Opening Burpsuite}
    \label{fig:1}
\end{figure}
\subsection{Opening Browser for Locally hosted Django Project}

% \subsubsection*{Syntax}
% \begin{verbatim}
% $
% \end{verbatim}

% \subsubsection*{Command}
% \begin{verbatim}
% $
% \end{verbatim}

% \subsubsection*{Purpose}

% \subsubsection*{Output}
\begin{figure}[H]
    \centering
    \includegraphics[width=0.95\textwidth]{burpsuite (2).png}
    \caption{Opening Burpsuite}
    \label{fig:1}
\end{figure}
\subsection{Entering Credentials and Interecepting}

% \subsubsection*{Syntax}
% \begin{verbatim}
% $
% \end{verbatim}

% \subsubsection*{Command}
% \begin{verbatim}
% $
% \end{verbatim}

% \subsubsection*{Purpose}

% \subsubsection*{Output}
\begin{figure}[H]
    \centering
    \includegraphics[width=0.95\textwidth]{burpsuite (3).png}
    \caption{Opening Burpsuite}
    \label{fig:1}
\end{figure}
\subsection{Setting Payloads by selecting Username and Password}

% \subsubsection*{Syntax}
% \begin{verbatim}
% $
% \end{verbatim}

% \subsubsection*{Command}
% \begin{verbatim}
% $
% \end{verbatim}

% \subsubsection*{Purpose}

% \subsubsection*{Output}
\begin{figure}[H]
    \centering
    \includegraphics[width=0.95\textwidth]{burpsuite (4).png}
    \caption{Opening Burpsuite}
    \label{fig:1}
\end{figure}
\subsection{Choosing Cluster Bomb}

% \subsubsection*{Syntax}
% \begin{verbatim}
% $
% \end{verbatim}

% \subsubsection*{Command}
% \begin{verbatim}
% $
% \end{verbatim}

% \subsubsection*{Purpose}

% \subsubsection*{Output}
\begin{figure}[H]
    \centering
    \includegraphics[width=0.95\textwidth]{burpsuite (5).png}
    \caption{Opening Burpsuite}
    \label{fig:1}
\end{figure}
\subsection{Setting Password Brute Force Attack lists}

% \subsubsection*{Syntax}
% \begin{verbatim}
% $
% \end{verbatim}

% \subsubsection*{Command}
% \begin{verbatim}
% $
% \end{verbatim}

% \subsubsection*{Purpose}

% \subsubsection*{Output}
\begin{figure}[H]
    \centering
    \includegraphics[width=0.95\textwidth]{burpsuite (6).png}
    \caption{Opening Burpsuite}
    \label{fig:1}
\end{figure}
\subsection{Setting Password Brute Force Attack lists}

% \subsubsection*{Syntax}
% \begin{verbatim}
% $
% \end{verbatim}

% \subsubsection*{Command}
% \begin{verbatim}
% $
% \end{verbatim}

% \subsubsection*{Purpose}

% \subsubsection*{Output}
\begin{figure}[H]
    \centering
    \includegraphics[width=0.95\textwidth]{burpsuite (7).png}
    \caption{Opening Burpsuite}
    \label{fig:1}
\end{figure}
\subsection{Setting Usernames list from brute forcer}

% \subsubsection*{Syntax}
% \begin{verbatim}
% $
% \end{verbatim}

% \subsubsection*{Command}
% \begin{verbatim}
% $
% \end{verbatim}

% \subsubsection*{Purpose}

% \subsubsection*{Output}
\begin{figure}[H]
    \centering
    \includegraphics[width=0.95\textwidth]{burpsuite (8).png}
    \caption{Opening Burpsuite}
    \label{fig:1}
\end{figure}
\subsection{Executing Attack}

% \subsubsection*{Syntax}
% \begin{verbatim}
% $
% \end{verbatim}

% \subsubsection*{Command}
% \begin{verbatim}
% $
% \end{verbatim}

% \subsubsection*{Purpose}

% \subsubsection*{Output}
\begin{figure}[H]
    \centering
    \includegraphics[width=0.95\textwidth]{burpsuite (9).png}
    \caption{Opening Burpsuite}
    \label{fig:1}
\end{figure}
\subsection{Executing Attack, as seen on django server console logs}

% \subsubsection*{Syntax}
% \begin{verbatim}
% $
% \end{verbatim}

% \subsubsection*{Command}
% \begin{verbatim}
% $
% \end{verbatim}

% \subsubsection*{Purpose}

% \subsubsection*{Output}
\begin{figure}[H]
    \centering
    \includegraphics[width=0.95\textwidth]{burpsuite (10).png}
    \caption{Opening Burpsuite}
    \label{fig:1}
\end{figure}
\subsection{Finding Correct password by noticing change in length}

% \subsubsection*{Syntax}
% \begin{verbatim}
% $
% \end{verbatim}

% \subsubsection*{Command}
% \begin{verbatim}
% $
% \end{verbatim}

% \subsubsection*{Purpose}

% \subsubsection*{Output}
\begin{figure}[H]
    \centering
    \includegraphics[width=0.95\textwidth]{burpsuite (11).png}
    \caption{Opening Burpsuite}
    \label{fig:1}
\end{figure}
\subsection{Correct password found}

% \subsubsection*{Syntax}
% \begin{verbatim}
% $
% \end{verbatim}

% \subsubsection*{Command}
% \begin{verbatim}
% $
% \end{verbatim}

% \subsubsection*{Purpose}

% \subsubsection*{Output}
\begin{figure}[H]
    \centering
    \includegraphics[width=0.95\textwidth]{burpsuite (12).png}
    \caption{Opening Burpsuite}
    \label{fig:1}
\end{figure}


\section{Platform}
\textbf{Operating System}: Arch Linux X8664 \\
\textbf{IDEs or Text Editors Used}: Visual Studio Code\\
% \textbf{Compilers or Interpreters}: Python 3.10.1\\

% \section{Code}
% \lstinputlisting[language=Python, caption="DSA Signature Validity using PyCrypto Library"]{../Programs/Assignment_7/dsa using lib.py}

\section*{FAQs}
\begin{enumerate}
    \item Features of Burp Suite:
    \begin{itemize}
        \item Web vulnerability scanning
        \item Intercepting proxy
        \item Application scanning
    \end{itemize}
    
    \item Three versions of Burp Suite:
    \begin{itemize}
        \item Community Edition (free)
        \item Professional Edition (paid)
        \item Enterprise Edition (paid with additional team features)
    \end{itemize}
    
    \item Safety of Burp Suite:
    \begin{itemize}
        \item Burp Suite is safe for ethical hacking and security testing when used responsibly and legally.
    \end{itemize}
    
    \item Purpose of the proxy tab in Burp Suite:
    \begin{itemize}
        \item Allows intercepting and modifying HTTP/S requests and responses for analyzing and manipulating web traffic during security testing.
    \end{itemize}
\end{enumerate}


\section{Conclusion}
In this assignment, we learned about the Burpsuite tool and performed a brute force attack using Burpsuite. We set up the attack by intercepting the login request, setting up the payloads, and executing the attack. We observed the attack in action and found the correct password by noticing the change in the response length. This exercise helped us understand the concept of brute force attacks and how they can be used to compromise security.
\clearpage

\pagebreak


\end{document}