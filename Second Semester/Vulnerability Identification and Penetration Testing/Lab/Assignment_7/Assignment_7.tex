% This is a Basic Assignment Paper but with like Code and stuff allowed in it, there is also url, hyperlinks from contents included. 

\documentclass[11pt]{article}

% Preamble

\usepackage[margin=1in]{geometry}
\usepackage{amsfonts, amsmath, amssymb, amsthm}
\usepackage{fancyhdr, float, graphicx}
\usepackage[utf8]{inputenc} % Required for inputting international characters
\usepackage[T1]{fontenc} % Output font encoding for international characters
% \usepackage{fouriernc} % Use the New Century Schoolbook font
\usepackage[nottoc, notlot, notlof]{tocbibind}
\usepackage{listings}
\usepackage{xcolor}
\usepackage{blindtext}
\usepackage{hyperref}
\definecolor{codepurple}{rgb}{0.58,0,0.82}
\hypersetup{
    colorlinks=true,
    linkcolor=black,
    filecolor=black,      
    urlcolor=codepurple,
    pdfpagemode=FullScreen,
    }

\definecolor{codegreen}{rgb}{0,0.6,0}
\definecolor{codegray}{rgb}{0.5,0.5,0.5}
\definecolor{backcolour}{rgb}{0.95,0.95,0.92}

\lstdefinestyle{mystyle}{
    backgroundcolor=\color{backcolour},   
    commentstyle=\color{codegreen},
    keywordstyle=\color{magenta},
    numberstyle=\tiny\color{codegray},
    stringstyle=\color{codepurple},
    basicstyle=\ttfamily\footnotesize,
    breakatwhitespace=false,         
    breaklines=true,                 
    captionpos=b,                    
    keepspaces=true,                 
    numbers=left,                    
    numbersep=5pt,                  
    showspaces=false,                
    showstringspaces=false,
    showtabs=false,                  
    tabsize=2
}

\lstset{style=mystyle}

% Header and Footer
\pagestyle{fancy}
\fancyhead{}
\fancyfoot{}
\fancyhead[L]{\textit{\Large{Vulnerability Identification and Penetration Testing}}}
\fancyhead[R]{\textit{Krishnaraj T}}
\fancyfoot[C]{\thepage}
\renewcommand{\footrulewidth}{1pt}
\newtheorem{thm}{Theorem}
\newtheorem{dfn}[thm]{Definition}


% Other Doc Editing
% \parindent 0ex
%\renewcommand{\baselinestretch}{1.5}

\begin{document}

\begin{titlepage}
    \centering

    %---------------------------NAMES-------------------------------

    \huge\textsc{
        MIT World Peace University
    }\\

    \vspace{0.75\baselineskip} % space after Uni Name

    \LARGE{
        Vulnerability Identification and Penetration Testing\\
        Third Year B. Tech, Semester 6
    }

    \vfill % space after Sub Name

    %--------------------------TITLE-------------------------------

    \rule{\textwidth}{1.6pt}\vspace*{-\baselineskip}\vspace*{2pt}
    \rule{\textwidth}{0.6pt}
    \vspace{0.75\baselineskip} % Whitespace above the title



    \huge{\textsc{
            Information Gathering with Scanning Tools
        }} \\



    \vspace{0.5\baselineskip} % Whitespace below the title
    \rule{\textwidth}{0.6pt}\vspace*{-\baselineskip}\vspace*{2.8pt}
    \rule{\textwidth}{1.6pt}

    \vspace{1\baselineskip} % Whitespace after the title block

    %--------------------------SUBTITLE --------------------------	

    \LARGE\textsc{
        Assignment 7
    } % Subtitle or further description
    \vfill

    %--------------------------AUTHOR-------------------------------

    Prepared By
    \vspace{0.5\baselineskip} % Whitespace before the editors

    \Large{
        Krishnaraj Thadesar \\
        Cyber Security and Forensics\\
        Batch A1, PA 10
    }


    \vspace{0.5\baselineskip} % Whitespace below the editor list
    \today

\end{titlepage}


\tableofcontents
\thispagestyle{empty}
\clearpage

\setcounter{page}{1}

\section{Aim}
To gather information about a target system using various scanning tools and techniques that include port scanners, network scanners etc.
\section{Objectives}
\begin{enumerate}
    \item To understand the concept of information gathering and reconnaissance.
    \item To explore various scanning tools and techniques for network reconnaissance.
    \item To perform network scanning and enumeration using tools like Nmap, Angry IP Scanner, and Metasploit.
    \item To analyze the results of network scanning and identify potential vulnerabilities in target systems.
    \item To understand the importance of information gathering in penetration testing and vulnerability assessment.
\end{enumerate}

\section{Theory}
\section{whois}

\subsection{Uses}
\begin{itemize}
    \item Retrieves registration information for domain names and IP addresses.
    \item Helps identify the owner, registrar, and contact details associated with a domain or IP.
    \item Provides valuable information for network troubleshooting and security analysis.
\end{itemize}

\subsection{Advantages}
\begin{itemize}
    \item Provides comprehensive domain and IP registration details.
    \item Helps in identifying potential security threats and malicious actors.
    \item Useful for verifying ownership and contact information for legitimate purposes.
\end{itemize}

\subsection{Disadvantages}
\begin{itemize}
    \item Limited effectiveness for domains with privacy protection services.
    \item Accuracy of information may vary depending on the registrar's data accuracy.
    \item Requires caution in handling sensitive contact information to avoid misuse or privacy violations.
\end{itemize}

\section{Angry IP Scanner}

\subsection{Uses}
\begin{itemize}
    \item Scans IP addresses and ports to discover active hosts and services on a network.
    \item Provides a quick and simple way to identify network devices and potential vulnerabilities.
    \item Offers fast scanning capabilities and a user-friendly interface for network reconnaissance.
\end{itemize}

\subsection{Advantages}
\begin{itemize}
    \item Cross-platform compatibility, supporting Windows, macOS, and Linux.
    \item Lightweight and easy to use, suitable for both beginners and experienced users.
    \item Offers customization options and advanced features for network scanning and analysis.
\end{itemize}

\subsection{Disadvantages}
\begin{itemize}
    \item Limited scanning options compared to more advanced tools like Nmap.
    \item May produce false positives or miss certain vulnerabilities in complex network environments.
    \item Lacks advanced features for detailed vulnerability assessment and exploitation.
\end{itemize}

\section{Metasploit}

\subsection{Uses}
\begin{itemize}
    \item Penetration testing framework for identifying and exploiting vulnerabilities in target systems.
    \item Provides a wide range of exploit modules, payloads, and post-exploitation tools.
    \item Used by security professionals and researchers for ethical hacking, red teaming, and vulnerability assessment.
\end{itemize}

\subsection{Mega Ping}
\begin{itemize}
    \item \textit{This subsection does not appear to be relevant to Metasploit.}
\end{itemize}

\subsection{Disadvantages}
\begin{itemize}
    \item Steep learning curve for beginners due to its complexity and extensive feature set.
    \item Requires caution and ethical considerations to avoid misuse and potential harm.
    \item May trigger false positives or lead to unintended consequences if not used carefully.
\end{itemize}

\section{Netscan Tools Pro}

\subsection{Uses}
\begin{itemize}
    \item Comprehensive network scanning tool for discovering hosts, open ports, and services.
    \item Offers a wide range of scanning techniques and customization options.
    \item Provides detailed reports and analysis for network security assessment and troubleshooting.
\end{itemize}

\subsection{Advantages}
\begin{itemize}
    \item User-friendly interface with intuitive features and workflows.
    \item Supports various scanning methods, including TCP, UDP, and SYN scanning.
    \item Offers real-time monitoring and reporting capabilities for network administrators.
\end{itemize}

\subsection{Disadvantages}
\begin{itemize}
    \item Paid software, may require a license for full functionality.
    \item Limited platform support compared to open-source alternatives.
    \item May lack some advanced features available in other commercial network scanning tools.
\end{itemize}


\section{Implementation}

\subsection{whois}

\subsubsection*{Syntax}
\begin{verbatim}
$ whois <domain_name>
\end{verbatim}

\subsubsection*{Command}
\begin{verbatim}
$ whois mitwpu.edu.in
\end{verbatim}

\subsubsection*{Purpose}
This command retrieves registration information for the specified domain name, including the owner, registrar, and contact details associated with the domain.
\subsubsection*{Output}
\begin{figure}[H]
    \centering
    \includegraphics[width=0.99\textwidth]{whois.png}
    \caption{Output of the Command}
\end{figure}
\subsection{whois}

\subsubsection*{Syntax}
\begin{verbatim}
$ whois <domain_name>
\end{verbatim}

\subsubsection*{Command}
\begin{verbatim}
$ whois whois.com
\end{verbatim}

\subsubsection*{Output}
\begin{figure}[H]
    \centering
    \includegraphics[width=0.99\textwidth]{whois (1).png}
    \caption{Output of the Command}
\end{figure}

\subsection{Angry IP Scanner}
\begin{figure}[H]
    \centering
    \includegraphics[width=0.99\textwidth]{angryip.png}
    \caption{Output of the Command}
\end{figure}

\subsection{Metasploit}

\subsubsection*{Command}
\begin{verbatim}
$ use auxiliary/scanner/portscan/syn
$ set interface eth0
$ set ports 80
$ set rhosts 192.168.114.1
$ set threads 50
\end{verbatim}

\subsubsection*{Purpose}
These commands configure Metasploit to perform a SYN port scan on the specified target IP address (
192.168.114.1 ) using port 80 and 50 threads for parallel scanning.
\subsubsection*{Output}
\begin{figure}[H]
    \centering
    \includegraphics[width=0.99\textwidth]{metasploit tcp (2).png}
    \caption{Output of the Command}
\end{figure}
\begin{figure}[H]
    \centering
    \includegraphics[width=0.99\textwidth]{metasploit tcp (3).png}
    \caption{Output of the Command}
\end{figure}

\subsection{Ports in Megaping}

\subsubsection*{Output}
\begin{figure}[H]
    \centering
    \includegraphics[width=0.99\textwidth]{megaping .png}
    \caption{Output of the Command}
\end{figure}

\subsection{IP Scanner in Megaping}
\begin{figure}[H]
    \centering
    \includegraphics[width=0.99\textwidth]{megaping (2).png}
    \caption{Output of the Command}
\end{figure}
\subsection{TraceRoute in Megaping}
\begin{figure}[H]
    \centering
    \includegraphics[width=0.99\textwidth]{megaping (3).png}
    \caption{Output of the Command}
\end{figure}
\subsection{IP Scanner in Range in Megaping}
\begin{figure}[H]
    \centering
    \includegraphics[width=0.99\textwidth]{megaping (4).png}
    \caption{Output of the Command}
\end{figure}
\subsection{NetScanTools Pro Demo Ping Scan Results in Browser}
\begin{figure}[H]
    \centering
    \includegraphics[width=0.99\textwidth]{statdemo (2).png}
    \caption{Output of the Command}
\end{figure}


\section{Platform}
\textbf{Operating System}: Arch Linux X8664 \\
\textbf{IDEs or Text Editors Used}: Visual Studio Code\\
% \textbf{Compilers or Interpreters}: Python 3.10.1\\

% \section{Code}
% \lstinputlisting[language=Python, caption="DSA Signature Validity using PyCrypto Library"]{../Programs/Assignment_7/dsa using lib.py}

\section{FAQs}

\begin{enumerate}
    \item \textbf{What is an Angry IP Scanner?}
          \begin{itemize}
              \item Angry IP Scanner: A cross-platform network scanner used to scan IP addresses and ports, providing information about active hosts and services.
              \item Offers fast scanning capabilities and a user-friendly interface for network reconnaissance.
          \end{itemize}

    \item \textbf{What is dpkg? Explain the purpose of it.}
          \begin{itemize}
              \item dpkg: Debian Package Manager, used for installing, removing, and managing software packages on Debian-based Linux distributions.
              \item Provides a command-line interface for package management, ensuring system stability and dependency resolution.
          \end{itemize}

    \item \textbf{Enlist the various Port scanning tools.}
          \begin{itemize}
              \item Nmap: A versatile network scanner with various scanning techniques.
              \item Masscan: High-speed TCP port scanner.
              \item Zmap: Fast Internet-scale network scanner.
              \item Unicornscan: Asynchronous TCP and UDP port scanner.
              \item Hping: Command-line packet crafter, sender, and analyzer.
          \end{itemize}

    \item \textbf{What are the various modules provided by Metasploit?}
          \begin{itemize}
              \item Exploits: Code to exploit vulnerabilities in target systems.
              \item Payloads: Code to deliver malicious payloads to compromised systems.
              \item Auxiliary: Modules for various tasks like information gathering, brute-forcing, and scanning.
              \item Post: Modules for post-exploitation activities like privilege escalation and data exfiltration.
              \item Encoders: Modules for encoding payloads to evade detection.
          \end{itemize}
\end{enumerate}

\section{Conclusion}
In this assignment, we explored various scanning tools and techniques for network reconnaissance, including whois, Angry IP Scanner, Metasploit, MegaPing, and NetScanTools Pro. These tools provide valuable information for identifying active hosts, open ports, and services on a network, helping security professionals and researchers in penetration testing, vulnerability assessment, and network troubleshooting. By analyzing the results of network scanning, we can identify potential vulnerabilities in target systems and take appropriate measures to secure them.
\clearpage

\pagebreak


\end{document}