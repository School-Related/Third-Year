% This is a Basic Assignment Paper but with like Code and stuff allowed in it, there is also url, hyperlinks from contents included. 

\documentclass[11pt]{article}

% Preamble

\usepackage[margin=1in]{geometry}
\usepackage{amsfonts, amsmath, amssymb, amsthm}
\usepackage{fancyhdr, float, graphicx}
\usepackage[utf8]{inputenc} % Required for inputting international characters
\usepackage[T1]{fontenc} % Output font encoding for international characters
% \usepackage{fouriernc} % Use the New Century Schoolbook font
\usepackage[nottoc, notlot, notlof]{tocbibind}
\usepackage{listings}
\usepackage{xcolor}
\usepackage{blindtext}
\usepackage{hyperref}
\definecolor{codepurple}{rgb}{0.58,0,0.82}
\hypersetup{
    colorlinks=true,
    linkcolor=black,
    filecolor=black,      
    urlcolor=codepurple,
    pdfpagemode=FullScreen,
    }

\definecolor{codegreen}{rgb}{0,0.6,0}
\definecolor{codegray}{rgb}{0.5,0.5,0.5}
\definecolor{backcolour}{rgb}{0.95,0.95,0.92}

\lstdefinestyle{mystyle}{
    backgroundcolor=\color{backcolour},   
    commentstyle=\color{codegreen},
    keywordstyle=\color{magenta},
    numberstyle=\tiny\color{codegray},
    stringstyle=\color{codepurple},
    basicstyle=\ttfamily\footnotesize,
    breakatwhitespace=false,         
    breaklines=true,                 
    captionpos=b,                    
    keepspaces=true,                 
    numbers=left,                    
    numbersep=5pt,                  
    showspaces=false,                
    showstringspaces=false,
    showtabs=false,                  
    tabsize=2
}

\lstset{style=mystyle}

% Header and Footer
\pagestyle{fancy}
\fancyhead{}
\fancyfoot{}
\fancyhead[L]{\textit{\Large{Vulnerability Identification and Penetration Testing}}}
\fancyhead[R]{\textit{Krishnaraj T}}
\fancyfoot[C]{\thepage}
\renewcommand{\footrulewidth}{1pt}
\newtheorem{thm}{Theorem}
\newtheorem{dfn}[thm]{Definition}


% Other Doc Editing
% \parindent 0ex
%\renewcommand{\baselinestretch}{1.5}

\begin{document}

\begin{titlepage}
    \centering

    %---------------------------NAMES-------------------------------

    \huge\textsc{
        MIT World Peace University
    }\\

    \vspace{0.75\baselineskip} % space after Uni Name

    \LARGE{
        Vulnerability Identification and Penetration Testing\\
        Third Year B. Tech, Semester 6
    }

    \vfill % space after Sub Name

    %--------------------------TITLE-------------------------------

    \rule{\textwidth}{1.6pt}\vspace*{-\baselineskip}\vspace*{2pt}
    \rule{\textwidth}{0.6pt}
    \vspace{0.75\baselineskip} % Whitespace above the title



    \huge{\textsc{
            Network Service Scanning with nmap in XML and HTML Format
        }} \\



    \vspace{0.5\baselineskip} % Whitespace below the title
    \rule{\textwidth}{0.6pt}\vspace*{-\baselineskip}\vspace*{2.8pt}
    \rule{\textwidth}{1.6pt}

    \vspace{1\baselineskip} % Whitespace after the title block

    %--------------------------SUBTITLE --------------------------	

    \LARGE\textsc{
        Assignment 4
    } % Subtitle or further description
    \vfill

    %--------------------------AUTHOR-------------------------------

    Prepared By
    \vspace{0.5\baselineskip} % Whitespace before the editors

    \Large{
        Krishnaraj Thadesar \\
        Cyber Security and Forensics\\
        Batch A1, PA 10
    }


    \vspace{0.5\baselineskip} % Whitespace below the editor list
    \today

\end{titlepage}


\tableofcontents
\thispagestyle{empty}
\clearpage

\setcounter{page}{1}

\section{Aim}
To Discover Network service to Organize and sort through Nmap scan output
\section{Objectives}
\begin{enumerate}
    \item To Discover Network service
    \item To Organize and sort through Nmap scan output
    \item To Generate Nmap scan output in XML and HTML format
\end{enumerate}

\section{Theory}

\subsection{XML Format}
XML stands for eXtensible Markup Language. It was designed to store and transport data. It was designed to be both human- and machine-readable. XML plays an important role in many different IT systems. It is used to store data, to configure programs, and to create user interfaces. XML is often used for distributing data over the Internet. It is important to note that XML is not a replacement for HTML. XML and HTML were designed with different goals.

\subsection{Uses}
\begin{itemize}
    \item Storing and transporting data in a structured format.
    \item Configuring programs and defining settings in a readable manner.
    \item Creating user interfaces and defining document structures.
    \item Exchanging data between different systems and platforms.
    \item Representing data hierarchies and relationships in a standardized format.
\end{itemize}

\subsection{Advantages}
\begin{itemize}
    \item Human-readable format, making it easy to understand and modify by developers and users.
    \item Machine-readable format, allowing for automated processing and interoperability between systems.
    \item Platform-independent, enabling data exchange between different operating systems and software applications.
    \item Extensibility, allowing users to define custom tags and structures to meet specific requirements.
    \item Well-defined syntax and rules, ensuring consistency and reliability in data representation.
\end{itemize}

\subsection{Disadvantages}
\begin{itemize}
    \item Verbosity, as XML documents can become large and complex due to the use of tags and attributes.
    \item Overhead, as XML parsing and processing may require additional computational resources compared to other data formats.
    \item Lack of native support for complex data types and structures, leading to the need for custom solutions or additional standards (e.g., XML Schema).
    \item Limited support for binary data, as XML is primarily designed for text-based data representation.
    \item Potential security risks, such as XML External Entity (XXE) attacks, if not properly validated and sanitized.
\end{itemize}


\section{Implementation}
\subsection{}

\subsubsection*{Syntax}
\begin{verbatim}
$
\end{verbatim}

\subsubsection*{Command}
\begin{verbatim}
$
\end{verbatim}

\subsubsection*{Purpose}

\subsubsection*{Output}
\begin{figure}[H]
    \centering
    \includegraphics[width=0.99\textwidth]{a4.png}
    \caption{Get IP Address}
    \label{fig:1}
\end{figure}

\subsection{}

\subsubsection*{Syntax}
\begin{verbatim}
$
\end{verbatim}

\subsubsection*{Command}
\begin{verbatim}
$
\end{verbatim}

\subsubsection*{Purpose}

\subsubsection*{Output}
\begin{figure}[H]
    \centering
    \includegraphics[width=0.99\textwidth]{a4 (1).png}
    \caption{Get IP Address}
    \label{fig:1}
\end{figure}
\subsection{}

\subsubsection*{Syntax}
\begin{verbatim}
$
\end{verbatim}

\subsubsection*{Command}
\begin{verbatim}
$
\end{verbatim}

\subsubsection*{Purpose}

\subsubsection*{Output}
\begin{figure}[H]
    \centering
    \includegraphics[width=0.99\textwidth]{a4 (2).png}
    \caption{Get IP Address}
    \label{fig:1}
\end{figure}
\subsection{}

\subsubsection*{Syntax}
\begin{verbatim}
$
\end{verbatim}

\subsubsection*{Command}
\begin{verbatim}
$
\end{verbatim}

\subsubsection*{Purpose}

\subsubsection*{Output}
\begin{figure}[H]
    \centering
    \includegraphics[width=0.99\textwidth]{a4 (3).png}
    \caption{Get IP Address}
    \label{fig:1}
\end{figure}
\subsection{}

\subsubsection*{Syntax}
\begin{verbatim}
$
\end{verbatim}

\subsubsection*{Command}
\begin{verbatim}
$
\end{verbatim}

\subsubsection*{Purpose}

\subsubsection*{Output}
\begin{figure}[H]
    \centering
    \includegraphics[width=0.99\textwidth]{a4 (4).png}
    \caption{Get IP Address}
    \label{fig:1}
\end{figure}
\subsection{}

\subsubsection*{Syntax}
\begin{verbatim}
$
\end{verbatim}

\subsubsection*{Command}
\begin{verbatim}
$
\end{verbatim}

\subsubsection*{Purpose}

\subsubsection*{Output}
\begin{figure}[H]
    \centering
    \includegraphics[width=0.99\textwidth]{a4 (5).png}
    \caption{Get IP Address}
    \label{fig:1}
\end{figure}
\subsection{}

\subsubsection*{Syntax}
\begin{verbatim}
$
\end{verbatim}

\subsubsection*{Command}
\begin{verbatim}
$
\end{verbatim}

\subsubsection*{Purpose}

\subsubsection*{Output}
\begin{figure}[H]
    \centering
    \includegraphics[width=0.99\textwidth]{a4 (6).png}
    \caption{Get IP Address}
    \label{fig:1}
\end{figure}
\subsection{}

\subsubsection*{Syntax}
\begin{verbatim}
$
\end{verbatim}

\subsubsection*{Command}
\begin{verbatim}
$
\end{verbatim}

\subsubsection*{Purpose}

\subsubsection*{Output}
\begin{figure}[H]
    \centering
    \includegraphics[width=0.99\textwidth]{a4 (7).png}
    \caption{Get IP Address}
    \label{fig:1}
\end{figure}


\section{Platform}
\textbf{Operating System}: Arch Linux X8664 \\
\textbf{IDEs or Text Editors Used}: Visual Studio Code\\
% \textbf{Compilers or Interpreters}: Python 3.10.1\\

% \section{Code}
% \lstinputlisting[language=Python, caption="DSA Signature Validity using PyCrypto Library"]{../Programs/Assignment_7/dsa using lib.py}

\section{FAQs}
\begin{enumerate}
    \item \textbf{What is the meaning of subnet?}
          \begin{itemize}
              \item Subnet: Division of an IP network for efficient resource management.
              \item Facilitates organization and management of network devices and addresses.
          \end{itemize}

    \item \textbf{Explain subnet mask? Meaning of /8, /16, /24, /32?}
          \begin{itemize}
              \item Subnet Mask: Binary pattern dividing IP address into network and host portions.
              \item /8, /16, /24, /32 denote network size, representing the number of bits in the subnet mask.
          \end{itemize}

    \item \textbf{What is NSE? Explain it.}
          \begin{itemize}
              \item NSE (Nmap Scripting Engine): Automates Nmap's functionality for network reconnaissance and exploitation.
              \item Provides a framework for writing and executing scripts to enhance scanning capabilities.
          \end{itemize}

    \item \textbf{Different flags for output supported by nmap.}
          \begin{itemize}
              \item Nmap Output Flags: Include -oN (normal), -oG (grepable), -oX (XML), and -oA (all formats).
              \item Allow users to customize the format and content of Nmap scan results for analysis and reporting.
          \end{itemize}

    \item \textbf{What is xsltproc? Explain it.}
          \begin{itemize}
              \item xsltproc: Command-line tool for transforming XML documents using XSLT stylesheets.
              \item Facilitates conversion of XML data into different formats such as HTML, text, or other XML formats.
          \end{itemize}

    \item \textbf{Why to see the output in HTML format than XML?}
          \begin{itemize}
              \item HTML Output: Provides visually appealing presentation of Nmap scan results compared to XML.
              \item Allows for easier interpretation and analysis of scan data through structured formatting and styling.
          \end{itemize}
\end{enumerate}


\section{Conclusion}
In this assignment, we learned about the importance of network service scanning and organizing Nmap scan output. We explored the generation of Nmap scan results in XML and HTML formats to facilitate data analysis and reporting. By leveraging the capabilities of Nmap and related tools, we can enhance our understanding of network vulnerabilities and security risks. This assignment provided valuable insights into the practical aspects of vulnerability identification and penetration testing, which are essential skills for cybersecurity professionals.
\clearpage

\pagebreak


\end{document}