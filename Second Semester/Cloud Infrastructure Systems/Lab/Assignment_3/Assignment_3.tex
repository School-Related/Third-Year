% This is a Basic Assignment Paper but with like Code and stuff allowed in it, there is also url, hyperlinks from contents included. 

\documentclass[11pt]{article}

% Preamble

\usepackage[margin=1in]{geometry}
\usepackage{amsfonts, amsmath, amssymb, amsthm}
\usepackage{fancyhdr, float, graphicx}
\usepackage[utf8]{inputenc} % Required for inputting international characters
\usepackage[T1]{fontenc} % Output font encoding for international characters
\usepackage{fouriernc} % Use the New Century Schoolbook font
\usepackage[nottoc, notlot, notlof]{tocbibind}
\usepackage{listings}
\usepackage{xcolor}
\usepackage{blindtext}
\usepackage{hyperref}
\definecolor{codepurple}{rgb}{0.58,0,0.82}
\hypersetup{
    colorlinks=true,
    linkcolor=black,
    filecolor=black,      
    urlcolor=codepurple,
    pdfpagemode=FullScreen,
    }

\definecolor{codegreen}{rgb}{0,0.6,0}
\definecolor{codegray}{rgb}{0.5,0.5,0.5}
\definecolor{backcolour}{rgb}{0.95,0.95,0.92}

\lstdefinestyle{mystyle}{
    backgroundcolor=\color{backcolour},   
    commentstyle=\color{codegreen},
    keywordstyle=\color{magenta},
    numberstyle=\tiny\color{codegray},
    stringstyle=\color{codepurple},
    basicstyle=\ttfamily\footnotesize,
    breakatwhitespace=false,         
    breaklines=true,                 
    captionpos=b,                    
    keepspaces=true,                 
    numbers=left,                    
    numbersep=5pt,                  
    showspaces=false,                
    showstringspaces=false,
    showtabs=false,                  
    tabsize=2
}

\lstset{style=mystyle}

% Header and Footer
\pagestyle{fancy}
\fancyhead{}
\fancyfoot{}
\fancyhead[L]{\textit{\Large{Cloud Infrastructure and Security - TY. B. Tech}}}
\fancyhead[R]{\textit{Krishnaraj T}}
\fancyfoot[C]{\thepage}
\renewcommand{\footrulewidth}{1pt}
\newtheorem{thm}{Theorem}
\newtheorem{dfn}[thm]{Definition}


% Other Doc Editing
% \parindent 0ex
%\renewcommand{\baselinestretch}{1.5}

\begin{document}

\begin{titlepage}
    \centering

    %---------------------------NAMES-------------------------------

    \huge\textsc{
        MIT World Peace University
    }\\

    \vspace{0.75\baselineskip} % space after Uni Name

    \LARGE{
        Cloud Infrastructure and Security\\
        Third Year B. Tech, Semester 6
    }

    \vfill % space after Sub Name

    %--------------------------TITLE-------------------------------

    \rule{\textwidth}{1.6pt}\vspace*{-\baselineskip}\vspace*{2pt}
    \rule{\textwidth}{0.6pt}
    \vspace{0.75\baselineskip} % Whitespace above the title



    \huge{\textsc{
            Deployment on AWS and Load Balancing
        }} \\



    \vspace{0.5\baselineskip} % Whitespace below the title
    \rule{\textwidth}{0.6pt}\vspace*{-\baselineskip}\vspace*{2.8pt}
    \rule{\textwidth}{1.6pt}

    \vspace{1\baselineskip} % Whitespace after the title block

    %--------------------------SUBTITLE --------------------------	

    \LARGE\textsc{
        Assignment 3
    } % Subtitle or further description
    \vfill

    %--------------------------AUTHOR-------------------------------

    Prepared By
    \vspace{0.5\baselineskip} % Whitespace before the editors

    \Large{
        Krishnaraj Thadesar \\
        Cyber Security and Forensics\\
        Batch A1, PA 10
    }


    \vspace{0.5\baselineskip} % Whitespace below the editor list
    \today

\end{titlepage}


\tableofcontents
\thispagestyle{empty}
\clearpage

\setcounter{page}{1}

\section{Aim}
Create an account on AWS. Deploy a Website admission portal on EC2 Service.
Configure the traffic rules for the server for a specific need. Creation of application load
balancer.

\section{Objectives}
\begin{enumerate}
    \item To get acquainted with AWS platform.
    \item To get familiar with EC2 Service
    \item Learn how to configure the traffic rules.
    \item To understand creation of application load balancer
\end{enumerate}

\section{Theory}
\subsection{Introduction to AWS (Amazon Web Services)}
Amazon Web Services (AWS) is a comprehensive, evolving cloud computing platform provided by Amazon. It provides a mix of infrastructure as a service (IaaS), platform as a service (PaaS) and packaged software as a service (SaaS) offerings. AWS services can offer an organization tools such as compute power, database storage and content delivery services. Moving to the cloud with AWS can help organizations cut costs, improve efficiency and innovate at a faster pace.

\subsection{Features of AWS}

\begin{enumerate}
    \item \textbf{Compute:} AWS offers a wide range of compute services, including Amazon Elastic Compute Cloud (EC2), AWS Lambda, Amazon Elastic Container Service (ECS), and more. These services provide scalable compute capacity for running applications and workloads.
    \item \textbf{Storage:} AWS provides various storage services, such as Amazon Simple Storage Service (S3), Amazon Elastic Block Store (EBS), Amazon Glacier, and more. These services offer scalable, durable, and cost-effective storage solutions for data storage and backup.
    \item \textbf{Database:} AWS offers a range of database services, including Amazon Relational Database Service (RDS), Amazon DynamoDB, Amazon Redshift, and more. These services provide managed database solutions for different use cases and workloads.
    \item \textbf{Networking:} AWS provides networking services such as Amazon Virtual Private Cloud (VPC), Amazon Route 53, AWS Direct Connect, and more. These services enable users to build secure and scalable network architectures for their applications.
    \item \textbf{Security:} AWS offers a wide range of security services, including AWS Identity and Access Management (IAM), AWS Key Management Service (KMS), AWS WAF, and more. These services help users secure their applications, data, and infrastructure in the cloud.
    \item \textbf{Management Tools:} AWS provides management tools such as AWS CloudFormation, AWS CloudWatch, AWS Config, and more. These tools help users automate, monitor, and manage their AWS resources and services.
    \item \textbf{Analytics:} AWS offers analytics services such as Amazon Kinesis, Amazon EMR, Amazon Redshift, and more. These services help users collect, process, analyze, and visualize data to derive insights and make informed decisions.
    \item \textbf{Machine Learning:} AWS provides machine learning services such as Amazon SageMaker, Amazon Rekognition, Amazon Comprehend, and more. These services enable users to build, train, and deploy machine learning models at scale.
    \item \textbf{Internet of Things (IoT):} AWS offers IoT services such as AWS IoT Core, AWS IoT Greengrass, AWS IoT Device Management, and more. These services help users connect, manage, and secure IoT devices and applications.
    \item \textbf{Serverless Computing:} AWS provides serverless computing services such as AWS Lambda, Amazon API Gateway, AWS Step Functions, and more. These services enable users to build and deploy applications without managing servers.
    \item \textbf{Containers:} AWS offers container services such as Amazon Elastic Container Service (ECS), Amazon Elastic Kubernetes Service (EKS), AWS Fargate, and more. These services help users run and manage containerized applications at scale.
    \item \textbf{DevOps:} AWS provides DevOps services such as AWS CodePipeline, AWS CodeBuild, AWS CodeDeploy, and more. These services help users automate software development and deployment processes.
    \item \textbf{Artificial Intelligence (AI):} AWS offers AI services such as Amazon Lex, Amazon Polly, Amazon Translate, and more. These services enable users to build AI-powered applications and services.
    \item \textbf{Blockchain:} AWS provides blockchain services such as Amazon Managed Blockchain, Amazon Quantum Ledger Database (QLDB), and more. These services help users build scalable and secure blockchain applications.
    \item \textbf{Augmented Reality (AR) and Virtual Reality (VR):} AWS offers AR and VR services such as Amazon Sumerian, Amazon S3 Glacier Deep Archive, and more. These services help users create immersive AR and VR experiences.
    \item \textbf{Game Development:} AWS provides game development services such as Amazon GameLift, Amazon Lumberyard, and more. These services help users build, deploy, and scale games in the cloud.
\end{enumerate}

\subsection{Steps /procedure to follow for an AWS account creation}

\begin{enumerate}
    \item \textbf{Sign Up for AWS:} Go to the AWS website and click on the "Create an AWS Account" button. Follow the on-screen instructions to create a new AWS account.
    \item \textbf{Provide Account Information:} Enter your email address, password, and account name. Click on the "Continue" button to proceed.
    \item \textbf{Contact Information:} Enter your contact information, including name, address, and phone number. Click on the "Create Account and Continue" button.
    \item \textbf{Payment Information:} Enter your payment information, including credit card details. AWS may charge a small amount to verify your identity.
    \item \textbf{Identity Verification:} AWS will verify your identity by sending a verification code to your phone number. Enter the code to complete the verification process.
    \item \textbf{Choose a Support Plan:} Choose a support plan based on your requirements. You can choose the Basic (free) plan or opt for a paid plan for additional support.
    \item \textbf{Confirmation:} Review the account details and click on the "Create Account and Continue" button to complete the account creation process.
    \item \textbf{Sign In:} Sign in to your new AWS account using your email address and password. You can now access the AWS Management Console and start using AWS services.
\end{enumerate}

\subsection{Briefing on EC2 Service}

Amazon Elastic Compute Cloud (Amazon EC2) is a web service that provides resizable compute capacity in the cloud. It is designed to make web-scale cloud computing easier for developers. Amazon EC2's simple web service interface allows you to obtain and configure capacity with minimal friction. It provides you with complete control of your computing resources and lets you run on Amazon's proven computing environment.

\subsubsection{Features}
\begin{enumerate}
    \item \textbf{Virtual Servers:} Amazon EC2 provides virtual servers, known as instances, that can be launched in minutes. You can choose from a variety of instance types with different compute, memory, and storage configurations.
    \item \textbf{Scalability:} Amazon EC2 allows you to scale your compute capacity up or down based on your application's needs. You can launch multiple instances to handle increased traffic or reduce the number of instances during off-peak hours.
    \item \textbf{Security:} Amazon EC2 provides security features such as security groups, network access control lists (ACLs), and key pairs to secure your instances. You can also use AWS Identity and Access Management (IAM) to control access to your resources.
    \item \textbf{Integration:} Amazon EC2 integrates with other AWS services such as Amazon S3, Amazon RDS, and Amazon VPC. You can easily deploy applications that use multiple AWS services and resources.
    \item \textbf{Monitoring:} Amazon EC2 provides monitoring and logging features through Amazon CloudWatch. You can monitor your instances' performance, set alarms, and view logs to troubleshoot issues.
    \item \textbf{Auto Scaling:} Amazon EC2 Auto Scaling allows you to automatically adjust the number of instances in your Auto Scaling group based on demand. You can define scaling policies to scale in or out based on metrics such as CPU utilization or request count.
    \item \textbf{Load Balancing:} Amazon EC2 works with Elastic Load Balancing to distribute incoming traffic across multiple instances. You can use Application Load Balancers or Network Load Balancers to achieve high availability and fault tolerance.
    \item \textbf{Pricing:} Amazon EC2 offers a pay-as-you-go pricing model with no upfront costs or long-term contracts. You only pay for the compute capacity you use, and you can choose from various pricing options based on your requirements.
\end{enumerate}

\subsection{Deploying the website using EC2 service}
% figures

\subsection{Configure the traffic rules for the server}

Steps:
\begin{enumerate}
    \item Accessing the Security Group:

          There are two primary ways to configure traffic rules for an EC2 instance:
          \begin{enumerate}
              \item Using the AWS Management Console:
                    \begin{enumerate}
                        \item Go to the Amazon EC2 service in the AWS Management Console (https://aws.amazon.com/).
                        \item In the navigation pane on the left, select the option for Security Groups.
                        \item Choose the security group you want to modify. This security group will be assigned to your EC2 instance and controls the inbound and outbound traffic.
                    \end{enumerate}
              \item Using the AWS CLI:
                    AWS provides a Command Line Interface (CLI) for managing various AWS services, including EC2 security groups. You can find instructions and reference guides for the AWS CLI here: https://aws.amazon.com/cli/
          \end{enumerate}

    \item Adding Inbound Rules:
          \begin{enumerate}
              \item Inbound rules define which types of traffic are allowed to reach your EC2 instance from the internet or your VPC (Virtual Private Cloud).
              \item To add an inbound rule, click on the "Edit inbound rules" button in the AWS Management Console.
          \end{enumerate}

    \item Specifying Rule Details:

          When adding an inbound rule, you'll need to specify several details:
          \begin{enumerate}
              \item Protocol: Choose the communication protocol for the traffic (e.g., TCP, UDP, ICMP).
              \item Port Range: Specify the port(s) on your EC2 instance that this rule applies to. For example, port 22 for SSH access or port 80 for web traffic. You can also open a range of ports if needed.
              \item Source: Define the source of the traffic. You can allow traffic from anywhere (0.0.0.0/0 for IPv4 or ::/0 for IPv6) or restrict it to specific IP addresses, security groups, or VPCs.
          \end{enumerate}

    \item Adding Outbound Rules (Optional):
          \begin{enumerate}
              \item Outbound rules define what type of traffic your EC2 instance can initiate towards the internet or other resources. By default, outbound traffic is allowed, but you can configure restrictions if needed.
              \item The process for adding outbound rules is similar to adding inbound rules. You'll specify the protocol, port range, and destination for the outbound traffic.
          \end{enumerate}

    \item Saving the Changes:

          Once you've added or modified the desired rules, click on the "Save" button to apply the changes to the security group.

\end{enumerate}

Important Considerations:


Security Best Practices: It's recommended to follow security best practices when configuring traffic rules. Don't open unnecessary ports or allow access from unrestricted sources.

Specific Needs: Tailor the rules to your specific application or service running on the EC2 instance.

Testing: After making changes to the security group, it's advisable to test your application or service to ensure it functions as expected with the new rules in place.

\subsection{Creating Application load balancer}
\begin{enumerate}
    \item \textbf{Prepare Your EC2 Instances:}
          \begin{itemize}
              \item Launch or ensure you have EC2 instances running your application.
              \item Make sure your application is properly configured and running on these instances.
          \end{itemize}

    \item \textbf{Create a Target Group:}
          \begin{itemize}
              \item Go to the Amazon EC2 dashboard.
              \item Under "Load Balancing," navigate to "Target Groups."
              \item Click on "Create target group."
              \item Specify a name for your target group, protocol, port, and VPC.
              \item Define the health checks for your target instances.
          \end{itemize}

    \item \textbf{Create an Application Load Balancer:}
          \begin{itemize}
              \item In the EC2 dashboard, under "Load Balancing," navigate to "Load Balancers."
              \item Click on "Create Load Balancer."
              \item Choose "Application Load Balancer" as the type.
              \item Configure the load balancer settings, including name, listeners, availability zones, and security settings.
              \item Associate the previously created target group with the load balancer.
          \end{itemize}

    \item \textbf{Configure Listener Rules:}
          \begin{itemize}
              \item Define the listener rules to route traffic to the appropriate target group based on the request characteristics (e.g., path patterns, host headers).
          \end{itemize}

    \item \textbf{Update DNS Records (Optional):}
          \begin{itemize}
              \item If you're using a custom domain, update your DNS records to point to the DNS name of your load balancer.
          \end{itemize}

    \item \textbf{Test Your Load Balancer:}
          \begin{itemize}
              \item Ensure that your load balancer is routing traffic to your EC2 instances correctly.
              \item Test your application to verify that it is functioning as expected with the load balancer in place.
          \end{itemize}

    \item \textbf{Monitor and Scale:}
          \begin{itemize}
              \item Monitor the performance and health of your load balancer and EC2 instances using Amazon CloudWatch metrics.
              \item Configure auto-scaling policies to automatically adjust the number of EC2 instances based on demand.
          \end{itemize}
\end{enumerate}

\section{Platform}
\textbf{Operating System}: Windows 11 \\
\textbf{IDEs or Text Editors Used}: Visual Studio Code\\
% \textbf{Compilers or Interpreters}: Python 3.10.1\\

% \section{Code}
% \lstinputlisting[language=Python, caption="DSA Signature Validity using PyCrypto Library"]{../Programs/Assignment_7/dsa using lib.py}

\section{FAQs}


\section{Conclusion}

In this assignment, we have learned about web services, PaaS tools, and the deployment of web services using Python on Cyclic. We have explored the different types of web services, including SOAP and RESTful services, and the standards and protocols used in web service development. We have also discussed the architecture of web services, the components involved, and the security mechanisms used to protect sensitive data. Finally, we have demonstrated the deployment of a web service using FastAPI and Cyclic, showcasing the process of creating, deploying, and accessing web services on a cloud platform.
\clearpage

\pagebreak
\begin{thebibliography}{}

    \bibitem{CloudConceptsOverview}
    IBM Cloud. Cloud Computing Concepts Overview. Accessed from: \url{https://www.ibm.com/cloud/learn/cloud-computing-concepts}.

\end{thebibliography}

\end{document}