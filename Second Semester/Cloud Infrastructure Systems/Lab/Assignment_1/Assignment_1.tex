% This is a Basic Assignment Paper but with like Code and stuff allowed in it, there is also url, hyperlinks from contents included. 

\documentclass[11pt]{article}

% Preamble

\usepackage[margin=1in]{geometry}
\usepackage{amsfonts, amsmath, amssymb, amsthm}
\usepackage{fancyhdr, float, graphicx}
\usepackage[utf8]{inputenc} % Required for inputting international characters
\usepackage[T1]{fontenc} % Output font encoding for international characters
\usepackage{fouriernc} % Use the New Century Schoolbook font
\usepackage[nottoc, notlot, notlof]{tocbibind}
\usepackage{listings}
\usepackage{xcolor}
\usepackage{blindtext}
\usepackage{hyperref}
\definecolor{codepurple}{rgb}{0.58,0,0.82}
\hypersetup{
    colorlinks=true,
    linkcolor=black,
    filecolor=black,      
    urlcolor=codepurple,
    pdfpagemode=FullScreen,
    }

\definecolor{codegreen}{rgb}{0,0.6,0}
\definecolor{codegray}{rgb}{0.5,0.5,0.5}
\definecolor{backcolour}{rgb}{0.95,0.95,0.92}

\lstdefinestyle{mystyle}{
    backgroundcolor=\color{backcolour},   
    commentstyle=\color{codegreen},
    keywordstyle=\color{magenta},
    numberstyle=\tiny\color{codegray},
    stringstyle=\color{codepurple},
    basicstyle=\ttfamily\footnotesize,
    breakatwhitespace=false,         
    breaklines=true,                 
    captionpos=b,                    
    keepspaces=true,                 
    numbers=left,                    
    numbersep=5pt,                  
    showspaces=false,                
    showstringspaces=false,
    showtabs=false,                  
    tabsize=2
}

\lstset{style=mystyle}

% Header and Footer
\pagestyle{fancy}
\fancyhead{}
\fancyfoot{}
\fancyhead[L]{\textit{\Large{Cloud Infrastructure and Security - TY. B. Tech}}}
\fancyhead[R]{\textit{Krishnaraj T}}
\fancyfoot[C]{\thepage}
\renewcommand{\footrulewidth}{1pt}
\newtheorem{thm}{Theorem}
\newtheorem{dfn}[thm]{Definition}


% Other Doc Editing
% \parindent 0ex
%\renewcommand{\baselinestretch}{1.5}

\begin{document}

\begin{titlepage}
    \centering

    %---------------------------NAMES-------------------------------

    \huge\textsc{
        MIT World Peace University
    }\\

    \vspace{0.75\baselineskip} % space after Uni Name

    \LARGE{
        Cloud Infrastructure and Security\\
        Third Year B. Tech, Semester 6
    }

    \vfill % space after Sub Name

    %--------------------------TITLE-------------------------------

    \rule{\textwidth}{1.6pt}\vspace*{-\baselineskip}\vspace*{2pt}
    \rule{\textwidth}{0.6pt}
    \vspace{0.75\baselineskip} % Whitespace above the title



    \huge{\textsc{
            Installation of Ubuntu Server on VMWare Workstation
        }} \\



    \vspace{0.5\baselineskip} % Whitespace below the title
    \rule{\textwidth}{0.6pt}\vspace*{-\baselineskip}\vspace*{2.8pt}
    \rule{\textwidth}{1.6pt}

    \vspace{1\baselineskip} % Whitespace after the title block

    %--------------------------SUBTITLE --------------------------	

    \LARGE\textsc{
        Assignment 1
    } % Subtitle or further description
    \vfill

    %--------------------------AUTHOR-------------------------------

    Prepared By
    \vspace{0.5\baselineskip} % Whitespace before the editors

    \Large{
        Krishnaraj Thadesar \\
        Cyber Security and Forensics\\
        Batch A1, PA 10
    }


    \vspace{0.5\baselineskip} % Whitespace below the editor list
    \today

\end{titlepage}


\tableofcontents
\thispagestyle{empty}
\clearpage

\setcounter{page}{1}

\section{Aim}
Install VM-Ware Workstation on a windows platform and deploy an Ubuntu server VM as per
requirement.

\section{Objectives}
\begin{enumerate}
    \item To get familiar with virtualization.
    \item Understand the use of VM ware workstation.
    \item Learn how to deploy Ubuntu on VM-Ware workstation.
\end{enumerate}

\section{Theory}

\subsection{Introduction to VMware and Virtualization}

\subsubsection{VMware}

VMware is a virtualization and cloud computing software provider based in Palo Alto, California. Founded in 1998, VMware is a subsidiary of Dell Technologies. The company's core product is a hypervisor that enables multiple operating systems to run on a single physical machine. VMware also offers a range of cloud services and software-defined data center solutions.

\begin{figure}[H]
    \centering
    \includegraphics[width=.95\textwidth]{vmware.png}
    \caption{VMWare Workstation 12 Pro}
\end{figure}

\subsubsection{Virtualization}

Virtualization is the process of creating a virtual version of a physical resource, such as a server, storage device, or network. It allows multiple virtual instances of the same resource to run on a single physical machine, enabling greater efficiency and flexibility in managing computing resources. Virtualization is a key technology in cloud computing, enabling the dynamic allocation of computing resources and the efficient use of hardware.


\subsubsection{Benefits of Virtualization}


\begin{enumerate}
    \item \textit{Resource Optimization:} Virtualization optimizes hardware utilization, leading to cost savings.
    \item \textit{Isolation:} It enhances security by isolating applications and workloads.
    \item \textit{Flexibility:} Virtualization provides flexibility in scaling resources up or down based on demand.
    \item \textit{Energy Efficiency:} Running multiple virtual machines on a single server reduces energy consumption.
    \item \textit{Snapshot and Cloning:} Virtualization allows for easy creation of snapshots and cloning for testing and backup purposes.
\end{enumerate}


\subsection{Drawbacks of Virtualization}

\begin{enumerate}
    \item \textit{Overhead:} Virtualization introduces some overhead due to the virtualization layer.
    \item \textit{Complexity:} Managing virtualized environments can be complex, requiring specialized skills.
    \item \textit{Dependency on Host:} Virtual machines are dependent on the stability and security of the host system.
    \item \textit{Licensing Costs:} Some virtualization solutions may involve licensing costs.
    \item \textit{Performance:} In certain high-performance scenarios, there may be a slight performance impact.
\end{enumerate}


\subsection{Steps or Procedure to follow for Installation}

\begin{figure}[H]
    \centering
    \includegraphics[width=.95\textwidth]{screenshots/1.png}
    \caption{}
\end{figure}
\begin{figure}[H]
    \centering
    \includegraphics[width=.95\textwidth]{screenshots/2.png}
    \caption{}
\end{figure}
\begin{figure}[H]
    \centering
    \includegraphics[width=.95\textwidth]{screenshots/3.png}
    \caption{}
\end{figure}
\begin{figure}[H]
    \centering
    \includegraphics[width=.95\textwidth]{screenshots/4.png}
    \caption{}
\end{figure}
\begin{figure}[H]
    \centering
    \includegraphics[width=.95\textwidth]{screenshots/5.png}
    \caption{}
\end{figure}
\begin{figure}[H]
    \centering
    \includegraphics[width=.95\textwidth]{screenshots/6.png}
    \caption{}
\end{figure}
\begin{figure}[H]
    \centering
    \includegraphics[width=.95\textwidth]{screenshots/7.png}
    \caption{}
\end{figure}
\begin{figure}[H]
    \centering
    \includegraphics[width=.95\textwidth]{screenshots/8.png}
    \caption{}
\end{figure}
\begin{figure}[H]
    \centering
    \includegraphics[width=.95\textwidth]{screenshots/9.png}
    \caption{}
\end{figure}
\begin{figure}[H]
    \centering
    \includegraphics[width=.95\textwidth]{screenshots/10.png}
    \caption{}
\end{figure}
\begin{figure}[H]
    \centering
    \includegraphics[width=.95\textwidth]{screenshots/11.png}
    \caption{}
\end{figure}
\begin{figure}[H]
    \centering
    \includegraphics[width=.95\textwidth]{screenshots/12.png}
    \caption{}
\end{figure}
\begin{figure}[H]
    \centering
    \includegraphics[width=.95\textwidth]{screenshots/13.png}
    \caption{}
\end{figure}
\begin{figure}[H]
    \centering
    \includegraphics[width=.95\textwidth]{screenshots/14.png}
    \caption{}
\end{figure}
\begin{figure}[H]
    \centering
    \includegraphics[width=.95\textwidth]{screenshots/15.png}
    \caption{}
\end{figure}
\begin{figure}[H]
    \centering
    \includegraphics[width=.95\textwidth]{screenshots/16.png}
    \caption{}
\end{figure}
\begin{figure}[H]
    \centering
    \includegraphics[width=.95\textwidth]{screenshots/17.png}
    \caption{}
\end{figure}
\begin{figure}[H]
    \centering
    \includegraphics[width=.95\textwidth]{screenshots/18.png}
    \caption{}
\end{figure}
\begin{figure}[H]
    \centering
    \includegraphics[width=.95\textwidth]{screenshots/19.png}
    \caption{}
\end{figure}
\begin{figure}[H]
    \centering
    \includegraphics[width=.95\textwidth]{screenshots/20.png}
    \caption{}
\end{figure}

\subsection{Features of Ubuntu Server}

\begin{enumerate}
    \item \textit{Lightweight:} Ubuntu Server is lightweight and optimized for server environments.
    \item \textit{Security:} Ubuntu Server provides robust security features and regular updates.
    \item \textit{Stability:} Ubuntu Server is known for its stability and reliability.
    \item \textit{Scalability and Flexibility:} Ubuntu Server is highly scalable and flexible, allowing for easy customization.
    \item \textit{Community Support:} Ubuntu Server has a large community of users and developers, providing support and resources.
    \item \textit{Cloud Integration:} Ubuntu Server integrates well with cloud platforms and services.
    \item \textit{Package Management:} Ubuntu Server uses the APT package management system for easy software installation and updates.
\end{enumerate}

\section{Platform}
\textbf{Operating System}: Windows 11 \\
\textbf{IDEs or Text Editors Used}: Visual Studio Code\\
% \textbf{Compilers or Interpreters}: Python 3.10.1\\

% \section{Code}
% \lstinputlisting[language=Python, caption="DSA Signature Validity using PyCrypto Library"]{../Programs/Assignment_7/dsa using lib.py}

\section{FAQs}

\begin{enumerate}
    \item \textbf{Explain the concept of Virtual Machine and How it differs from Physical Machine:}
          \begin{itemize}
              \item \textbf{Virtual Machine (VM):} A virtual machine is a software-based emulation of a physical computer that runs an operating system and applications. It operates in an isolated environment from the host system and can be configured with specific hardware resources.
              \item \textbf{Differences from Physical Machine:}
                    \begin{itemize}
                        \item \textbf{Hardware Independence:} VMs are not tied to specific physical hardware and can be easily moved or replicated across different host systems.
                        \item \textbf{Isolation:} Each VM operates independently from other VMs and the host system, providing a secure and isolated environment for running applications.
                        \item \textbf{Resource Allocation:} VMs can be allocated specific amounts of CPU, memory, storage, and network resources, allowing for efficient utilization and optimization of hardware resources.
                        \item \textbf{Snapshots and Cloning:} VMs support features like snapshots and cloning, allowing users to capture the state of a VM at a particular point in time and create identical copies for testing or backup purposes.
                    \end{itemize}
          \end{itemize}

    \item \textbf{Software or Packages for Ubuntu Server VM Experimentation:}
          \begin{itemize}
              \item \textbf{Apache Web Server:} For hosting websites or web applications.
              \item \textbf{MySQL or PostgreSQL:} Relational database management systems for storing and managing data.
              \item \textbf{Node.js or Python:} Programming languages and runtime environments for developing and running server-side applications.
              \item \textbf{Docker:} Containerization platform for packaging and deploying applications.
              \item \textbf{Git:} Version control system for managing project code.
              \item \textbf{OpenSSH:} Secure shell protocol for remote access and administration of the server.
              \item \textbf{Nginx:} Web server and reverse proxy for serving static content or load balancing.
          \end{itemize}

    \item \textbf{Configuration Options for Ubuntu Server VM in VMware Workstation:}
          \begin{itemize}
              \item \textbf{Hardware Resources:} Allocate CPU cores, RAM, disk space, and network adapters to the VM.
              \item \textbf{Operating System Installation:} Select the Ubuntu Server ISO image for installation.
              \item \textbf{Network Settings:} Choose between bridged, NAT, or host-only networking modes.
              \item \textbf{Storage Options:} Create virtual disks and configure storage settings such as disk type (e.g., SATA, SCSI), size, and location.
              \item \textbf{VM Hardware Compatibility:} Choose the hardware compatibility level for the VM, which determines the virtual hardware features available.
              \item \textbf{Integration Features:} Enable or disable features like VMware Tools, which enhance VM performance and integration with the host system.
          \end{itemize}

    \item \textbf{Utilization of Hardware Virtualization Technologies by VMware Workstation:}
          \begin{itemize}
              \item \textbf{Processor Virtualization:} VMware Workstation leverages hardware-assisted virtualization technologies such as Intel VT-x and AMD-V to offload virtualization tasks to the CPU, improving performance and efficiency.
              \item \textbf{Memory Management:} Hardware virtualization assists in efficiently managing memory resources, allowing VMs to access physical memory directly and reducing overhead associated with memory management.
              \item \textbf{I/O Virtualization:} Virtualization technologies optimize I/O operations by providing direct access to physical devices from within the VM, enhancing disk and network performance.
              \item \textbf{Virtualization Extensions:} VMware Workstation supports hardware virtualization extensions such as Intel VT-d and AMD-Vi for improved security and isolation of VMs.
          \end{itemize}

    \item \textbf{Handling Installation of Ubuntu Server OS in VMware Workstation:}
          \begin{itemize}
              \item \textbf{Create New Virtual Machine:} Use the New Virtual Machine Wizard to set up a new VM.
              \item \textbf{Select Guest Operating System:} Choose Ubuntu Server as the guest OS.
              \item \textbf{Allocate Resources:} Assign CPU cores, memory, and storage for the VM.
              \item \textbf{Configure Network:} Select network settings such as bridged, NAT, or host-only.
              \item \textbf{Install Ubuntu Server:} Mount the Ubuntu Server ISO image and boot the VM to begin the installation process.
              \item \textbf{Follow Installation Steps:} Proceed with the installation by following the on-screen prompts, including disk partitioning, user account setup, and package selection.
              \item \textbf{Install VMware Tools:} After installation, install VMware Tools to enhance VM performance and integration with the host system.
          \end{itemize}
\end{enumerate}

\section{Conclusion}
In this assignment, we have learnt about VMWare and how to use it. We have also learnt how to install Ubuntu Server on VMWare Workstation. This will help us in understanding virtualization and cloud computing concepts better.

\clearpage

\pagebreak
\begin{thebibliography}{}
    \bibitem{googlecloud}
    Google Cloud Platform. \textit{Google Cloud Platform}. Available at: \url{https://cloud.google.com/}.

    \bibitem{openshift}
    Red Hat. \textit{Red Hat OpenShift}. Available at: \url{https://www.openshift.com/}.

    \bibitem{azure}
    Microsoft Azure. \textit{Microsoft Azure}. Available at: \url{https://azure.microsoft.com/}.

    \bibitem{aws}
    Amazon Web
\end{thebibliography}

\end{document}