% This is a Basic Assignment Paper but with like Code and stuff allowed in it, there is also url, hyperlinks from contents included. 

\documentclass[11pt]{article}

% Preamble

\usepackage[margin=1in]{geometry}
\usepackage{amsfonts, amsmath, amssymb, amsthm}
\usepackage{fancyhdr, float, graphicx}
\usepackage[utf8]{inputenc} % Required for inputting international characters
\usepackage[T1]{fontenc} % Output font encoding for international characters
\usepackage{fouriernc} % Use the New Century Schoolbook font
\usepackage[nottoc, notlot, notlof]{tocbibind}
\usepackage{listings}
\usepackage{xcolor}
\usepackage{blindtext}
\usepackage{hyperref}
\definecolor{codepurple}{rgb}{0.58,0,0.82}
\hypersetup{
    colorlinks=true,
    linkcolor=black,
    filecolor=black,      
    urlcolor=codepurple,
    pdfpagemode=FullScreen,
    }

\definecolor{codegreen}{rgb}{0,0.6,0}
\definecolor{codegray}{rgb}{0.5,0.5,0.5}
\definecolor{backcolour}{rgb}{0.95,0.95,0.92}

\lstdefinestyle{mystyle}{
    backgroundcolor=\color{backcolour},   
    commentstyle=\color{codegreen},
    keywordstyle=\color{magenta},
    numberstyle=\tiny\color{codegray},
    stringstyle=\color{codepurple},
    basicstyle=\ttfamily\footnotesize,
    breakatwhitespace=false,         
    breaklines=true,                 
    captionpos=b,                    
    keepspaces=true,                 
    numbers=left,                    
    numbersep=5pt,                  
    showspaces=false,                
    showstringspaces=false,
    showtabs=false,                  
    tabsize=2
}

\lstset{style=mystyle}

% Header and Footer
\pagestyle{fancy}
\fancyhead{}
\fancyfoot{}
\fancyhead[L]{\textit{\Large{Digital Forensics and Investigation - TY. B. Tech}}}
\fancyhead[R]{\textit{Krishnaraj T}}
\fancyfoot[C]{\thepage}
\renewcommand{\footrulewidth}{1pt}
\newtheorem{thm}{Theorem}
\newtheorem{dfn}[thm]{Definition}


% Other Doc Editing
% \parindent 0ex
%\renewcommand{\baselinestretch}{1.5}

\begin{document}

\begin{titlepage}
    \centering

    %---------------------------NAMES-------------------------------

    \huge\textsc{
        MIT World Peace University
    }\\

    \vspace{0.75\baselineskip} % space after Uni Name

    \LARGE{
        Cloud Infrastructure and Security\\
        Third Year B. Tech, Semester 6
    }

    \vfill % space after Sub Name

    %--------------------------TITLE-------------------------------

    \rule{\textwidth}{1.6pt}\vspace*{-\baselineskip}\vspace*{2pt}
    \rule{\textwidth}{0.6pt}
    \vspace{0.75\baselineskip} % Whitespace above the title



    \huge{\textsc{
            Cloud Computing and Virtualization
        }} \\



    \vspace{0.5\baselineskip} % Whitespace below the title
    \rule{\textwidth}{0.6pt}\vspace*{-\baselineskip}\vspace*{2.8pt}
    \rule{\textwidth}{1.6pt}

    \vspace{1\baselineskip} % Whitespace after the title block

    %--------------------------SUBTITLE --------------------------	

    \LARGE\textsc{
        Study Assignment
    } % Subtitle or further description
    \vfill

    %--------------------------AUTHOR-------------------------------

    Prepared By
    \vspace{0.5\baselineskip} % Whitespace before the editors

    \Large{
        Krishnaraj Thadesar \\
        Cyber Security and Forensics\\
        Batch A1, PA 10
    }


    \vspace{0.5\baselineskip} % Whitespace below the editor list
    \today

\end{titlepage}


\tableofcontents
\thispagestyle{empty}
\clearpage

\setcounter{page}{1}

\section{Aim}
To learn about Cloud Computing and Virtualization.

\section{Objectives}
\begin{enumerate}
    \item To get familiar with cloud vendors and supported services. 
    \item To understand concept of virtualization. 
    \item To have a comparitive study of cloud services provided by different cloud vendors. 
\end{enumerate}

\section{Theory}

\subsection{Cloud}

\begin{dfn}
    \textbf{Cloud Platforms}: A cloud platform is a set of hardware and software components that are used to run applications and services in the cloud. It provides the runtime environment for applications and services that are deployed on top of it. A cloud platform also provides a set of services that are used by applications running on it. 
\end{dfn}

\subsection{Need for Cloud}
\begin{enumerate}
    \item \textbf{Cost Reduction} - Cloud computing is often cheaper than traditional hosting services. It reduces the need for physical hardware, maintenance, bandwidth, and other resources.
    \item \textbf{Scalability} - Cloud computing allows you to scale your resources to meet the demands of your business. You can scale your resources up or down depending on your needs.
    \item \textbf{Reliability} - Cloud computing is more reliable than traditional hosting services. It reduces the need for physical hardware, maintenance, bandwidth, and other resources.
    \item \textbf{Mobility} - Cloud computing allows you to scale your resources to meet the demands of your business. You can scale your resources up or down depending on your needs.
    \item \textbf{Performance} - Cloud computing is more reliable than traditional hosting services. It reduces the need for physical hardware, maintenance, bandwidth, and other resources.
    \item \textbf{Security} - Cloud computing allows you to scale your resources to meet the demands of your business. You can scale your resources up or down depending on your needs.
    \item \textbf{Control} - Cloud computing is more reliable than traditional hosting services. It reduces the need for physical hardware, maintenance, bandwidth, and other resources.
    \item \textbf{Competitiveness} - Cloud computing allows you to scale your resources to meet the demands of your business. You can scale your resources up or down depending on your needs.
\end{enumerate}

\section{Cloud Vendors}

Cloud vendors are companies that provide cloud computing services to businesses and individuals. They offer a variety of services, including infrastructure as a service (IaaS), platform as a service (PaaS), and software as a service (SaaS). Cloud vendors are also known as cloud service providers (CSPs).

\section{Cloud Services}

Cloud computing is a model for enabling ubiquitous, convenient, on-demand network access to a shared pool of configurable computing resources (e.g., networks, servers, storage, applications, and services) that can be rapidly provisioned and released with minimal management effort or service provider interaction.

\subsection{Infrastructure as a Service (IaaS)}
IaaS refers to a cloud computing model where virtualized computing resources are provided over the internet. In this model, users can rent virtual machines and other fundamental computing resources on a pay-as-you-go basis.

\textbf{Features:}
\begin{enumerate}
    \item \textit{Scalability:} IaaS allows for the dynamic scaling of resources, enabling users to adjust computing capacity based on demand.
    \item \textit{Virtualization:} It involves the use of virtualization technologies, such as virtual machines, storage, and networks.
    \item \textit{Self-Service:} Users have control over their infrastructure, being able to provision and manage resources independently.
    \item \textit{Cost-Efficiency:} Users pay for the resources they use, reducing the need for upfront capital investment.
    \item \textit{Network Connectivity:} IaaS providers offer robust networking options, allowing users to establish secure connections between resources.
\end{enumerate}

\subsection{Platform as a Service (PaaS)}
PaaS is a cloud computing model that provides a platform allowing customers to develop, run, and manage applications without dealing with the complexity of building and maintaining the underlying infrastructure.

\textbf{Features:}
\begin{enumerate}
    \item \textit{Development Tools:} PaaS provides a set of tools and services that streamline the application development process.
    \item \textit{Automated Deployment:} It automates application deployment, reducing the time and effort required for the deployment process.
    \item \textit{Scalability:} Similar to IaaS, PaaS offers scalability to accommodate varying workloads.
    \item \textit{Middleware Services:} PaaS often includes middleware services like databases, messaging systems, and caching.
    \item \textit{Collaboration:} PaaS facilitates collaboration among development teams through shared development environments and tools.
\end{enumerate}

\subsection{Software as a Service (SaaS)}
SaaS delivers software applications over the internet on a subscription basis, eliminating the need for users to install, maintain, and update the software locally.

\textbf{Features:}
\begin{enumerate}
    \item \textit{Accessibility:} SaaS applications are accessible from any device with an internet connection, promoting remote collaboration.
    \item \textit{Automatic Updates:} Users benefit from automatic software updates, ensuring they always have access to the latest features and security patches.
    \item \textit{Subscription Model:} SaaS follows a subscription-based pricing model, often reducing upfront costs for users.
    \item \textit{Multi-Tenancy:} Multiple users can access and use the same instance of the application without interference.
    \item \textit{Data Security:} SaaS providers implement robust security measures to protect user data, often including encryption and regular security audits.
\end{enumerate}


\section{Virtualization}

\begin{dfn}
    \textbf{Cloud Platforms}: A cloud platform is a set of hardware and software components that are used to run applications and services in the cloud. It provides the runtime environment for applications and services that are deployed on top of it. A cloud platform also provides a set of services that are used by applications running on it. 
\end{dfn}

\subsection{Need for Virtualization}
Virtualization is crucial in modern computing environments for several reasons.

\textbf{Reasons:}
\begin{enumerate}
    \item \textit{Resource Utilization:} Virtualization allows for efficient utilization of hardware resources by running multiple virtual machines on a single physical server.
    \item \textit{Isolation:} It provides a level of isolation between different virtual machines, enhancing security and preventing interference.
    \item \textit{Flexibility:} Virtualization enables the easy creation and deployment of virtual machines, providing flexibility in managing computing resources.
    \item \textit{Cost Reduction:} By consolidating servers through virtualization, organizations can reduce hardware and operational costs.
    \item \textit{Disaster Recovery:} Virtualization facilitates quick and efficient disaster recovery through the use of snapshots and backup copies of virtual machines.
\end{enumerate}

\subsection{Pros and Cons of Virtualization}
Understanding the advantages and disadvantages of virtualization is essential for informed decision-making.

\textbf{Pros:}
\begin{enumerate}
    \item \textit{Resource Optimization:} Virtualization optimizes hardware utilization, leading to cost savings.
    \item \textit{Isolation:} It enhances security by isolating applications and workloads.
    \item \textit{Flexibility:} Virtualization provides flexibility in scaling resources up or down based on demand.
    \item \textit{Energy Efficiency:} Running multiple virtual machines on a single server reduces energy consumption.
    \item \textit{Snapshot and Cloning:} Virtualization allows for easy creation of snapshots and cloning for testing and backup purposes.
\end{enumerate}

\textbf{Cons:}
\begin{enumerate}
    \item \textit{Overhead:} Virtualization introduces some overhead due to the virtualization layer.
    \item \textit{Complexity:} Managing virtualized environments can be complex, requiring specialized skills.
    \item \textit{Dependency on Host:} Virtual machines are dependent on the stability and security of the host system.
    \item \textit{Licensing Costs:} Some virtualization solutions may involve licensing costs.
    \item \textit{Performance:} In certain high-performance scenarios, there may be a slight performance impact.
\end{enumerate}

\section{Comparative Study of Cloud Vendors}

\subsection{Amazon Web Services (AWS)}
AWS is a leading cloud service provider, offering a comprehensive suite of services.

\textbf{Key Points:}
\begin{enumerate}
    \item \textit{Extensive Service Portfolio:} AWS provides a vast array of services, including computing power, storage, databases, machine learning, and more.
    \item \textit{Global Reach:} AWS has a widespread global infrastructure, ensuring low-latency access to resources from various regions.
    \item \textit{Market Dominance:} AWS is a market leader with a large customer base, making it a reliable choice for enterprises.
    \item \textit{Pricing Options:} AWS offers various pricing models, allowing users to choose between on-demand, reserved, and spot instances.
    \item \textit{Security and Compliance:} AWS adheres to stringent security standards and provides tools for compliance management.
\end{enumerate}

\subsection{Microsoft Azure}
Azure, Microsoft's cloud platform, offers a diverse set of services and integration with Microsoft products.

\textbf{Key Points:}
\begin{enumerate}
    \item \textit{Hybrid Cloud Capabilities:} Azure supports hybrid cloud deployments, integrating on-premises solutions with cloud services.
    \item \textit{Enterprise Integration:} Azure seamlessly integrates with Microsoft products like Windows Server, Active Directory, and SQL Server.
    \item \textit{AI and ML Services:} Azure provides advanced AI and machine learning services, empowering data-driven decision-making.
    \item \textit{Developer-Friendly:} Azure supports multiple programming languages and frameworks, making it accessible for developers.
    \item \textit{Global Presence:} Azure has a widespread global network of data centers, ensuring reliable and scalable services.
\end{enumerate}

\subsection{Google Cloud Platform (GCP)}
GCP is known for its strong focus on data analytics, machine learning, and open-source technologies.

\textbf{Key Points:}
\begin{enumerate}
    \item \textit{Data and Analytics:} GCP excels in data analytics, offering BigQuery and other powerful data processing tools.
    \item \textit{Machine Learning:} GCP provides robust machine learning and AI services, including TensorFlow and AutoML.
    \item \textit{Open Source Embrace:} GCP embraces open-source technologies, supporting a wide range of open-source projects.
    \item \textit{Networking Infrastructure:} GCP offers a high-performance global network infrastructure for low-latency access.
    \item \textit{Cost Management:} GCP provides transparent pricing with sustained use discounts and custom pricing options.
\end{enumerate}


\section{Platform}
\textbf{Operating System}: Windows 11 \\
\textbf{IDEs or Text Editors Used}: Visual Studio Code\\
% \textbf{Compilers or Interpreters}: Python 3.10.1\\

% \section{Code}
% \lstinputlisting[language=Python, caption="DSA Signature Validity using PyCrypto Library"]{../Programs/Assignment_7/dsa using lib.py}

\section{Conclusion}
In this assignment, we learned about Cloud Computing and Virtualization. We also learned about various cloud vendors and their services. We also learned about the need for virtualization and its pros and cons. Finally, we had a comparative study of cloud vendors.

\clearpage

\pagebreak
\begin{thebibliography}{}

    \bibitem{CloudConceptsOverview}
    Cloud Computing Concepts Overview.
    Accessed from: \url{https://www.ibm.com/cloud/learn/cloud-computing-concepts}

    \bibitem{VirtualizationBenefits}
    Benefits of Virtualization.
    Accessed from: \url{https://www.vmware.com/topics/glossary/content/virtualization}

\end{thebibliography}

\end{document}