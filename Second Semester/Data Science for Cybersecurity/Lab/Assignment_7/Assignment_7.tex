% This is a Basic Assignment Paper but with like Code and stuff allowed in it, there is also url, hyperlinks from contents included. 

\documentclass[11pt]{article}

% Preamble

\usepackage[margin=1in]{geometry}
\usepackage{amsfonts, amsmath, amssymb, amsthm}
\usepackage{fancyhdr, float, graphicx}
\usepackage[utf8]{inputenc} % Required for inputting international characters
\usepackage[T1]{fontenc} % Output font encoding for international characters
% \usepackage{fouriernc} % Use the New Century Schoolbook font
\usepackage[nottoc, notlot, notlof]{tocbibind}
\usepackage{listings}
\usepackage{xcolor}
\usepackage{blindtext}
\usepackage{hyperref}
\definecolor{codepurple}{rgb}{0.58,0,0.82}
\hypersetup{
    colorlinks=true,
    linkcolor=black,
    filecolor=black,      
    urlcolor=codepurple,
    pdfpagemode=FullScreen,
    }

\definecolor{codegreen}{rgb}{0,0.6,0}
\definecolor{codegray}{rgb}{0.5,0.5,0.5}
\definecolor{backcolour}{rgb}{0.95,0.95,0.92}

\lstdefinestyle{mystyle}{
    backgroundcolor=\color{backcolour},   
    commentstyle=\color{codegreen},
    keywordstyle=\color{magenta},
    numberstyle=\tiny\color{codegray},
    stringstyle=\color{codepurple},
    basicstyle=\ttfamily\footnotesize,
    breakatwhitespace=false,         
    breaklines=true,                 
    captionpos=b,                    
    keepspaces=true,                 
    numbers=left,                    
    numbersep=5pt,                  
    showspaces=false,                
    showstringspaces=false,
    showtabs=false,                  
    tabsize=2
}

\lstset{style=mystyle}

% Header and Footer
\pagestyle{fancy}
\fancyhead{}
\fancyfoot{}
\fancyhead[L]{\textit{\Large{Data Science for Cybersecurity and Forensics}}}
\fancyhead[R]{\textit{Krishnaraj T}}
\fancyfoot[C]{\thepage}
\renewcommand{\footrulewidth}{1pt}
\newtheorem{thm}{Theorem}
\newtheorem{dfn}[thm]{Definition}


\usepackage[breakable]{tcolorbox}
\usepackage{parskip} % Stop auto-indenting (to mimic markdown behaviour)


% Basic figure setup, for now with no caption control since it's done
% automatically by Pandoc (which extracts ![](path) syntax from Markdown).
\usepackage{graphicx}
% Keep aspect ratio if custom image width or height is specified
\setkeys{Gin}{keepaspectratio}
% Maintain compatibility with old templates. Remove in nbconvert 6.0
\let\Oldincludegraphics\includegraphics
% Ensure that by default, figures have no caption (until we provide a
% proper Figure object with a Caption API and a way to capture that
% in the conversion process - todo).
% \usepackage{caption}
% \DeclareCaptionFormat{nocaption}{}
% \captionsetup{format=nocaption,aboveskip=0pt,belowskip=0pt}

\usepackage{float}
\floatplacement{figure}{H} % forces figures to be placed at the correct location
\usepackage{xcolor} % Allow colors to be defined
\usepackage{enumerate} % Needed for markdown enumerations to work
\usepackage{geometry} % Used to adjust the document margins
\usepackage{amsmath} % Equations
\usepackage{amssymb} % Equations
\usepackage{textcomp} % defines textquotesingle
% Hack from http://tex.stackexchange.com/a/47451/13684:
\AtBeginDocument{%
    \def\PYZsq{\textquotesingle}% Upright quotes in Pygmentized code
}
\usepackage{upquote} % Upright quotes for verbatim code
\usepackage{eurosym} % defines \euro

\usepackage{iftex}
\ifPDFTeX
    \usepackage[T1]{fontenc}
    \IfFileExists{alphabeta.sty}{
            \usepackage{alphabeta}
        }{
            \usepackage[mathletters]{ucs}
            \usepackage[utf8x]{inputenc}
        }
\else
    \usepackage{fontspec}
    \usepackage{unicode-math}
\fi

\usepackage{fancyvrb} % verbatim replacement that allows latex
\usepackage{grffile} % extends the file name processing of package graphics
                        % to support a larger range
\makeatletter % fix for old versions of grffile with XeLaTeX
\@ifpackagelater{grffile}{2019/11/01}
{
    % Do nothing on new versions
}
{
    \def\Gread@@xetex#1{%
    \IfFileExists{"\Gin@base".bb}%
    {\Gread@eps{\Gin@base.bb}}%
    {\Gread@@xetex@aux#1}%
    }
}
\makeatother
\usepackage[Export]{adjustbox} % Used to constrain images to a maximum size
\adjustboxset{max size={0.9\linewidth}{0.9\paperheight}}

% The hyperref package gives us a pdf with properly built
% internal navigation ('pdf bookmarks' for the table of contents,
% internal cross-reference links, web links for URLs, etc.)
\usepackage{hyperref}
% The default LaTeX title has an obnoxious amount of whitespace. By default,
% titling removes some of it. It also provides customization options.
\usepackage{titling}
\usepackage{longtable} % longtable support required by pandoc >1.10
\usepackage{booktabs}  % table support for pandoc > 1.12.2
\usepackage{array}     % table support for pandoc >= 2.11.3
\usepackage{calc}      % table minipage width calculation for pandoc >= 2.11.1
\usepackage[inline]{enumitem} % IRkernel/repr support (it uses the enumerate* environment)
\usepackage[normalem]{ulem} % ulem is needed to support strikethroughs (\sout)
                            % normalem makes italics be italics, not underlines
\usepackage{soul}      % strikethrough (\st) support for pandoc >= 3.0.0
\usepackage{mathrsfs}



% Colors for the hyperref package
\definecolor{urlcolor}{rgb}{0,.145,.698}
\definecolor{linkcolor}{rgb}{.71,0.21,0.01}
\definecolor{citecolor}{rgb}{.12,.54,.11}

% ANSI colors
\definecolor{ansi-black}{HTML}{3E424D}
\definecolor{ansi-black-intense}{HTML}{282C36}
\definecolor{ansi-red}{HTML}{E75C58}
\definecolor{ansi-red-intense}{HTML}{B22B31}
\definecolor{ansi-green}{HTML}{00A250}
\definecolor{ansi-green-intense}{HTML}{007427}
\definecolor{ansi-yellow}{HTML}{DDB62B}
\definecolor{ansi-yellow-intense}{HTML}{B27D12}
\definecolor{ansi-blue}{HTML}{208FFB}
\definecolor{ansi-blue-intense}{HTML}{0065CA}
\definecolor{ansi-magenta}{HTML}{D160C4}
\definecolor{ansi-magenta-intense}{HTML}{A03196}
\definecolor{ansi-cyan}{HTML}{60C6C8}
\definecolor{ansi-cyan-intense}{HTML}{258F8F}
\definecolor{ansi-white}{HTML}{C5C1B4}
\definecolor{ansi-white-intense}{HTML}{A1A6B2}
\definecolor{ansi-default-inverse-fg}{HTML}{FFFFFF}
\definecolor{ansi-default-inverse-bg}{HTML}{000000}

% common color for the border for error outputs.
\definecolor{outerrorbackground}{HTML}{FFDFDF}

% commands and environments needed by pandoc snippets
% extracted from the output of `pandoc -s`
\providecommand{\tightlist}{%
    \setlength{\itemsep}{0pt}\setlength{\parskip}{0pt}}
\DefineVerbatimEnvironment{Highlighting}{Verbatim}{commandchars=\\\{\}}
% Add ',fontsize=\small' for more characters per line
\newenvironment{Shaded}{}{}
\newcommand{\KeywordTok}[1]{\textcolor[rgb]{0.00,0.44,0.13}{\textbf{{#1}}}}
\newcommand{\DataTypeTok}[1]{\textcolor[rgb]{0.56,0.13,0.00}{{#1}}}
\newcommand{\DecValTok}[1]{\textcolor[rgb]{0.25,0.63,0.44}{{#1}}}
\newcommand{\BaseNTok}[1]{\textcolor[rgb]{0.25,0.63,0.44}{{#1}}}
\newcommand{\FloatTok}[1]{\textcolor[rgb]{0.25,0.63,0.44}{{#1}}}
\newcommand{\CharTok}[1]{\textcolor[rgb]{0.25,0.44,0.63}{{#1}}}
\newcommand{\StringTok}[1]{\textcolor[rgb]{0.25,0.44,0.63}{{#1}}}
\newcommand{\CommentTok}[1]{\textcolor[rgb]{0.38,0.63,0.69}{\textit{{#1}}}}
\newcommand{\OtherTok}[1]{\textcolor[rgb]{0.00,0.44,0.13}{{#1}}}
\newcommand{\AlertTok}[1]{\textcolor[rgb]{1.00,0.00,0.00}{\textbf{{#1}}}}
\newcommand{\FunctionTok}[1]{\textcolor[rgb]{0.02,0.16,0.49}{{#1}}}
\newcommand{\RegionMarkerTok}[1]{{#1}}
\newcommand{\ErrorTok}[1]{\textcolor[rgb]{1.00,0.00,0.00}{\textbf{{#1}}}}
\newcommand{\NormalTok}[1]{{#1}}

% Additional commands for more recent versions of Pandoc
\newcommand{\ConstantTok}[1]{\textcolor[rgb]{0.53,0.00,0.00}{{#1}}}
\newcommand{\SpecialCharTok}[1]{\textcolor[rgb]{0.25,0.44,0.63}{{#1}}}
\newcommand{\VerbatimStringTok}[1]{\textcolor[rgb]{0.25,0.44,0.63}{{#1}}}
\newcommand{\SpecialStringTok}[1]{\textcolor[rgb]{0.73,0.40,0.53}{{#1}}}
\newcommand{\ImportTok}[1]{{#1}}
\newcommand{\DocumentationTok}[1]{\textcolor[rgb]{0.73,0.13,0.13}{\textit{{#1}}}}
\newcommand{\AnnotationTok}[1]{\textcolor[rgb]{0.38,0.63,0.69}{\textbf{\textit{{#1}}}}}
\newcommand{\CommentVarTok}[1]{\textcolor[rgb]{0.38,0.63,0.69}{\textbf{\textit{{#1}}}}}
\newcommand{\VariableTok}[1]{\textcolor[rgb]{0.10,0.09,0.49}{{#1}}}
\newcommand{\ControlFlowTok}[1]{\textcolor[rgb]{0.00,0.44,0.13}{\textbf{{#1}}}}
\newcommand{\OperatorTok}[1]{\textcolor[rgb]{0.40,0.40,0.40}{{#1}}}
\newcommand{\BuiltInTok}[1]{{#1}}
\newcommand{\ExtensionTok}[1]{{#1}}
\newcommand{\PreprocessorTok}[1]{\textcolor[rgb]{0.74,0.48,0.00}{{#1}}}
\newcommand{\AttributeTok}[1]{\textcolor[rgb]{0.49,0.56,0.16}{{#1}}}
\newcommand{\InformationTok}[1]{\textcolor[rgb]{0.38,0.63,0.69}{\textbf{\textit{{#1}}}}}
\newcommand{\WarningTok}[1]{\textcolor[rgb]{0.38,0.63,0.69}{\textbf{\textit{{#1}}}}}


% Define a nice break command that doesn't care if a line doesn't already
% exist.
\def\br{\hspace*{\fill} \\* }
% Math Jax compatibility definitions
\def\gt{>}
\def\lt{<}
\let\Oldtex\TeX
\let\Oldlatex\LaTeX
\renewcommand{\TeX}{\textrm{\Oldtex}}
\renewcommand{\LaTeX}{\textrm{\Oldlatex}}  
    
    
    
    
    
    
% Pygments definitions
\makeatletter
\def\PY@reset{\let\PY@it=\relax \let\PY@bf=\relax%
    \let\PY@ul=\relax \let\PY@tc=\relax%
    \let\PY@bc=\relax \let\PY@ff=\relax}
\def\PY@tok#1{\csname PY@tok@#1\endcsname}
\def\PY@toks#1+{\ifx\relax#1\empty\else%
    \PY@tok{#1}\expandafter\PY@toks\fi}
\def\PY@do#1{\PY@bc{\PY@tc{\PY@ul{%
    \PY@it{\PY@bf{\PY@ff{#1}}}}}}}
\def\PY#1#2{\PY@reset\PY@toks#1+\relax+\PY@do{#2}}

\@namedef{PY@tok@w}{\def\PY@tc##1{\textcolor[rgb]{0.73,0.73,0.73}{##1}}}
\@namedef{PY@tok@c}{\let\PY@it=\textit\def\PY@tc##1{\textcolor[rgb]{0.24,0.48,0.48}{##1}}}
\@namedef{PY@tok@cp}{\def\PY@tc##1{\textcolor[rgb]{0.61,0.40,0.00}{##1}}}
\@namedef{PY@tok@k}{\let\PY@bf=\textbf\def\PY@tc##1{\textcolor[rgb]{0.00,0.50,0.00}{##1}}}
\@namedef{PY@tok@kp}{\def\PY@tc##1{\textcolor[rgb]{0.00,0.50,0.00}{##1}}}
\@namedef{PY@tok@kt}{\def\PY@tc##1{\textcolor[rgb]{0.69,0.00,0.25}{##1}}}
\@namedef{PY@tok@o}{\def\PY@tc##1{\textcolor[rgb]{0.40,0.40,0.40}{##1}}}
\@namedef{PY@tok@ow}{\let\PY@bf=\textbf\def\PY@tc##1{\textcolor[rgb]{0.67,0.13,1.00}{##1}}}
\@namedef{PY@tok@nb}{\def\PY@tc##1{\textcolor[rgb]{0.00,0.50,0.00}{##1}}}
\@namedef{PY@tok@nf}{\def\PY@tc##1{\textcolor[rgb]{0.00,0.00,1.00}{##1}}}
\@namedef{PY@tok@nc}{\let\PY@bf=\textbf\def\PY@tc##1{\textcolor[rgb]{0.00,0.00,1.00}{##1}}}
\@namedef{PY@tok@nn}{\let\PY@bf=\textbf\def\PY@tc##1{\textcolor[rgb]{0.00,0.00,1.00}{##1}}}
\@namedef{PY@tok@ne}{\let\PY@bf=\textbf\def\PY@tc##1{\textcolor[rgb]{0.80,0.25,0.22}{##1}}}
\@namedef{PY@tok@nv}{\def\PY@tc##1{\textcolor[rgb]{0.10,0.09,0.49}{##1}}}
\@namedef{PY@tok@no}{\def\PY@tc##1{\textcolor[rgb]{0.53,0.00,0.00}{##1}}}
\@namedef{PY@tok@nl}{\def\PY@tc##1{\textcolor[rgb]{0.46,0.46,0.00}{##1}}}
\@namedef{PY@tok@ni}{\let\PY@bf=\textbf\def\PY@tc##1{\textcolor[rgb]{0.44,0.44,0.44}{##1}}}
\@namedef{PY@tok@na}{\def\PY@tc##1{\textcolor[rgb]{0.41,0.47,0.13}{##1}}}
\@namedef{PY@tok@nt}{\let\PY@bf=\textbf\def\PY@tc##1{\textcolor[rgb]{0.00,0.50,0.00}{##1}}}
\@namedef{PY@tok@nd}{\def\PY@tc##1{\textcolor[rgb]{0.67,0.13,1.00}{##1}}}
\@namedef{PY@tok@s}{\def\PY@tc##1{\textcolor[rgb]{0.73,0.13,0.13}{##1}}}
\@namedef{PY@tok@sd}{\let\PY@it=\textit\def\PY@tc##1{\textcolor[rgb]{0.73,0.13,0.13}{##1}}}
\@namedef{PY@tok@si}{\let\PY@bf=\textbf\def\PY@tc##1{\textcolor[rgb]{0.64,0.35,0.47}{##1}}}
\@namedef{PY@tok@se}{\let\PY@bf=\textbf\def\PY@tc##1{\textcolor[rgb]{0.67,0.36,0.12}{##1}}}
\@namedef{PY@tok@sr}{\def\PY@tc##1{\textcolor[rgb]{0.64,0.35,0.47}{##1}}}
\@namedef{PY@tok@ss}{\def\PY@tc##1{\textcolor[rgb]{0.10,0.09,0.49}{##1}}}
\@namedef{PY@tok@sx}{\def\PY@tc##1{\textcolor[rgb]{0.00,0.50,0.00}{##1}}}
\@namedef{PY@tok@m}{\def\PY@tc##1{\textcolor[rgb]{0.40,0.40,0.40}{##1}}}
\@namedef{PY@tok@gh}{\let\PY@bf=\textbf\def\PY@tc##1{\textcolor[rgb]{0.00,0.00,0.50}{##1}}}
\@namedef{PY@tok@gu}{\let\PY@bf=\textbf\def\PY@tc##1{\textcolor[rgb]{0.50,0.00,0.50}{##1}}}
\@namedef{PY@tok@gd}{\def\PY@tc##1{\textcolor[rgb]{0.63,0.00,0.00}{##1}}}
\@namedef{PY@tok@gi}{\def\PY@tc##1{\textcolor[rgb]{0.00,0.52,0.00}{##1}}}
\@namedef{PY@tok@gr}{\def\PY@tc##1{\textcolor[rgb]{0.89,0.00,0.00}{##1}}}
\@namedef{PY@tok@ge}{\let\PY@it=\textit}
\@namedef{PY@tok@gs}{\let\PY@bf=\textbf}
\@namedef{PY@tok@ges}{\let\PY@bf=\textbf\let\PY@it=\textit}
\@namedef{PY@tok@gp}{\let\PY@bf=\textbf\def\PY@tc##1{\textcolor[rgb]{0.00,0.00,0.50}{##1}}}
\@namedef{PY@tok@go}{\def\PY@tc##1{\textcolor[rgb]{0.44,0.44,0.44}{##1}}}
\@namedef{PY@tok@gt}{\def\PY@tc##1{\textcolor[rgb]{0.00,0.27,0.87}{##1}}}
\@namedef{PY@tok@err}{\def\PY@bc##1{{\setlength{\fboxsep}{\string -\fboxrule}\fcolorbox[rgb]{1.00,0.00,0.00}{1,1,1}{\strut ##1}}}}
\@namedef{PY@tok@kc}{\let\PY@bf=\textbf\def\PY@tc##1{\textcolor[rgb]{0.00,0.50,0.00}{##1}}}
\@namedef{PY@tok@kd}{\let\PY@bf=\textbf\def\PY@tc##1{\textcolor[rgb]{0.00,0.50,0.00}{##1}}}
\@namedef{PY@tok@kn}{\let\PY@bf=\textbf\def\PY@tc##1{\textcolor[rgb]{0.00,0.50,0.00}{##1}}}
\@namedef{PY@tok@kr}{\let\PY@bf=\textbf\def\PY@tc##1{\textcolor[rgb]{0.00,0.50,0.00}{##1}}}
\@namedef{PY@tok@bp}{\def\PY@tc##1{\textcolor[rgb]{0.00,0.50,0.00}{##1}}}
\@namedef{PY@tok@fm}{\def\PY@tc##1{\textcolor[rgb]{0.00,0.00,1.00}{##1}}}
\@namedef{PY@tok@vc}{\def\PY@tc##1{\textcolor[rgb]{0.10,0.09,0.49}{##1}}}
\@namedef{PY@tok@vg}{\def\PY@tc##1{\textcolor[rgb]{0.10,0.09,0.49}{##1}}}
\@namedef{PY@tok@vi}{\def\PY@tc##1{\textcolor[rgb]{0.10,0.09,0.49}{##1}}}
\@namedef{PY@tok@vm}{\def\PY@tc##1{\textcolor[rgb]{0.10,0.09,0.49}{##1}}}
\@namedef{PY@tok@sa}{\def\PY@tc##1{\textcolor[rgb]{0.73,0.13,0.13}{##1}}}
\@namedef{PY@tok@sb}{\def\PY@tc##1{\textcolor[rgb]{0.73,0.13,0.13}{##1}}}
\@namedef{PY@tok@sc}{\def\PY@tc##1{\textcolor[rgb]{0.73,0.13,0.13}{##1}}}
\@namedef{PY@tok@dl}{\def\PY@tc##1{\textcolor[rgb]{0.73,0.13,0.13}{##1}}}
\@namedef{PY@tok@s2}{\def\PY@tc##1{\textcolor[rgb]{0.73,0.13,0.13}{##1}}}
\@namedef{PY@tok@sh}{\def\PY@tc##1{\textcolor[rgb]{0.73,0.13,0.13}{##1}}}
\@namedef{PY@tok@s1}{\def\PY@tc##1{\textcolor[rgb]{0.73,0.13,0.13}{##1}}}
\@namedef{PY@tok@mb}{\def\PY@tc##1{\textcolor[rgb]{0.40,0.40,0.40}{##1}}}
\@namedef{PY@tok@mf}{\def\PY@tc##1{\textcolor[rgb]{0.40,0.40,0.40}{##1}}}
\@namedef{PY@tok@mh}{\def\PY@tc##1{\textcolor[rgb]{0.40,0.40,0.40}{##1}}}
\@namedef{PY@tok@mi}{\def\PY@tc##1{\textcolor[rgb]{0.40,0.40,0.40}{##1}}}
\@namedef{PY@tok@il}{\def\PY@tc##1{\textcolor[rgb]{0.40,0.40,0.40}{##1}}}
\@namedef{PY@tok@mo}{\def\PY@tc##1{\textcolor[rgb]{0.40,0.40,0.40}{##1}}}
\@namedef{PY@tok@ch}{\let\PY@it=\textit\def\PY@tc##1{\textcolor[rgb]{0.24,0.48,0.48}{##1}}}
\@namedef{PY@tok@cm}{\let\PY@it=\textit\def\PY@tc##1{\textcolor[rgb]{0.24,0.48,0.48}{##1}}}
\@namedef{PY@tok@cpf}{\let\PY@it=\textit\def\PY@tc##1{\textcolor[rgb]{0.24,0.48,0.48}{##1}}}
\@namedef{PY@tok@c1}{\let\PY@it=\textit\def\PY@tc##1{\textcolor[rgb]{0.24,0.48,0.48}{##1}}}
\@namedef{PY@tok@cs}{\let\PY@it=\textit\def\PY@tc##1{\textcolor[rgb]{0.24,0.48,0.48}{##1}}}

\def\PYZbs{\char`\\}
\def\PYZus{\char`\_}
\def\PYZob{\char`\{}
\def\PYZcb{\char`\}}
\def\PYZca{\char`\^}
\def\PYZam{\char`\&}
\def\PYZlt{\char`\<}
\def\PYZgt{\char`\>}
\def\PYZsh{\char`\#}
\def\PYZpc{\char`\%}
\def\PYZdl{\char`\$}
\def\PYZhy{\char`\-}
\def\PYZsq{\char`\'}
\def\PYZdq{\char`\"}
\def\PYZti{\char`\~}
% for compatibility with earlier versions
\def\PYZat{@}
\def\PYZlb{[}
\def\PYZrb{]}
\makeatother


    % For linebreaks inside Verbatim environment from package fancyvrb.
    \makeatletter
        \newbox\Wrappedcontinuationbox
        \newbox\Wrappedvisiblespacebox
        \newcommand*\Wrappedvisiblespace {\textcolor{red}{\textvisiblespace}}
        \newcommand*\Wrappedcontinuationsymbol {\textcolor{red}{\llap{\tiny$\m@th\hookrightarrow$}}}
        \newcommand*\Wrappedcontinuationindent {3ex }
        \newcommand*\Wrappedafterbreak {\kern\Wrappedcontinuationindent\copy\Wrappedcontinuationbox}
        % Take advantage of the already applied Pygments mark-up to insert
        % potential linebreaks for TeX processing.
        %        {, <, #, %, $, ' and ": go to next line.
        %        _, }, ^, &, >, - and ~: stay at end of broken line.
        % Use of \textquotesingle for straight quote.
        \newcommand*\Wrappedbreaksatspecials {%
            \def\PYGZus{\discretionary{\char`\_}{\Wrappedafterbreak}{\char`\_}}%
            \def\PYGZob{\discretionary{}{\Wrappedafterbreak\char`\{}{\char`\{}}%
            \def\PYGZcb{\discretionary{\char`\}}{\Wrappedafterbreak}{\char`\}}}%
            \def\PYGZca{\discretionary{\char`\^}{\Wrappedafterbreak}{\char`\^}}%
            \def\PYGZam{\discretionary{\char`\&}{\Wrappedafterbreak}{\char`\&}}%
            \def\PYGZlt{\discretionary{}{\Wrappedafterbreak\char`\<}{\char`\<}}%
            \def\PYGZgt{\discretionary{\char`\>}{\Wrappedafterbreak}{\char`\>}}%
            \def\PYGZsh{\discretionary{}{\Wrappedafterbreak\char`\#}{\char`\#}}%
            \def\PYGZpc{\discretionary{}{\Wrappedafterbreak\char`\%}{\char`\%}}%
            \def\PYGZdl{\discretionary{}{\Wrappedafterbreak\char`\$}{\char`\$}}%
            \def\PYGZhy{\discretionary{\char`\-}{\Wrappedafterbreak}{\char`\-}}%
            \def\PYGZsq{\discretionary{}{\Wrappedafterbreak\textquotesingle}{\textquotesingle}}%
            \def\PYGZdq{\discretionary{}{\Wrappedafterbreak\char`\"}{\char`\"}}%
            \def\PYGZti{\discretionary{\char`\~}{\Wrappedafterbreak}{\char`\~}}%
        }
        % Some characters . , ; ? ! / are not pygmentized.
        % This macro makes them "active" and they will insert potential linebreaks
        \newcommand*\Wrappedbreaksatpunct {%
            \lccode`\~`\.\lowercase{\def~}{\discretionary{\hbox{\char`\.}}{\Wrappedafterbreak}{\hbox{\char`\.}}}%
            \lccode`\~`\,\lowercase{\def~}{\discretionary{\hbox{\char`\,}}{\Wrappedafterbreak}{\hbox{\char`\,}}}%
            \lccode`\~`\;\lowercase{\def~}{\discretionary{\hbox{\char`\;}}{\Wrappedafterbreak}{\hbox{\char`\;}}}%
            \lccode`\~`\:\lowercase{\def~}{\discretionary{\hbox{\char`\:}}{\Wrappedafterbreak}{\hbox{\char`\:}}}%
            \lccode`\~`\?\lowercase{\def~}{\discretionary{\hbox{\char`\?}}{\Wrappedafterbreak}{\hbox{\char`\?}}}%
            \lccode`\~`\!\lowercase{\def~}{\discretionary{\hbox{\char`\!}}{\Wrappedafterbreak}{\hbox{\char`\!}}}%
            \lccode`\~`\/\lowercase{\def~}{\discretionary{\hbox{\char`\/}}{\Wrappedafterbreak}{\hbox{\char`\/}}}%
            \catcode`\.\active
            \catcode`\,\active
            \catcode`\;\active
            \catcode`\:\active
            \catcode`\?\active
            \catcode`\!\active
            \catcode`\/\active
            \lccode`\~`\~
        }
    \makeatother

    \let\OriginalVerbatim=\Verbatim
    \makeatletter
    \renewcommand{\Verbatim}[1][1]{%
        %\parskip\z@skip
        \sbox\Wrappedcontinuationbox {\Wrappedcontinuationsymbol}%
        \sbox\Wrappedvisiblespacebox {\FV@SetupFont\Wrappedvisiblespace}%
        \def\FancyVerbFormatLine ##1{\hsize\linewidth
            \vtop{\raggedright\hyphenpenalty\z@\exhyphenpenalty\z@
                \doublehyphendemerits\z@\finalhyphendemerits\z@
                \strut ##1\strut}%
        }%
        % If the linebreak is at a space, the latter will be displayed as visible
        % space at end of first line, and a continuation symbol starts next line.
        % Stretch/shrink are however usually zero for typewriter font.
        \def\FV@Space {%
            \nobreak\hskip\z@ plus\fontdimen3\font minus\fontdimen4\font
            \discretionary{\copy\Wrappedvisiblespacebox}{\Wrappedafterbreak}
            {\kern\fontdimen2\font}%
        }%

        % Allow breaks at special characters using \PYG... macros.
        \Wrappedbreaksatspecials
        % Breaks at punctuation characters . , ; ? ! and / need catcode=\active
        \OriginalVerbatim[#1,codes*=\Wrappedbreaksatpunct]%
    }
    \makeatother

    % Exact colors from NB
    \definecolor{incolor}{HTML}{303F9F}
    \definecolor{outcolor}{HTML}{D84315}
    \definecolor{cellborder}{HTML}{CFCFCF}
    \definecolor{cellbackground}{HTML}{F7F7F7}

    % prompt
    \makeatletter
    \newcommand{\boxspacing}{\kern\kvtcb@left@rule\kern\kvtcb@boxsep}
    \makeatother
    \newcommand{\prompt}[4]{
        {\ttfamily\llap{{\color{#2}[#3]:\hspace{3pt}#4}}\vspace{-\baselineskip}}
    }
    

    
    % Prevent overflowing lines due to hard-to-break entities
    \sloppy
    % Setup hyperref package
    % \hypersetup{
    %   breaklinks=true,  % so long urls are correctly broken across lines
    %   colorlinks=true,
    %   urlcolor=urlcolor,
    %   linkcolor=linkcolor,
    %   citecolor=citecolor,
    %   }
    % Slightly bigger margins than the latex defaults
    
    \geometry{verbose,tmargin=1in,bmargin=1in,lmargin=1in,rmargin=1in}
    
  


% Other Doc Editing
% \parindent 0ex
%\renewcommand{\baselinestretch}{1.5}

\begin{document}

\begin{titlepage}
    \centering

    %---------------------------NAMES-------------------------------

    \huge\textsc{
        MIT World Peace University
    }\\

    \vspace{0.75\baselineskip} % space after Uni Name

    \LARGE{
        Data Science for Cybersecurity and Forensics\\
        Third Year B. Tech, Semester 6
    }

    \vfill % space after Sub Name

    %--------------------------TITLE-------------------------------

    \rule{\textwidth}{1.6pt}\vspace*{-\baselineskip}\vspace*{2pt}
    \rule{\textwidth}{0.6pt}
    \vspace{0.75\baselineskip} % Whitespace above the title



    \huge{\textsc{
            Implementation of K-Means Clustering in Python
        }} \\



    \vspace{0.5\baselineskip} % Whitespace below the title
    \rule{\textwidth}{0.6pt}\vspace*{-\baselineskip}\vspace*{2.8pt}
    \rule{\textwidth}{1.6pt}

    \vspace{1\baselineskip} % Whitespace after the title block

    %--------------------------SUBTITLE --------------------------	

    \LARGE\textsc{
        Assignment 7
    } % Subtitle or further description
    \vfill

    %--------------------------AUTHOR-------------------------------

    Prepared By
    \vspace{0.5\baselineskip} % Whitespace before the editors

    \Large{
        Krishnaraj Thadesar \\
        Cyber Security and Forensics\\
        Batch A1, PA 10
    }


    \vspace{0.5\baselineskip} % Whitespace below the editor list
    \today

\end{titlepage}


\tableofcontents
\thispagestyle{empty}
\clearpage

\setcounter{page}{1}

\section{Aim}
Implement K-Means Clustering in Python using IOT Based Attacks dataset.

\section{Objectives}
\begin{enumerate}
    \item To understand the concept of K-Means Clustering.
    \item To implement K-Means Clustering in Python.
\end{enumerate}

\section{Theory}
\subsection{K-Means Clustering}

K-Means clustering is a popular unsupervised machine learning algorithm used for partitioning a dataset into K distinct clusters. The algorithm aims to minimize the variance within each cluster and maximize the variance between clusters.

\begin{figure}[H]
    \centering
    \includegraphics[width=.85\textwidth]{kmeans.jpg}
    \caption{K Means Clustering Example}
\end{figure}
\subsection{K-Means Clustering Algorithm}

The K-Means clustering algorithm can be summarized in the following steps:

\begin{enumerate}
    \item \textbf{Initialization}: Randomly initialize K cluster centroids.
    \item \textbf{Assign Data Points to Nearest Centroid}: Assign each data point to the cluster with the nearest centroid.
    \item \textbf{Update Centroids}: Recalculate the centroids of each cluster based on the mean of the data points assigned to that cluster.
    \item \textbf{Repeat Steps 2 and 3}: Iterate the process of assigning data points to clusters and updating centroids until convergence.
    \item \textbf{Convergence Criteria}: The algorithm converges when the centroids no longer change significantly between iterations or when a specified number of iterations is reached.
\end{enumerate}

The objective function of K-Means clustering can be defined as minimizing the within-cluster sum of squared distances:

\[
    \underset{C}{\operatorname{argmin}} \sum_{i=1}^{K} \sum_{x \in C_{i}}\|x - \mu_{i}\|^2
\]

Where:
- \(C\) represents the set of clusters.
- \(C_{i}\) represents the data points assigned to cluster \(i\).
- \(\mu_{i}\) represents the centroid of cluster \(i\).
- \(\|\cdot\|\) denotes the Euclidean distance.

The algorithm converges to a locally optimal solution, and the quality of the clustering depends on the initial placement of centroids and the choice of K. K-Means is efficient and scalable for large datasets but may converge to suboptimal solutions depending on the initial centroids and data distribution.

\section{Procedure}
\begin{enumerate}
    \item Import the required python packages.
    \item Load the dataset.
    \item Data analysis.
    \item Split the dataset into dependent/independent variables.
    \item Split data into Train/Test sets.
    \item Train the regression model.
    \item Predict the result.
\end{enumerate}

\section{Platform}
\textbf{Operating System}: Windows 11 \\
\textbf{IDEs or Text Editors Used}: Visual Studio Code\\
\textbf{Compilers or Interpreters}: Python 3.10.1\\

\section{Requirements}
\begin{lstlisting}
python==3.10.1
matplotlib==3.8.3
numpy==1.26.4
pandas==2.2.2
seaborn==0.13.2
\end{lstlisting}


\section{Code}

    
    \begin{tcolorbox}[breakable, size=fbox, boxrule=1pt, pad at break*=1mm,colback=cellbackground, colframe=cellborder]
\prompt{In}{incolor}{95}{\boxspacing}
\begin{Verbatim}[commandchars=\\\{\}]
\PY{c+c1}{\PYZsh{} implementing k means clustering on iot based attacks dataset}
\PY{c+c1}{\PYZsh{} importing required libraries}
\PY{k+kn}{import} \PY{n+nn}{pandas} \PY{k}{as} \PY{n+nn}{pd}
\PY{k+kn}{import} \PY{n+nn}{numpy} \PY{k}{as} \PY{n+nn}{np}
\PY{k+kn}{import} \PY{n+nn}{matplotlib}\PY{n+nn}{.}\PY{n+nn}{pyplot} \PY{k}{as} \PY{n+nn}{plt}
\PY{k+kn}{from} \PY{n+nn}{sklearn}\PY{n+nn}{.}\PY{n+nn}{cluster} \PY{k+kn}{import} \PY{n}{KMeans}
\PY{k+kn}{from} \PY{n+nn}{sklearn}\PY{n+nn}{.}\PY{n+nn}{preprocessing} \PY{k+kn}{import} \PY{n}{StandardScaler}
\end{Verbatim}
\end{tcolorbox}

    \begin{tcolorbox}[breakable, size=fbox, boxrule=1pt, pad at break*=1mm,colback=cellbackground, colframe=cellborder]
\prompt{In}{incolor}{96}{\boxspacing}
\begin{Verbatim}[commandchars=\\\{\}]
\PY{c+c1}{\PYZsh{} importing the dataset}
\PY{n}{df} \PY{o}{=} \PY{n}{pd}\PY{o}{.}\PY{n}{read\PYZus{}csv}\PY{p}{(}\PY{l+s+s2}{\PYZdq{}}\PY{l+s+s2}{part\PYZhy{}00000\PYZhy{}363d1ba3\PYZhy{}8ab5\PYZhy{}4f96\PYZhy{}bc25\PYZhy{}4d5862db7cb9\PYZhy{}c000.csv}\PY{l+s+s2}{\PYZdq{}}\PY{p}{)}
\PY{n}{df}\PY{o}{.}\PY{n}{head}\PY{p}{(}\PY{p}{)}
\end{Verbatim}
\end{tcolorbox}

            \begin{tcolorbox}[breakable, size=fbox, boxrule=.5pt, pad at break*=1mm, opacityfill=0]
\prompt{Out}{outcolor}{96}{\boxspacing}
\begin{Verbatim}[commandchars=\\\{\}]
   flow\_duration  Header\_Length  Protocol Type  Duration         Rate  \textbackslash{}
0       0.000000          54.00           6.00     64.00     0.329807
1       0.000000          57.04           6.33     64.00     4.290556
2       0.000000           0.00           1.00     64.00    33.396799
3       0.328175       76175.00          17.00     64.00  4642.133010
4       0.117320         101.73           6.11     65.91     6.202211

         Srate  Drate  fin\_flag\_number  syn\_flag\_number  rst\_flag\_number  {\ldots}  \textbackslash{}
0     0.329807    0.0              1.0              0.0              1.0  {\ldots}
1     4.290556    0.0              0.0              0.0              0.0  {\ldots}
2    33.396799    0.0              0.0              0.0              0.0  {\ldots}
3  4642.133010    0.0              0.0              0.0              0.0  {\ldots}
4     6.202211    0.0              0.0              1.0              0.0  {\ldots}

         Std  Tot size           IAT  Number   Magnitue     Radius  \textbackslash{}
0   0.000000     54.00  8.334383e+07     9.5  10.392305   0.000000
1   2.822973     57.04  8.292607e+07     9.5  10.464666   4.010353
2   0.000000     42.00  8.312799e+07     9.5   9.165151   0.000000
3   0.000000     50.00  8.301570e+07     9.5  10.000000   0.000000
4  23.113111     57.88  8.297300e+07     9.5  11.346876  32.716243

    Covariance  Variance  Weight             label
0     0.000000      0.00  141.55  DDoS-RSTFINFlood
1   160.987842      0.05  141.55     DoS-TCP\_Flood
2     0.000000      0.00  141.55   DDoS-ICMP\_Flood
3     0.000000      0.00  141.55     DoS-UDP\_Flood
4  3016.808286      0.19  141.55     DoS-SYN\_Flood

[5 rows x 47 columns]
\end{Verbatim}
\end{tcolorbox}
        
    \begin{tcolorbox}[breakable, size=fbox, boxrule=1pt, pad at break*=1mm,colback=cellbackground, colframe=cellborder]
\prompt{In}{incolor}{97}{\boxspacing}
\begin{Verbatim}[commandchars=\\\{\}]
\PY{n}{df}\PY{o}{.}\PY{n}{info}\PY{p}{(}\PY{p}{)}
\end{Verbatim}
\end{tcolorbox}

    \begin{Verbatim}[commandchars=\\\{\}]
<class 'pandas.core.frame.DataFrame'>
RangeIndex: 238687 entries, 0 to 238686
Data columns (total 47 columns):
 \#   Column           Non-Null Count   Dtype
---  ------           --------------   -----
 0   flow\_duration    238687 non-null  float64
 1   Header\_Length    238687 non-null  float64
 2   Protocol Type    238687 non-null  float64
 3   Duration         238687 non-null  float64
 4   Rate             238687 non-null  float64
 5   Srate            238687 non-null  float64
 6   Drate            238687 non-null  float64
 7   fin\_flag\_number  238687 non-null  float64
 8   syn\_flag\_number  238687 non-null  float64
 9   rst\_flag\_number  238687 non-null  float64
 10  psh\_flag\_number  238687 non-null  float64
 11  ack\_flag\_number  238687 non-null  float64
 12  ece\_flag\_number  238687 non-null  float64
 13  cwr\_flag\_number  238687 non-null  float64
 14  ack\_count        238687 non-null  float64
 15  syn\_count        238687 non-null  float64
 16  fin\_count        238687 non-null  float64
 17  urg\_count        238687 non-null  float64
 18  rst\_count        238687 non-null  float64
 19  HTTP             238687 non-null  float64
 20  HTTPS            238687 non-null  float64
 21  DNS              238687 non-null  float64
 22  Telnet           238687 non-null  float64
 23  SMTP             238687 non-null  float64
 24  SSH              238687 non-null  float64
 25  IRC              238687 non-null  float64
 26  TCP              238687 non-null  float64
 27  UDP              238687 non-null  float64
 28  DHCP             238687 non-null  float64
 29  ARP              238687 non-null  float64
 30  ICMP             238687 non-null  float64
 31  IPv              238687 non-null  float64
 32  LLC              238687 non-null  float64
 33  Tot sum          238687 non-null  float64
 34  Min              238687 non-null  float64
 35  Max              238687 non-null  float64
 36  AVG              238687 non-null  float64
 37  Std              238687 non-null  float64
 38  Tot size         238687 non-null  float64
 39  IAT              238687 non-null  float64
 40  Number           238687 non-null  float64
 41  Magnitue         238687 non-null  float64
 42  Radius           238687 non-null  float64
 43  Covariance       238687 non-null  float64
 44  Variance         238687 non-null  float64
 45  Weight           238687 non-null  float64
 46  label            238687 non-null  object
dtypes: float64(46), object(1)
memory usage: 85.6+ MB
    \end{Verbatim}

    \begin{tcolorbox}[breakable, size=fbox, boxrule=1pt, pad at break*=1mm,colback=cellbackground, colframe=cellborder]
\prompt{In}{incolor}{98}{\boxspacing}
\begin{Verbatim}[commandchars=\\\{\}]
\PY{c+c1}{\PYZsh{} lets remove columns that we dont need.}
\PY{n}{df}\PY{o}{.}\PY{n}{columns}
\end{Verbatim}
\end{tcolorbox}

            \begin{tcolorbox}[breakable, size=fbox, boxrule=.5pt, pad at break*=1mm, opacityfill=0]
\prompt{Out}{outcolor}{98}{\boxspacing}
\begin{Verbatim}[commandchars=\\\{\}]
Index(['flow\_duration', 'Header\_Length', 'Protocol Type', 'Duration', 'Rate',
       'Srate', 'Drate', 'fin\_flag\_number', 'syn\_flag\_number',
       'rst\_flag\_number', 'psh\_flag\_number', 'ack\_flag\_number',
       'ece\_flag\_number', 'cwr\_flag\_number', 'ack\_count', 'syn\_count',
       'fin\_count', 'urg\_count', 'rst\_count', 'HTTP', 'HTTPS', 'DNS', 'Telnet',
       'SMTP', 'SSH', 'IRC', 'TCP', 'UDP', 'DHCP', 'ARP', 'ICMP', 'IPv', 'LLC',
       'Tot sum', 'Min', 'Max', 'AVG', 'Std', 'Tot size', 'IAT', 'Number',
       'Magnitue', 'Radius', 'Covariance', 'Variance', 'Weight', 'label'],
      dtype='object')
\end{Verbatim}
\end{tcolorbox}
        
    \begin{tcolorbox}[breakable, size=fbox, boxrule=1pt, pad at break*=1mm,colback=cellbackground, colframe=cellborder]
\prompt{In}{incolor}{ }{\boxspacing}
\begin{Verbatim}[commandchars=\\\{\}]

\end{Verbatim}
\end{tcolorbox}

    \begin{tcolorbox}[breakable, size=fbox, boxrule=1pt, pad at break*=1mm,colback=cellbackground, colframe=cellborder]
\prompt{In}{incolor}{99}{\boxspacing}
\begin{Verbatim}[commandchars=\\\{\}]
\PY{n}{df}\PY{o}{.}\PY{n}{drop}\PY{p}{(}
    \PY{n}{columns}\PY{o}{=}\PY{p}{[}
        \PY{l+s+s2}{\PYZdq{}}\PY{l+s+s2}{Drate}\PY{l+s+s2}{\PYZdq{}}\PY{p}{,}
        \PY{l+s+s2}{\PYZdq{}}\PY{l+s+s2}{fin\PYZus{}flag\PYZus{}number}\PY{l+s+s2}{\PYZdq{}}\PY{p}{,}
        \PY{l+s+s2}{\PYZdq{}}\PY{l+s+s2}{syn\PYZus{}flag\PYZus{}number}\PY{l+s+s2}{\PYZdq{}}\PY{p}{,}
        \PY{l+s+s2}{\PYZdq{}}\PY{l+s+s2}{rst\PYZus{}flag\PYZus{}number}\PY{l+s+s2}{\PYZdq{}}\PY{p}{,}
        \PY{l+s+s2}{\PYZdq{}}\PY{l+s+s2}{psh\PYZus{}flag\PYZus{}number}\PY{l+s+s2}{\PYZdq{}}\PY{p}{,}
        \PY{l+s+s2}{\PYZdq{}}\PY{l+s+s2}{ack\PYZus{}flag\PYZus{}number}\PY{l+s+s2}{\PYZdq{}}\PY{p}{,}
        \PY{l+s+s2}{\PYZdq{}}\PY{l+s+s2}{ece\PYZus{}flag\PYZus{}number}\PY{l+s+s2}{\PYZdq{}}\PY{p}{,}
        \PY{l+s+s2}{\PYZdq{}}\PY{l+s+s2}{cwr\PYZus{}flag\PYZus{}number}\PY{l+s+s2}{\PYZdq{}}\PY{p}{,}
        \PY{l+s+s2}{\PYZdq{}}\PY{l+s+s2}{ack\PYZus{}count}\PY{l+s+s2}{\PYZdq{}}\PY{p}{,}
        \PY{l+s+s2}{\PYZdq{}}\PY{l+s+s2}{syn\PYZus{}count}\PY{l+s+s2}{\PYZdq{}}\PY{p}{,}
        \PY{l+s+s2}{\PYZdq{}}\PY{l+s+s2}{fin\PYZus{}count}\PY{l+s+s2}{\PYZdq{}}\PY{p}{,}
        \PY{l+s+s2}{\PYZdq{}}\PY{l+s+s2}{urg\PYZus{}count}\PY{l+s+s2}{\PYZdq{}}\PY{p}{,}
        \PY{l+s+s2}{\PYZdq{}}\PY{l+s+s2}{rst\PYZus{}count}\PY{l+s+s2}{\PYZdq{}}\PY{p}{,}
        \PY{l+s+s2}{\PYZdq{}}\PY{l+s+s2}{HTTP}\PY{l+s+s2}{\PYZdq{}}\PY{p}{,}
        \PY{l+s+s2}{\PYZdq{}}\PY{l+s+s2}{HTTPS}\PY{l+s+s2}{\PYZdq{}}\PY{p}{,}
        \PY{l+s+s2}{\PYZdq{}}\PY{l+s+s2}{DNS}\PY{l+s+s2}{\PYZdq{}}\PY{p}{,}
        \PY{l+s+s2}{\PYZdq{}}\PY{l+s+s2}{Telnet}\PY{l+s+s2}{\PYZdq{}}\PY{p}{,}
        \PY{l+s+s2}{\PYZdq{}}\PY{l+s+s2}{SMTP}\PY{l+s+s2}{\PYZdq{}}\PY{p}{,}
        \PY{l+s+s2}{\PYZdq{}}\PY{l+s+s2}{SSH}\PY{l+s+s2}{\PYZdq{}}\PY{p}{,}
        \PY{l+s+s2}{\PYZdq{}}\PY{l+s+s2}{IRC}\PY{l+s+s2}{\PYZdq{}}\PY{p}{,}
        \PY{l+s+s2}{\PYZdq{}}\PY{l+s+s2}{TCP}\PY{l+s+s2}{\PYZdq{}}\PY{p}{,}
        \PY{l+s+s2}{\PYZdq{}}\PY{l+s+s2}{UDP}\PY{l+s+s2}{\PYZdq{}}\PY{p}{,}
        \PY{l+s+s2}{\PYZdq{}}\PY{l+s+s2}{DHCP}\PY{l+s+s2}{\PYZdq{}}\PY{p}{,}
        \PY{l+s+s2}{\PYZdq{}}\PY{l+s+s2}{ARP}\PY{l+s+s2}{\PYZdq{}}\PY{p}{,}
        \PY{l+s+s2}{\PYZdq{}}\PY{l+s+s2}{ICMP}\PY{l+s+s2}{\PYZdq{}}\PY{p}{,}
        \PY{l+s+s2}{\PYZdq{}}\PY{l+s+s2}{IPv}\PY{l+s+s2}{\PYZdq{}}\PY{p}{,}
        \PY{l+s+s2}{\PYZdq{}}\PY{l+s+s2}{LLC}\PY{l+s+s2}{\PYZdq{}}\PY{p}{,}
        \PY{l+s+s2}{\PYZdq{}}\PY{l+s+s2}{Tot sum}\PY{l+s+s2}{\PYZdq{}}\PY{p}{,}
        \PY{l+s+s2}{\PYZdq{}}\PY{l+s+s2}{Min}\PY{l+s+s2}{\PYZdq{}}\PY{p}{,}
        \PY{l+s+s2}{\PYZdq{}}\PY{l+s+s2}{Max}\PY{l+s+s2}{\PYZdq{}}\PY{p}{,}
        \PY{l+s+s2}{\PYZdq{}}\PY{l+s+s2}{AVG}\PY{l+s+s2}{\PYZdq{}}\PY{p}{,}
        \PY{l+s+s2}{\PYZdq{}}\PY{l+s+s2}{Std}\PY{l+s+s2}{\PYZdq{}}\PY{p}{,}
        \PY{l+s+s2}{\PYZdq{}}\PY{l+s+s2}{Tot size}\PY{l+s+s2}{\PYZdq{}}\PY{p}{,}
        \PY{l+s+s2}{\PYZdq{}}\PY{l+s+s2}{IAT}\PY{l+s+s2}{\PYZdq{}}\PY{p}{,}
        \PY{l+s+s2}{\PYZdq{}}\PY{l+s+s2}{Number}\PY{l+s+s2}{\PYZdq{}}\PY{p}{,}
        \PY{l+s+s2}{\PYZdq{}}\PY{l+s+s2}{Magnitue}\PY{l+s+s2}{\PYZdq{}}\PY{p}{,}
        \PY{l+s+s2}{\PYZdq{}}\PY{l+s+s2}{Radius}\PY{l+s+s2}{\PYZdq{}}\PY{p}{,}
        \PY{l+s+s2}{\PYZdq{}}\PY{l+s+s2}{Covariance}\PY{l+s+s2}{\PYZdq{}}\PY{p}{,}
        \PY{l+s+s2}{\PYZdq{}}\PY{l+s+s2}{Variance}\PY{l+s+s2}{\PYZdq{}}\PY{p}{,}
        \PY{l+s+s2}{\PYZdq{}}\PY{l+s+s2}{Weight}\PY{l+s+s2}{\PYZdq{}}\PY{p}{,}
    \PY{p}{]}\PY{p}{,}
    \PY{n}{inplace}\PY{o}{=}\PY{k+kc}{True}\PY{p}{,}
\PY{p}{)}
\end{Verbatim}
\end{tcolorbox}

    \begin{tcolorbox}[breakable, size=fbox, boxrule=1pt, pad at break*=1mm,colback=cellbackground, colframe=cellborder]
\prompt{In}{incolor}{100}{\boxspacing}
\begin{Verbatim}[commandchars=\\\{\}]
\PY{c+c1}{\PYZsh{} let us now visualize scatter plots of the data}
\PY{k+kn}{import} \PY{n+nn}{seaborn} \PY{k}{as} \PY{n+nn}{sns}

\PY{c+c1}{\PYZsh{} before that lets scale the data down, just take the first 1000 rows}
\PY{n}{scaled\PYZus{}df} \PY{o}{=} \PY{n}{df}\PY{p}{[}\PY{p}{:}\PY{l+m+mi}{1000}\PY{p}{]}
\end{Verbatim}
\end{tcolorbox}

    \begin{tcolorbox}[breakable, size=fbox, boxrule=1pt, pad at break*=1mm,colback=cellbackground, colframe=cellborder]
\prompt{In}{incolor}{101}{\boxspacing}
\begin{Verbatim}[commandchars=\\\{\}]
\PY{n}{sns}\PY{o}{.}\PY{n}{relplot}\PY{p}{(}\PY{n}{df}\PY{p}{,} \PY{n}{y}\PY{o}{=}\PY{l+s+s2}{\PYZdq{}}\PY{l+s+s2}{flow\PYZus{}duration}\PY{l+s+s2}{\PYZdq{}}\PY{p}{,} \PY{n}{x}\PY{o}{=}\PY{n}{df}\PY{o}{.}\PY{n}{index}\PY{p}{,} \PY{n}{kind}\PY{o}{=}\PY{l+s+s2}{\PYZdq{}}\PY{l+s+s2}{scatter}\PY{l+s+s2}{\PYZdq{}}\PY{p}{,} \PY{n}{hue}\PY{o}{=}\PY{l+s+s2}{\PYZdq{}}\PY{l+s+s2}{label}\PY{l+s+s2}{\PYZdq{}}\PY{p}{)}
\end{Verbatim}
\end{tcolorbox}

            \begin{tcolorbox}[breakable, size=fbox, boxrule=.5pt, pad at break*=1mm, opacityfill=0]
\prompt{Out}{outcolor}{101}{\boxspacing}
\begin{Verbatim}[commandchars=\\\{\}]
<seaborn.axisgrid.FacetGrid at 0x2d6006d7520>
\end{Verbatim}
\end{tcolorbox}
        
    \begin{center}
    \adjustimage{max size={0.9\linewidth}{0.9\paperheight}}{output_7_1.png}
    \end{center}
    { \hspace*{\fill} \\}
    
    \begin{tcolorbox}[breakable, size=fbox, boxrule=1pt, pad at break*=1mm,colback=cellbackground, colframe=cellborder]
\prompt{In}{incolor}{102}{\boxspacing}
\begin{Verbatim}[commandchars=\\\{\}]
\PY{c+c1}{\PYZsh{} lets go through the dataset, and remove those rows with labels that occur very less times to remove noise}

\PY{n}{df}\PY{p}{[}\PY{l+s+s2}{\PYZdq{}}\PY{l+s+s2}{label}\PY{l+s+s2}{\PYZdq{}}\PY{p}{]}\PY{o}{.}\PY{n}{value\PYZus{}counts}\PY{p}{(}\PY{p}{)}
\end{Verbatim}
\end{tcolorbox}

            \begin{tcolorbox}[breakable, size=fbox, boxrule=.5pt, pad at break*=1mm, opacityfill=0]
\prompt{Out}{outcolor}{102}{\boxspacing}
\begin{Verbatim}[commandchars=\\\{\}]
label
DDoS-ICMP\_Flood            36554
DDoS-UDP\_Flood             27626
DDoS-TCP\_Flood             23149
DDoS-PSHACK\_Flood          21210
DDoS-SYN\_Flood             20739
DDoS-RSTFINFlood           20669
DDoS-SynonymousIP\_Flood    18189
DoS-UDP\_Flood              16957
DoS-TCP\_Flood              13630
DoS-SYN\_Flood              10275
BenignTraffic               5600
Mirai-greeth\_flood          5016
Mirai-udpplain              4661
Mirai-greip\_flood           3758
DDoS-ICMP\_Fragmentation     2377
MITM-ArpSpoofing            1614
DDoS-ACK\_Fragmentation      1505
DDoS-UDP\_Fragmentation      1484
DNS\_Spoofing                 925
Recon-HostDiscovery          697
Recon-OSScan                 517
Recon-PortScan               430
DoS-HTTP\_Flood               414
VulnerabilityScan            210
DDoS-HTTP\_Flood              169
DDoS-SlowLoris               106
DictionaryBruteForce          63
SqlInjection                  31
BrowserHijacking              30
CommandInjection              28
Backdoor\_Malware              22
XSS                           18
Uploading\_Attack               8
Recon-PingSweep                6
Name: count, dtype: int64
\end{Verbatim}
\end{tcolorbox}
        
    \begin{tcolorbox}[breakable, size=fbox, boxrule=1pt, pad at break*=1mm,colback=cellbackground, colframe=cellborder]
\prompt{In}{incolor}{103}{\boxspacing}
\begin{Verbatim}[commandchars=\\\{\}]
\PY{c+c1}{\PYZsh{} lets keep the top 5 labels and remove the rest}
\PY{n}{top\PYZus{}labels} \PY{o}{=} \PY{n}{df}\PY{p}{[}\PY{l+s+s2}{\PYZdq{}}\PY{l+s+s2}{label}\PY{l+s+s2}{\PYZdq{}}\PY{p}{]}\PY{o}{.}\PY{n}{value\PYZus{}counts}\PY{p}{(}\PY{p}{)}\PY{o}{.}\PY{n}{head}\PY{p}{(}\PY{l+m+mi}{5}\PY{p}{)}\PY{o}{.}\PY{n}{index}
\PY{n}{df} \PY{o}{=} \PY{n}{df}\PY{p}{[}\PY{n}{df}\PY{p}{[}\PY{l+s+s2}{\PYZdq{}}\PY{l+s+s2}{label}\PY{l+s+s2}{\PYZdq{}}\PY{p}{]}\PY{o}{.}\PY{n}{isin}\PY{p}{(}\PY{n}{top\PYZus{}labels}\PY{p}{)}\PY{p}{]}
\end{Verbatim}
\end{tcolorbox}

    \begin{tcolorbox}[breakable, size=fbox, boxrule=1pt, pad at break*=1mm,colback=cellbackground, colframe=cellborder]
\prompt{In}{incolor}{104}{\boxspacing}
\begin{Verbatim}[commandchars=\\\{\}]
\PY{n}{df}\PY{o}{.}\PY{n}{shape}
\end{Verbatim}
\end{tcolorbox}

            \begin{tcolorbox}[breakable, size=fbox, boxrule=.5pt, pad at break*=1mm, opacityfill=0]
\prompt{Out}{outcolor}{104}{\boxspacing}
\begin{Verbatim}[commandchars=\\\{\}]
(129278, 7)
\end{Verbatim}
\end{tcolorbox}
        
    \begin{tcolorbox}[breakable, size=fbox, boxrule=1pt, pad at break*=1mm,colback=cellbackground, colframe=cellborder]
\prompt{In}{incolor}{105}{\boxspacing}
\begin{Verbatim}[commandchars=\\\{\}]
\PY{c+c1}{\PYZsh{} lets try plotting again}
\PY{n}{sns}\PY{o}{.}\PY{n}{relplot}\PY{p}{(}\PY{n}{df}\PY{p}{,} \PY{n}{y}\PY{o}{=}\PY{l+s+s2}{\PYZdq{}}\PY{l+s+s2}{flow\PYZus{}duration}\PY{l+s+s2}{\PYZdq{}}\PY{p}{,} \PY{n}{x}\PY{o}{=}\PY{n}{df}\PY{o}{.}\PY{n}{index}\PY{p}{,} \PY{n}{kind}\PY{o}{=}\PY{l+s+s2}{\PYZdq{}}\PY{l+s+s2}{scatter}\PY{l+s+s2}{\PYZdq{}}\PY{p}{,} \PY{n}{hue}\PY{o}{=}\PY{l+s+s2}{\PYZdq{}}\PY{l+s+s2}{label}\PY{l+s+s2}{\PYZdq{}}\PY{p}{)}
\end{Verbatim}
\end{tcolorbox}

            \begin{tcolorbox}[breakable, size=fbox, boxrule=.5pt, pad at break*=1mm, opacityfill=0]
\prompt{Out}{outcolor}{105}{\boxspacing}
\begin{Verbatim}[commandchars=\\\{\}]
<seaborn.axisgrid.FacetGrid at 0x2d651613af0>
\end{Verbatim}
\end{tcolorbox}
        
    \begin{center}
    \adjustimage{max size={0.9\linewidth}{0.9\paperheight}}{output_11_1.png}
    \end{center}
    { \hspace*{\fill} \\}
    
    \begin{tcolorbox}[breakable, size=fbox, boxrule=1pt, pad at break*=1mm,colback=cellbackground, colframe=cellborder]
\prompt{In}{incolor}{127}{\boxspacing}
\begin{Verbatim}[commandchars=\\\{\}]
\PY{c+c1}{\PYZsh{} lets remove flow\PYZus{}duration rows with z scores more than 3}
\PY{k+kn}{from} \PY{n+nn}{scipy} \PY{k+kn}{import} \PY{n}{stats}

\PY{n}{z} \PY{o}{=} \PY{n}{np}\PY{o}{.}\PY{n}{abs}\PY{p}{(}\PY{n}{stats}\PY{o}{.}\PY{n}{zscore}\PY{p}{(}\PY{n}{df}\PY{p}{[}\PY{l+s+s2}{\PYZdq{}}\PY{l+s+s2}{flow\PYZus{}duration}\PY{l+s+s2}{\PYZdq{}}\PY{p}{]}\PY{p}{)}\PY{p}{)}
\PY{n}{non\PYZus{}outlier\PYZus{}df} \PY{o}{=} \PY{n}{df}\PY{p}{[}\PY{p}{(}\PY{n}{z} \PY{o}{\PYZlt{}} \PY{l+m+mf}{0.0115}\PY{p}{)}\PY{p}{]}
\PY{n}{non\PYZus{}outlier\PYZus{}df}\PY{o}{.}\PY{n}{shape}
\end{Verbatim}
\end{tcolorbox}

            \begin{tcolorbox}[breakable, size=fbox, boxrule=.5pt, pad at break*=1mm, opacityfill=0]
\prompt{Out}{outcolor}{127}{\boxspacing}
\begin{Verbatim}[commandchars=\\\{\}]
(34489, 7)
\end{Verbatim}
\end{tcolorbox}
        
    \begin{tcolorbox}[breakable, size=fbox, boxrule=1pt, pad at break*=1mm,colback=cellbackground, colframe=cellborder]
\prompt{In}{incolor}{128}{\boxspacing}
\begin{Verbatim}[commandchars=\\\{\}]
\PY{c+c1}{\PYZsh{} lets plot a distribution}
\PY{n}{sns}\PY{o}{.}\PY{n}{displot}\PY{p}{(}\PY{n}{non\PYZus{}outlier\PYZus{}df}\PY{p}{[}\PY{l+s+s2}{\PYZdq{}}\PY{l+s+s2}{flow\PYZus{}duration}\PY{l+s+s2}{\PYZdq{}}\PY{p}{]}\PY{p}{,} \PY{n}{kind}\PY{o}{=}\PY{l+s+s2}{\PYZdq{}}\PY{l+s+s2}{kde}\PY{l+s+s2}{\PYZdq{}}\PY{p}{,} \PY{n}{fill}\PY{o}{=}\PY{k+kc}{True}\PY{p}{)}
\end{Verbatim}
\end{tcolorbox}

            \begin{tcolorbox}[breakable, size=fbox, boxrule=.5pt, pad at break*=1mm, opacityfill=0]
\prompt{Out}{outcolor}{128}{\boxspacing}
\begin{Verbatim}[commandchars=\\\{\}]
<seaborn.axisgrid.FacetGrid at 0x2d62991a350>
\end{Verbatim}
\end{tcolorbox}
        
    \begin{center}
    \adjustimage{max size={0.9\linewidth}{0.9\paperheight}}{output_13_1.png}
    \end{center}
    { \hspace*{\fill} \\}
    
    \begin{tcolorbox}[breakable, size=fbox, boxrule=1pt, pad at break*=1mm,colback=cellbackground, colframe=cellborder]
\prompt{In}{incolor}{129}{\boxspacing}
\begin{Verbatim}[commandchars=\\\{\}]
\PY{c+c1}{\PYZsh{} lets try plotting again}
\PY{n}{sns}\PY{o}{.}\PY{n}{relplot}\PY{p}{(}
    \PY{n}{non\PYZus{}outlier\PYZus{}df}\PY{p}{,}
    \PY{n}{y}\PY{o}{=}\PY{l+s+s2}{\PYZdq{}}\PY{l+s+s2}{flow\PYZus{}duration}\PY{l+s+s2}{\PYZdq{}}\PY{p}{,}
    \PY{n}{x}\PY{o}{=}\PY{n}{non\PYZus{}outlier\PYZus{}df}\PY{o}{.}\PY{n}{index}\PY{p}{,}
    \PY{n}{kind}\PY{o}{=}\PY{l+s+s2}{\PYZdq{}}\PY{l+s+s2}{scatter}\PY{l+s+s2}{\PYZdq{}}\PY{p}{,}
    \PY{n}{hue}\PY{o}{=}\PY{l+s+s2}{\PYZdq{}}\PY{l+s+s2}{label}\PY{l+s+s2}{\PYZdq{}}\PY{p}{,}
\PY{p}{)}
\end{Verbatim}
\end{tcolorbox}

            \begin{tcolorbox}[breakable, size=fbox, boxrule=.5pt, pad at break*=1mm, opacityfill=0]
\prompt{Out}{outcolor}{129}{\boxspacing}
\begin{Verbatim}[commandchars=\\\{\}]
<seaborn.axisgrid.FacetGrid at 0x2d644e29030>
\end{Verbatim}
\end{tcolorbox}
        
    \begin{center}
    \adjustimage{max size={0.9\linewidth}{0.9\paperheight}}{output_14_1.png}
    \end{center}
    { \hspace*{\fill} \\}
    
    \begin{tcolorbox}[breakable, size=fbox, boxrule=1pt, pad at break*=1mm,colback=cellbackground, colframe=cellborder]
\prompt{In}{incolor}{131}{\boxspacing}
\begin{Verbatim}[commandchars=\\\{\}]
\PY{c+c1}{\PYZsh{} lets now try with only 1000 rows}
\PY{n}{scaled\PYZus{}df} \PY{o}{=} \PY{n}{non\PYZus{}outlier\PYZus{}df}\PY{p}{[}\PY{p}{:}\PY{l+m+mi}{100}\PY{p}{]}

\PY{c+c1}{\PYZsh{} lets plot}
\PY{n}{sns}\PY{o}{.}\PY{n}{relplot}\PY{p}{(}
    \PY{n}{scaled\PYZus{}df}\PY{p}{,} \PY{n}{y}\PY{o}{=}\PY{l+s+s2}{\PYZdq{}}\PY{l+s+s2}{flow\PYZus{}duration}\PY{l+s+s2}{\PYZdq{}}\PY{p}{,} \PY{n}{x}\PY{o}{=}\PY{n}{scaled\PYZus{}df}\PY{o}{.}\PY{n}{index}\PY{p}{,} \PY{n}{kind}\PY{o}{=}\PY{l+s+s2}{\PYZdq{}}\PY{l+s+s2}{scatter}\PY{l+s+s2}{\PYZdq{}}\PY{p}{,} \PY{n}{hue}\PY{o}{=}\PY{l+s+s2}{\PYZdq{}}\PY{l+s+s2}{label}\PY{l+s+s2}{\PYZdq{}}
\PY{p}{)}
\end{Verbatim}
\end{tcolorbox}

            \begin{tcolorbox}[breakable, size=fbox, boxrule=.5pt, pad at break*=1mm, opacityfill=0]
\prompt{Out}{outcolor}{131}{\boxspacing}
\begin{Verbatim}[commandchars=\\\{\}]
<seaborn.axisgrid.FacetGrid at 0x2d64f866d40>
\end{Verbatim}
\end{tcolorbox}
        
    \begin{center}
    \adjustimage{max size={0.9\linewidth}{0.9\paperheight}}{output_15_1.png}
    \end{center}
    { \hspace*{\fill} \\}
    
    \begin{tcolorbox}[breakable, size=fbox, boxrule=1pt, pad at break*=1mm,colback=cellbackground, colframe=cellborder]
\prompt{In}{incolor}{135}{\boxspacing}
\begin{Verbatim}[commandchars=\\\{\}]
\PY{c+c1}{\PYZsh{} we still cant see a nice cluster here, lets try and remove ddos udp flood}
\PY{n}{scaled\PYZus{}df} \PY{o}{=} \PY{n}{non\PYZus{}outlier\PYZus{}df}\PY{p}{[}\PY{n}{non\PYZus{}outlier\PYZus{}df}\PY{p}{[}\PY{l+s+s2}{\PYZdq{}}\PY{l+s+s2}{label}\PY{l+s+s2}{\PYZdq{}}\PY{p}{]} \PY{o}{!=} \PY{l+s+s2}{\PYZdq{}}\PY{l+s+s2}{DDoS\PYZhy{}UDP\PYZus{}Flood}\PY{l+s+s2}{\PYZdq{}}\PY{p}{]}
\end{Verbatim}
\end{tcolorbox}

    \begin{tcolorbox}[breakable, size=fbox, boxrule=1pt, pad at break*=1mm,colback=cellbackground, colframe=cellborder]
\prompt{In}{incolor}{138}{\boxspacing}
\begin{Verbatim}[commandchars=\\\{\}]
\PY{c+c1}{\PYZsh{} lets plot again}
\PY{n}{scaled\PYZus{}df} \PY{o}{=} \PY{n}{scaled\PYZus{}df}\PY{p}{[}\PY{p}{:}\PY{l+m+mi}{100}\PY{p}{]}
\PY{n}{sns}\PY{o}{.}\PY{n}{relplot}\PY{p}{(}
    \PY{n}{scaled\PYZus{}df}\PY{p}{,} \PY{n}{y}\PY{o}{=}\PY{l+s+s2}{\PYZdq{}}\PY{l+s+s2}{flow\PYZus{}duration}\PY{l+s+s2}{\PYZdq{}}\PY{p}{,} \PY{n}{x}\PY{o}{=}\PY{n}{scaled\PYZus{}df}\PY{o}{.}\PY{n}{index}\PY{p}{,} \PY{n}{kind}\PY{o}{=}\PY{l+s+s2}{\PYZdq{}}\PY{l+s+s2}{scatter}\PY{l+s+s2}{\PYZdq{}}\PY{p}{,} \PY{n}{hue}\PY{o}{=}\PY{l+s+s2}{\PYZdq{}}\PY{l+s+s2}{label}\PY{l+s+s2}{\PYZdq{}}
\PY{p}{)}
\end{Verbatim}
\end{tcolorbox}

            \begin{tcolorbox}[breakable, size=fbox, boxrule=.5pt, pad at break*=1mm, opacityfill=0]
\prompt{Out}{outcolor}{138}{\boxspacing}
\begin{Verbatim}[commandchars=\\\{\}]
<seaborn.axisgrid.FacetGrid at 0x2d651567850>
\end{Verbatim}
\end{tcolorbox}
        
    \begin{center}
    \adjustimage{max size={0.9\linewidth}{0.9\paperheight}}{output_17_1.png}
    \end{center}
    { \hspace*{\fill} \\}
    
    \begin{tcolorbox}[breakable, size=fbox, boxrule=1pt, pad at break*=1mm,colback=cellbackground, colframe=cellborder]
\prompt{In}{incolor}{139}{\boxspacing}
\begin{Verbatim}[commandchars=\\\{\}]
\PY{c+c1}{\PYZsh{} lets try other features}
\PY{n}{df}\PY{o}{.}\PY{n}{head}\PY{p}{(}\PY{p}{)}
\end{Verbatim}
\end{tcolorbox}

            \begin{tcolorbox}[breakable, size=fbox, boxrule=.5pt, pad at break*=1mm, opacityfill=0]
\prompt{Out}{outcolor}{139}{\boxspacing}
\begin{Verbatim}[commandchars=\\\{\}]
    flow\_duration  Header\_Length  Protocol Type  Duration       Rate  \textbackslash{}
2        0.000000           0.00           1.00     64.00  33.396799
9        0.000000          54.20           6.00     64.00  11.243547
10       0.223192          61.54           6.11     64.64   9.087882
11       0.000000          54.00           6.00     64.00  17.333181
12       0.000000           0.00           1.00     75.46   0.000000

        Srate              label
2   33.396799    DDoS-ICMP\_Flood
9   11.243547     DDoS-SYN\_Flood
10   9.087882  DDoS-PSHACK\_Flood
11  17.333181     DDoS-TCP\_Flood
12   0.000000    DDoS-ICMP\_Flood
\end{Verbatim}
\end{tcolorbox}
        
    \begin{tcolorbox}[breakable, size=fbox, boxrule=1pt, pad at break*=1mm,colback=cellbackground, colframe=cellborder]
\prompt{In}{incolor}{141}{\boxspacing}
\begin{Verbatim}[commandchars=\\\{\}]
\PY{c+c1}{\PYZsh{} lets plot rate just like flow\PYZus{}duration}
\PY{n}{df}\PY{o}{.}\PY{n}{shape}
\end{Verbatim}
\end{tcolorbox}

            \begin{tcolorbox}[breakable, size=fbox, boxrule=.5pt, pad at break*=1mm, opacityfill=0]
\prompt{Out}{outcolor}{141}{\boxspacing}
\begin{Verbatim}[commandchars=\\\{\}]
(129278, 7)
\end{Verbatim}
\end{tcolorbox}
        
    \begin{tcolorbox}[breakable, size=fbox, boxrule=1pt, pad at break*=1mm,colback=cellbackground, colframe=cellborder]
\prompt{In}{incolor}{143}{\boxspacing}
\begin{Verbatim}[commandchars=\\\{\}]
\PY{n}{sns}\PY{o}{.}\PY{n}{relplot}\PY{p}{(}\PY{n}{df}\PY{p}{,} \PY{n}{y}\PY{o}{=}\PY{l+s+s2}{\PYZdq{}}\PY{l+s+s2}{Rate}\PY{l+s+s2}{\PYZdq{}}\PY{p}{,} \PY{n}{x}\PY{o}{=}\PY{n}{df}\PY{o}{.}\PY{n}{index}\PY{p}{,} \PY{n}{kind}\PY{o}{=}\PY{l+s+s2}{\PYZdq{}}\PY{l+s+s2}{scatter}\PY{l+s+s2}{\PYZdq{}}\PY{p}{,} \PY{n}{hue}\PY{o}{=}\PY{l+s+s2}{\PYZdq{}}\PY{l+s+s2}{label}\PY{l+s+s2}{\PYZdq{}}\PY{p}{)}
\end{Verbatim}
\end{tcolorbox}

            \begin{tcolorbox}[breakable, size=fbox, boxrule=.5pt, pad at break*=1mm, opacityfill=0]
\prompt{Out}{outcolor}{143}{\boxspacing}
\begin{Verbatim}[commandchars=\\\{\}]
<seaborn.axisgrid.FacetGrid at 0x2d61a8e65c0>
\end{Verbatim}
\end{tcolorbox}
        
    \begin{center}
    \adjustimage{max size={0.9\linewidth}{0.9\paperheight}}{output_20_1.png}
    \end{center}
    { \hspace*{\fill} \\}
    
    \begin{tcolorbox}[breakable, size=fbox, boxrule=1pt, pad at break*=1mm,colback=cellbackground, colframe=cellborder]
\prompt{In}{incolor}{154}{\boxspacing}
\begin{Verbatim}[commandchars=\\\{\}]
\PY{c+c1}{\PYZsh{} its not great but there are some clusters, we can try here.}
\PY{c+c1}{\PYZsh{} lets try implementing k means}

\PY{c+c1}{\PYZsh{} lets try with 5 clusters}
\PY{n}{kmeans} \PY{o}{=} \PY{n}{KMeans}\PY{p}{(}\PY{n}{n\PYZus{}clusters}\PY{o}{=}\PY{l+m+mi}{5}\PY{p}{)}
\PY{n}{kmeans}\PY{o}{.}\PY{n}{fit}\PY{p}{(}\PY{n}{df}\PY{p}{[}\PY{p}{[}\PY{l+s+s2}{\PYZdq{}}\PY{l+s+s2}{Rate}\PY{l+s+s2}{\PYZdq{}}\PY{p}{]}\PY{p}{]}\PY{p}{)}

\PY{c+c1}{\PYZsh{} lets get the labels}
\PY{n}{df}\PY{p}{[}\PY{l+s+s2}{\PYZdq{}}\PY{l+s+s2}{cluster}\PY{l+s+s2}{\PYZdq{}}\PY{p}{]} \PY{o}{=} \PY{n}{kmeans}\PY{o}{.}\PY{n}{labels\PYZus{}}

\PY{c+c1}{\PYZsh{} lets plot}
\PY{n}{sns}\PY{o}{.}\PY{n}{relplot}\PY{p}{(}
    \PY{n}{df}\PY{p}{,}
    \PY{n}{y}\PY{o}{=}\PY{l+s+s2}{\PYZdq{}}\PY{l+s+s2}{Rate}\PY{l+s+s2}{\PYZdq{}}\PY{p}{,}
    \PY{n}{x}\PY{o}{=}\PY{n}{df}\PY{o}{.}\PY{n}{index}\PY{p}{,}
    \PY{n}{kind}\PY{o}{=}\PY{l+s+s2}{\PYZdq{}}\PY{l+s+s2}{scatter}\PY{l+s+s2}{\PYZdq{}}\PY{p}{,}
    \PY{n}{hue}\PY{o}{=}\PY{l+s+s2}{\PYZdq{}}\PY{l+s+s2}{cluster}\PY{l+s+s2}{\PYZdq{}}\PY{p}{,}
    \PY{n}{palette}\PY{o}{=}\PY{l+s+s2}{\PYZdq{}}\PY{l+s+s2}{viridis}\PY{l+s+s2}{\PYZdq{}}\PY{p}{,}
    \PY{n}{legend}\PY{o}{=}\PY{l+s+s2}{\PYZdq{}}\PY{l+s+s2}{full}\PY{l+s+s2}{\PYZdq{}}\PY{p}{,}
    \PY{n}{style}\PY{o}{=}\PY{l+s+s2}{\PYZdq{}}\PY{l+s+s2}{label}\PY{l+s+s2}{\PYZdq{}}\PY{p}{,}
\PY{p}{)}

\PY{c+c1}{\PYZsh{} titles}
\PY{n}{plt}\PY{o}{.}\PY{n}{title}\PY{p}{(}\PY{l+s+s2}{\PYZdq{}}\PY{l+s+s2}{K Means Clustering for Rate of Attacks into 3 Clusters}\PY{l+s+s2}{\PYZdq{}}\PY{p}{)}
\PY{n}{plt}\PY{o}{.}\PY{n}{xlabel}\PY{p}{(}\PY{l+s+s2}{\PYZdq{}}\PY{l+s+s2}{Index}\PY{l+s+s2}{\PYZdq{}}\PY{p}{)}
\PY{n}{plt}\PY{o}{.}\PY{n}{ylabel}\PY{p}{(}\PY{l+s+s2}{\PYZdq{}}\PY{l+s+s2}{Rate}\PY{l+s+s2}{\PYZdq{}}\PY{p}{)}
\end{Verbatim}
\end{tcolorbox}

            \begin{tcolorbox}[breakable, size=fbox, boxrule=.5pt, pad at break*=1mm, opacityfill=0]
\prompt{Out}{outcolor}{154}{\boxspacing}
\begin{Verbatim}[commandchars=\\\{\}]
Text(49.60998958333333, 0.5, 'Rate')
\end{Verbatim}
\end{tcolorbox}
        
    \begin{center}
    \adjustimage{max size={0.9\linewidth}{0.9\paperheight}}{output_21_1.png}
    \end{center}
    { \hspace*{\fill} \\}
    
    \begin{tcolorbox}[breakable, size=fbox, boxrule=1pt, pad at break*=1mm,colback=cellbackground, colframe=cellborder]
\prompt{In}{incolor}{153}{\boxspacing}
\begin{Verbatim}[commandchars=\\\{\}]
\PY{c+c1}{\PYZsh{} lets try with 3 clusters}
\PY{n}{kmeans} \PY{o}{=} \PY{n}{KMeans}\PY{p}{(}\PY{n}{n\PYZus{}clusters}\PY{o}{=}\PY{l+m+mi}{3}\PY{p}{)}
\PY{n}{kmeans}\PY{o}{.}\PY{n}{fit}\PY{p}{(}\PY{n}{df}\PY{p}{[}\PY{p}{[}\PY{l+s+s2}{\PYZdq{}}\PY{l+s+s2}{Rate}\PY{l+s+s2}{\PYZdq{}}\PY{p}{]}\PY{p}{]}\PY{p}{)}

\PY{c+c1}{\PYZsh{} lets get the labels}
\PY{n}{df}\PY{p}{[}\PY{l+s+s2}{\PYZdq{}}\PY{l+s+s2}{cluster}\PY{l+s+s2}{\PYZdq{}}\PY{p}{]} \PY{o}{=} \PY{n}{kmeans}\PY{o}{.}\PY{n}{labels\PYZus{}}

\PY{c+c1}{\PYZsh{} lets plot}
\PY{n}{sns}\PY{o}{.}\PY{n}{relplot}\PY{p}{(}
    \PY{n}{df}\PY{p}{,}
    \PY{n}{y}\PY{o}{=}\PY{l+s+s2}{\PYZdq{}}\PY{l+s+s2}{Rate}\PY{l+s+s2}{\PYZdq{}}\PY{p}{,}
    \PY{n}{x}\PY{o}{=}\PY{n}{df}\PY{o}{.}\PY{n}{index}\PY{p}{,}
    \PY{n}{kind}\PY{o}{=}\PY{l+s+s2}{\PYZdq{}}\PY{l+s+s2}{scatter}\PY{l+s+s2}{\PYZdq{}}\PY{p}{,}
    \PY{n}{hue}\PY{o}{=}\PY{l+s+s2}{\PYZdq{}}\PY{l+s+s2}{cluster}\PY{l+s+s2}{\PYZdq{}}\PY{p}{,}
    \PY{n}{palette}\PY{o}{=}\PY{l+s+s2}{\PYZdq{}}\PY{l+s+s2}{viridis}\PY{l+s+s2}{\PYZdq{}}\PY{p}{,}
    \PY{n}{legend}\PY{o}{=}\PY{l+s+s2}{\PYZdq{}}\PY{l+s+s2}{full}\PY{l+s+s2}{\PYZdq{}}\PY{p}{,}
    \PY{n}{style}\PY{o}{=}\PY{l+s+s2}{\PYZdq{}}\PY{l+s+s2}{label}\PY{l+s+s2}{\PYZdq{}}\PY{p}{,}
\PY{p}{)}

\PY{c+c1}{\PYZsh{} titles}
\PY{n}{plt}\PY{o}{.}\PY{n}{title}\PY{p}{(}\PY{l+s+s2}{\PYZdq{}}\PY{l+s+s2}{K Means Clustering for Rate of Attacks into 3 Clusters}\PY{l+s+s2}{\PYZdq{}}\PY{p}{)}
\PY{n}{plt}\PY{o}{.}\PY{n}{xlabel}\PY{p}{(}\PY{l+s+s2}{\PYZdq{}}\PY{l+s+s2}{Index}\PY{l+s+s2}{\PYZdq{}}\PY{p}{)}
\PY{n}{plt}\PY{o}{.}\PY{n}{ylabel}\PY{p}{(}\PY{l+s+s2}{\PYZdq{}}\PY{l+s+s2}{Rate}\PY{l+s+s2}{\PYZdq{}}\PY{p}{)}
\end{Verbatim}
\end{tcolorbox}

            \begin{tcolorbox}[breakable, size=fbox, boxrule=.5pt, pad at break*=1mm, opacityfill=0]
\prompt{Out}{outcolor}{153}{\boxspacing}
\begin{Verbatim}[commandchars=\\\{\}]
Text(49.60998958333333, 0.5, 'Rate')
\end{Verbatim}
\end{tcolorbox}
        
    \begin{center}
    \adjustimage{max size={0.9\linewidth}{0.9\paperheight}}{output_22_1.png}
    \end{center}
    { \hspace*{\fill} \\}
    
    \begin{tcolorbox}[breakable, size=fbox, boxrule=1pt, pad at break*=1mm,colback=cellbackground, colframe=cellborder]
\prompt{In}{incolor}{156}{\boxspacing}
\begin{Verbatim}[commandchars=\\\{\}]
\PY{n}{scaled\PYZus{}df} \PY{o}{=} \PY{n}{non\PYZus{}outlier\PYZus{}df}\PY{p}{[}\PY{p}{:}\PY{l+m+mi}{100}\PY{p}{]}

\PY{c+c1}{\PYZsh{} we could also try it on flow\PYZus{}duration}
\PY{n}{kmeans} \PY{o}{=} \PY{n}{KMeans}\PY{p}{(}\PY{n}{n\PYZus{}clusters}\PY{o}{=}\PY{l+m+mi}{5}\PY{p}{)}
\PY{n}{kmeans}\PY{o}{.}\PY{n}{fit}\PY{p}{(}\PY{n}{scaled\PYZus{}df}\PY{p}{[}\PY{p}{[}\PY{l+s+s2}{\PYZdq{}}\PY{l+s+s2}{flow\PYZus{}duration}\PY{l+s+s2}{\PYZdq{}}\PY{p}{]}\PY{p}{]}\PY{p}{)}

\PY{c+c1}{\PYZsh{} lets get the labels}
\PY{n}{scaled\PYZus{}df}\PY{p}{[}\PY{l+s+s2}{\PYZdq{}}\PY{l+s+s2}{cluster}\PY{l+s+s2}{\PYZdq{}}\PY{p}{]} \PY{o}{=} \PY{n}{kmeans}\PY{o}{.}\PY{n}{labels\PYZus{}}

\PY{c+c1}{\PYZsh{} lets plot}
\PY{n}{sns}\PY{o}{.}\PY{n}{relplot}\PY{p}{(}
    \PY{n}{scaled\PYZus{}df}\PY{p}{,}
    \PY{n}{y}\PY{o}{=}\PY{l+s+s2}{\PYZdq{}}\PY{l+s+s2}{flow\PYZus{}duration}\PY{l+s+s2}{\PYZdq{}}\PY{p}{,}
    \PY{n}{x}\PY{o}{=}\PY{n}{scaled\PYZus{}df}\PY{o}{.}\PY{n}{index}\PY{p}{,}
    \PY{n}{kind}\PY{o}{=}\PY{l+s+s2}{\PYZdq{}}\PY{l+s+s2}{scatter}\PY{l+s+s2}{\PYZdq{}}\PY{p}{,}
    \PY{n}{hue}\PY{o}{=}\PY{l+s+s2}{\PYZdq{}}\PY{l+s+s2}{cluster}\PY{l+s+s2}{\PYZdq{}}\PY{p}{,}
    \PY{n}{palette}\PY{o}{=}\PY{l+s+s2}{\PYZdq{}}\PY{l+s+s2}{viridis}\PY{l+s+s2}{\PYZdq{}}\PY{p}{,}
    \PY{n}{legend}\PY{o}{=}\PY{l+s+s2}{\PYZdq{}}\PY{l+s+s2}{full}\PY{l+s+s2}{\PYZdq{}}\PY{p}{,}
    \PY{n}{style}\PY{o}{=}\PY{l+s+s2}{\PYZdq{}}\PY{l+s+s2}{label}\PY{l+s+s2}{\PYZdq{}}\PY{p}{,}
\PY{p}{)}
\end{Verbatim}
\end{tcolorbox}

    \begin{Verbatim}[commandchars=\\\{\}]
C:\textbackslash{}Users\textbackslash{}Krishnaraj\textbackslash{}AppData\textbackslash{}Local\textbackslash{}Temp\textbackslash{}ipykernel\_43912\textbackslash{}3087714870.py:8:
SettingWithCopyWarning:
A value is trying to be set on a copy of a slice from a DataFrame.
Try using .loc[row\_indexer,col\_indexer] = value instead

See the caveats in the documentation: https://pandas.pydata.org/pandas-
docs/stable/user\_guide/indexing.html\#returning-a-view-versus-a-copy
  scaled\_df["cluster"] = kmeans.labels\_
    \end{Verbatim}

            \begin{tcolorbox}[breakable, size=fbox, boxrule=.5pt, pad at break*=1mm, opacityfill=0]
\prompt{Out}{outcolor}{156}{\boxspacing}
\begin{Verbatim}[commandchars=\\\{\}]
<seaborn.axisgrid.FacetGrid at 0x2d6821e67d0>
\end{Verbatim}
\end{tcolorbox}
        
    \begin{center}
    \adjustimage{max size={0.9\linewidth}{0.9\paperheight}}{output_23_2.png}
    \end{center}
    { \hspace*{\fill} \\}
    
    \begin{tcolorbox}[breakable, size=fbox, boxrule=1pt, pad at break*=1mm,colback=cellbackground, colframe=cellborder]
\prompt{In}{incolor}{157}{\boxspacing}
\begin{Verbatim}[commandchars=\\\{\}]
\PY{c+c1}{\PYZsh{} so that concludes our k means clustering analysis on features of an IOT dataset.}
\end{Verbatim}
\end{tcolorbox}

    \begin{tcolorbox}[breakable, size=fbox, boxrule=1pt, pad at break*=1mm,colback=cellbackground, colframe=cellborder]
\prompt{In}{incolor}{159}{\boxspacing}
\begin{Verbatim}[commandchars=\\\{\}]
\PY{c+c1}{\PYZsh{} credits to https://www.unb.ca/cic/datasets/iotdataset\PYZhy{}2022.html for the dataset}
\PY{c+c1}{\PYZsh{} Citation: Sajjad Dadkhah, Hassan Mahdikhani, Priscilla Kyei Danso, Alireza Zohourian, Kevin Anh Truong, Ali A. Ghorbani, “Towards the development of a realistic multidimensional IoT profiling dataset”, Submitted to: The 19th Annual International Conference on Privacy, Security \PYZam{} Trust (PST2022) August 22\PYZhy{}24, 2022, Fredericton, Canada.}

\PY{c+c1}{\PYZsh{} analysis by: Krishnaraj T}
\end{Verbatim}
\end{tcolorbox}

    \begin{tcolorbox}[breakable, size=fbox, boxrule=1pt, pad at break*=1mm,colback=cellbackground, colframe=cellborder]
\prompt{In}{incolor}{ }{\boxspacing}
\begin{Verbatim}[commandchars=\\\{\}]

\end{Verbatim}
\end{tcolorbox}


    % Add a bibliography block to the postdoc
    
    
    
    \begin{tcolorbox}[breakable, size=fbox, boxrule=1pt, pad at break*=1mm,colback=cellbackground, colframe=cellborder]
        \prompt{In}{incolor}{95}{\boxspacing}
        \begin{Verbatim}[commandchars=\\\{\}]
        \PY{c+c1}{\PYZsh{} implementing k means clustering on iot based attacks dataset}
        \PY{c+c1}{\PYZsh{} importing required libraries}
        \PY{k+kn}{import} \PY{n+nn}{pandas} \PY{k}{as} \PY{n+nn}{pd}
        \PY{k+kn}{import} \PY{n+nn}{numpy} \PY{k}{as} \PY{n+nn}{np}
        \PY{k+kn}{import} \PY{n+nn}{matplotlib}\PY{n+nn}{.}\PY{n+nn}{pyplot} \PY{k}{as} \PY{n+nn}{plt}
        \PY{k+kn}{from} \PY{n+nn}{sklearn}\PY{n+nn}{.}\PY{n+nn}{cluster} \PY{k+kn}{import} \PY{n}{KMeans}
        \PY{k+kn}{from} \PY{n+nn}{sklearn}\PY{n+nn}{.}\PY{n+nn}{preprocessing} \PY{k+kn}{import} \PY{n}{StandardScaler}
        \end{Verbatim}
        \end{tcolorbox}
        
            \begin{tcolorbox}[breakable, size=fbox, boxrule=1pt, pad at break*=1mm,colback=cellbackground, colframe=cellborder]
        \prompt{In}{incolor}{96}{\boxspacing}
        \begin{Verbatim}[commandchars=\\\{\}]
        \PY{c+c1}{\PYZsh{} importing the dataset}
        \PY{n}{df} \PY{o}{=} \PY{n}{pd}\PY{o}{.}\PY{n}{read\PYZus{}csv}\PY{p}{(}\PY{l+s+s2}{\PYZdq{}}\PY{l+s+s2}{part\PYZhy{}00000\PYZhy{}363d1ba3\PYZhy{}8ab5\PYZhy{}4f96\PYZhy{}bc25\PYZhy{}4d5862db7cb9\PYZhy{}c000.csv}\PY{l+s+s2}{\PYZdq{}}\PY{p}{)}
        \PY{n}{df}\PY{o}{.}\PY{n}{head}\PY{p}{(}\PY{p}{)}
        \end{Verbatim}
        \end{tcolorbox}
        
                    \begin{tcolorbox}[breakable, size=fbox, boxrule=.5pt, pad at break*=1mm, opacityfill=0]
        \prompt{Out}{outcolor}{96}{\boxspacing}
        \begin{Verbatim}[commandchars=\\\{\}]
           flow\_duration  Header\_Length  Protocol Type  Duration         Rate  \textbackslash{}
        0       0.000000          54.00           6.00     64.00     0.329807
        1       0.000000          57.04           6.33     64.00     4.290556
        2       0.000000           0.00           1.00     64.00    33.396799
        3       0.328175       76175.00          17.00     64.00  4642.133010
        4       0.117320         101.73           6.11     65.91     6.202211
        
                 Srate  Drate  fin\_flag\_number  syn\_flag\_number  rst\_flag\_number  {\ldots}  \textbackslash{}
        0     0.329807    0.0              1.0              0.0              1.0  {\ldots}
        1     4.290556    0.0              0.0              0.0              0.0  {\ldots}
        2    33.396799    0.0              0.0              0.0              0.0  {\ldots}
        3  4642.133010    0.0              0.0              0.0              0.0  {\ldots}
        4     6.202211    0.0              0.0              1.0              0.0  {\ldots}
        
                 Std  Tot size           IAT  Number   Magnitue     Radius  \textbackslash{}
        0   0.000000     54.00  8.334383e+07     9.5  10.392305   0.000000
        1   2.822973     57.04  8.292607e+07     9.5  10.464666   4.010353
        2   0.000000     42.00  8.312799e+07     9.5   9.165151   0.000000
        3   0.000000     50.00  8.301570e+07     9.5  10.000000   0.000000
        4  23.113111     57.88  8.297300e+07     9.5  11.346876  32.716243
        
            Covariance  Variance  Weight             label
        0     0.000000      0.00  141.55  DDoS-RSTFINFlood
        1   160.987842      0.05  141.55     DoS-TCP\_Flood
        2     0.000000      0.00  141.55   DDoS-ICMP\_Flood
        3     0.000000      0.00  141.55     DoS-UDP\_Flood
        4  3016.808286      0.19  141.55     DoS-SYN\_Flood
        
        [5 rows x 47 columns]
        \end{Verbatim}
        \end{tcolorbox}
                
            \begin{tcolorbox}[breakable, size=fbox, boxrule=1pt, pad at break*=1mm,colback=cellbackground, colframe=cellborder]
        \prompt{In}{incolor}{97}{\boxspacing}
        \begin{Verbatim}[commandchars=\\\{\}]
        \PY{n}{df}\PY{o}{.}\PY{n}{info}\PY{p}{(}\PY{p}{)}
        \end{Verbatim}
        \end{tcolorbox}
        
            \begin{Verbatim}[commandchars=\\\{\}]
        <class 'pandas.core.frame.DataFrame'>
        RangeIndex: 238687 entries, 0 to 238686
        Data columns (total 47 columns):
         \#   Column           Non-Null Count   Dtype
        ---  ------           --------------   -----
         0   flow\_duration    238687 non-null  float64
         1   Header\_Length    238687 non-null  float64
         2   Protocol Type    238687 non-null  float64
         3   Duration         238687 non-null  float64
         4   Rate             238687 non-null  float64
         5   Srate            238687 non-null  float64
         6   Drate            238687 non-null  float64
         7   fin\_flag\_number  238687 non-null  float64
         8   syn\_flag\_number  238687 non-null  float64
         9   rst\_flag\_number  238687 non-null  float64
         10  psh\_flag\_number  238687 non-null  float64
         11  ack\_flag\_number  238687 non-null  float64
         12  ece\_flag\_number  238687 non-null  float64
         13  cwr\_flag\_number  238687 non-null  float64
         14  ack\_count        238687 non-null  float64
         15  syn\_count        238687 non-null  float64
         16  fin\_count        238687 non-null  float64
         17  urg\_count        238687 non-null  float64
         18  rst\_count        238687 non-null  float64
         19  HTTP             238687 non-null  float64
         20  HTTPS            238687 non-null  float64
         21  DNS              238687 non-null  float64
         22  Telnet           238687 non-null  float64
         23  SMTP             238687 non-null  float64
         24  SSH              238687 non-null  float64
         25  IRC              238687 non-null  float64
         26  TCP              238687 non-null  float64
         27  UDP              238687 non-null  float64
         28  DHCP             238687 non-null  float64
         29  ARP              238687 non-null  float64
         30  ICMP             238687 non-null  float64
         31  IPv              238687 non-null  float64
         32  LLC              238687 non-null  float64
         33  Tot sum          238687 non-null  float64
         34  Min              238687 non-null  float64
         35  Max              238687 non-null  float64
         36  AVG              238687 non-null  float64
         37  Std              238687 non-null  float64
         38  Tot size         238687 non-null  float64
         39  IAT              238687 non-null  float64
         40  Number           238687 non-null  float64
         41  Magnitue         238687 non-null  float64
         42  Radius           238687 non-null  float64
         43  Covariance       238687 non-null  float64
         44  Variance         238687 non-null  float64
         45  Weight           238687 non-null  float64
         46  label            238687 non-null  object
        dtypes: float64(46), object(1)
        memory usage: 85.6+ MB
            \end{Verbatim}
        
            \begin{tcolorbox}[breakable, size=fbox, boxrule=1pt, pad at break*=1mm,colback=cellbackground, colframe=cellborder]
        \prompt{In}{incolor}{98}{\boxspacing}
        \begin{Verbatim}[commandchars=\\\{\}]
        \PY{c+c1}{\PYZsh{} lets remove columns that we dont need.}
        \PY{n}{df}\PY{o}{.}\PY{n}{columns}
        \end{Verbatim}
        \end{tcolorbox}
        
                    \begin{tcolorbox}[breakable, size=fbox, boxrule=.5pt, pad at break*=1mm, opacityfill=0]
        \prompt{Out}{outcolor}{98}{\boxspacing}
        \begin{Verbatim}[commandchars=\\\{\}]
        Index(['flow\_duration', 'Header\_Length', 'Protocol Type', 'Duration', 'Rate',
               'Srate', 'Drate', 'fin\_flag\_number', 'syn\_flag\_number',
               'rst\_flag\_number', 'psh\_flag\_number', 'ack\_flag\_number',
               'ece\_flag\_number', 'cwr\_flag\_number', 'ack\_count', 'syn\_count',
               'fin\_count', 'urg\_count', 'rst\_count', 'HTTP', 'HTTPS', 'DNS', 'Telnet',
               'SMTP', 'SSH', 'IRC', 'TCP', 'UDP', 'DHCP', 'ARP', 'ICMP', 'IPv', 'LLC',
               'Tot sum', 'Min', 'Max', 'AVG', 'Std', 'Tot size', 'IAT', 'Number',
               'Magnitue', 'Radius', 'Covariance', 'Variance', 'Weight', 'label'],
              dtype='object')
        \end{Verbatim}
        \end{tcolorbox}
                
            \begin{tcolorbox}[breakable, size=fbox, boxrule=1pt, pad at break*=1mm,colback=cellbackground, colframe=cellborder]
        \prompt{In}{incolor}{ }{\boxspacing}
        \begin{Verbatim}[commandchars=\\\{\}]
        
        \end{Verbatim}
        \end{tcolorbox}
        
            \begin{tcolorbox}[breakable, size=fbox, boxrule=1pt, pad at break*=1mm,colback=cellbackground, colframe=cellborder]
        \prompt{In}{incolor}{99}{\boxspacing}
        \begin{Verbatim}[commandchars=\\\{\}]
        \PY{n}{df}\PY{o}{.}\PY{n}{drop}\PY{p}{(}
            \PY{n}{columns}\PY{o}{=}\PY{p}{[}
                \PY{l+s+s2}{\PYZdq{}}\PY{l+s+s2}{Drate}\PY{l+s+s2}{\PYZdq{}}\PY{p}{,}
                \PY{l+s+s2}{\PYZdq{}}\PY{l+s+s2}{fin\PYZus{}flag\PYZus{}number}\PY{l+s+s2}{\PYZdq{}}\PY{p}{,}
                \PY{l+s+s2}{\PYZdq{}}\PY{l+s+s2}{syn\PYZus{}flag\PYZus{}number}\PY{l+s+s2}{\PYZdq{}}\PY{p}{,}
                \PY{l+s+s2}{\PYZdq{}}\PY{l+s+s2}{rst\PYZus{}flag\PYZus{}number}\PY{l+s+s2}{\PYZdq{}}\PY{p}{,}
                \PY{l+s+s2}{\PYZdq{}}\PY{l+s+s2}{psh\PYZus{}flag\PYZus{}number}\PY{l+s+s2}{\PYZdq{}}\PY{p}{,}
                \PY{l+s+s2}{\PYZdq{}}\PY{l+s+s2}{ack\PYZus{}flag\PYZus{}number}\PY{l+s+s2}{\PYZdq{}}\PY{p}{,}
                \PY{l+s+s2}{\PYZdq{}}\PY{l+s+s2}{ece\PYZus{}flag\PYZus{}number}\PY{l+s+s2}{\PYZdq{}}\PY{p}{,}
                \PY{l+s+s2}{\PYZdq{}}\PY{l+s+s2}{cwr\PYZus{}flag\PYZus{}number}\PY{l+s+s2}{\PYZdq{}}\PY{p}{,}
                \PY{l+s+s2}{\PYZdq{}}\PY{l+s+s2}{ack\PYZus{}count}\PY{l+s+s2}{\PYZdq{}}\PY{p}{,}
                \PY{l+s+s2}{\PYZdq{}}\PY{l+s+s2}{syn\PYZus{}count}\PY{l+s+s2}{\PYZdq{}}\PY{p}{,}
                \PY{l+s+s2}{\PYZdq{}}\PY{l+s+s2}{fin\PYZus{}count}\PY{l+s+s2}{\PYZdq{}}\PY{p}{,}
                \PY{l+s+s2}{\PYZdq{}}\PY{l+s+s2}{urg\PYZus{}count}\PY{l+s+s2}{\PYZdq{}}\PY{p}{,}
                \PY{l+s+s2}{\PYZdq{}}\PY{l+s+s2}{rst\PYZus{}count}\PY{l+s+s2}{\PYZdq{}}\PY{p}{,}
                \PY{l+s+s2}{\PYZdq{}}\PY{l+s+s2}{HTTP}\PY{l+s+s2}{\PYZdq{}}\PY{p}{,}
                \PY{l+s+s2}{\PYZdq{}}\PY{l+s+s2}{HTTPS}\PY{l+s+s2}{\PYZdq{}}\PY{p}{,}
                \PY{l+s+s2}{\PYZdq{}}\PY{l+s+s2}{DNS}\PY{l+s+s2}{\PYZdq{}}\PY{p}{,}
                \PY{l+s+s2}{\PYZdq{}}\PY{l+s+s2}{Telnet}\PY{l+s+s2}{\PYZdq{}}\PY{p}{,}
                \PY{l+s+s2}{\PYZdq{}}\PY{l+s+s2}{SMTP}\PY{l+s+s2}{\PYZdq{}}\PY{p}{,}
                \PY{l+s+s2}{\PYZdq{}}\PY{l+s+s2}{SSH}\PY{l+s+s2}{\PYZdq{}}\PY{p}{,}
                \PY{l+s+s2}{\PYZdq{}}\PY{l+s+s2}{IRC}\PY{l+s+s2}{\PYZdq{}}\PY{p}{,}
                \PY{l+s+s2}{\PYZdq{}}\PY{l+s+s2}{TCP}\PY{l+s+s2}{\PYZdq{}}\PY{p}{,}
                \PY{l+s+s2}{\PYZdq{}}\PY{l+s+s2}{UDP}\PY{l+s+s2}{\PYZdq{}}\PY{p}{,}
                \PY{l+s+s2}{\PYZdq{}}\PY{l+s+s2}{DHCP}\PY{l+s+s2}{\PYZdq{}}\PY{p}{,}
                \PY{l+s+s2}{\PYZdq{}}\PY{l+s+s2}{ARP}\PY{l+s+s2}{\PYZdq{}}\PY{p}{,}
                \PY{l+s+s2}{\PYZdq{}}\PY{l+s+s2}{ICMP}\PY{l+s+s2}{\PYZdq{}}\PY{p}{,}
                \PY{l+s+s2}{\PYZdq{}}\PY{l+s+s2}{IPv}\PY{l+s+s2}{\PYZdq{}}\PY{p}{,}
                \PY{l+s+s2}{\PYZdq{}}\PY{l+s+s2}{LLC}\PY{l+s+s2}{\PYZdq{}}\PY{p}{,}
                \PY{l+s+s2}{\PYZdq{}}\PY{l+s+s2}{Tot sum}\PY{l+s+s2}{\PYZdq{}}\PY{p}{,}
                \PY{l+s+s2}{\PYZdq{}}\PY{l+s+s2}{Min}\PY{l+s+s2}{\PYZdq{}}\PY{p}{,}
                \PY{l+s+s2}{\PYZdq{}}\PY{l+s+s2}{Max}\PY{l+s+s2}{\PYZdq{}}\PY{p}{,}
                \PY{l+s+s2}{\PYZdq{}}\PY{l+s+s2}{AVG}\PY{l+s+s2}{\PYZdq{}}\PY{p}{,}
                \PY{l+s+s2}{\PYZdq{}}\PY{l+s+s2}{Std}\PY{l+s+s2}{\PYZdq{}}\PY{p}{,}
                \PY{l+s+s2}{\PYZdq{}}\PY{l+s+s2}{Tot size}\PY{l+s+s2}{\PYZdq{}}\PY{p}{,}
                \PY{l+s+s2}{\PYZdq{}}\PY{l+s+s2}{IAT}\PY{l+s+s2}{\PYZdq{}}\PY{p}{,}
                \PY{l+s+s2}{\PYZdq{}}\PY{l+s+s2}{Number}\PY{l+s+s2}{\PYZdq{}}\PY{p}{,}
                \PY{l+s+s2}{\PYZdq{}}\PY{l+s+s2}{Magnitue}\PY{l+s+s2}{\PYZdq{}}\PY{p}{,}
                \PY{l+s+s2}{\PYZdq{}}\PY{l+s+s2}{Radius}\PY{l+s+s2}{\PYZdq{}}\PY{p}{,}
                \PY{l+s+s2}{\PYZdq{}}\PY{l+s+s2}{Covariance}\PY{l+s+s2}{\PYZdq{}}\PY{p}{,}
                \PY{l+s+s2}{\PYZdq{}}\PY{l+s+s2}{Variance}\PY{l+s+s2}{\PYZdq{}}\PY{p}{,}
                \PY{l+s+s2}{\PYZdq{}}\PY{l+s+s2}{Weight}\PY{l+s+s2}{\PYZdq{}}\PY{p}{,}
            \PY{p}{]}\PY{p}{,}
            \PY{n}{inplace}\PY{o}{=}\PY{k+kc}{True}\PY{p}{,}
        \PY{p}{)}
        \end{Verbatim}
        \end{tcolorbox}
        
            \begin{tcolorbox}[breakable, size=fbox, boxrule=1pt, pad at break*=1mm,colback=cellbackground, colframe=cellborder]
        \prompt{In}{incolor}{100}{\boxspacing}
        \begin{Verbatim}[commandchars=\\\{\}]
        \PY{c+c1}{\PYZsh{} let us now visualize scatter plots of the data}
        \PY{k+kn}{import} \PY{n+nn}{seaborn} \PY{k}{as} \PY{n+nn}{sns}
        
        \PY{c+c1}{\PYZsh{} before that lets scale the data down, just take the first 1000 rows}
        \PY{n}{scaled\PYZus{}df} \PY{o}{=} \PY{n}{df}\PY{p}{[}\PY{p}{:}\PY{l+m+mi}{1000}\PY{p}{]}
        \end{Verbatim}
        \end{tcolorbox}
        
            \begin{tcolorbox}[breakable, size=fbox, boxrule=1pt, pad at break*=1mm,colback=cellbackground, colframe=cellborder]
        \prompt{In}{incolor}{101}{\boxspacing}
        \begin{Verbatim}[commandchars=\\\{\}]
        \PY{n}{sns}\PY{o}{.}\PY{n}{relplot}\PY{p}{(}\PY{n}{df}\PY{p}{,} \PY{n}{y}\PY{o}{=}\PY{l+s+s2}{\PYZdq{}}\PY{l+s+s2}{flow\PYZus{}duration}\PY{l+s+s2}{\PYZdq{}}\PY{p}{,} \PY{n}{x}\PY{o}{=}\PY{n}{df}\PY{o}{.}\PY{n}{index}\PY{p}{,} \PY{n}{kind}\PY{o}{=}\PY{l+s+s2}{\PYZdq{}}\PY{l+s+s2}{scatter}\PY{l+s+s2}{\PYZdq{}}\PY{p}{,} \PY{n}{hue}\PY{o}{=}\PY{l+s+s2}{\PYZdq{}}\PY{l+s+s2}{label}\PY{l+s+s2}{\PYZdq{}}\PY{p}{)}
        \end{Verbatim}
        \end{tcolorbox}
        
                    \begin{tcolorbox}[breakable, size=fbox, boxrule=.5pt, pad at break*=1mm, opacityfill=0]
        \prompt{Out}{outcolor}{101}{\boxspacing}
        \begin{Verbatim}[commandchars=\\\{\}]
        <seaborn.axisgrid.FacetGrid at 0x2d6006d7520>
        \end{Verbatim}
        \end{tcolorbox}
                
            \begin{center}
            \adjustimage{max size={0.9\linewidth}{0.9\paperheight}}{output_7_1.png}
            \end{center}
            { \hspace*{\fill} \\}
            
            \begin{tcolorbox}[breakable, size=fbox, boxrule=1pt, pad at break*=1mm,colback=cellbackground, colframe=cellborder]
        \prompt{In}{incolor}{102}{\boxspacing}
        \begin{Verbatim}[commandchars=\\\{\}]
        \PY{c+c1}{\PYZsh{} lets go through the dataset, and remove those rows with labels that occur very less times to remove noise}
        
        \PY{n}{df}\PY{p}{[}\PY{l+s+s2}{\PYZdq{}}\PY{l+s+s2}{label}\PY{l+s+s2}{\PYZdq{}}\PY{p}{]}\PY{o}{.}\PY{n}{value\PYZus{}counts}\PY{p}{(}\PY{p}{)}
        \end{Verbatim}
        \end{tcolorbox}
        
                    \begin{tcolorbox}[breakable, size=fbox, boxrule=.5pt, pad at break*=1mm, opacityfill=0]
        \prompt{Out}{outcolor}{102}{\boxspacing}
        \begin{Verbatim}[commandchars=\\\{\}]
        label
        DDoS-ICMP\_Flood            36554
        DDoS-UDP\_Flood             27626
        DDoS-TCP\_Flood             23149
        DDoS-PSHACK\_Flood          21210
        DDoS-SYN\_Flood             20739
        DDoS-RSTFINFlood           20669
        DDoS-SynonymousIP\_Flood    18189
        DoS-UDP\_Flood              16957
        DoS-TCP\_Flood              13630
        DoS-SYN\_Flood              10275
        BenignTraffic               5600
        Mirai-greeth\_flood          5016
        Mirai-udpplain              4661
        Mirai-greip\_flood           3758
        DDoS-ICMP\_Fragmentation     2377
        MITM-ArpSpoofing            1614
        DDoS-ACK\_Fragmentation      1505
        DDoS-UDP\_Fragmentation      1484
        DNS\_Spoofing                 925
        Recon-HostDiscovery          697
        Recon-OSScan                 517
        Recon-PortScan               430
        DoS-HTTP\_Flood               414
        VulnerabilityScan            210
        DDoS-HTTP\_Flood              169
        DDoS-SlowLoris               106
        DictionaryBruteForce          63
        SqlInjection                  31
        BrowserHijacking              30
        CommandInjection              28
        Backdoor\_Malware              22
        XSS                           18
        Uploading\_Attack               8
        Recon-PingSweep                6
        Name: count, dtype: int64
        \end{Verbatim}
        \end{tcolorbox}
                
            \begin{tcolorbox}[breakable, size=fbox, boxrule=1pt, pad at break*=1mm,colback=cellbackground, colframe=cellborder]
        \prompt{In}{incolor}{103}{\boxspacing}
        \begin{Verbatim}[commandchars=\\\{\}]
        \PY{c+c1}{\PYZsh{} lets keep the top 5 labels and remove the rest}
        \PY{n}{top\PYZus{}labels} \PY{o}{=} \PY{n}{df}\PY{p}{[}\PY{l+s+s2}{\PYZdq{}}\PY{l+s+s2}{label}\PY{l+s+s2}{\PYZdq{}}\PY{p}{]}\PY{o}{.}\PY{n}{value\PYZus{}counts}\PY{p}{(}\PY{p}{)}\PY{o}{.}\PY{n}{head}\PY{p}{(}\PY{l+m+mi}{5}\PY{p}{)}\PY{o}{.}\PY{n}{index}
        \PY{n}{df} \PY{o}{=} \PY{n}{df}\PY{p}{[}\PY{n}{df}\PY{p}{[}\PY{l+s+s2}{\PYZdq{}}\PY{l+s+s2}{label}\PY{l+s+s2}{\PYZdq{}}\PY{p}{]}\PY{o}{.}\PY{n}{isin}\PY{p}{(}\PY{n}{top\PYZus{}labels}\PY{p}{)}\PY{p}{]}
        \end{Verbatim}
        \end{tcolorbox}
        
            \begin{tcolorbox}[breakable, size=fbox, boxrule=1pt, pad at break*=1mm,colback=cellbackground, colframe=cellborder]
        \prompt{In}{incolor}{104}{\boxspacing}
        \begin{Verbatim}[commandchars=\\\{\}]
        \PY{n}{df}\PY{o}{.}\PY{n}{shape}
        \end{Verbatim}
        \end{tcolorbox}
        
                    \begin{tcolorbox}[breakable, size=fbox, boxrule=.5pt, pad at break*=1mm, opacityfill=0]
        \prompt{Out}{outcolor}{104}{\boxspacing}
        \begin{Verbatim}[commandchars=\\\{\}]
        (129278, 7)
        \end{Verbatim}
        \end{tcolorbox}
                
            \begin{tcolorbox}[breakable, size=fbox, boxrule=1pt, pad at break*=1mm,colback=cellbackground, colframe=cellborder]
        \prompt{In}{incolor}{105}{\boxspacing}
        \begin{Verbatim}[commandchars=\\\{\}]
        \PY{c+c1}{\PYZsh{} lets try plotting again}
        \PY{n}{sns}\PY{o}{.}\PY{n}{relplot}\PY{p}{(}\PY{n}{df}\PY{p}{,} \PY{n}{y}\PY{o}{=}\PY{l+s+s2}{\PYZdq{}}\PY{l+s+s2}{flow\PYZus{}duration}\PY{l+s+s2}{\PYZdq{}}\PY{p}{,} \PY{n}{x}\PY{o}{=}\PY{n}{df}\PY{o}{.}\PY{n}{index}\PY{p}{,} \PY{n}{kind}\PY{o}{=}\PY{l+s+s2}{\PYZdq{}}\PY{l+s+s2}{scatter}\PY{l+s+s2}{\PYZdq{}}\PY{p}{,} \PY{n}{hue}\PY{o}{=}\PY{l+s+s2}{\PYZdq{}}\PY{l+s+s2}{label}\PY{l+s+s2}{\PYZdq{}}\PY{p}{)}
        \end{Verbatim}
        \end{tcolorbox}
        
                    \begin{tcolorbox}[breakable, size=fbox, boxrule=.5pt, pad at break*=1mm, opacityfill=0]
        \prompt{Out}{outcolor}{105}{\boxspacing}
        \begin{Verbatim}[commandchars=\\\{\}]
        <seaborn.axisgrid.FacetGrid at 0x2d651613af0>
        \end{Verbatim}
        \end{tcolorbox}
                
            \begin{center}
            \adjustimage{max size={0.9\linewidth}{0.9\paperheight}}{output_11_1.png}
            \end{center}
            { \hspace*{\fill} \\}
            
            \begin{tcolorbox}[breakable, size=fbox, boxrule=1pt, pad at break*=1mm,colback=cellbackground, colframe=cellborder]
        \prompt{In}{incolor}{127}{\boxspacing}
        \begin{Verbatim}[commandchars=\\\{\}]
        \PY{c+c1}{\PYZsh{} lets remove flow\PYZus{}duration rows with z scores more than 3}
        \PY{k+kn}{from} \PY{n+nn}{scipy} \PY{k+kn}{import} \PY{n}{stats}
        
        \PY{n}{z} \PY{o}{=} \PY{n}{np}\PY{o}{.}\PY{n}{abs}\PY{p}{(}\PY{n}{stats}\PY{o}{.}\PY{n}{zscore}\PY{p}{(}\PY{n}{df}\PY{p}{[}\PY{l+s+s2}{\PYZdq{}}\PY{l+s+s2}{flow\PYZus{}duration}\PY{l+s+s2}{\PYZdq{}}\PY{p}{]}\PY{p}{)}\PY{p}{)}
        \PY{n}{non\PYZus{}outlier\PYZus{}df} \PY{o}{=} \PY{n}{df}\PY{p}{[}\PY{p}{(}\PY{n}{z} \PY{o}{\PYZlt{}} \PY{l+m+mf}{0.0115}\PY{p}{)}\PY{p}{]}
        \PY{n}{non\PYZus{}outlier\PYZus{}df}\PY{o}{.}\PY{n}{shape}
        \end{Verbatim}
        \end{tcolorbox}
        
                    \begin{tcolorbox}[breakable, size=fbox, boxrule=.5pt, pad at break*=1mm, opacityfill=0]
        \prompt{Out}{outcolor}{127}{\boxspacing}
        \begin{Verbatim}[commandchars=\\\{\}]
        (34489, 7)
        \end{Verbatim}
        \end{tcolorbox}
                
            \begin{tcolorbox}[breakable, size=fbox, boxrule=1pt, pad at break*=1mm,colback=cellbackground, colframe=cellborder]
        \prompt{In}{incolor}{128}{\boxspacing}
        \begin{Verbatim}[commandchars=\\\{\}]
        \PY{c+c1}{\PYZsh{} lets plot a distribution}
        \PY{n}{sns}\PY{o}{.}\PY{n}{displot}\PY{p}{(}\PY{n}{non\PYZus{}outlier\PYZus{}df}\PY{p}{[}\PY{l+s+s2}{\PYZdq{}}\PY{l+s+s2}{flow\PYZus{}duration}\PY{l+s+s2}{\PYZdq{}}\PY{p}{]}\PY{p}{,} \PY{n}{kind}\PY{o}{=}\PY{l+s+s2}{\PYZdq{}}\PY{l+s+s2}{kde}\PY{l+s+s2}{\PYZdq{}}\PY{p}{,} \PY{n}{fill}\PY{o}{=}\PY{k+kc}{True}\PY{p}{)}
        \end{Verbatim}
        \end{tcolorbox}
        
                    \begin{tcolorbox}[breakable, size=fbox, boxrule=.5pt, pad at break*=1mm, opacityfill=0]
        \prompt{Out}{outcolor}{128}{\boxspacing}
        \begin{Verbatim}[commandchars=\\\{\}]
        <seaborn.axisgrid.FacetGrid at 0x2d62991a350>
        \end{Verbatim}
        \end{tcolorbox}
                
            \begin{center}
            \adjustimage{max size={0.9\linewidth}{0.9\paperheight}}{output_13_1.png}
            \end{center}
            { \hspace*{\fill} \\}
            
            \begin{tcolorbox}[breakable, size=fbox, boxrule=1pt, pad at break*=1mm,colback=cellbackground, colframe=cellborder]
        \prompt{In}{incolor}{129}{\boxspacing}
        \begin{Verbatim}[commandchars=\\\{\}]
        \PY{c+c1}{\PYZsh{} lets try plotting again}
        \PY{n}{sns}\PY{o}{.}\PY{n}{relplot}\PY{p}{(}
            \PY{n}{non\PYZus{}outlier\PYZus{}df}\PY{p}{,}
            \PY{n}{y}\PY{o}{=}\PY{l+s+s2}{\PYZdq{}}\PY{l+s+s2}{flow\PYZus{}duration}\PY{l+s+s2}{\PYZdq{}}\PY{p}{,}
            \PY{n}{x}\PY{o}{=}\PY{n}{non\PYZus{}outlier\PYZus{}df}\PY{o}{.}\PY{n}{index}\PY{p}{,}
            \PY{n}{kind}\PY{o}{=}\PY{l+s+s2}{\PYZdq{}}\PY{l+s+s2}{scatter}\PY{l+s+s2}{\PYZdq{}}\PY{p}{,}
            \PY{n}{hue}\PY{o}{=}\PY{l+s+s2}{\PYZdq{}}\PY{l+s+s2}{label}\PY{l+s+s2}{\PYZdq{}}\PY{p}{,}
        \PY{p}{)}
        \end{Verbatim}
        \end{tcolorbox}
        
                    \begin{tcolorbox}[breakable, size=fbox, boxrule=.5pt, pad at break*=1mm, opacityfill=0]
        \prompt{Out}{outcolor}{129}{\boxspacing}
        \begin{Verbatim}[commandchars=\\\{\}]
        <seaborn.axisgrid.FacetGrid at 0x2d644e29030>
        \end{Verbatim}
        \end{tcolorbox}
                
            \begin{center}
            \adjustimage{max size={0.9\linewidth}{0.9\paperheight}}{output_14_1.png}
            \end{center}
            { \hspace*{\fill} \\}
            
            \begin{tcolorbox}[breakable, size=fbox, boxrule=1pt, pad at break*=1mm,colback=cellbackground, colframe=cellborder]
        \prompt{In}{incolor}{131}{\boxspacing}
        \begin{Verbatim}[commandchars=\\\{\}]
        \PY{c+c1}{\PYZsh{} lets now try with only 1000 rows}
        \PY{n}{scaled\PYZus{}df} \PY{o}{=} \PY{n}{non\PYZus{}outlier\PYZus{}df}\PY{p}{[}\PY{p}{:}\PY{l+m+mi}{100}\PY{p}{]}
        
        \PY{c+c1}{\PYZsh{} lets plot}
        \PY{n}{sns}\PY{o}{.}\PY{n}{relplot}\PY{p}{(}
            \PY{n}{scaled\PYZus{}df}\PY{p}{,} \PY{n}{y}\PY{o}{=}\PY{l+s+s2}{\PYZdq{}}\PY{l+s+s2}{flow\PYZus{}duration}\PY{l+s+s2}{\PYZdq{}}\PY{p}{,} \PY{n}{x}\PY{o}{=}\PY{n}{scaled\PYZus{}df}\PY{o}{.}\PY{n}{index}\PY{p}{,} \PY{n}{kind}\PY{o}{=}\PY{l+s+s2}{\PYZdq{}}\PY{l+s+s2}{scatter}\PY{l+s+s2}{\PYZdq{}}\PY{p}{,} \PY{n}{hue}\PY{o}{=}\PY{l+s+s2}{\PYZdq{}}\PY{l+s+s2}{label}\PY{l+s+s2}{\PYZdq{}}
        \PY{p}{)}
        \end{Verbatim}
        \end{tcolorbox}
        
                    \begin{tcolorbox}[breakable, size=fbox, boxrule=.5pt, pad at break*=1mm, opacityfill=0]
        \prompt{Out}{outcolor}{131}{\boxspacing}
        \begin{Verbatim}[commandchars=\\\{\}]
        <seaborn.axisgrid.FacetGrid at 0x2d64f866d40>
        \end{Verbatim}
        \end{tcolorbox}
                
            \begin{center}
            \adjustimage{max size={0.9\linewidth}{0.9\paperheight}}{output_15_1.png}
            \end{center}
            { \hspace*{\fill} \\}
            
            \begin{tcolorbox}[breakable, size=fbox, boxrule=1pt, pad at break*=1mm,colback=cellbackground, colframe=cellborder]
        \prompt{In}{incolor}{135}{\boxspacing}
        \begin{Verbatim}[commandchars=\\\{\}]
        \PY{c+c1}{\PYZsh{} we still cant see a nice cluster here, lets try and remove ddos udp flood}
        \PY{n}{scaled\PYZus{}df} \PY{o}{=} \PY{n}{non\PYZus{}outlier\PYZus{}df}\PY{p}{[}\PY{n}{non\PYZus{}outlier\PYZus{}df}\PY{p}{[}\PY{l+s+s2}{\PYZdq{}}\PY{l+s+s2}{label}\PY{l+s+s2}{\PYZdq{}}\PY{p}{]} \PY{o}{!=} \PY{l+s+s2}{\PYZdq{}}\PY{l+s+s2}{DDoS\PYZhy{}UDP\PYZus{}Flood}\PY{l+s+s2}{\PYZdq{}}\PY{p}{]}
        \end{Verbatim}
        \end{tcolorbox}
        
            \begin{tcolorbox}[breakable, size=fbox, boxrule=1pt, pad at break*=1mm,colback=cellbackground, colframe=cellborder]
        \prompt{In}{incolor}{138}{\boxspacing}
        \begin{Verbatim}[commandchars=\\\{\}]
        \PY{c+c1}{\PYZsh{} lets plot again}
        \PY{n}{scaled\PYZus{}df} \PY{o}{=} \PY{n}{scaled\PYZus{}df}\PY{p}{[}\PY{p}{:}\PY{l+m+mi}{100}\PY{p}{]}
        \PY{n}{sns}\PY{o}{.}\PY{n}{relplot}\PY{p}{(}
            \PY{n}{scaled\PYZus{}df}\PY{p}{,} \PY{n}{y}\PY{o}{=}\PY{l+s+s2}{\PYZdq{}}\PY{l+s+s2}{flow\PYZus{}duration}\PY{l+s+s2}{\PYZdq{}}\PY{p}{,} \PY{n}{x}\PY{o}{=}\PY{n}{scaled\PYZus{}df}\PY{o}{.}\PY{n}{index}\PY{p}{,} \PY{n}{kind}\PY{o}{=}\PY{l+s+s2}{\PYZdq{}}\PY{l+s+s2}{scatter}\PY{l+s+s2}{\PYZdq{}}\PY{p}{,} \PY{n}{hue}\PY{o}{=}\PY{l+s+s2}{\PYZdq{}}\PY{l+s+s2}{label}\PY{l+s+s2}{\PYZdq{}}
        \PY{p}{)}
        \end{Verbatim}
        \end{tcolorbox}
        
                    \begin{tcolorbox}[breakable, size=fbox, boxrule=.5pt, pad at break*=1mm, opacityfill=0]
        \prompt{Out}{outcolor}{138}{\boxspacing}
        \begin{Verbatim}[commandchars=\\\{\}]
        <seaborn.axisgrid.FacetGrid at 0x2d651567850>
        \end{Verbatim}
        \end{tcolorbox}
                
            \begin{center}
            \adjustimage{max size={0.9\linewidth}{0.9\paperheight}}{output_17_1.png}
            \end{center}
            { \hspace*{\fill} \\}
            
            \begin{tcolorbox}[breakable, size=fbox, boxrule=1pt, pad at break*=1mm,colback=cellbackground, colframe=cellborder]
        \prompt{In}{incolor}{139}{\boxspacing}
        \begin{Verbatim}[commandchars=\\\{\}]
        \PY{c+c1}{\PYZsh{} lets try other features}
        \PY{n}{df}\PY{o}{.}\PY{n}{head}\PY{p}{(}\PY{p}{)}
        \end{Verbatim}
        \end{tcolorbox}
        
                    \begin{tcolorbox}[breakable, size=fbox, boxrule=.5pt, pad at break*=1mm, opacityfill=0]
        \prompt{Out}{outcolor}{139}{\boxspacing}
        \begin{Verbatim}[commandchars=\\\{\}]
            flow\_duration  Header\_Length  Protocol Type  Duration       Rate  \textbackslash{}
        2        0.000000           0.00           1.00     64.00  33.396799
        9        0.000000          54.20           6.00     64.00  11.243547
        10       0.223192          61.54           6.11     64.64   9.087882
        11       0.000000          54.00           6.00     64.00  17.333181
        12       0.000000           0.00           1.00     75.46   0.000000
        
                Srate              label
        2   33.396799    DDoS-ICMP\_Flood
        9   11.243547     DDoS-SYN\_Flood
        10   9.087882  DDoS-PSHACK\_Flood
        11  17.333181     DDoS-TCP\_Flood
        12   0.000000    DDoS-ICMP\_Flood
        \end{Verbatim}
        \end{tcolorbox}
                
            \begin{tcolorbox}[breakable, size=fbox, boxrule=1pt, pad at break*=1mm,colback=cellbackground, colframe=cellborder]
        \prompt{In}{incolor}{141}{\boxspacing}
        \begin{Verbatim}[commandchars=\\\{\}]
        \PY{c+c1}{\PYZsh{} lets plot rate just like flow\PYZus{}duration}
        \PY{n}{df}\PY{o}{.}\PY{n}{shape}
        \end{Verbatim}
        \end{tcolorbox}
        
                    \begin{tcolorbox}[breakable, size=fbox, boxrule=.5pt, pad at break*=1mm, opacityfill=0]
        \prompt{Out}{outcolor}{141}{\boxspacing}
        \begin{Verbatim}[commandchars=\\\{\}]
        (129278, 7)
        \end{Verbatim}
        \end{tcolorbox}
                
            \begin{tcolorbox}[breakable, size=fbox, boxrule=1pt, pad at break*=1mm,colback=cellbackground, colframe=cellborder]
        \prompt{In}{incolor}{143}{\boxspacing}
        \begin{Verbatim}[commandchars=\\\{\}]
        \PY{n}{sns}\PY{o}{.}\PY{n}{relplot}\PY{p}{(}\PY{n}{df}\PY{p}{,} \PY{n}{y}\PY{o}{=}\PY{l+s+s2}{\PYZdq{}}\PY{l+s+s2}{Rate}\PY{l+s+s2}{\PYZdq{}}\PY{p}{,} \PY{n}{x}\PY{o}{=}\PY{n}{df}\PY{o}{.}\PY{n}{index}\PY{p}{,} \PY{n}{kind}\PY{o}{=}\PY{l+s+s2}{\PYZdq{}}\PY{l+s+s2}{scatter}\PY{l+s+s2}{\PYZdq{}}\PY{p}{,} \PY{n}{hue}\PY{o}{=}\PY{l+s+s2}{\PYZdq{}}\PY{l+s+s2}{label}\PY{l+s+s2}{\PYZdq{}}\PY{p}{)}
        \end{Verbatim}
        \end{tcolorbox}
        
                    \begin{tcolorbox}[breakable, size=fbox, boxrule=.5pt, pad at break*=1mm, opacityfill=0]
        \prompt{Out}{outcolor}{143}{\boxspacing}
        \begin{Verbatim}[commandchars=\\\{\}]
        <seaborn.axisgrid.FacetGrid at 0x2d61a8e65c0>
        \end{Verbatim}
        \end{tcolorbox}
                
            \begin{center}
            \adjustimage{max size={0.9\linewidth}{0.9\paperheight}}{output_20_1.png}
            \end{center}
            { \hspace*{\fill} \\}
            
            \begin{tcolorbox}[breakable, size=fbox, boxrule=1pt, pad at break*=1mm,colback=cellbackground, colframe=cellborder]
        \prompt{In}{incolor}{154}{\boxspacing}
        \begin{Verbatim}[commandchars=\\\{\}]
        \PY{c+c1}{\PYZsh{} its not great but there are some clusters, we can try here.}
        \PY{c+c1}{\PYZsh{} lets try implementing k means}
        
        \PY{c+c1}{\PYZsh{} lets try with 5 clusters}
        \PY{n}{kmeans} \PY{o}{=} \PY{n}{KMeans}\PY{p}{(}\PY{n}{n\PYZus{}clusters}\PY{o}{=}\PY{l+m+mi}{5}\PY{p}{)}
        \PY{n}{kmeans}\PY{o}{.}\PY{n}{fit}\PY{p}{(}\PY{n}{df}\PY{p}{[}\PY{p}{[}\PY{l+s+s2}{\PYZdq{}}\PY{l+s+s2}{Rate}\PY{l+s+s2}{\PYZdq{}}\PY{p}{]}\PY{p}{]}\PY{p}{)}
        
        \PY{c+c1}{\PYZsh{} lets get the labels}
        \PY{n}{df}\PY{p}{[}\PY{l+s+s2}{\PYZdq{}}\PY{l+s+s2}{cluster}\PY{l+s+s2}{\PYZdq{}}\PY{p}{]} \PY{o}{=} \PY{n}{kmeans}\PY{o}{.}\PY{n}{labels\PYZus{}}
        
        \PY{c+c1}{\PYZsh{} lets plot}
        \PY{n}{sns}\PY{o}{.}\PY{n}{relplot}\PY{p}{(}
            \PY{n}{df}\PY{p}{,}
            \PY{n}{y}\PY{o}{=}\PY{l+s+s2}{\PYZdq{}}\PY{l+s+s2}{Rate}\PY{l+s+s2}{\PYZdq{}}\PY{p}{,}
            \PY{n}{x}\PY{o}{=}\PY{n}{df}\PY{o}{.}\PY{n}{index}\PY{p}{,}
            \PY{n}{kind}\PY{o}{=}\PY{l+s+s2}{\PYZdq{}}\PY{l+s+s2}{scatter}\PY{l+s+s2}{\PYZdq{}}\PY{p}{,}
            \PY{n}{hue}\PY{o}{=}\PY{l+s+s2}{\PYZdq{}}\PY{l+s+s2}{cluster}\PY{l+s+s2}{\PYZdq{}}\PY{p}{,}
            \PY{n}{palette}\PY{o}{=}\PY{l+s+s2}{\PYZdq{}}\PY{l+s+s2}{viridis}\PY{l+s+s2}{\PYZdq{}}\PY{p}{,}
            \PY{n}{legend}\PY{o}{=}\PY{l+s+s2}{\PYZdq{}}\PY{l+s+s2}{full}\PY{l+s+s2}{\PYZdq{}}\PY{p}{,}
            \PY{n}{style}\PY{o}{=}\PY{l+s+s2}{\PYZdq{}}\PY{l+s+s2}{label}\PY{l+s+s2}{\PYZdq{}}\PY{p}{,}
        \PY{p}{)}
        
        \PY{c+c1}{\PYZsh{} titles}
        \PY{n}{plt}\PY{o}{.}\PY{n}{title}\PY{p}{(}\PY{l+s+s2}{\PYZdq{}}\PY{l+s+s2}{K Means Clustering for Rate of Attacks into 3 Clusters}\PY{l+s+s2}{\PYZdq{}}\PY{p}{)}
        \PY{n}{plt}\PY{o}{.}\PY{n}{xlabel}\PY{p}{(}\PY{l+s+s2}{\PYZdq{}}\PY{l+s+s2}{Index}\PY{l+s+s2}{\PYZdq{}}\PY{p}{)}
        \PY{n}{plt}\PY{o}{.}\PY{n}{ylabel}\PY{p}{(}\PY{l+s+s2}{\PYZdq{}}\PY{l+s+s2}{Rate}\PY{l+s+s2}{\PYZdq{}}\PY{p}{)}
        \end{Verbatim}
        \end{tcolorbox}
        
                    \begin{tcolorbox}[breakable, size=fbox, boxrule=.5pt, pad at break*=1mm, opacityfill=0]
        \prompt{Out}{outcolor}{154}{\boxspacing}
        \begin{Verbatim}[commandchars=\\\{\}]
        Text(49.60998958333333, 0.5, 'Rate')
        \end{Verbatim}
        \end{tcolorbox}
                
            \begin{center}
            \adjustimage{max size={0.9\linewidth}{0.9\paperheight}}{output_21_1.png}
            \end{center}
            { \hspace*{\fill} \\}
            
            \begin{tcolorbox}[breakable, size=fbox, boxrule=1pt, pad at break*=1mm,colback=cellbackground, colframe=cellborder]
        \prompt{In}{incolor}{153}{\boxspacing}
        \begin{Verbatim}[commandchars=\\\{\}]
        \PY{c+c1}{\PYZsh{} lets try with 3 clusters}
        \PY{n}{kmeans} \PY{o}{=} \PY{n}{KMeans}\PY{p}{(}\PY{n}{n\PYZus{}clusters}\PY{o}{=}\PY{l+m+mi}{3}\PY{p}{)}
        \PY{n}{kmeans}\PY{o}{.}\PY{n}{fit}\PY{p}{(}\PY{n}{df}\PY{p}{[}\PY{p}{[}\PY{l+s+s2}{\PYZdq{}}\PY{l+s+s2}{Rate}\PY{l+s+s2}{\PYZdq{}}\PY{p}{]}\PY{p}{]}\PY{p}{)}
        
        \PY{c+c1}{\PYZsh{} lets get the labels}
        \PY{n}{df}\PY{p}{[}\PY{l+s+s2}{\PYZdq{}}\PY{l+s+s2}{cluster}\PY{l+s+s2}{\PYZdq{}}\PY{p}{]} \PY{o}{=} \PY{n}{kmeans}\PY{o}{.}\PY{n}{labels\PYZus{}}
        
        \PY{c+c1}{\PYZsh{} lets plot}
        \PY{n}{sns}\PY{o}{.}\PY{n}{relplot}\PY{p}{(}
            \PY{n}{df}\PY{p}{,}
            \PY{n}{y}\PY{o}{=}\PY{l+s+s2}{\PYZdq{}}\PY{l+s+s2}{Rate}\PY{l+s+s2}{\PYZdq{}}\PY{p}{,}
            \PY{n}{x}\PY{o}{=}\PY{n}{df}\PY{o}{.}\PY{n}{index}\PY{p}{,}
            \PY{n}{kind}\PY{o}{=}\PY{l+s+s2}{\PYZdq{}}\PY{l+s+s2}{scatter}\PY{l+s+s2}{\PYZdq{}}\PY{p}{,}
            \PY{n}{hue}\PY{o}{=}\PY{l+s+s2}{\PYZdq{}}\PY{l+s+s2}{cluster}\PY{l+s+s2}{\PYZdq{}}\PY{p}{,}
            \PY{n}{palette}\PY{o}{=}\PY{l+s+s2}{\PYZdq{}}\PY{l+s+s2}{viridis}\PY{l+s+s2}{\PYZdq{}}\PY{p}{,}
            \PY{n}{legend}\PY{o}{=}\PY{l+s+s2}{\PYZdq{}}\PY{l+s+s2}{full}\PY{l+s+s2}{\PYZdq{}}\PY{p}{,}
            \PY{n}{style}\PY{o}{=}\PY{l+s+s2}{\PYZdq{}}\PY{l+s+s2}{label}\PY{l+s+s2}{\PYZdq{}}\PY{p}{,}
        \PY{p}{)}
        
        \PY{c+c1}{\PYZsh{} titles}
        \PY{n}{plt}\PY{o}{.}\PY{n}{title}\PY{p}{(}\PY{l+s+s2}{\PYZdq{}}\PY{l+s+s2}{K Means Clustering for Rate of Attacks into 3 Clusters}\PY{l+s+s2}{\PYZdq{}}\PY{p}{)}
        \PY{n}{plt}\PY{o}{.}\PY{n}{xlabel}\PY{p}{(}\PY{l+s+s2}{\PYZdq{}}\PY{l+s+s2}{Index}\PY{l+s+s2}{\PYZdq{}}\PY{p}{)}
        \PY{n}{plt}\PY{o}{.}\PY{n}{ylabel}\PY{p}{(}\PY{l+s+s2}{\PYZdq{}}\PY{l+s+s2}{Rate}\PY{l+s+s2}{\PYZdq{}}\PY{p}{)}
        \end{Verbatim}
        \end{tcolorbox}
        
                    \begin{tcolorbox}[breakable, size=fbox, boxrule=.5pt, pad at break*=1mm, opacityfill=0]
        \prompt{Out}{outcolor}{153}{\boxspacing}
        \begin{Verbatim}[commandchars=\\\{\}]
        Text(49.60998958333333, 0.5, 'Rate')
        \end{Verbatim}
        \end{tcolorbox}
                
            \begin{center}
            \adjustimage{max size={0.9\linewidth}{0.9\paperheight}}{output_22_1.png}
            \end{center}
            { \hspace*{\fill} \\}
            
            \begin{tcolorbox}[breakable, size=fbox, boxrule=1pt, pad at break*=1mm,colback=cellbackground, colframe=cellborder]
        \prompt{In}{incolor}{156}{\boxspacing}
        \begin{Verbatim}[commandchars=\\\{\}]
        \PY{n}{scaled\PYZus{}df} \PY{o}{=} \PY{n}{non\PYZus{}outlier\PYZus{}df}\PY{p}{[}\PY{p}{:}\PY{l+m+mi}{100}\PY{p}{]}
        
        \PY{c+c1}{\PYZsh{} we could also try it on flow\PYZus{}duration}
        \PY{n}{kmeans} \PY{o}{=} \PY{n}{KMeans}\PY{p}{(}\PY{n}{n\PYZus{}clusters}\PY{o}{=}\PY{l+m+mi}{5}\PY{p}{)}
        \PY{n}{kmeans}\PY{o}{.}\PY{n}{fit}\PY{p}{(}\PY{n}{scaled\PYZus{}df}\PY{p}{[}\PY{p}{[}\PY{l+s+s2}{\PYZdq{}}\PY{l+s+s2}{flow\PYZus{}duration}\PY{l+s+s2}{\PYZdq{}}\PY{p}{]}\PY{p}{]}\PY{p}{)}
        
        \PY{c+c1}{\PYZsh{} lets get the labels}
        \PY{n}{scaled\PYZus{}df}\PY{p}{[}\PY{l+s+s2}{\PYZdq{}}\PY{l+s+s2}{cluster}\PY{l+s+s2}{\PYZdq{}}\PY{p}{]} \PY{o}{=} \PY{n}{kmeans}\PY{o}{.}\PY{n}{labels\PYZus{}}
        
        \PY{c+c1}{\PYZsh{} lets plot}
        \PY{n}{sns}\PY{o}{.}\PY{n}{relplot}\PY{p}{(}
            \PY{n}{scaled\PYZus{}df}\PY{p}{,}
            \PY{n}{y}\PY{o}{=}\PY{l+s+s2}{\PYZdq{}}\PY{l+s+s2}{flow\PYZus{}duration}\PY{l+s+s2}{\PYZdq{}}\PY{p}{,}
            \PY{n}{x}\PY{o}{=}\PY{n}{scaled\PYZus{}df}\PY{o}{.}\PY{n}{index}\PY{p}{,}
            \PY{n}{kind}\PY{o}{=}\PY{l+s+s2}{\PYZdq{}}\PY{l+s+s2}{scatter}\PY{l+s+s2}{\PYZdq{}}\PY{p}{,}
            \PY{n}{hue}\PY{o}{=}\PY{l+s+s2}{\PYZdq{}}\PY{l+s+s2}{cluster}\PY{l+s+s2}{\PYZdq{}}\PY{p}{,}
            \PY{n}{palette}\PY{o}{=}\PY{l+s+s2}{\PYZdq{}}\PY{l+s+s2}{viridis}\PY{l+s+s2}{\PYZdq{}}\PY{p}{,}
            \PY{n}{legend}\PY{o}{=}\PY{l+s+s2}{\PYZdq{}}\PY{l+s+s2}{full}\PY{l+s+s2}{\PYZdq{}}\PY{p}{,}
            \PY{n}{style}\PY{o}{=}\PY{l+s+s2}{\PYZdq{}}\PY{l+s+s2}{label}\PY{l+s+s2}{\PYZdq{}}\PY{p}{,}
        \PY{p}{)}
        \end{Verbatim}
        \end{tcolorbox}
        
            \begin{Verbatim}[commandchars=\\\{\}]
        C:\textbackslash{}Users\textbackslash{}Krishnaraj\textbackslash{}AppData\textbackslash{}Local\textbackslash{}Temp\textbackslash{}ipykernel\_43912\textbackslash{}3087714870.py:8:
        SettingWithCopyWarning:
        A value is trying to be set on a copy of a slice from a DataFrame.
        Try using .loc[row\_indexer,col\_indexer] = value instead
        
        See the caveats in the documentation: https://pandas.pydata.org/pandas-
        docs/stable/user\_guide/indexing.html\#returning-a-view-versus-a-copy
          scaled\_df["cluster"] = kmeans.labels\_
            \end{Verbatim}
        
                    \begin{tcolorbox}[breakable, size=fbox, boxrule=.5pt, pad at break*=1mm, opacityfill=0]
        \prompt{Out}{outcolor}{156}{\boxspacing}
        \begin{Verbatim}[commandchars=\\\{\}]
        <seaborn.axisgrid.FacetGrid at 0x2d6821e67d0>
        \end{Verbatim}
        \end{tcolorbox}
                
            \begin{center}
            \adjustimage{max size={0.9\linewidth}{0.9\paperheight}}{output_23_2.png}
            \end{center}
            { \hspace*{\fill} \\}
            
            \begin{tcolorbox}[breakable, size=fbox, boxrule=1pt, pad at break*=1mm,colback=cellbackground, colframe=cellborder]
        \prompt{In}{incolor}{157}{\boxspacing}
        \begin{Verbatim}[commandchars=\\\{\}]
        \PY{c+c1}{\PYZsh{} so that concludes our k means clustering analysis on features of an IOT dataset.}
        \end{Verbatim}
        \end{tcolorbox}
        
            \begin{tcolorbox}[breakable, size=fbox, boxrule=1pt, pad at break*=1mm,colback=cellbackground, colframe=cellborder]
        \prompt{In}{incolor}{159}{\boxspacing}
        \begin{Verbatim}[commandchars=\\\{\}]
        \PY{c+c1}{\PYZsh{} credits to https://www.unb.ca/cic/datasets/iotdataset\PYZhy{}2022.html for the dataset}
        \PY{c+c1}{\PYZsh{} Citation: Sajjad Dadkhah, Hassan Mahdikhani, Priscilla Kyei Danso, Alireza Zohourian, Kevin Anh Truong, Ali A. Ghorbani, “Towards the development of a realistic multidimensional IoT profiling dataset”, Submitted to: The 19th Annual International Conference on Privacy, Security \PYZam{} Trust (PST2022) August 22\PYZhy{}24, 2022, Fredericton, Canada.}
        
        \PY{c+c1}{\PYZsh{} analysis by: Krishnaraj T}
        \end{Verbatim}
        \end{tcolorbox}
        
            \begin{tcolorbox}[breakable, size=fbox, boxrule=1pt, pad at break*=1mm,colback=cellbackground, colframe=cellborder]
        \prompt{In}{incolor}{ }{\boxspacing}
        \begin{Verbatim}[commandchars=\\\{\}]
        
        \end{Verbatim}
        \end{tcolorbox}
        
        
            % Add a bibliography block to the postdoc
            
            
\clearpage
\section{FAQs}

\begin{enumerate}
    \item \textbf{What do you understand by K-Means method?}
          \begin{itemize}
              \item K-Means is a popular clustering algorithm used in machine learning and data mining.
              \item It aims to partition a dataset into K distinct, non-overlapping clusters, where each data point belongs to the cluster with the nearest mean.
              \item The algorithm iteratively assigns each data point to the nearest cluster centroid and recalculates the centroids until convergence.
              \item K-Means is commonly used for data exploration, pattern recognition, and image segmentation tasks.
          \end{itemize}

    \item \textbf{Discuss on how can we get data from IoT devices for Cyber Security?}
          \begin{itemize}
              \item IoT (Internet of Things) devices generate vast amounts of data that can be leveraged for cybersecurity purposes.
              \item Data from IoT devices can be obtained through various means, including direct data collection from sensors, network traffic monitoring, and device logs.
              \item Security protocols and standards such as MQTT (Message Queuing Telemetry Transport) and HTTPS (Hypertext Transfer Protocol Secure) can be used to securely transmit data from IoT devices to centralized servers or cloud platforms.
              \item Data preprocessing techniques, such as data cleaning, normalization, and feature extraction, may be applied to IoT data to prepare it for cybersecurity analysis.
          \end{itemize}

    \item \textbf{Can K-Means method be used for Anomaly Detection? Explain how?}
          \begin{itemize}
              \item While K-Means is primarily a clustering algorithm, it can be adapted for anomaly detection in certain scenarios.
              \item One approach is to use K-Means to cluster normal data points and identify clusters with fewer data points, which may indicate anomalies.
              \item Another approach is to calculate the distance of each data point to its nearest cluster centroid and flag data points with distances above a certain threshold as anomalies.
              \item However, K-Means may not be suitable for detecting complex or nonlinear anomalies, and other anomaly detection techniques such as Isolation Forest or One-Class SVM may be more appropriate in such cases.
          \end{itemize}
\end{enumerate}


\section{Conclusion}
In this Assignment, we implemented the K-Means Clustering algorithm in Python using the IOT Based Attacks dataset. We loaded the dataset, performed data analysis, split the data into dependent and independent variables, and trained the K-Means model. We visualized the clusters and analyzed the results. K-Means clustering is a powerful technique for partitioning data into distinct clusters based on similarity and can be applied to various domains, including cybersecurity and forensics.
\clearpage

\pagebreak
% \begin{thebibliography}{}

%     \bibitem{CloudConceptsOverview}
%     Cloud Computing Concepts Overview.
%     Accessed from: \url{https://www.ibm.com/cloud/learn/cloud-computing-concepts}

%     \bibitem{VirtualizationBenefits}
%     Benefits of Virtualization.
%     Accessed from: \url{https://www.vmware.com/topics/glossary/content/virtualization}

% \end{thebibliography}

\end{document}