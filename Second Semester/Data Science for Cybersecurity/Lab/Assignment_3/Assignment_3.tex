% This is a Basic Assignment Paper but with like Code and stuff allowed in it, there is also url, hyperlinks from contents included. 

\documentclass[11pt]{article}

% Preamble

\usepackage[margin=1in]{geometry}
\usepackage{amsfonts, amsmath, amssymb, amsthm}
\usepackage{fancyhdr, float, graphicx}
\usepackage[utf8]{inputenc} % Required for inputting international characters
\usepackage[T1]{fontenc} % Output font encoding for international characters
% \usepackage{fouriernc} % Use the New Century Schoolbook font
\usepackage[nottoc, notlot, notlof]{tocbibind}
\usepackage{listings}
\usepackage{xcolor}
\usepackage{blindtext}
\usepackage{hyperref}
\definecolor{codepurple}{rgb}{0.58,0,0.82}
\hypersetup{
    colorlinks=true,
    linkcolor=black,
    filecolor=black,      
    urlcolor=codepurple,
    pdfpagemode=FullScreen,
    }

\definecolor{codegreen}{rgb}{0,0.6,0}
\definecolor{codegray}{rgb}{0.5,0.5,0.5}
\definecolor{backcolour}{rgb}{0.95,0.95,0.92}

\lstdefinestyle{mystyle}{
    backgroundcolor=\color{backcolour},   
    commentstyle=\color{codegreen},
    keywordstyle=\color{magenta},
    numberstyle=\tiny\color{codegray},
    stringstyle=\color{codepurple},
    basicstyle=\ttfamily\footnotesize,
    breakatwhitespace=false,         
    breaklines=true,                 
    captionpos=b,                    
    keepspaces=true,                 
    numbers=left,                    
    numbersep=5pt,                  
    showspaces=false,                
    showstringspaces=false,
    showtabs=false,                  
    tabsize=2
}

\lstset{style=mystyle}

% Header and Footer
\pagestyle{fancy}
\fancyhead{}
\fancyfoot{}
\fancyhead[L]{\textit{\Large{Data Science for Cybersecurity and Forensics}}}
\fancyhead[R]{\textit{Krishnaraj T}}
\fancyfoot[C]{\thepage}
\renewcommand{\footrulewidth}{1pt}
\newtheorem{thm}{Theorem}
\newtheorem{dfn}[thm]{Definition}


\usepackage[breakable]{tcolorbox}
\usepackage{parskip} % Stop auto-indenting (to mimic markdown behaviour)


% Basic figure setup, for now with no caption control since it's done
% automatically by Pandoc (which extracts ![](path) syntax from Markdown).
\usepackage{graphicx}
% Keep aspect ratio if custom image width or height is specified
\setkeys{Gin}{keepaspectratio}
% Maintain compatibility with old templates. Remove in nbconvert 6.0
\let\Oldincludegraphics\includegraphics
% Ensure that by default, figures have no caption (until we provide a
% proper Figure object with a Caption API and a way to capture that
% in the conversion process - todo).
\usepackage{caption}
\DeclareCaptionFormat{nocaption}{}
\captionsetup{format=nocaption,aboveskip=0pt,belowskip=0pt}

\usepackage{float}
\floatplacement{figure}{H} % forces figures to be placed at the correct location
\usepackage{xcolor} % Allow colors to be defined
\usepackage{enumerate} % Needed for markdown enumerations to work
\usepackage{geometry} % Used to adjust the document margins
\usepackage{amsmath} % Equations
\usepackage{amssymb} % Equations
\usepackage{textcomp} % defines textquotesingle
% Hack from http://tex.stackexchange.com/a/47451/13684:
\AtBeginDocument{%
    \def\PYZsq{\textquotesingle}% Upright quotes in Pygmentized code
}
\usepackage{upquote} % Upright quotes for verbatim code
\usepackage{eurosym} % defines \euro

\usepackage{iftex}
\ifPDFTeX
    \usepackage[T1]{fontenc}
    \IfFileExists{alphabeta.sty}{
            \usepackage{alphabeta}
        }{
            \usepackage[mathletters]{ucs}
            \usepackage[utf8x]{inputenc}
        }
\else
    \usepackage{fontspec}
    \usepackage{unicode-math}
\fi

\usepackage{fancyvrb} % verbatim replacement that allows latex
\usepackage{grffile} % extends the file name processing of package graphics
                        % to support a larger range
\makeatletter % fix for old versions of grffile with XeLaTeX
\@ifpackagelater{grffile}{2019/11/01}
{
    % Do nothing on new versions
}
{
    \def\Gread@@xetex#1{%
    \IfFileExists{"\Gin@base".bb}%
    {\Gread@eps{\Gin@base.bb}}%
    {\Gread@@xetex@aux#1}%
    }
}
\makeatother
\usepackage[Export]{adjustbox} % Used to constrain images to a maximum size
\adjustboxset{max size={0.9\linewidth}{0.9\paperheight}}

% The hyperref package gives us a pdf with properly built
% internal navigation ('pdf bookmarks' for the table of contents,
% internal cross-reference links, web links for URLs, etc.)
\usepackage{hyperref}
% The default LaTeX title has an obnoxious amount of whitespace. By default,
% titling removes some of it. It also provides customization options.
\usepackage{titling}
\usepackage{longtable} % longtable support required by pandoc >1.10
\usepackage{booktabs}  % table support for pandoc > 1.12.2
\usepackage{array}     % table support for pandoc >= 2.11.3
\usepackage{calc}      % table minipage width calculation for pandoc >= 2.11.1
\usepackage[inline]{enumitem} % IRkernel/repr support (it uses the enumerate* environment)
\usepackage[normalem]{ulem} % ulem is needed to support strikethroughs (\sout)
                            % normalem makes italics be italics, not underlines
\usepackage{soul}      % strikethrough (\st) support for pandoc >= 3.0.0
\usepackage{mathrsfs}



% Colors for the hyperref package
\definecolor{urlcolor}{rgb}{0,.145,.698}
\definecolor{linkcolor}{rgb}{.71,0.21,0.01}
\definecolor{citecolor}{rgb}{.12,.54,.11}

% ANSI colors
\definecolor{ansi-black}{HTML}{3E424D}
\definecolor{ansi-black-intense}{HTML}{282C36}
\definecolor{ansi-red}{HTML}{E75C58}
\definecolor{ansi-red-intense}{HTML}{B22B31}
\definecolor{ansi-green}{HTML}{00A250}
\definecolor{ansi-green-intense}{HTML}{007427}
\definecolor{ansi-yellow}{HTML}{DDB62B}
\definecolor{ansi-yellow-intense}{HTML}{B27D12}
\definecolor{ansi-blue}{HTML}{208FFB}
\definecolor{ansi-blue-intense}{HTML}{0065CA}
\definecolor{ansi-magenta}{HTML}{D160C4}
\definecolor{ansi-magenta-intense}{HTML}{A03196}
\definecolor{ansi-cyan}{HTML}{60C6C8}
\definecolor{ansi-cyan-intense}{HTML}{258F8F}
\definecolor{ansi-white}{HTML}{C5C1B4}
\definecolor{ansi-white-intense}{HTML}{A1A6B2}
\definecolor{ansi-default-inverse-fg}{HTML}{FFFFFF}
\definecolor{ansi-default-inverse-bg}{HTML}{000000}

% common color for the border for error outputs.
\definecolor{outerrorbackground}{HTML}{FFDFDF}

% commands and environments needed by pandoc snippets
% extracted from the output of `pandoc -s`
\providecommand{\tightlist}{%
    \setlength{\itemsep}{0pt}\setlength{\parskip}{0pt}}
\DefineVerbatimEnvironment{Highlighting}{Verbatim}{commandchars=\\\{\}}
% Add ',fontsize=\small' for more characters per line
\newenvironment{Shaded}{}{}
\newcommand{\KeywordTok}[1]{\textcolor[rgb]{0.00,0.44,0.13}{\textbf{{#1}}}}
\newcommand{\DataTypeTok}[1]{\textcolor[rgb]{0.56,0.13,0.00}{{#1}}}
\newcommand{\DecValTok}[1]{\textcolor[rgb]{0.25,0.63,0.44}{{#1}}}
\newcommand{\BaseNTok}[1]{\textcolor[rgb]{0.25,0.63,0.44}{{#1}}}
\newcommand{\FloatTok}[1]{\textcolor[rgb]{0.25,0.63,0.44}{{#1}}}
\newcommand{\CharTok}[1]{\textcolor[rgb]{0.25,0.44,0.63}{{#1}}}
\newcommand{\StringTok}[1]{\textcolor[rgb]{0.25,0.44,0.63}{{#1}}}
\newcommand{\CommentTok}[1]{\textcolor[rgb]{0.38,0.63,0.69}{\textit{{#1}}}}
\newcommand{\OtherTok}[1]{\textcolor[rgb]{0.00,0.44,0.13}{{#1}}}
\newcommand{\AlertTok}[1]{\textcolor[rgb]{1.00,0.00,0.00}{\textbf{{#1}}}}
\newcommand{\FunctionTok}[1]{\textcolor[rgb]{0.02,0.16,0.49}{{#1}}}
\newcommand{\RegionMarkerTok}[1]{{#1}}
\newcommand{\ErrorTok}[1]{\textcolor[rgb]{1.00,0.00,0.00}{\textbf{{#1}}}}
\newcommand{\NormalTok}[1]{{#1}}

% Additional commands for more recent versions of Pandoc
\newcommand{\ConstantTok}[1]{\textcolor[rgb]{0.53,0.00,0.00}{{#1}}}
\newcommand{\SpecialCharTok}[1]{\textcolor[rgb]{0.25,0.44,0.63}{{#1}}}
\newcommand{\VerbatimStringTok}[1]{\textcolor[rgb]{0.25,0.44,0.63}{{#1}}}
\newcommand{\SpecialStringTok}[1]{\textcolor[rgb]{0.73,0.40,0.53}{{#1}}}
\newcommand{\ImportTok}[1]{{#1}}
\newcommand{\DocumentationTok}[1]{\textcolor[rgb]{0.73,0.13,0.13}{\textit{{#1}}}}
\newcommand{\AnnotationTok}[1]{\textcolor[rgb]{0.38,0.63,0.69}{\textbf{\textit{{#1}}}}}
\newcommand{\CommentVarTok}[1]{\textcolor[rgb]{0.38,0.63,0.69}{\textbf{\textit{{#1}}}}}
\newcommand{\VariableTok}[1]{\textcolor[rgb]{0.10,0.09,0.49}{{#1}}}
\newcommand{\ControlFlowTok}[1]{\textcolor[rgb]{0.00,0.44,0.13}{\textbf{{#1}}}}
\newcommand{\OperatorTok}[1]{\textcolor[rgb]{0.40,0.40,0.40}{{#1}}}
\newcommand{\BuiltInTok}[1]{{#1}}
\newcommand{\ExtensionTok}[1]{{#1}}
\newcommand{\PreprocessorTok}[1]{\textcolor[rgb]{0.74,0.48,0.00}{{#1}}}
\newcommand{\AttributeTok}[1]{\textcolor[rgb]{0.49,0.56,0.16}{{#1}}}
\newcommand{\InformationTok}[1]{\textcolor[rgb]{0.38,0.63,0.69}{\textbf{\textit{{#1}}}}}
\newcommand{\WarningTok}[1]{\textcolor[rgb]{0.38,0.63,0.69}{\textbf{\textit{{#1}}}}}


% Define a nice break command that doesn't care if a line doesn't already
% exist.
\def\br{\hspace*{\fill} \\* }
% Math Jax compatibility definitions
\def\gt{>}
\def\lt{<}
\let\Oldtex\TeX
\let\Oldlatex\LaTeX
\renewcommand{\TeX}{\textrm{\Oldtex}}
\renewcommand{\LaTeX}{\textrm{\Oldlatex}}  
    
    
    
    
    
    
% Pygments definitions
\makeatletter
\def\PY@reset{\let\PY@it=\relax \let\PY@bf=\relax%
    \let\PY@ul=\relax \let\PY@tc=\relax%
    \let\PY@bc=\relax \let\PY@ff=\relax}
\def\PY@tok#1{\csname PY@tok@#1\endcsname}
\def\PY@toks#1+{\ifx\relax#1\empty\else%
    \PY@tok{#1}\expandafter\PY@toks\fi}
\def\PY@do#1{\PY@bc{\PY@tc{\PY@ul{%
    \PY@it{\PY@bf{\PY@ff{#1}}}}}}}
\def\PY#1#2{\PY@reset\PY@toks#1+\relax+\PY@do{#2}}

\@namedef{PY@tok@w}{\def\PY@tc##1{\textcolor[rgb]{0.73,0.73,0.73}{##1}}}
\@namedef{PY@tok@c}{\let\PY@it=\textit\def\PY@tc##1{\textcolor[rgb]{0.24,0.48,0.48}{##1}}}
\@namedef{PY@tok@cp}{\def\PY@tc##1{\textcolor[rgb]{0.61,0.40,0.00}{##1}}}
\@namedef{PY@tok@k}{\let\PY@bf=\textbf\def\PY@tc##1{\textcolor[rgb]{0.00,0.50,0.00}{##1}}}
\@namedef{PY@tok@kp}{\def\PY@tc##1{\textcolor[rgb]{0.00,0.50,0.00}{##1}}}
\@namedef{PY@tok@kt}{\def\PY@tc##1{\textcolor[rgb]{0.69,0.00,0.25}{##1}}}
\@namedef{PY@tok@o}{\def\PY@tc##1{\textcolor[rgb]{0.40,0.40,0.40}{##1}}}
\@namedef{PY@tok@ow}{\let\PY@bf=\textbf\def\PY@tc##1{\textcolor[rgb]{0.67,0.13,1.00}{##1}}}
\@namedef{PY@tok@nb}{\def\PY@tc##1{\textcolor[rgb]{0.00,0.50,0.00}{##1}}}
\@namedef{PY@tok@nf}{\def\PY@tc##1{\textcolor[rgb]{0.00,0.00,1.00}{##1}}}
\@namedef{PY@tok@nc}{\let\PY@bf=\textbf\def\PY@tc##1{\textcolor[rgb]{0.00,0.00,1.00}{##1}}}
\@namedef{PY@tok@nn}{\let\PY@bf=\textbf\def\PY@tc##1{\textcolor[rgb]{0.00,0.00,1.00}{##1}}}
\@namedef{PY@tok@ne}{\let\PY@bf=\textbf\def\PY@tc##1{\textcolor[rgb]{0.80,0.25,0.22}{##1}}}
\@namedef{PY@tok@nv}{\def\PY@tc##1{\textcolor[rgb]{0.10,0.09,0.49}{##1}}}
\@namedef{PY@tok@no}{\def\PY@tc##1{\textcolor[rgb]{0.53,0.00,0.00}{##1}}}
\@namedef{PY@tok@nl}{\def\PY@tc##1{\textcolor[rgb]{0.46,0.46,0.00}{##1}}}
\@namedef{PY@tok@ni}{\let\PY@bf=\textbf\def\PY@tc##1{\textcolor[rgb]{0.44,0.44,0.44}{##1}}}
\@namedef{PY@tok@na}{\def\PY@tc##1{\textcolor[rgb]{0.41,0.47,0.13}{##1}}}
\@namedef{PY@tok@nt}{\let\PY@bf=\textbf\def\PY@tc##1{\textcolor[rgb]{0.00,0.50,0.00}{##1}}}
\@namedef{PY@tok@nd}{\def\PY@tc##1{\textcolor[rgb]{0.67,0.13,1.00}{##1}}}
\@namedef{PY@tok@s}{\def\PY@tc##1{\textcolor[rgb]{0.73,0.13,0.13}{##1}}}
\@namedef{PY@tok@sd}{\let\PY@it=\textit\def\PY@tc##1{\textcolor[rgb]{0.73,0.13,0.13}{##1}}}
\@namedef{PY@tok@si}{\let\PY@bf=\textbf\def\PY@tc##1{\textcolor[rgb]{0.64,0.35,0.47}{##1}}}
\@namedef{PY@tok@se}{\let\PY@bf=\textbf\def\PY@tc##1{\textcolor[rgb]{0.67,0.36,0.12}{##1}}}
\@namedef{PY@tok@sr}{\def\PY@tc##1{\textcolor[rgb]{0.64,0.35,0.47}{##1}}}
\@namedef{PY@tok@ss}{\def\PY@tc##1{\textcolor[rgb]{0.10,0.09,0.49}{##1}}}
\@namedef{PY@tok@sx}{\def\PY@tc##1{\textcolor[rgb]{0.00,0.50,0.00}{##1}}}
\@namedef{PY@tok@m}{\def\PY@tc##1{\textcolor[rgb]{0.40,0.40,0.40}{##1}}}
\@namedef{PY@tok@gh}{\let\PY@bf=\textbf\def\PY@tc##1{\textcolor[rgb]{0.00,0.00,0.50}{##1}}}
\@namedef{PY@tok@gu}{\let\PY@bf=\textbf\def\PY@tc##1{\textcolor[rgb]{0.50,0.00,0.50}{##1}}}
\@namedef{PY@tok@gd}{\def\PY@tc##1{\textcolor[rgb]{0.63,0.00,0.00}{##1}}}
\@namedef{PY@tok@gi}{\def\PY@tc##1{\textcolor[rgb]{0.00,0.52,0.00}{##1}}}
\@namedef{PY@tok@gr}{\def\PY@tc##1{\textcolor[rgb]{0.89,0.00,0.00}{##1}}}
\@namedef{PY@tok@ge}{\let\PY@it=\textit}
\@namedef{PY@tok@gs}{\let\PY@bf=\textbf}
\@namedef{PY@tok@ges}{\let\PY@bf=\textbf\let\PY@it=\textit}
\@namedef{PY@tok@gp}{\let\PY@bf=\textbf\def\PY@tc##1{\textcolor[rgb]{0.00,0.00,0.50}{##1}}}
\@namedef{PY@tok@go}{\def\PY@tc##1{\textcolor[rgb]{0.44,0.44,0.44}{##1}}}
\@namedef{PY@tok@gt}{\def\PY@tc##1{\textcolor[rgb]{0.00,0.27,0.87}{##1}}}
\@namedef{PY@tok@err}{\def\PY@bc##1{{\setlength{\fboxsep}{\string -\fboxrule}\fcolorbox[rgb]{1.00,0.00,0.00}{1,1,1}{\strut ##1}}}}
\@namedef{PY@tok@kc}{\let\PY@bf=\textbf\def\PY@tc##1{\textcolor[rgb]{0.00,0.50,0.00}{##1}}}
\@namedef{PY@tok@kd}{\let\PY@bf=\textbf\def\PY@tc##1{\textcolor[rgb]{0.00,0.50,0.00}{##1}}}
\@namedef{PY@tok@kn}{\let\PY@bf=\textbf\def\PY@tc##1{\textcolor[rgb]{0.00,0.50,0.00}{##1}}}
\@namedef{PY@tok@kr}{\let\PY@bf=\textbf\def\PY@tc##1{\textcolor[rgb]{0.00,0.50,0.00}{##1}}}
\@namedef{PY@tok@bp}{\def\PY@tc##1{\textcolor[rgb]{0.00,0.50,0.00}{##1}}}
\@namedef{PY@tok@fm}{\def\PY@tc##1{\textcolor[rgb]{0.00,0.00,1.00}{##1}}}
\@namedef{PY@tok@vc}{\def\PY@tc##1{\textcolor[rgb]{0.10,0.09,0.49}{##1}}}
\@namedef{PY@tok@vg}{\def\PY@tc##1{\textcolor[rgb]{0.10,0.09,0.49}{##1}}}
\@namedef{PY@tok@vi}{\def\PY@tc##1{\textcolor[rgb]{0.10,0.09,0.49}{##1}}}
\@namedef{PY@tok@vm}{\def\PY@tc##1{\textcolor[rgb]{0.10,0.09,0.49}{##1}}}
\@namedef{PY@tok@sa}{\def\PY@tc##1{\textcolor[rgb]{0.73,0.13,0.13}{##1}}}
\@namedef{PY@tok@sb}{\def\PY@tc##1{\textcolor[rgb]{0.73,0.13,0.13}{##1}}}
\@namedef{PY@tok@sc}{\def\PY@tc##1{\textcolor[rgb]{0.73,0.13,0.13}{##1}}}
\@namedef{PY@tok@dl}{\def\PY@tc##1{\textcolor[rgb]{0.73,0.13,0.13}{##1}}}
\@namedef{PY@tok@s2}{\def\PY@tc##1{\textcolor[rgb]{0.73,0.13,0.13}{##1}}}
\@namedef{PY@tok@sh}{\def\PY@tc##1{\textcolor[rgb]{0.73,0.13,0.13}{##1}}}
\@namedef{PY@tok@s1}{\def\PY@tc##1{\textcolor[rgb]{0.73,0.13,0.13}{##1}}}
\@namedef{PY@tok@mb}{\def\PY@tc##1{\textcolor[rgb]{0.40,0.40,0.40}{##1}}}
\@namedef{PY@tok@mf}{\def\PY@tc##1{\textcolor[rgb]{0.40,0.40,0.40}{##1}}}
\@namedef{PY@tok@mh}{\def\PY@tc##1{\textcolor[rgb]{0.40,0.40,0.40}{##1}}}
\@namedef{PY@tok@mi}{\def\PY@tc##1{\textcolor[rgb]{0.40,0.40,0.40}{##1}}}
\@namedef{PY@tok@il}{\def\PY@tc##1{\textcolor[rgb]{0.40,0.40,0.40}{##1}}}
\@namedef{PY@tok@mo}{\def\PY@tc##1{\textcolor[rgb]{0.40,0.40,0.40}{##1}}}
\@namedef{PY@tok@ch}{\let\PY@it=\textit\def\PY@tc##1{\textcolor[rgb]{0.24,0.48,0.48}{##1}}}
\@namedef{PY@tok@cm}{\let\PY@it=\textit\def\PY@tc##1{\textcolor[rgb]{0.24,0.48,0.48}{##1}}}
\@namedef{PY@tok@cpf}{\let\PY@it=\textit\def\PY@tc##1{\textcolor[rgb]{0.24,0.48,0.48}{##1}}}
\@namedef{PY@tok@c1}{\let\PY@it=\textit\def\PY@tc##1{\textcolor[rgb]{0.24,0.48,0.48}{##1}}}
\@namedef{PY@tok@cs}{\let\PY@it=\textit\def\PY@tc##1{\textcolor[rgb]{0.24,0.48,0.48}{##1}}}

\def\PYZbs{\char`\\}
\def\PYZus{\char`\_}
\def\PYZob{\char`\{}
\def\PYZcb{\char`\}}
\def\PYZca{\char`\^}
\def\PYZam{\char`\&}
\def\PYZlt{\char`\<}
\def\PYZgt{\char`\>}
\def\PYZsh{\char`\#}
\def\PYZpc{\char`\%}
\def\PYZdl{\char`\$}
\def\PYZhy{\char`\-}
\def\PYZsq{\char`\'}
\def\PYZdq{\char`\"}
\def\PYZti{\char`\~}
% for compatibility with earlier versions
\def\PYZat{@}
\def\PYZlb{[}
\def\PYZrb{]}
\makeatother


    % For linebreaks inside Verbatim environment from package fancyvrb.
    \makeatletter
        \newbox\Wrappedcontinuationbox
        \newbox\Wrappedvisiblespacebox
        \newcommand*\Wrappedvisiblespace {\textcolor{red}{\textvisiblespace}}
        \newcommand*\Wrappedcontinuationsymbol {\textcolor{red}{\llap{\tiny$\m@th\hookrightarrow$}}}
        \newcommand*\Wrappedcontinuationindent {3ex }
        \newcommand*\Wrappedafterbreak {\kern\Wrappedcontinuationindent\copy\Wrappedcontinuationbox}
        % Take advantage of the already applied Pygments mark-up to insert
        % potential linebreaks for TeX processing.
        %        {, <, #, %, $, ' and ": go to next line.
        %        _, }, ^, &, >, - and ~: stay at end of broken line.
        % Use of \textquotesingle for straight quote.
        \newcommand*\Wrappedbreaksatspecials {%
            \def\PYGZus{\discretionary{\char`\_}{\Wrappedafterbreak}{\char`\_}}%
            \def\PYGZob{\discretionary{}{\Wrappedafterbreak\char`\{}{\char`\{}}%
            \def\PYGZcb{\discretionary{\char`\}}{\Wrappedafterbreak}{\char`\}}}%
            \def\PYGZca{\discretionary{\char`\^}{\Wrappedafterbreak}{\char`\^}}%
            \def\PYGZam{\discretionary{\char`\&}{\Wrappedafterbreak}{\char`\&}}%
            \def\PYGZlt{\discretionary{}{\Wrappedafterbreak\char`\<}{\char`\<}}%
            \def\PYGZgt{\discretionary{\char`\>}{\Wrappedafterbreak}{\char`\>}}%
            \def\PYGZsh{\discretionary{}{\Wrappedafterbreak\char`\#}{\char`\#}}%
            \def\PYGZpc{\discretionary{}{\Wrappedafterbreak\char`\%}{\char`\%}}%
            \def\PYGZdl{\discretionary{}{\Wrappedafterbreak\char`\$}{\char`\$}}%
            \def\PYGZhy{\discretionary{\char`\-}{\Wrappedafterbreak}{\char`\-}}%
            \def\PYGZsq{\discretionary{}{\Wrappedafterbreak\textquotesingle}{\textquotesingle}}%
            \def\PYGZdq{\discretionary{}{\Wrappedafterbreak\char`\"}{\char`\"}}%
            \def\PYGZti{\discretionary{\char`\~}{\Wrappedafterbreak}{\char`\~}}%
        }
        % Some characters . , ; ? ! / are not pygmentized.
        % This macro makes them "active" and they will insert potential linebreaks
        \newcommand*\Wrappedbreaksatpunct {%
            \lccode`\~`\.\lowercase{\def~}{\discretionary{\hbox{\char`\.}}{\Wrappedafterbreak}{\hbox{\char`\.}}}%
            \lccode`\~`\,\lowercase{\def~}{\discretionary{\hbox{\char`\,}}{\Wrappedafterbreak}{\hbox{\char`\,}}}%
            \lccode`\~`\;\lowercase{\def~}{\discretionary{\hbox{\char`\;}}{\Wrappedafterbreak}{\hbox{\char`\;}}}%
            \lccode`\~`\:\lowercase{\def~}{\discretionary{\hbox{\char`\:}}{\Wrappedafterbreak}{\hbox{\char`\:}}}%
            \lccode`\~`\?\lowercase{\def~}{\discretionary{\hbox{\char`\?}}{\Wrappedafterbreak}{\hbox{\char`\?}}}%
            \lccode`\~`\!\lowercase{\def~}{\discretionary{\hbox{\char`\!}}{\Wrappedafterbreak}{\hbox{\char`\!}}}%
            \lccode`\~`\/\lowercase{\def~}{\discretionary{\hbox{\char`\/}}{\Wrappedafterbreak}{\hbox{\char`\/}}}%
            \catcode`\.\active
            \catcode`\,\active
            \catcode`\;\active
            \catcode`\:\active
            \catcode`\?\active
            \catcode`\!\active
            \catcode`\/\active
            \lccode`\~`\~
        }
    \makeatother

    \let\OriginalVerbatim=\Verbatim
    \makeatletter
    \renewcommand{\Verbatim}[1][1]{%
        %\parskip\z@skip
        \sbox\Wrappedcontinuationbox {\Wrappedcontinuationsymbol}%
        \sbox\Wrappedvisiblespacebox {\FV@SetupFont\Wrappedvisiblespace}%
        \def\FancyVerbFormatLine ##1{\hsize\linewidth
            \vtop{\raggedright\hyphenpenalty\z@\exhyphenpenalty\z@
                \doublehyphendemerits\z@\finalhyphendemerits\z@
                \strut ##1\strut}%
        }%
        % If the linebreak is at a space, the latter will be displayed as visible
        % space at end of first line, and a continuation symbol starts next line.
        % Stretch/shrink are however usually zero for typewriter font.
        \def\FV@Space {%
            \nobreak\hskip\z@ plus\fontdimen3\font minus\fontdimen4\font
            \discretionary{\copy\Wrappedvisiblespacebox}{\Wrappedafterbreak}
            {\kern\fontdimen2\font}%
        }%

        % Allow breaks at special characters using \PYG... macros.
        \Wrappedbreaksatspecials
        % Breaks at punctuation characters . , ; ? ! and / need catcode=\active
        \OriginalVerbatim[#1,codes*=\Wrappedbreaksatpunct]%
    }
    \makeatother

    % Exact colors from NB
    \definecolor{incolor}{HTML}{303F9F}
    \definecolor{outcolor}{HTML}{D84315}
    \definecolor{cellborder}{HTML}{CFCFCF}
    \definecolor{cellbackground}{HTML}{F7F7F7}

    % prompt
    \makeatletter
    \newcommand{\boxspacing}{\kern\kvtcb@left@rule\kern\kvtcb@boxsep}
    \makeatother
    \newcommand{\prompt}[4]{
        {\ttfamily\llap{{\color{#2}[#3]:\hspace{3pt}#4}}\vspace{-\baselineskip}}
    }
    

    
    % Prevent overflowing lines due to hard-to-break entities
    \sloppy
    % Setup hyperref package
    % \hypersetup{
    %   breaklinks=true,  % so long urls are correctly broken across lines
    %   colorlinks=true,
    %   urlcolor=urlcolor,
    %   linkcolor=linkcolor,
    %   citecolor=citecolor,
    %   }
    % Slightly bigger margins than the latex defaults
    
    \geometry{verbose,tmargin=1in,bmargin=1in,lmargin=1in,rmargin=1in}
    
  


% Other Doc Editing
% \parindent 0ex
%\renewcommand{\baselinestretch}{1.5}

\begin{document}

\begin{titlepage}
    \centering

    %---------------------------NAMES-------------------------------

    \huge\textsc{
        MIT World Peace University
    }\\

    \vspace{0.75\baselineskip} % space after Uni Name

    \LARGE{
        Data Science for Cybersecurity and Forensics\\
        Third Year B. Tech, Semester 6
    }

    \vfill % space after Sub Name

    %--------------------------TITLE-------------------------------

    \rule{\textwidth}{1.6pt}\vspace*{-\baselineskip}\vspace*{2pt}
    \rule{\textwidth}{0.6pt}
    \vspace{0.75\baselineskip} % Whitespace above the title



    \huge{\textsc{
            Statistical Approaches in Data Science
        }} \\



    \vspace{0.5\baselineskip} % Whitespace below the title
    \rule{\textwidth}{0.6pt}\vspace*{-\baselineskip}\vspace*{2.8pt}
    \rule{\textwidth}{1.6pt}

    \vspace{1\baselineskip} % Whitespace after the title block

    %--------------------------SUBTITLE --------------------------	

    \LARGE\textsc{
        Assignment 3
    } % Subtitle or further description
    \vfill

    %--------------------------AUTHOR-------------------------------

    Prepared By
    \vspace{0.5\baselineskip} % Whitespace before the editors

    \Large{
        Krishnaraj Thadesar \\
        Cyber Security and Forensics\\
        Batch A1, PA 10
    }


    \vspace{0.5\baselineskip} % Whitespace below the editor list
    \today

\end{titlepage}


\tableofcontents
\thispagestyle{empty}
\clearpage

\setcounter{page}{1}

\section{Aim}
Learning some statistical approaches that are used in data science.

\section{Objectives}
\begin{enumerate}
    \item Write a Python program to implement central tendency for housing data.
    \item Using python compute variance in the weather.
    \item Compute variance in the weather to find best time to visit New Delhi(or any city).
    \item Using histogram find the best time to visit Delhi (or any)s on any dataset.
\end{enumerate}

\section{Theory}

\subsection{Types of Statistics}
Statistics can be broadly categorized into two main types: descriptive statistics and inferential statistics.

\subsection{Descriptive Statistics}
Descriptive statistics involve methods for summarizing and describing the features of a dataset. It provides insights into the central tendency, variability, and distribution of the data.

\subsubsection{Measures of Central Tendency}
Measures of central tendency are statistics that describe the center or average of a dataset. Common measures of central tendency include the mean, median, and mode.
\begin{itemize}
    \item \textbf{Mean}: The arithmetic average of a set of values, calculated by summing all the values and dividing by the number of observations.

          Formula: \[ \bar{x} = \frac{\sum_{i=1}^{n} x_i}{n} \]
    \item \textbf{Median}: The middle value in a dataset when the values are arranged in ascending order. It divides the dataset into two equal halves.

          Formula: \[ \text{Median} = \begin{cases} x_{(n+1)/2} & \text{if } n \text{ is odd} \\ \frac{x_{n/2} + x_{n/2 + 1}}{2} & \text{if } n \text{ is even} \end{cases} \]
    \item \textbf{Mode}: The value that appears most frequently in a dataset.

          Formula: \[ \text{Mode} = \text{value with highest frequency} \]
\end{itemize}

\subsubsection{Measures of Dispersion}
Measures of dispersion quantify the spread or variability of the data points in a dataset. They provide information about how the data is distributed around the central tendency.
\begin{itemize}
    \item \textbf{Range}: The difference between the maximum and minimum values in a dataset.

          Formula:
          \[ \text{Range} = \text{Maximum value} - \text{Minimum value} \]
    \item \textbf{Variance}: The average of the squared differences from the mean. It measures the average distance of each data point from the mean.

          Formula:
          \[ \text{Variance} = \frac{\sum_{i=1}^{n} (x_i - \bar{x})^2}{n} \]
    \item \textbf{Standard Deviation}: The square root of the variance. It provides a measure of the dispersion of data points around the mean.

          Formula:
          \[ \text{Standard Deviation} = \sqrt{\text{Variance}} \]
\end{itemize}

\subsection{Inferential Statistics}

Inferential statistics involve methods for making predictions or inferences about a population based on a sample of data. It uses probability theory to draw conclusions about the population parameters.

\subsubsection{Hypothesis Testing}
Hypothesis testing is a statistical method used to determine whether there is enough evidence to reject a null hypothesis in favor of an alternative hypothesis. It involves setting up a null hypothesis and an alternative hypothesis, collecting data, and using statistical tests to make a decision.

\subsubsection{Regression Analysis}

Regression analysis is a statistical technique used to model the relationship between a dependent variable and one or more independent variables. It helps in understanding how the value of the dependent variable changes when one or more independent variables are varied.

\subsubsection{Correlation Analysis}

Correlation analysis is a statistical method used to measure the strength and direction of the relationship between two variables. It helps in understanding how changes in one variable are associated with changes in another variable.

\section{Platform}
\textbf{Operating System}: Windows 11 \\
\textbf{IDEs or Text Editors Used}: Visual Studio Code\\
\textbf{Compilers or Interpreters}: Python 3.10.1\\

\section{Requirements}
\begin{lstlisting}
python==3.10.1
matplotlib==3.8.3
numpy==1.26.4
pandas==2.2.2
seaborn==0.13.2
\end{lstlisting}
\section{Code}

    

    
\begin{tcolorbox}[breakable, size=fbox, boxrule=1pt, pad at break*=1mm,colback=cellbackground, colframe=cellborder]
    \prompt{In}{incolor}{26}{\boxspacing}
    \begin{Verbatim}[commandchars=\\\{\}]
    \PY{c+c1}{\PYZsh{} import libraries}
    \PY{k+kn}{import} \PY{n+nn}{numpy} \PY{k}{as} \PY{n+nn}{np}
    \PY{k+kn}{import} \PY{n+nn}{pandas} \PY{k}{as} \PY{n+nn}{pd}
    \PY{k+kn}{import} \PY{n+nn}{matplotlib}\PY{n+nn}{.}\PY{n+nn}{pyplot} \PY{k}{as} \PY{n+nn}{plt}
    \PY{k+kn}{import} \PY{n+nn}{seaborn} \PY{k}{as} \PY{n+nn}{sns}
    
    \PY{n}{sns}\PY{o}{.}\PY{n}{set\PYZus{}theme}\PY{p}{(}\PY{n}{style}\PY{o}{=}\PY{l+s+s2}{\PYZdq{}}\PY{l+s+s2}{darkgrid}\PY{l+s+s2}{\PYZdq{}}\PY{p}{)}
    \end{Verbatim}
    \end{tcolorbox}
    
        \begin{tcolorbox}[breakable, size=fbox, boxrule=1pt, pad at break*=1mm,colback=cellbackground, colframe=cellborder]
    \prompt{In}{incolor}{7}{\boxspacing}
    \begin{Verbatim}[commandchars=\\\{\}]
    \PY{c+c1}{\PYZsh{} load dataset}
    \PY{n}{df} \PY{o}{=} \PY{n}{pd}\PY{o}{.}\PY{n}{read\PYZus{}csv}\PY{p}{(}\PY{l+s+s2}{\PYZdq{}}\PY{l+s+s2}{data.csv}\PY{l+s+s2}{\PYZdq{}}\PY{p}{)}
    \PY{n}{df}\PY{o}{.}\PY{n}{head}\PY{p}{(}\PY{p}{)}
    \end{Verbatim}
    \end{tcolorbox}
    
                \begin{tcolorbox}[breakable, size=fbox, boxrule=.5pt, pad at break*=1mm, opacityfill=0]
    \prompt{Out}{outcolor}{7}{\boxspacing}
    \begin{Verbatim}[commandchars=\\\{\}]
       Unnamed: 0      Zone  State   City                  Name          Type  \textbackslash{}
    0           0  Northern  Delhi  Delhi            India Gate  War Memorial
    1           1  Northern  Delhi  Delhi        Humayun's Tomb          Tomb
    2           2  Northern  Delhi  Delhi     Akshardham Temple        Temple
    3           3  Northern  Delhi  Delhi  Waste to Wonder Park    Theme Park
    4           4  Northern  Delhi  Delhi         Jantar Mantar   Observatory
    
      Establishment Year  time needed to visit in hrs  Google review rating  \textbackslash{}
    0               1921                          0.5                   4.6
    1               1572                          2.0                   4.5
    2               2005                          5.0                   4.6
    3               2019                          2.0                   4.1
    4               1724                          2.0                   4.2
    
       Entrance Fee in INR Airport with 50km Radius Weekly Off   Significance  \textbackslash{}
    0                    0                      Yes        NaN     Historical
    1                   30                      Yes        NaN     Historical
    2                   60                      Yes        NaN      Religious
    3                   50                      Yes     Monday  Environmental
    4                   15                      Yes        NaN     Scientific
    
      DSLR Allowed  Number of google review in lakhs Best Time to visit
    0          Yes                              2.60            Evening
    1          Yes                              0.40          Afternoon
    2           No                              0.40          Afternoon
    3          Yes                              0.27            Evening
    4          Yes                              0.31            Morning
    \end{Verbatim}
    \end{tcolorbox}
            
        \subsection{Pre Processing}\label{pre-processing}
    
        \begin{tcolorbox}[breakable, size=fbox, boxrule=1pt, pad at break*=1mm,colback=cellbackground, colframe=cellborder]
    \prompt{In}{incolor}{8}{\boxspacing}
    \begin{Verbatim}[commandchars=\\\{\}]
    \PY{c+c1}{\PYZsh{} lets remove the unnamed column}
    \PY{n}{df} \PY{o}{=} \PY{n}{df}\PY{o}{.}\PY{n}{drop}\PY{p}{(}\PY{l+s+s2}{\PYZdq{}}\PY{l+s+s2}{Unnamed: 0}\PY{l+s+s2}{\PYZdq{}}\PY{p}{,} \PY{n}{axis}\PY{o}{=}\PY{l+m+mi}{1}\PY{p}{)}
    \PY{n}{df}\PY{o}{.}\PY{n}{columns}
    \end{Verbatim}
    \end{tcolorbox}
    
                \begin{tcolorbox}[breakable, size=fbox, boxrule=.5pt, pad at break*=1mm, opacityfill=0]
    \prompt{Out}{outcolor}{8}{\boxspacing}
    \begin{Verbatim}[commandchars=\\\{\}]
    Index(['Zone', 'State', 'City', 'Name', 'Type', 'Establishment Year',
           'time needed to visit in hrs', 'Google review rating',
           'Entrance Fee in INR', 'Airport with 50km Radius', 'Weekly Off',
           'Significance', 'DSLR Allowed', 'Number of google review in lakhs',
           'Best Time to visit'],
          dtype='object')
    \end{Verbatim}
    \end{tcolorbox}
            
        \begin{tcolorbox}[breakable, size=fbox, boxrule=1pt, pad at break*=1mm,colback=cellbackground, colframe=cellborder]
    \prompt{In}{incolor}{19}{\boxspacing}
    \begin{Verbatim}[commandchars=\\\{\}]
    \PY{c+c1}{\PYZsh{} lets remove null values}
    \PY{n+nb}{print}\PY{p}{(}\PY{n}{df}\PY{o}{.}\PY{n}{isnull}\PY{p}{(}\PY{p}{)}\PY{o}{.}\PY{n}{sum}\PY{p}{(}\PY{p}{)}\PY{p}{)}
    \end{Verbatim}
    \end{tcolorbox}
    
        \begin{Verbatim}[commandchars=\\\{\}]
    Zone                                  0
    State                                 0
    City                                  0
    Name                                  0
    Type                                  0
    Establishment Year                    0
    time needed to visit in hrs           0
    Google review rating                  0
    Entrance Fee in INR                   0
    Airport with 50km Radius              0
    Weekly Off                          293
    Significance                          0
    DSLR Allowed                          0
    Number of google review in lakhs      0
    Best Time to visit                    0
    dtype: int64
        \end{Verbatim}
    
        \begin{tcolorbox}[breakable, size=fbox, boxrule=1pt, pad at break*=1mm,colback=cellbackground, colframe=cellborder]
    \prompt{In}{incolor}{30}{\boxspacing}
    \begin{Verbatim}[commandchars=\\\{\}]
    \PY{c+c1}{\PYZsh{} lets find outliers for ratings column, using z score}
    \PY{k+kn}{from} \PY{n+nn}{scipy} \PY{k+kn}{import} \PY{n}{stats}
    
    \PY{n}{z} \PY{o}{=} \PY{n}{np}\PY{o}{.}\PY{n}{abs}\PY{p}{(}\PY{n}{stats}\PY{o}{.}\PY{n}{zscore}\PY{p}{(}\PY{n}{df}\PY{p}{[}\PY{l+s+s2}{\PYZdq{}}\PY{l+s+s2}{Google review rating}\PY{l+s+s2}{\PYZdq{}}\PY{p}{]}\PY{p}{)}\PY{p}{)}
    
    \PY{c+c1}{\PYZsh{} lets remove outliers}
    \PY{n}{df} \PY{o}{=} \PY{n}{df}\PY{p}{[}\PY{p}{(}\PY{n}{z} \PY{o}{\PYZlt{}} \PY{l+m+mi}{3}\PY{p}{)}\PY{p}{]}
    \PY{n}{df}\PY{o}{.}\PY{n}{shape}
    \end{Verbatim}
    \end{tcolorbox}
    
                \begin{tcolorbox}[breakable, size=fbox, boxrule=.5pt, pad at break*=1mm, opacityfill=0]
    \prompt{Out}{outcolor}{30}{\boxspacing}
    \begin{Verbatim}[commandchars=\\\{\}]
    (324, 14)
    \end{Verbatim}
    \end{tcolorbox}
            
        \begin{tcolorbox}[breakable, size=fbox, boxrule=1pt, pad at break*=1mm,colback=cellbackground, colframe=cellborder]
    \prompt{In}{incolor}{20}{\boxspacing}
    \begin{Verbatim}[commandchars=\\\{\}]
    \PY{c+c1}{\PYZsh{} given weekly off is mostly empty, lets remove it}
    \PY{n}{df} \PY{o}{=} \PY{n}{df}\PY{o}{.}\PY{n}{drop}\PY{p}{(}\PY{l+s+s2}{\PYZdq{}}\PY{l+s+s2}{Weekly Off}\PY{l+s+s2}{\PYZdq{}}\PY{p}{,} \PY{n}{axis}\PY{o}{=}\PY{l+m+mi}{1}\PY{p}{)}
    \PY{n}{df}\PY{o}{.}\PY{n}{head}\PY{p}{(}\PY{p}{)}
    \end{Verbatim}
    \end{tcolorbox}
    
                \begin{tcolorbox}[breakable, size=fbox, boxrule=.5pt, pad at break*=1mm, opacityfill=0]
    \prompt{Out}{outcolor}{20}{\boxspacing}
    \begin{Verbatim}[commandchars=\\\{\}]
           Zone  State   City                  Name          Type  \textbackslash{}
    0  Northern  Delhi  Delhi            India Gate  War Memorial
    1  Northern  Delhi  Delhi        Humayun's Tomb          Tomb
    2  Northern  Delhi  Delhi     Akshardham Temple        Temple
    3  Northern  Delhi  Delhi  Waste to Wonder Park    Theme Park
    4  Northern  Delhi  Delhi         Jantar Mantar   Observatory
    
      Establishment Year  time needed to visit in hrs  Google review rating  \textbackslash{}
    0               1921                          0.5                   4.6
    1               1572                          2.0                   4.5
    2               2005                          5.0                   4.6
    3               2019                          2.0                   4.1
    4               1724                          2.0                   4.2
    
       Entrance Fee in INR Airport with 50km Radius   Significance DSLR Allowed  \textbackslash{}
    0                    0                      Yes     Historical          Yes
    1                   30                      Yes     Historical          Yes
    2                   60                      Yes      Religious           No
    3                   50                      Yes  Environmental          Yes
    4                   15                      Yes     Scientific          Yes
    
       Number of google review in lakhs Best Time to visit
    0                              2.60            Evening
    1                              0.40          Afternoon
    2                              0.40          Afternoon
    3                              0.27            Evening
    4                              0.31            Morning
    \end{Verbatim}
    \end{tcolorbox}
            
        \subsection{EDA}\label{eda}
    
        \begin{tcolorbox}[breakable, size=fbox, boxrule=1pt, pad at break*=1mm,colback=cellbackground, colframe=cellborder]
    \prompt{In}{incolor}{31}{\boxspacing}
    \begin{Verbatim}[commandchars=\\\{\}]
    \PY{c+c1}{\PYZsh{} lets see the data types of the columns}
    \PY{n}{df}\PY{o}{.}\PY{n}{dtypes}
    \end{Verbatim}
    \end{tcolorbox}
    
                \begin{tcolorbox}[breakable, size=fbox, boxrule=.5pt, pad at break*=1mm, opacityfill=0]
    \prompt{Out}{outcolor}{31}{\boxspacing}
    \begin{Verbatim}[commandchars=\\\{\}]
    Zone                                 object
    State                                object
    City                                 object
    Name                                 object
    Type                                 object
    Establishment Year                   object
    time needed to visit in hrs         float64
    Google review rating                float64
    Entrance Fee in INR                   int64
    Airport with 50km Radius             object
    Significance                         object
    DSLR Allowed                         object
    Number of google review in lakhs    float64
    Best Time to visit                   object
    dtype: object
    \end{Verbatim}
    \end{tcolorbox}
            
        \begin{tcolorbox}[breakable, size=fbox, boxrule=1pt, pad at break*=1mm,colback=cellbackground, colframe=cellborder]
    \prompt{In}{incolor}{32}{\boxspacing}
    \begin{Verbatim}[commandchars=\\\{\}]
    \PY{n}{df}\PY{o}{.}\PY{n}{info}\PY{p}{(}\PY{p}{)}
    \end{Verbatim}
    \end{tcolorbox}
    
        \begin{Verbatim}[commandchars=\\\{\}]
    <class 'pandas.core.frame.DataFrame'>
    Index: 324 entries, 0 to 324
    Data columns (total 14 columns):
     \#   Column                            Non-Null Count  Dtype
    ---  ------                            --------------  -----
     0   Zone                              324 non-null    object
     1   State                             324 non-null    object
     2   City                              324 non-null    object
     3   Name                              324 non-null    object
     4   Type                              324 non-null    object
     5   Establishment Year                324 non-null    object
     6   time needed to visit in hrs       324 non-null    float64
     7   Google review rating              324 non-null    float64
     8   Entrance Fee in INR               324 non-null    int64
     9   Airport with 50km Radius          324 non-null    object
     10  Significance                      324 non-null    object
     11  DSLR Allowed                      324 non-null    object
     12  Number of google review in lakhs  324 non-null    float64
     13  Best Time to visit                324 non-null    object
    dtypes: float64(3), int64(1), object(10)
    memory usage: 38.0+ KB
        \end{Verbatim}
    
        \begin{tcolorbox}[breakable, size=fbox, boxrule=1pt, pad at break*=1mm,colback=cellbackground, colframe=cellborder]
    \prompt{In}{incolor}{33}{\boxspacing}
    \begin{Verbatim}[commandchars=\\\{\}]
    \PY{n}{df}\PY{o}{.}\PY{n}{describe}\PY{p}{(}\PY{p}{)}
    \end{Verbatim}
    \end{tcolorbox}
    
                \begin{tcolorbox}[breakable, size=fbox, boxrule=.5pt, pad at break*=1mm, opacityfill=0]
    \prompt{Out}{outcolor}{33}{\boxspacing}
    \begin{Verbatim}[commandchars=\\\{\}]
           time needed to visit in hrs  Google review rating  Entrance Fee in INR  \textbackslash{}
    count                   324.000000            324.000000           324.000000
    mean                      1.797840              4.495679           112.620370
    std                       0.956497              0.214591           528.554154
    min                       0.500000              3.700000             0.000000
    25\%                       1.000000              4.400000             0.000000
    50\%                       1.500000              4.500000             0.000000
    75\%                       2.000000              4.600000            36.250000
    max                       7.000000              4.900000          7500.000000
    
           Number of google review in lakhs
    count                        324.000000
    mean                           0.406767
    std                            0.646965
    min                            0.010000
    25\%                            0.059000
    50\%                            0.165000
    75\%                            0.492500
    max                            7.400000
    \end{Verbatim}
    \end{tcolorbox}
            
        \begin{tcolorbox}[breakable, size=fbox, boxrule=1pt, pad at break*=1mm,colback=cellbackground, colframe=cellborder]
    \prompt{In}{incolor}{34}{\boxspacing}
    \begin{Verbatim}[commandchars=\\\{\}]
    \PY{c+c1}{\PYZsh{} lets see the highest rated places to visit, sorting by Google Review Rating}
    \PY{n}{df}\PY{o}{.}\PY{n}{sort\PYZus{}values}\PY{p}{(}\PY{l+s+s2}{\PYZdq{}}\PY{l+s+s2}{Google review rating}\PY{l+s+s2}{\PYZdq{}}\PY{p}{,} \PY{n}{ascending}\PY{o}{=}\PY{k+kc}{False}\PY{p}{)}\PY{o}{.}\PY{n}{head}\PY{p}{(}\PY{p}{)}
    \end{Verbatim}
    \end{tcolorbox}
    
                \begin{tcolorbox}[breakable, size=fbox, boxrule=.5pt, pad at break*=1mm, opacityfill=0]
    \prompt{Out}{outcolor}{34}{\boxspacing}
    \begin{Verbatim}[commandchars=\\\{\}]
             Zone           State           City                             Name  \textbackslash{}
    92   Northern          Punjab       Amritsar  Golden Temple (Harmandir Sahib)
    196  Northern          Ladakh            Leh                      Pangong Tso
    72    Western         Gujarat  Rann of Kutch                       Rann Utsav
    71    Western         Gujarat        Somnath                   Somnath Temple
    145   Central  Madhya Pradesh         Orchha                      Orchha Fort
    
                   Type Establishment Year  time needed to visit in hrs  \textbackslash{}
    92   Religious Site               1604                          1.5
    196            Lake            Unknown                          2.0
    72         Cultural            Unknown                          3.0
    71           Temple               1951                          2.0
    145            Fort               1500                          1.5
    
         Google review rating  Entrance Fee in INR Airport with 50km Radius  \textbackslash{}
    92                    4.9                    0                      Yes
    196                   4.9                   20                      Yes
    72                    4.9                 7500                      Yes
    71                    4.8                    0                       No
    145                   4.8                   10                       No
    
        Significance DSLR Allowed  Number of google review in lakhs  \textbackslash{}
    92     Spiritual          Yes                              1.90
    196       Nature          Yes                              0.15
    72      Cultural          Yes                              0.10
    71     Religious           No                              0.39
    145   Historical          Yes                              0.10
    
        Best Time to visit
    92                 All
    196            Morning
    72             Evening
    71             Morning
    145          Afternoon
    \end{Verbatim}
    \end{tcolorbox}
            
        \begin{tcolorbox}[breakable, size=fbox, boxrule=1pt, pad at break*=1mm,colback=cellbackground, colframe=cellborder]
    \prompt{In}{incolor}{39}{\boxspacing}
    \begin{Verbatim}[commandchars=\\\{\}]
    \PY{c+c1}{\PYZsh{} lets see which zone of India is most rated, being categorical, lets make a pie chart to see which has the most number of ratings}
    \PY{n}{df}\PY{p}{[}\PY{l+s+s2}{\PYZdq{}}\PY{l+s+s2}{Zone}\PY{l+s+s2}{\PYZdq{}}\PY{p}{]}\PY{o}{.}\PY{n}{value\PYZus{}counts}\PY{p}{(}\PY{p}{)}\PY{o}{.}\PY{n}{plot}\PY{o}{.}\PY{n}{pie}\PY{p}{(}\PY{n}{autopct}\PY{o}{=}\PY{l+s+s2}{\PYZdq{}}\PY{l+s+si}{\PYZpc{}1.1f}\PY{l+s+si}{\PYZpc{}\PYZpc{}}\PY{l+s+s2}{\PYZdq{}}\PY{p}{)}
    \end{Verbatim}
    \end{tcolorbox}
    
                \begin{tcolorbox}[breakable, size=fbox, boxrule=.5pt, pad at break*=1mm, opacityfill=0]
    \prompt{Out}{outcolor}{39}{\boxspacing}
    \begin{Verbatim}[commandchars=\\\{\}]
    <Axes: ylabel='count'>
    \end{Verbatim}
    \end{tcolorbox}
            
        \begin{center}
        \adjustimage{max size={0.9\linewidth}{0.9\paperheight}}{output_12_1.png}
        \end{center}
        { \hspace*{\fill} \\}
        
        \begin{tcolorbox}[breakable, size=fbox, boxrule=1pt, pad at break*=1mm,colback=cellbackground, colframe=cellborder]
    \prompt{In}{incolor}{46}{\boxspacing}
    \begin{Verbatim}[commandchars=\\\{\}]
    \PY{c+c1}{\PYZsh{} lets plot zone vs rating}
    \PY{n}{plt}\PY{o}{.}\PY{n}{figure}\PY{p}{(}\PY{n}{figsize}\PY{o}{=}\PY{p}{(}\PY{l+m+mi}{10}\PY{p}{,} \PY{l+m+mi}{6}\PY{p}{)}\PY{p}{)}
    \PY{n}{sns}\PY{o}{.}\PY{n}{boxplot}\PY{p}{(}\PY{n}{x}\PY{o}{=}\PY{l+s+s2}{\PYZdq{}}\PY{l+s+s2}{Zone}\PY{l+s+s2}{\PYZdq{}}\PY{p}{,} \PY{n}{y}\PY{o}{=}\PY{l+s+s2}{\PYZdq{}}\PY{l+s+s2}{Google review rating}\PY{l+s+s2}{\PYZdq{}}\PY{p}{,} \PY{n}{data}\PY{o}{=}\PY{n}{df}\PY{p}{,} \PY{n}{ax}\PY{o}{=}\PY{n}{plt}\PY{o}{.}\PY{n}{gca}\PY{p}{(}\PY{p}{)}\PY{p}{)}
    \PY{c+c1}{\PYZsh{} title and labels}
    \PY{n}{plt}\PY{o}{.}\PY{n}{title}\PY{p}{(}\PY{l+s+s2}{\PYZdq{}}\PY{l+s+s2}{Zone vs Google Review Rating}\PY{l+s+s2}{\PYZdq{}}\PY{p}{)}
    \PY{n}{plt}\PY{o}{.}\PY{n}{xlabel}\PY{p}{(}\PY{l+s+s2}{\PYZdq{}}\PY{l+s+s2}{Zone}\PY{l+s+s2}{\PYZdq{}}\PY{p}{)}
    \PY{n}{plt}\PY{o}{.}\PY{n}{ylabel}\PY{p}{(}\PY{l+s+s2}{\PYZdq{}}\PY{l+s+s2}{Google Review Rating}\PY{l+s+s2}{\PYZdq{}}\PY{p}{)}
    \end{Verbatim}
    \end{tcolorbox}
    
                \begin{tcolorbox}[breakable, size=fbox, boxrule=.5pt, pad at break*=1mm, opacityfill=0]
    \prompt{Out}{outcolor}{46}{\boxspacing}
    \begin{Verbatim}[commandchars=\\\{\}]
    Text(0, 0.5, 'Google Review Rating')
    \end{Verbatim}
    \end{tcolorbox}
            
        \begin{center}
        \adjustimage{max size={0.9\linewidth}{0.9\paperheight}}{output_13_1.png}
        \end{center}
        { \hspace*{\fill} \\}
        
        \begin{tcolorbox}[breakable, size=fbox, boxrule=1pt, pad at break*=1mm,colback=cellbackground, colframe=cellborder]
    \prompt{In}{incolor}{47}{\boxspacing}
    \begin{Verbatim}[commandchars=\\\{\}]
    \PY{n}{sns}\PY{o}{.}\PY{n}{catplot}\PY{p}{(}\PY{n}{data}\PY{o}{=}\PY{n}{df}\PY{p}{,} \PY{n}{x}\PY{o}{=}\PY{l+s+s2}{\PYZdq{}}\PY{l+s+s2}{Zone}\PY{l+s+s2}{\PYZdq{}}\PY{p}{,} \PY{n}{y}\PY{o}{=}\PY{l+s+s2}{\PYZdq{}}\PY{l+s+s2}{Google review rating}\PY{l+s+s2}{\PYZdq{}}\PY{p}{,} \PY{n}{kind}\PY{o}{=}\PY{l+s+s2}{\PYZdq{}}\PY{l+s+s2}{bar}\PY{l+s+s2}{\PYZdq{}}\PY{p}{,} \PY{n}{hue}\PY{o}{=}\PY{l+s+s2}{\PYZdq{}}\PY{l+s+s2}{Zone}\PY{l+s+s2}{\PYZdq{}}\PY{p}{)}
    \PY{c+c1}{\PYZsh{} title and labels}
    \PY{n}{plt}\PY{o}{.}\PY{n}{title}\PY{p}{(}\PY{l+s+s2}{\PYZdq{}}\PY{l+s+s2}{Zone vs Google Review Rating}\PY{l+s+s2}{\PYZdq{}}\PY{p}{)}
    \PY{n}{plt}\PY{o}{.}\PY{n}{xlabel}\PY{p}{(}\PY{l+s+s2}{\PYZdq{}}\PY{l+s+s2}{Zone}\PY{l+s+s2}{\PYZdq{}}\PY{p}{)}
    \PY{n}{plt}\PY{o}{.}\PY{n}{ylabel}\PY{p}{(}\PY{l+s+s2}{\PYZdq{}}\PY{l+s+s2}{Google Review Rating}\PY{l+s+s2}{\PYZdq{}}\PY{p}{)}
    \end{Verbatim}
    \end{tcolorbox}
    
                \begin{tcolorbox}[breakable, size=fbox, boxrule=.5pt, pad at break*=1mm, opacityfill=0]
    \prompt{Out}{outcolor}{47}{\boxspacing}
    \begin{Verbatim}[commandchars=\\\{\}]
    Text(25.319444444444443, 0.5, 'Google Review Rating')
    \end{Verbatim}
    \end{tcolorbox}
            
        \begin{center}
        \adjustimage{max size={0.9\linewidth}{0.9\paperheight}}{output_14_1.png}
        \end{center}
        { \hspace*{\fill} \\}
        
        so we see that most of the zones are almost equally rated and there isnt
    much difference there.
    
    we can then perform a hypothesis test to see if the ratings are
    significantly different or not
    
        \begin{tcolorbox}[breakable, size=fbox, boxrule=1pt, pad at break*=1mm,colback=cellbackground, colframe=cellborder]
    \prompt{In}{incolor}{50}{\boxspacing}
    \begin{Verbatim}[commandchars=\\\{\}]
    \PY{c+c1}{\PYZsh{} null hypothesis: the ratings are not significantly different}
    \PY{c+c1}{\PYZsh{} alternate hypothesis: the ratings are significantly different}
    
    \PY{k+kn}{from} \PY{n+nn}{scipy}\PY{n+nn}{.}\PY{n+nn}{stats} \PY{k+kn}{import} \PY{n}{f\PYZus{}oneway}  \PY{c+c1}{\PYZsh{} anova test}
    
    \PY{c+c1}{\PYZsh{} lets get the ratings for each zone}
    \PY{n}{north} \PY{o}{=} \PY{n}{df}\PY{p}{[}\PY{n}{df}\PY{p}{[}\PY{l+s+s2}{\PYZdq{}}\PY{l+s+s2}{Zone}\PY{l+s+s2}{\PYZdq{}}\PY{p}{]} \PY{o}{==} \PY{l+s+s2}{\PYZdq{}}\PY{l+s+s2}{Northern}\PY{l+s+s2}{\PYZdq{}}\PY{p}{]}\PY{p}{[}\PY{l+s+s2}{\PYZdq{}}\PY{l+s+s2}{Google review rating}\PY{l+s+s2}{\PYZdq{}}\PY{p}{]}
    \PY{n}{south} \PY{o}{=} \PY{n}{df}\PY{p}{[}\PY{n}{df}\PY{p}{[}\PY{l+s+s2}{\PYZdq{}}\PY{l+s+s2}{Zone}\PY{l+s+s2}{\PYZdq{}}\PY{p}{]} \PY{o}{==} \PY{l+s+s2}{\PYZdq{}}\PY{l+s+s2}{Southern}\PY{l+s+s2}{\PYZdq{}}\PY{p}{]}\PY{p}{[}\PY{l+s+s2}{\PYZdq{}}\PY{l+s+s2}{Google review rating}\PY{l+s+s2}{\PYZdq{}}\PY{p}{]}
    \PY{n}{east} \PY{o}{=} \PY{n}{df}\PY{p}{[}\PY{n}{df}\PY{p}{[}\PY{l+s+s2}{\PYZdq{}}\PY{l+s+s2}{Zone}\PY{l+s+s2}{\PYZdq{}}\PY{p}{]} \PY{o}{==} \PY{l+s+s2}{\PYZdq{}}\PY{l+s+s2}{Eastern}\PY{l+s+s2}{\PYZdq{}}\PY{p}{]}\PY{p}{[}\PY{l+s+s2}{\PYZdq{}}\PY{l+s+s2}{Google review rating}\PY{l+s+s2}{\PYZdq{}}\PY{p}{]}
    \PY{n}{west} \PY{o}{=} \PY{n}{df}\PY{p}{[}\PY{n}{df}\PY{p}{[}\PY{l+s+s2}{\PYZdq{}}\PY{l+s+s2}{Zone}\PY{l+s+s2}{\PYZdq{}}\PY{p}{]} \PY{o}{==} \PY{l+s+s2}{\PYZdq{}}\PY{l+s+s2}{Western}\PY{l+s+s2}{\PYZdq{}}\PY{p}{]}\PY{p}{[}\PY{l+s+s2}{\PYZdq{}}\PY{l+s+s2}{Google review rating}\PY{l+s+s2}{\PYZdq{}}\PY{p}{]}
    \PY{n}{central} \PY{o}{=} \PY{n}{df}\PY{p}{[}\PY{n}{df}\PY{p}{[}\PY{l+s+s2}{\PYZdq{}}\PY{l+s+s2}{Zone}\PY{l+s+s2}{\PYZdq{}}\PY{p}{]} \PY{o}{==} \PY{l+s+s2}{\PYZdq{}}\PY{l+s+s2}{Central}\PY{l+s+s2}{\PYZdq{}}\PY{p}{]}\PY{p}{[}\PY{l+s+s2}{\PYZdq{}}\PY{l+s+s2}{Google review rating}\PY{l+s+s2}{\PYZdq{}}\PY{p}{]}
    \PY{n}{north\PYZus{}east} \PY{o}{=} \PY{n}{df}\PY{p}{[}\PY{n}{df}\PY{p}{[}\PY{l+s+s2}{\PYZdq{}}\PY{l+s+s2}{Zone}\PY{l+s+s2}{\PYZdq{}}\PY{p}{]} \PY{o}{==} \PY{l+s+s2}{\PYZdq{}}\PY{l+s+s2}{North Eastern}\PY{l+s+s2}{\PYZdq{}}\PY{p}{]}\PY{p}{[}\PY{l+s+s2}{\PYZdq{}}\PY{l+s+s2}{Google review rating}\PY{l+s+s2}{\PYZdq{}}\PY{p}{]}
    
    \PY{c+c1}{\PYZsh{} lets perform the test}
    \PY{n}{f\PYZus{}oneway}\PY{p}{(}\PY{n}{north}\PY{p}{,} \PY{n}{south}\PY{p}{,} \PY{n}{east}\PY{p}{,} \PY{n}{west}\PY{p}{,} \PY{n}{central}\PY{p}{,} \PY{n}{north\PYZus{}east}\PY{p}{)}
    \end{Verbatim}
    \end{tcolorbox}
    
                \begin{tcolorbox}[breakable, size=fbox, boxrule=.5pt, pad at break*=1mm, opacityfill=0]
    \prompt{Out}{outcolor}{50}{\boxspacing}
    \begin{Verbatim}[commandchars=\\\{\}]
    F\_onewayResult(statistic=1.598487683716663, pvalue=0.16009417988080096)
    \end{Verbatim}
    \end{tcolorbox}
            
        since p value is more than 0.05, we fail to reject the null hypothesis,
    which means the ratings are not significantly different
    
        \begin{tcolorbox}[breakable, size=fbox, boxrule=1pt, pad at break*=1mm,colback=cellbackground, colframe=cellborder]
    \prompt{In}{incolor}{52}{\boxspacing}
    \begin{Verbatim}[commandchars=\\\{\}]
    \PY{n}{df}\PY{o}{.}\PY{n}{head}\PY{p}{(}\PY{p}{)}
    \end{Verbatim}
    \end{tcolorbox}
    
                \begin{tcolorbox}[breakable, size=fbox, boxrule=.5pt, pad at break*=1mm, opacityfill=0]
    \prompt{Out}{outcolor}{52}{\boxspacing}
    \begin{Verbatim}[commandchars=\\\{\}]
           Zone  State   City                  Name          Type  \textbackslash{}
    0  Northern  Delhi  Delhi            India Gate  War Memorial
    1  Northern  Delhi  Delhi        Humayun's Tomb          Tomb
    2  Northern  Delhi  Delhi     Akshardham Temple        Temple
    3  Northern  Delhi  Delhi  Waste to Wonder Park    Theme Park
    4  Northern  Delhi  Delhi         Jantar Mantar   Observatory
    
      Establishment Year  time needed to visit in hrs  Google review rating  \textbackslash{}
    0               1921                          0.5                   4.6
    1               1572                          2.0                   4.5
    2               2005                          5.0                   4.6
    3               2019                          2.0                   4.1
    4               1724                          2.0                   4.2
    
       Entrance Fee in INR Airport with 50km Radius   Significance DSLR Allowed  \textbackslash{}
    0                    0                      Yes     Historical          Yes
    1                   30                      Yes     Historical          Yes
    2                   60                      Yes      Religious           No
    3                   50                      Yes  Environmental          Yes
    4                   15                      Yes     Scientific          Yes
    
       Number of google review in lakhs Best Time to visit
    0                              2.60            Evening
    1                              0.40          Afternoon
    2                              0.40          Afternoon
    3                              0.27            Evening
    4                              0.31            Morning
    \end{Verbatim}
    \end{tcolorbox}
            
        \begin{tcolorbox}[breakable, size=fbox, boxrule=1pt, pad at break*=1mm,colback=cellbackground, colframe=cellborder]
    \prompt{In}{incolor}{66}{\boxspacing}
    \begin{Verbatim}[commandchars=\\\{\}]
    \PY{c+c1}{\PYZsh{} lets see what places take the most time to visit}
    \PY{n}{df}\PY{o}{.}\PY{n}{sort\PYZus{}values}\PY{p}{(}\PY{l+s+s2}{\PYZdq{}}\PY{l+s+s2}{time needed to visit in hrs}\PY{l+s+s2}{\PYZdq{}}\PY{p}{,} \PY{n}{ascending}\PY{o}{=}\PY{k+kc}{False}\PY{p}{)}\PY{o}{.}\PY{n}{head}\PY{p}{(}\PY{p}{)}
    
    \PY{c+c1}{\PYZsh{} plottin a simple hist plot to see the relationship between time to visit and google review rating}
    \PY{n}{plt}\PY{o}{.}\PY{n}{figure}\PY{p}{(}\PY{n}{figsize}\PY{o}{=}\PY{p}{(}\PY{l+m+mi}{10}\PY{p}{,} \PY{l+m+mi}{6}\PY{p}{)}\PY{p}{)}
    \PY{n}{sns}\PY{o}{.}\PY{n}{histplot}\PY{p}{(}\PY{n}{y}\PY{o}{=}\PY{l+s+s2}{\PYZdq{}}\PY{l+s+s2}{time needed to visit in hrs}\PY{l+s+s2}{\PYZdq{}}\PY{p}{,} \PY{n}{x}\PY{o}{=}\PY{l+s+s2}{\PYZdq{}}\PY{l+s+s2}{Zone}\PY{l+s+s2}{\PYZdq{}}\PY{p}{,} \PY{n}{data}\PY{o}{=}\PY{n}{df}\PY{p}{,} \PY{n}{ax}\PY{o}{=}\PY{n}{plt}\PY{o}{.}\PY{n}{gca}\PY{p}{(}\PY{p}{)}\PY{p}{)}
    \PY{c+c1}{\PYZsh{} title and labels}
    \PY{n}{plt}\PY{o}{.}\PY{n}{title}\PY{p}{(}\PY{l+s+s2}{\PYZdq{}}\PY{l+s+s2}{Time to Visit vs Zone}\PY{l+s+s2}{\PYZdq{}}\PY{p}{)}
    \PY{n}{plt}\PY{o}{.}\PY{n}{xlabel}\PY{p}{(}\PY{l+s+s2}{\PYZdq{}}\PY{l+s+s2}{Zone}\PY{l+s+s2}{\PYZdq{}}\PY{p}{)}
    \PY{n}{plt}\PY{o}{.}\PY{n}{ylabel}\PY{p}{(}\PY{l+s+s2}{\PYZdq{}}\PY{l+s+s2}{Time Needed in Hours to Visit }\PY{l+s+s2}{\PYZdq{}}\PY{p}{)}
    \end{Verbatim}
    \end{tcolorbox}
    
                \begin{tcolorbox}[breakable, size=fbox, boxrule=.5pt, pad at break*=1mm, opacityfill=0]
    \prompt{Out}{outcolor}{66}{\boxspacing}
    \begin{Verbatim}[commandchars=\\\{\}]
    Text(0, 0.5, 'Time Needed in Hours to Visit ')
    \end{Verbatim}
    \end{tcolorbox}
            
        \begin{center}
        \adjustimage{max size={0.9\linewidth}{0.9\paperheight}}{output_19_1.png}
        \end{center}
        { \hspace*{\fill} \\}
        
        above graphs shows us that north and south often take around 2 hours,
    south takes the least time, while eastern, celtral and north eastern can
    be visited mostly in half to 3 hours.
    
        \begin{tcolorbox}[breakable, size=fbox, boxrule=1pt, pad at break*=1mm,colback=cellbackground, colframe=cellborder]
    \prompt{In}{incolor}{77}{\boxspacing}
    \begin{Verbatim}[commandchars=\\\{\}]
    \PY{c+c1}{\PYZsh{} lets see this better by binning the time needed to visit between 0 and 1, 1 and 2, 2 and 3, 3 and 4, 4 and 5}
    \PY{n}{df}\PY{p}{[}\PY{l+s+s2}{\PYZdq{}}\PY{l+s+s2}{time needed to visit in hrs}\PY{l+s+s2}{\PYZdq{}}\PY{p}{]}\PY{o}{.}\PY{n}{value\PYZus{}counts}\PY{p}{(}\PY{n}{bins}\PY{o}{=}\PY{l+m+mi}{5}\PY{p}{)}
    \PY{c+c1}{\PYZsh{} lets plot}
    \PY{n}{plt}\PY{o}{.}\PY{n}{figure}\PY{p}{(}\PY{n}{figsize}\PY{o}{=}\PY{p}{(}\PY{l+m+mi}{10}\PY{p}{,} \PY{l+m+mi}{6}\PY{p}{)}\PY{p}{)}
    \PY{n}{sns}\PY{o}{.}\PY{n}{histplot}\PY{p}{(}
        \PY{n}{x}\PY{o}{=}\PY{l+s+s2}{\PYZdq{}}\PY{l+s+s2}{time needed to visit in hrs}\PY{l+s+s2}{\PYZdq{}}\PY{p}{,}
        \PY{n}{data}\PY{o}{=}\PY{n}{df}\PY{p}{,}
        \PY{n}{bins}\PY{o}{=}\PY{l+m+mi}{5}\PY{p}{,}
        \PY{n}{ax}\PY{o}{=}\PY{n}{plt}\PY{o}{.}\PY{n}{gca}\PY{p}{(}\PY{p}{)}\PY{p}{,}
        \PY{n}{hue}\PY{o}{=}\PY{l+s+s2}{\PYZdq{}}\PY{l+s+s2}{Zone}\PY{l+s+s2}{\PYZdq{}}\PY{p}{,}
        \PY{n}{palette}\PY{o}{=}\PY{n}{sns}\PY{o}{.}\PY{n}{color\PYZus{}palette}\PY{p}{(}\PY{l+s+s2}{\PYZdq{}}\PY{l+s+s2}{husl}\PY{l+s+s2}{\PYZdq{}}\PY{p}{,} \PY{l+m+mi}{6}\PY{p}{)}\PY{p}{,}
    \PY{p}{)}
    \PY{c+c1}{\PYZsh{} title and labels}
    \PY{n}{plt}\PY{o}{.}\PY{n}{title}\PY{p}{(}\PY{l+s+s2}{\PYZdq{}}\PY{l+s+s2}{Time to Visit in Hours}\PY{l+s+s2}{\PYZdq{}}\PY{p}{)}
    \PY{n}{plt}\PY{o}{.}\PY{n}{xlabel}\PY{p}{(}\PY{l+s+s2}{\PYZdq{}}\PY{l+s+s2}{Time Needed in Hours to Visit }\PY{l+s+s2}{\PYZdq{}}\PY{p}{)}
    \PY{n}{plt}\PY{o}{.}\PY{n}{ylabel}\PY{p}{(}\PY{l+s+s2}{\PYZdq{}}\PY{l+s+s2}{Count}\PY{l+s+s2}{\PYZdq{}}\PY{p}{)}
    \end{Verbatim}
    \end{tcolorbox}
    
                \begin{tcolorbox}[breakable, size=fbox, boxrule=.5pt, pad at break*=1mm, opacityfill=0]
    \prompt{Out}{outcolor}{77}{\boxspacing}
    \begin{Verbatim}[commandchars=\\\{\}]
    Text(0, 0.5, 'Count')
    \end{Verbatim}
    \end{tcolorbox}
            
        \begin{center}
        \adjustimage{max size={0.9\linewidth}{0.9\paperheight}}{output_21_1.png}
        \end{center}
        { \hspace*{\fill} \\}
        
        this shows us that most places can be visted in half to 2 hours, while
    only some may take more. Places in south take the least time
    
        \begin{tcolorbox}[breakable, size=fbox, boxrule=1pt, pad at break*=1mm,colback=cellbackground, colframe=cellborder]
    \prompt{In}{incolor}{78}{\boxspacing}
    \begin{Verbatim}[commandchars=\\\{\}]
    \PY{n}{df}\PY{o}{.}\PY{n}{head}\PY{p}{(}\PY{p}{)}
    \end{Verbatim}
    \end{tcolorbox}
    
                \begin{tcolorbox}[breakable, size=fbox, boxrule=.5pt, pad at break*=1mm, opacityfill=0]
    \prompt{Out}{outcolor}{78}{\boxspacing}
    \begin{Verbatim}[commandchars=\\\{\}]
           Zone  State   City                  Name          Type  \textbackslash{}
    0  Northern  Delhi  Delhi            India Gate  War Memorial
    1  Northern  Delhi  Delhi        Humayun's Tomb          Tomb
    2  Northern  Delhi  Delhi     Akshardham Temple        Temple
    3  Northern  Delhi  Delhi  Waste to Wonder Park    Theme Park
    4  Northern  Delhi  Delhi         Jantar Mantar   Observatory
    
      Establishment Year  time needed to visit in hrs  Google review rating  \textbackslash{}
    0               1921                          0.5                   4.6
    1               1572                          2.0                   4.5
    2               2005                          5.0                   4.6
    3               2019                          2.0                   4.1
    4               1724                          2.0                   4.2
    
       Entrance Fee in INR Airport with 50km Radius   Significance DSLR Allowed  \textbackslash{}
    0                    0                      Yes     Historical          Yes
    1                   30                      Yes     Historical          Yes
    2                   60                      Yes      Religious           No
    3                   50                      Yes  Environmental          Yes
    4                   15                      Yes     Scientific          Yes
    
       Number of google review in lakhs Best Time to visit
    0                              2.60            Evening
    1                              0.40          Afternoon
    2                              0.40          Afternoon
    3                              0.27            Evening
    4                              0.31            Morning
    \end{Verbatim}
    \end{tcolorbox}
            
        \begin{tcolorbox}[breakable, size=fbox, boxrule=1pt, pad at break*=1mm,colback=cellbackground, colframe=cellborder]
    \prompt{In}{incolor}{82}{\boxspacing}
    \begin{Verbatim}[commandchars=\\\{\}]
    \PY{c+c1}{\PYZsh{} lets see what the best time to visit is with a bar chart}
    \PY{n}{df}\PY{p}{[}\PY{l+s+s2}{\PYZdq{}}\PY{l+s+s2}{Best Time to visit}\PY{l+s+s2}{\PYZdq{}}\PY{p}{]}\PY{o}{.}\PY{n}{value\PYZus{}counts}\PY{p}{(}\PY{p}{)}\PY{o}{.}\PY{n}{plot}\PY{o}{.}\PY{n}{bar}\PY{p}{(}\PY{p}{)}
    \PY{c+c1}{\PYZsh{} df[\PYZdq{}Best Time to visit\PYZdq{}].value\PYZus{}counts().plot.pie(autopct=\PYZdq{}\PYZpc{}1.1f\PYZpc{}\PYZpc{}\PYZdq{})}
    \end{Verbatim}
    \end{tcolorbox}
    
                \begin{tcolorbox}[breakable, size=fbox, boxrule=.5pt, pad at break*=1mm, opacityfill=0]
    \prompt{Out}{outcolor}{82}{\boxspacing}
    \begin{Verbatim}[commandchars=\\\{\}]
    <Axes: xlabel='Best Time to visit'>
    \end{Verbatim}
    \end{tcolorbox}
            
        \begin{center}
        \adjustimage{max size={0.9\linewidth}{0.9\paperheight}}{output_24_1.png}
        \end{center}
        { \hspace*{\fill} \\}
        
        \begin{tcolorbox}[breakable, size=fbox, boxrule=1pt, pad at break*=1mm,colback=cellbackground, colframe=cellborder]
    \prompt{In}{incolor}{84}{\boxspacing}
    \begin{Verbatim}[commandchars=\\\{\}]
    \PY{c+c1}{\PYZsh{} lets see what the best time to visit places in delhi are}
    \PY{n}{df}\PY{p}{[}\PY{n}{df}\PY{p}{[}\PY{l+s+s2}{\PYZdq{}}\PY{l+s+s2}{City}\PY{l+s+s2}{\PYZdq{}}\PY{p}{]} \PY{o}{==} \PY{l+s+s2}{\PYZdq{}}\PY{l+s+s2}{Delhi}\PY{l+s+s2}{\PYZdq{}}\PY{p}{]}\PY{p}{[}\PY{l+s+s2}{\PYZdq{}}\PY{l+s+s2}{Best Time to visit}\PY{l+s+s2}{\PYZdq{}}\PY{p}{]}\PY{o}{.}\PY{n}{value\PYZus{}counts}\PY{p}{(}\PY{p}{)}\PY{o}{.}\PY{n}{plot}\PY{o}{.}\PY{n}{pie}\PY{p}{(}
        \PY{n}{autopct}\PY{o}{=}\PY{l+s+s2}{\PYZdq{}}\PY{l+s+si}{\PYZpc{}1.1f}\PY{l+s+si}{\PYZpc{}\PYZpc{}}\PY{l+s+s2}{\PYZdq{}}
    \PY{p}{)}
    \end{Verbatim}
    \end{tcolorbox}
    
                \begin{tcolorbox}[breakable, size=fbox, boxrule=.5pt, pad at break*=1mm, opacityfill=0]
    \prompt{Out}{outcolor}{84}{\boxspacing}
    \begin{Verbatim}[commandchars=\\\{\}]
    <Axes: ylabel='count'>
    \end{Verbatim}
    \end{tcolorbox}
            
        \begin{center}
        \adjustimage{max size={0.9\linewidth}{0.9\paperheight}}{output_25_1.png}
        \end{center}
        { \hspace*{\fill} \\}
        
        \begin{tcolorbox}[breakable, size=fbox, boxrule=1pt, pad at break*=1mm,colback=cellbackground, colframe=cellborder]
    \prompt{In}{incolor}{85}{\boxspacing}
    \begin{Verbatim}[commandchars=\\\{\}]
    \PY{n}{df}\PY{o}{.}\PY{n}{head}\PY{p}{(}\PY{p}{)}
    \end{Verbatim}
    \end{tcolorbox}
    
                \begin{tcolorbox}[breakable, size=fbox, boxrule=.5pt, pad at break*=1mm, opacityfill=0]
    \prompt{Out}{outcolor}{85}{\boxspacing}
    \begin{Verbatim}[commandchars=\\\{\}]
           Zone  State   City                  Name          Type  \textbackslash{}
    0  Northern  Delhi  Delhi            India Gate  War Memorial
    1  Northern  Delhi  Delhi        Humayun's Tomb          Tomb
    2  Northern  Delhi  Delhi     Akshardham Temple        Temple
    3  Northern  Delhi  Delhi  Waste to Wonder Park    Theme Park
    4  Northern  Delhi  Delhi         Jantar Mantar   Observatory
    
      Establishment Year  time needed to visit in hrs  Google review rating  \textbackslash{}
    0               1921                          0.5                   4.6
    1               1572                          2.0                   4.5
    2               2005                          5.0                   4.6
    3               2019                          2.0                   4.1
    4               1724                          2.0                   4.2
    
       Entrance Fee in INR Airport with 50km Radius   Significance DSLR Allowed  \textbackslash{}
    0                    0                      Yes     Historical          Yes
    1                   30                      Yes     Historical          Yes
    2                   60                      Yes      Religious           No
    3                   50                      Yes  Environmental          Yes
    4                   15                      Yes     Scientific          Yes
    
       Number of google review in lakhs Best Time to visit
    0                              2.60            Evening
    1                              0.40          Afternoon
    2                              0.40          Afternoon
    3                              0.27            Evening
    4                              0.31            Morning
    \end{Verbatim}
    \end{tcolorbox}
            
        \begin{tcolorbox}[breakable, size=fbox, boxrule=1pt, pad at break*=1mm,colback=cellbackground, colframe=cellborder]
    \prompt{In}{incolor}{103}{\boxspacing}
    \begin{Verbatim}[commandchars=\\\{\}]
    \PY{n}{top\PYZus{}5\PYZus{}types} \PY{o}{=} \PY{n}{df}\PY{p}{[}\PY{l+s+s2}{\PYZdq{}}\PY{l+s+s2}{Type}\PY{l+s+s2}{\PYZdq{}}\PY{p}{]}\PY{o}{.}\PY{n}{value\PYZus{}counts}\PY{p}{(}\PY{p}{)}\PY{o}{.}\PY{n}{head}\PY{p}{(}\PY{p}{)}
    \PY{c+c1}{\PYZsh{} lets filter df by these}
    \PY{n}{df\PYZus{}top\PYZus{}5} \PY{o}{=} \PY{n}{df}\PY{p}{[}\PY{n}{df}\PY{p}{[}\PY{l+s+s2}{\PYZdq{}}\PY{l+s+s2}{Type}\PY{l+s+s2}{\PYZdq{}}\PY{p}{]}\PY{o}{.}\PY{n}{isin}\PY{p}{(}\PY{n}{top\PYZus{}5\PYZus{}types}\PY{o}{.}\PY{n}{index}\PY{p}{)}\PY{p}{]}
    \end{Verbatim}
    \end{tcolorbox}
    
        \begin{tcolorbox}[breakable, size=fbox, boxrule=1pt, pad at break*=1mm,colback=cellbackground, colframe=cellborder]
    \prompt{In}{incolor}{104}{\boxspacing}
    \begin{Verbatim}[commandchars=\\\{\}]
    \PY{c+c1}{\PYZsh{} lets see the top 5 kinds of places by making a pie chart}
    \PY{n}{df\PYZus{}top\PYZus{}5}\PY{p}{[}\PY{l+s+s2}{\PYZdq{}}\PY{l+s+s2}{Type}\PY{l+s+s2}{\PYZdq{}}\PY{p}{]}\PY{o}{.}\PY{n}{value\PYZus{}counts}\PY{p}{(}\PY{p}{)}\PY{o}{.}\PY{n}{plot}\PY{o}{.}\PY{n}{pie}\PY{p}{(}\PY{n}{autopct}\PY{o}{=}\PY{l+s+s2}{\PYZdq{}}\PY{l+s+si}{\PYZpc{}1.1f}\PY{l+s+si}{\PYZpc{}\PYZpc{}}\PY{l+s+s2}{\PYZdq{}}\PY{p}{)}
    \end{Verbatim}
    \end{tcolorbox}
    
                \begin{tcolorbox}[breakable, size=fbox, boxrule=.5pt, pad at break*=1mm, opacityfill=0]
    \prompt{Out}{outcolor}{104}{\boxspacing}
    \begin{Verbatim}[commandchars=\\\{\}]
    <Axes: ylabel='count'>
    \end{Verbatim}
    \end{tcolorbox}
            
        \begin{center}
        \adjustimage{max size={0.9\linewidth}{0.9\paperheight}}{output_28_1.png}
        \end{center}
        { \hspace*{\fill} \\}
        
        \begin{tcolorbox}[breakable, size=fbox, boxrule=1pt, pad at break*=1mm,colback=cellbackground, colframe=cellborder]
    \prompt{In}{incolor}{105}{\boxspacing}
    \begin{Verbatim}[commandchars=\\\{\}]
    \PY{c+c1}{\PYZsh{} lets now see which are the most highly rated}
    \PY{n}{sns}\PY{o}{.}\PY{n}{catplot}\PY{p}{(}\PY{n}{data}\PY{o}{=}\PY{n}{df\PYZus{}top\PYZus{}5}\PY{p}{,} \PY{n}{x}\PY{o}{=}\PY{l+s+s2}{\PYZdq{}}\PY{l+s+s2}{Type}\PY{l+s+s2}{\PYZdq{}}\PY{p}{,} \PY{n}{y}\PY{o}{=}\PY{l+s+s2}{\PYZdq{}}\PY{l+s+s2}{Google review rating}\PY{l+s+s2}{\PYZdq{}}\PY{p}{,} \PY{n}{kind}\PY{o}{=}\PY{l+s+s2}{\PYZdq{}}\PY{l+s+s2}{bar}\PY{l+s+s2}{\PYZdq{}}\PY{p}{,} \PY{n}{hue}\PY{o}{=}\PY{l+s+s2}{\PYZdq{}}\PY{l+s+s2}{Type}\PY{l+s+s2}{\PYZdq{}}\PY{p}{)}
    \end{Verbatim}
    \end{tcolorbox}
    
                \begin{tcolorbox}[breakable, size=fbox, boxrule=.5pt, pad at break*=1mm, opacityfill=0]
    \prompt{Out}{outcolor}{105}{\boxspacing}
    \begin{Verbatim}[commandchars=\\\{\}]
    <seaborn.axisgrid.FacetGrid at 0x1d85df57100>
    \end{Verbatim}
    \end{tcolorbox}
            
        \begin{center}
        \adjustimage{max size={0.9\linewidth}{0.9\paperheight}}{output_29_1.png}
        \end{center}
        { \hspace*{\fill} \\}
        
        \begin{tcolorbox}[breakable, size=fbox, boxrule=1pt, pad at break*=1mm,colback=cellbackground, colframe=cellborder]
    \prompt{In}{incolor}{ }{\boxspacing}
    \begin{Verbatim}[commandchars=\\\{\}]
    
    \end{Verbatim}
    \end{tcolorbox}
    
    
\clearpage
\section{FAQs}

\subsection{Question 1}
\begin{enumerate}
    \item \textbf{What do you understand by Statistics for Data science?}\\
          Statistics for data science involves the application of statistical methods and techniques to analyze, interpret, and derive insights from data. It encompasses a wide range of methods, including descriptive statistics, inferential statistics, and predictive modeling, to explore patterns, relationships, and trends within datasets. In data science, statistics plays a crucial role in data preprocessing, exploratory data analysis, hypothesis testing, and model evaluation, enabling data scientists to make informed decisions and derive actionable insights from data.

\end{enumerate}

\subsection{Question 2}
\begin{enumerate}
    \item \textbf{Do we need preprocessing to perform statistics for Data science? Justify, your answer}\\
          Yes, preprocessing is essential for performing statistics in data science. Preprocessing involves cleaning, transforming, and preparing raw data to make it suitable for statistical analysis. Without preprocessing, raw data may contain missing values, outliers, inconsistencies, or other irregularities that can affect the accuracy and reliability of statistical results. Preprocessing techniques such as handling missing values, outlier detection, data normalization, and feature engineering help ensure that the data meets the assumptions and requirements of statistical methods. By preprocessing the data, data scientists can improve the quality of statistical analysis, enhance the performance of models, and derive more accurate and meaningful insights from the data.
\end{enumerate}

\subsection{Question 3}
\begin{enumerate}
    \item \textbf{Describe the different Statistical approaches in Data science using Python?}\\
          In data science, various statistical approaches are used to analyze and model data using Python. Some common statistical approaches include:
          \begin{itemize}
              \item Descriptive Statistics: Summarizing and describing the features of a dataset using measures of central tendency, dispersion, and visualization techniques.
              \item Inferential Statistics: Making inferences and predictions about populations based on sample data using hypothesis testing, confidence intervals, and regression analysis.
              \item Predictive Modeling: Building predictive models to forecast future outcomes or classify data into different categories using techniques such as linear regression, logistic regression, decision trees, and ensemble methods.
              \item Time Series Analysis: Analyzing and forecasting time-series data using methods like autoregressive integrated moving average (ARIMA), seasonal decomposition, and exponential smoothing.
          \end{itemize}
          Python provides a wide range of libraries and tools for implementing these statistical approaches, including NumPy, pandas, SciPy, scikit-learn, and Statsmodels, making it a popular choice for statistical analysis in data science.
\end{enumerate}

\section{Conclusion}
In this assignment, we learned about various statistical approaches used in data science, including measures of central tendency, dispersion, hypothesis testing, regression analysis, and correlation analysis. We implemented these statistical concepts using Python and explored how they can be applied to analyze and interpret data. By understanding and applying statistical methods, data scientists can gain valuable insights from data, make informed decisions, and build predictive models to solve real-world problems.
\clearpage

\pagebreak
% \begin{thebibliography}{}

%     \bibitem{CloudConceptsOverview}
%     Cloud Computing Concepts Overview.
%     Accessed from: \url{https://www.ibm.com/cloud/learn/cloud-computing-concepts}

%     \bibitem{VirtualizationBenefits}
%     Benefits of Virtualization.
%     Accessed from: \url{https://www.vmware.com/topics/glossary/content/virtualization}

% \end{thebibliography}

\end{document}