% This is a Basic Assignment Paper but with like Code and stuff allowed in it, there is also url, hyperlinks from contents included. 

\documentclass[openany]{report}

% Preamble

\usepackage[margin=1in]{geometry}
\usepackage{amsfonts, amsmath, amssymb}
\usepackage{fancyhdr, float, graphicx}
\usepackage[utf8]{inputenc} % Required for inputting international characters
\usepackage[T1]{fontenc} % Output font encoding for international characters
\usepackage{fouriernc} % Use the New Century Schoolbook font
\usepackage[nottoc, notlot, notlof]{tocbibind}
\usepackage{listings}
\usepackage{xcolor}
\usepackage{blindtext}
\usepackage{hyperref}
\hypersetup{
    colorlinks=true,
    linkcolor=black,
    filecolor=magenta,      
    urlcolor=blue,
    pdfpagemode=FullScreen,
    }

\definecolor{codegreen}{rgb}{0,0.6,0}
\definecolor{codegray}{rgb}{0.5,0.5,0.5}
\definecolor{codepurple}{rgb}{0.58,0,0.82}
\definecolor{backcolour}{rgb}{0.95,0.95,0.92}

\lstdefinestyle{mystyle}{
    backgroundcolor=\color{backcolour},   
    commentstyle=\color{codegreen},
    keywordstyle=\color{magenta},
    numberstyle=\tiny\color{codegray},
    stringstyle=\color{codepurple},
    basicstyle=\ttfamily\footnotesize,
    breakatwhitespace=false,         
    breaklines=true,                 
    captionpos=b,                    
    keepspaces=true,                 
    numbers=left,                    
    numbersep=5pt,                  
    showspaces=false,                
    showstringspaces=false,
    showtabs=false,                  
    tabsize=2
}

\lstset{style=mystyle}

% Header and Footer
\pagestyle{fancy}
\fancyhead{}
\fancyfoot{}
\fancyhead[L]{\textit{\Large{Data Science For Cybersecurity and Forensics - 3nd Year B. Tech}}}
\fancyhead[R]{\textit{Krishnaraj T}}
\fancyfoot[C]{\thepage}
\renewcommand{\footrulewidth}{1pt}

% Other Doc Editing
% \parindent 0ex
%\renewcommand{\baselinestretch}{1.5}

\begin{document}

\begin{titlepage}
    \centering

    %---------------------------NAMES-------------------------------

    \huge\textsc{
        Dr. Vishwanath Karad MIT World Peace University, Pune
    }\\

    \vspace{0.75\baselineskip} % space after Uni Name

    \LARGE{
        Data Science for Cybersecurity\\
        Third Year B. Tech, Semester 6\\
    }

    \vfill % space after Sub Name

    %--------------------------TITLE-------------------------------

    \rule{\textwidth}{1.6pt}\vspace*{-\baselineskip}\vspace*{2pt}
    \rule{\textwidth}{0.6pt}
    \vspace{0.75\baselineskip} % Whitespace above the title



    \huge{\textsc{
            PC Usage Analyzer
        }} \\



    \vspace{0.5\baselineskip} % Whitespace below the title
    \rule{\textwidth}{0.6pt}\vspace*{-\baselineskip}\vspace*{2.8pt}
    \rule{\textwidth}{1.6pt}

    \vspace{1\baselineskip} % Whitespace after the title block

    %--------------------------SUBTITLE --------------------------	

    \LARGE\textsc{
        Mini Project Report
    } % Subtitle or further description

    %--------------------------AUTHOR-------------------------------

    \vspace{0.5\baselineskip} % Whitespace below the editor list
    Under the Guidance of\\
    \Large{
        \textbf{Prof. Sunita Warjri}
    }
    \vfill

    Prepared By
    \vspace{0.5\baselineskip} % Whitespace before the editors

    \Large{
        Krishnaraj Thadesar, PA10, 1032210888\\
    }
    \vspace{0.5\baselineskip} % Whitespace before the editors
    \LARGE{
        Department of School of Computer Engineering and Technology\\
        Maharashtra, India.\\
        2023-2024\\
    }
    % \vspace{0.5\baselineskip} % Whitespace below the editor list
    \today

\end{titlepage}


\tableofcontents
\thispagestyle{empty}
% \frontmatter
\clearpage

\chapter*{Acknowledgment}
\thispagestyle{empty}

I would like to express my deepest appreciation to all those who provided me the possibility to complete this report. A special gratitude I give to our mentor, Prof. Sunita Warjri, whose contribution in stimulating suggestions and encouragement, helped me to coordinate my project especially in writing this report.\\

Furthermore, I would also like to acknowledge with much appreciation the crucial role of the staff of MIT WPU, who gave the permission to use all required equipment and the necessary materials to complete the task. A special thanks goes to my team mates,who helped me enormously to assemble the parts and gave suggestion about the task of using the techniques of measurements.\\

I have to appreciate the guidance given by other supervisor as well as the panels especially in our project presentation that has improved our presentation skills thanks to their comment and advices.\\

I would also like to thank my parents for their wise counsel and sympathetic ear. You are always there for me. Finally, I wish to thank my friends for their support and encouragement throughout my study.



\section*{Name of Student}
\begin{enumerate}
    \item Krishnaraj Thadesar, PA10, 1032210888
\end{enumerate}

\thispagestyle{empty}
\clearpage

\chapter*{Abstract}
We spend a lot of time on our computers, and it can be interesting to see how we use them. This project aims to analyze the usage of a computer by monitoring the applications used, the time spent on each application, and the frequency of usage. The project will involve developing a tool that can track the user's activities on the computer and generate reports based on the data collected. The tool will provide insights into the user's behavior and help identify patterns in computer usage. The project will also explore the privacy implications of monitoring computer usage and discuss ways to protect user data. \\

The project will be implemented using Python and will involve developing scripts to capture and analyze computer usage data. The project will provide a valuable resource for users to understand their computer usage patterns and make informed decisions about their digital habits.\\


For ease of use, Django will be used to create a web interface for the tool, allowing users to view their computer usage data and generate reports. The project will also explore the ethical considerations of monitoring computer usage and discuss the implications of collecting and analyzing user data.\\


This project was intented as a mini project for the Data Science for Cybersecurity and Forensics course at Dr. Vishwanath Karad MIT World Peace University, Pune. The project aims to provide a practical application of data science techniques in the field of cybersecurity and forensics and to explore the potential of monitoring computer usage as a tool for improving digital habits and privacy awareness. But its root and inspiration was from having played a lot of games, and realizing that an analysis of the time spent on the computer could be interesting.

\section{Keywords}
Computer Usage, Monitoring, Analysis, Python, Privacy, Data Collection, Insights, Patterns, Digital Habits.

\thispagestyle{empty}
\clearpage

\listoffigures
\clearpage
\listoftables
\clearpage
\setcounter{page}{1}

\chapter{Introduction}

\section{Problem Statement}

\section{Need of the Project}


\chapter{Literature Survey}

\chapter{Methodology, Algorithms and Implementations}

\section{Methodology}

\section{Algorithms}

\section{Implementation}

\section{Platform}
\textbf{Operating System}: Windows 11 Pro x86 \\
\textbf{IDEs or Text Editors Used}: Visual Studio Code\\
\textbf{Compilers or Interpreters}: Python 3.10.1\\

\section{Screenshots}

\chapter{Code Review}

\chapter{Conclusion}

\chapter{Future Prospects}
\begin{enumerate}
    \item
\end{enumerate}

\clearpage
\begin{thebibliography}{10}
    \bibitem{phishing101}
    Jakobsson, M., \& Myers, S. (2007). Phishing and Countermeasures: Understanding the Increasing Problem of Electronic Identity Theft. Wiley Publishing.

\end{thebibliography}

\end{document}
